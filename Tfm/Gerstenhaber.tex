\documentclass[TFM.tex]{subfiles}

\begin{document}

%\hyphenation{equi-va-len-cia}\hyphenation{pro-pie-dad}\hyphenation{res-pec-ti-va-men-te}\hyphenation{sub-es-pa-cio}
\chapter{Hochschild cohomology of a ring}

In this chapter we're introducing the cohomology of a ring and showing that it has a \emph{Gerstenhaber algebra} structure, which we shall first define. The reader is referred to \cite{Gerstenhaber} for more details.  


\section{Gerstenhaber algebras}

We begin reminding that a \emph{graded algebra} is a graded $S$-module (for some ring $S$) which is also a graded ring with the same gradation. We shall denote a graded algebra $A=\sum_\lambda A_\lambda$, indexed by some additive group, generally the integers. We denote the degree of an element $a\in A$ by $|a|$. %By $A[1]$ we will denote the shift of gradation, i.e., the graded algebra where $A[1]_\lambda=A_{\lambda+1}$. 


\begin{defi}\label{defi1}
A graded algebra $A$ is a \emph{Gerstenhaber algebra} if one has:
\begin{enumerate}
\item[(1)] An associative, graded commutative multiplication:
\[
\cdot: A\times A\to A.
\]
For every $a,b\in A$ one has $a\cdot b=(-1)^{|a||b|}b\cdot a$.

\item[(2)] A graded Lie bracket 
\[
[,]:A\times A\to A.
\]
So one has, for every $a,b,c\in A$:
\begin{itemize}
\item $|[a,b]|=|a|+|b|-1$
\item $[b,a]=-(-1)^{(|a|-1)(|b|-1)}[a,b]$ (graded antisymmetry)
\item $(-1)^{(|a|-1)(|c|-1)}[[a,b],c]+(-1)^{(|b|-1)(|a|-1)}[[b,c],a]+(-1)^{(|c|-1)(|b|-1)}[[c,a],b]=0$ (graded Jacobi identity). 
\end{itemize}
\item[(3)]  $[,c]$ is a left derivation of $\cdot$ of degree $|c|-1$: %LO QUE PRUEBA GERSTENHABER ES CON EL PRODUCTO EN EL LADO IZQUIERDO PERO SUPONGO QUE POR LA CONMUTATIVIDAD Y ANTISIMETRÍA
\[
[a\cdot b,c]=[a,c]\cdot b+(-1)^{|a|(|c|-1)}a\cdot [b,c]
\]
¿ES ESTO EQUIVALENTE A SI LO QUE FIJO ES EL PRIMER ARGUMENTO?
\end{enumerate}
\end{defi}


\section{Cohomology of a ring and commutativity of cup product}

\begin{defi}Let $A$ be a ring and $P$ a two-sided $A$-module. We define a \emph{Hochschild $m$-cochain} $f^m$ of $A$ with coefficients in $P$ to be an $S$-module homormorphism $f^m: A^{\otimes m}\to P$. The module $\Hom_S(A^{\otimes m}, P)$ or all such $f^m$ will be denoted by $C^m(A,P)$, $S$ being understood. We indetify $C^0(A,P)$ with $P$.
\end{defi}

For every $m$ there is defined a homomorphism $\delta_m: C^m(A,P)\to C^{m+1}(A,P)$ defined by setting
\begin{align*}
\delta_m f(a_1\otimes\cdots\otimes a_{m+1})&=a_1f(a_2\otimes\cdots\otimes a_{m+1})\\
& +\sum_{i=1}^m(-1)^if(a_1\otimes\cdots\oplus a_{i-1}\otimes a_ia_{i+1}\oplus a_{i+1}\otimes\cdots a_{m+1})\\
& +(-1)^{m+1}f(a_1\otimes\cdots\otimes a_m)a_{m+1},
\end{align*}
and extending this definition linearly. It is the case that $\delta_{m+1}\delta_m=0$; $Z^m(A,P)$ is defined to be the kernel of $\delta_m$, $B^m(A,P)$ the image of $\delta_{m-1}$ for $m\geq 1$ and 0 for $m=0$. One has $B^m(A,P)\subseteq Z^m(A,P)$ and the $m$-th \emph{Hochschild cohomology module} $H^m(A,P)$ is defined to be $Z^m(A,P)/B^m(A,P)$. We will denote $C^*(A,P)$, $Z^*(A,P)$ and $H^*(A,P)$, respectively, the direct sums of the modules $C^m(A,P)$, $Z^m(A,P)$ and $H^m(A,P)$ for $m=0,1,2,\dots$. 

If $P$ is further an associative ring, then the multiplication in $P$ induces, for every $m$ and $n$, a homomorphism denoted $\smile$ and called the \emph{cup product} of $C^m(A,P)\otimes C^n(A,P)$ into $C^{m+n}(A,P)$ defined by setting for $f\in C^m(A,P)$ and $g\in C^n(A,P)$
\[
f\smile g(a_1\otimes\cdots\otimes a_m\otimes b_1\otimes\cdots\otimes b_n)=f(a_1\otimes\cdots\otimes a_m)g(b_1\otimes\cdots\otimes b_n).
\]

Under this multiplication $C^*(A,P)$ becomes an associative ring graded by the integers.

\begin{thm}(\cite[Corollary 1 of \textsection 7]{Gerstenhaber})
If $A$ is a ring, then the ring $\{H^*(A,A),\smile\}$ is a graded commutative ring, with grading given by dimension, i.e., if $[f]\in H^m(A,A)$ and $[g]\in H^n(A,A)$, then $[f]\smile [g] =(-1)^{mn}[g]\smile [f]$.
\end{thm} 

The cup product induces a module structure on $C^*(A,P)$ over $C^*(A,A)$ that induces a module structre on $H^*(A,P)$ over $H^*(A,A)$. Using Theorem 3 of \cite{Gerstenhaber} we get the following result.

\begin{thm}(\cite[Corollary 2 of \textsection 7]{Gerstenhaber})
Let $A$ a ring and $P$ a two-sided $A$-module. If either $[f]\in H^n(A,A)$ and $[g]\in H^m(A,P)$ or $[f]\in H^n(A,P)$ and $[g]\in H^m(A,A)$, then
\[
[f]\smile[g]=(-1)^{mn}[g]\smile [f].
\]
\end{thm}

According to \cite[Theorem 4 of \textsection 8]{Gerstenhaber} we can define a Lie bracket on $C^*(A,A)$ in such a way that $H^*(A,P)$ becomes a two sided module over $\{H^*(A,A),[,]\}$. Finally, we have

\begin{thm}(\cite[Corollary 2 of \textsection 8]{Gerstenhaber}
Let $A$ be a ring and $[f]$, $[g]$ and $[h]$ elements of $H^m(A,A)$, $H^n(A,A)$ and $H^p(A,A)$, respectively. Then
\[
[[f]\smile[g], [h]]=[[f],[h]]\smile [g]+(-1)^{m(p-1)}[f]\smile[[g],[h]].
\]
\end{thm}


According to \ref{defi1}, we now have that the Hochschild cohomology $H^*(A,A)$ is a Gerstenhaber algebra. 




\end{document}

%isotopía \url{https://link.springer.com/chapter/10.1007%2F978-94-015-9319-9_6}
%Orientation preserving (que las cartas conservan la orientación) https://math.stackexchange.com/questions/1319234/meaning-of-the-expression-orientation-preserving-homeomorphism 
%O sea, conserva la orientación como superficie