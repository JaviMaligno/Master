\documentclass[TFM.tex]{subfiles}

\begin{document}

%\hyphenation{equi-va-len-cia}\hyphenation{pro-pie-dad}\hyphenation{res-pec-ti-va-men-te}\hyphenation{sub-es-pa-cio}
\chapter{Little disks operad}

ESCRIBIR ALGO AQUÍ, INCLUYENDO ALGO DE QUE ALL MAPS ARE REQUIRED TO BE CONTINUOUS


\section{Operads}
First we give the general notion of operad and some properties of this object. EN PRINCIPIO LO DEFINO EN TOP Y YA PREGUNTO SI HACE FALTA QUE SEA EN UNA MONOIDAL SIMÉTRICA. SI HACE FALTA, HACER UNA SECCIÓN DE CATEGORÍAS SIMÉTRICAS MONOIDALES. SI NO, CITAR A DONALD YAU CON QUE SE PUEDE DEFINIR CON MÁS GENERALIDAD


NO PARECE QUE ESTA DEFINICIÓN CAPTURE LA COMPOSICIÓN DEL LITTLE DISKS OPERAD, PERO EN VERDAD SÍ, LO QUE FALTA ES RELLENA CON LA IDENTIDAD HASTA COMPLETAR LA ARIDAD DEL PRIMERO
\begin{defi}
A (non-symmetric) \emph{operad} is a tuple $(O, \circ, \mathbbm{1})$ consisting of the following data:
\begin{enumerate}[(1)]
\item a sequence of topological spaces $O=\{O(n)\}_{n\geq 0}$, whose elements are called \emph{$n$-ary operations}.
\item an element $\mathbbm{1}\in P(1)$ is called the \emph{identity},
\item for all positive integers $n,k_1,\dots, k_n$, a \emph{composition} map
\begin{align*}
\circ:P(n)\times P(k_1)\times\cdots\times P(k_n)&\to P(k_1+\cdots+k_n)\\
(\theta, \theta_1,\dots, \theta_n)&\to \theta\circ(\theta_1,\dots, \theta_n),
\end{align*}
satisfying the following axioms:
\begin{itemize}
\item \emph{indentity}: $\theta(1,\dots, 1)=1\circ\theta$,
\item \emph{associativiy}:
\end{itemize}
\end{enumerate}
\end{defi}

\url{https://en.wikipedia.org/wiki/Operad_theory}

We're treating operads only in the category of topological spaces, but they can be defined in a more general setting, namely, in any \emph{symmetric monoidal category} \cite{Yau}.

\section{Little disks operad}




\end{document}

%isotopía \url{https://link.springer.com/chapter/10.1007%2F978-94-015-9319-9_6}
%Orientation preserving (que las cartas conservan la orientación) https://math.stackexchange.com/questions/1319234/meaning-of-the-expression-orientation-preserving-homeomorphism 
%O sea, conserva la orientación como superficie