\documentclass[GSR.tex]{subfiles}

\begin{document}


%\hyphenation{equi-va-len-cia}\hyphenation{pro-pie-dad}\hyphenation{res-pec-ti-va-men-te}\hyphenation{sub-es-pa-cio}

\chapter{Subvariedades}

 
\section{Repaso de superficies}

\section{Subvariedades }
Sea $f:(M,g)\to(\overline{M},\overline{g})$ una inmersión isométrica, $\nabla:\Chi(M)\times\Chi(M)\to\Chi(M)$ la conexión de Levi-Civita de $M$ y $\overline{\nabla}:\Chi(\overline{M})\times\Chi(\overline{M})\to\Chi(\overline{M})$ la de $\overline{M}$. Vamos a ver qué relación existe entre ambas conexiones. Nos centramos en $U\subseteq M$ abierto tal que $f|_U:U\to\overline{M}$ sea un embebimiento. Si $X\in\Chi(U)$, existe $\overline{X}\in\Chi(M)$ tal que $\overline{X}\circ f=X$ en $U$. La relación que vamos a probar es $\nabla_X Y=(\overline{\nabla}_{\overline{X}}\overline{Y})^t$ (parte tangente), aunque en esta igualdad estamos haciendo algunas identificaciones.

Dado $\nabla_XY\in\Chi(U)$ y $p\in U$, $(\nabla_X Y)(p)\in T_pM$. Tomamos $(df)_p(\nabla_X)_p\in (df)_p(T_pM)$. Tomamos una extensión $\overline{\nabla}_{\tilde{X}}\tilde{Y}\in\Chi(\overline{M})$, de modo que si $q\in \overline{M}$, $(\overline{\nabla}_{\tilde{X}}\tilde{Y})(q)\in T_q\overline{M}$. En particular se tiene para $q=f(p)$, en cuyo caso $T_{f(p)}\overline{M}=T_pM\oplus T_pM^\perp$. Nos quedamos solo con la parte tangente $(\overline{\nabla}_{\tilde{X}}\tilde{Y})(f(p))^t\in(df)_p(T_pM)$. Formalmente tendremos $(df)_p(\nabla_XY(p))=(\overline{\nabla}_{\tilde{X}}\tilde{Y})(f(p))^t$, o equivalentemente, 
$\nabla_XY(p)=(df)^{-1}_p(\overline{\nabla}_{\tilde{X}}\tilde{Y})(f(p))^t$.

Para probar esta igualdad, vamos a ver que el miembro de la derecha define una conexión de Levi-Civita en $M$, de lo cual se desprenderá el resultado por unicidad. 

Definimos $T:\Chi(U)\times\Chi(U)\to\Chi(U)$ como $T(X,Y)=(df)^{-1}[(\overline{\nabla}_{\tilde{X}}\tilde{Y}\circ f)^t]$.

Probamos las cinco propiedades que definen la conexión de Levi-Civita:
\begin{enumerate}
\item Para $g\in C^{\infty}(M)$, tomamos $\overline{g}=g\circ f^{-1}$. 
$$T(X,gY)=(df)^{-1}(\overline{\nabla}_{\tilde{X}}\tilde{Y}\circ f)^t=(df)^{-1}([\overline{g}\overline{\nabla}_{\tilde{X}}\tilde{Y}+\tilde{X}(\overline{g})\tilde{Y}]\circ f)^t=(df)^{-1}([g\overline{\nabla}_{\tilde{X}}\tilde{Y}]^t+[(\tilde{X}\circ f)(\tilde{Y}\circ  f)]^t)=$$
$$gT(X,Y)+(df)^{-1}([\tilde{X}(\overline{g})\circ f)(\tilde{Y}\circ f)]^t)$$
Ahora, 

$$(\tilde{X}(\overline{g})\circ f)(p)=(d\overline{g})_{f(p)}(\tilde{X}f(p))=(d\overline{g})_{f(p)}((df)_p(X_p))=$$
$$d(\overline{g}\circ f)_p(X_p)=(df)_p(X_p)=X_p(g)=X(g)(p)$$
Con esto, $\tilde{X}(\overline{g})\circ f=X(g)$ y $\tilde{Y}\circ g=(df)(Y)$. De modo que 
$$(df)^{-1}([\tilde{X}(\overline{g})\circ f)(\tilde{Y}\circ f)]^t)=X(g)(df)^{-1}(df)(Y)=X(g)Y$$

\item
\item
\item
\item 
\end{enumerate}

Sea $f:(M^n,g)\to (M^m,\overline{g})$ una inmersión isométrica. Sea $O\subseteq M$ abierto tal que $f|_O$ es un embebimiento. Sean $X,Y\in\mathfrak{X}(O)$ y $\widetilde{X},\widetilde{Y}\in\mathfrak{X}(\overline{M})$ tales que $df(X)=\widetilde{X}\circ f$ y $df(Y)=\widetilde{Y}\circ f$.

Tenemos que para $p\in O$ y $f(p)\in\overline{M}$, $T_pM\oplus T_pM^\perp =T_{f(p)}M\ni (\overline{\nabla}_{\widetilde{X}}\widetilde{Y})(f(p))=(\overline{\nabla}_{\widetilde{X}}\widetilde{Y})(f(p))^t+(\overline{\nabla}_{\widetilde{X}}\widetilde{Y})(f(p))^N$. Al segundo término es a lo que llamamos \emph{2ª forma fundamental de $f|_O$}.

\begin{defi}
Sean $X,Y\in\Chi(O)$. $\sigma(X,Y)=\overline{\nabla}_{\widetilde{X}}\widetilde{Y}\circ f-(\overline{\nabla}_{\widetilde{X}}\widetilde{Y}\circ f)^t=\overline{\nabla}_{\widetilde{X}}\widetilde{Y}\circ f-\nabla_XY=(\overline{\nabla}_{\widetilde{X}}\widetilde{Y}\circ f)^N$. Para todo $p\in O$, $\sigma(X,Y)(p)=(\overline{\nabla}_{\widetilde{X}}\widetilde{Y})^N(f(p))\in T_pM^\perp$, luego $\sigma(X,Y)\in\Chi^\perp(f|_O)$ (esto es igual a lo que habíamos llamado 2ª forma fundamental). 

Vemos que $\sigma:\Chi(O)\times\Chi(O)\to\Chi^\perp(f|_O)$, $(X,Y)\mapsto\sigma(X,Y)=(\overline{\nabla}_{\widetilde{X}}\widetilde{Y}\circ f)-(df)(\nabla_X Y)$ 
\begin{enumerate}
\item Está bien definida.
\item Es bilineal. La aditividad es trivial. Sea $f_1\in\CC^{\infty}(O)$. Entonces , para $\overline{f}_1=f_1\circ f^{-1}$

$$
\sigma(f_1X,Y)=(\overline{\nabla}_{\overline{f}_1\widetilde{X}}\widetilde{Y}\circ f)-(df)(\nabla_{f_1X}Y)=(\overline{f}_1\overline{\nabla}_{\widetilde{X}}\widetilde{Y})\circ f-(df)(f_1\nabla_XY)=
$$
$$f_1(\overline{\nabla}_{\widetilde{X}}\widetilde{Y}\circ f)-f_1(df)(\nabla_XY)=f_\sigma(X,Y)$$

Ahora, en la otra componente
$$
\sigma(X,f_1Y)=\overline{\nabla}_{\widetilde{X}}(\overline{f}_1\widetilde{Y})\circ f-(df)(\nabla_X(f_1Y))=(\overline{f}_1\overline{\nabla}_{\widetilde{X}}\widetilde{Y} +\widetilde{X}(\overline{f}_1)\widetilde{Y})\circ f-(df)(f_1\nabla_XY+X(f_1)Y)=
$$
$$
f_1(\overline{\nabla}_{\widetilde{X}}\widetilde{Y}\circ f)+\widetilde{X}(\overline{f}_1)\circ f(\widetilde{Y}\circ f)-f_1(df)(\nabla_XY)-X(f_1)df(Y)
$$
Además $\widetilde{X}(\overline{f}_1)\circ f=X(f_1)$ y $\widetilde{Y}\circ f=(df)(Y)$, luego cancelando nos queda $f_1\sigma(X,Y)$. 
\item Es simétrica. Efectivamente
$$\sigma(X,Y)-\sigma(Y,X)=(\overline{\nabla}_{\widetilde{X}}\widetilde{Y}\circ f)^N-(\overline{\nabla}_{\widetilde{Y}}\widetilde{X}\circ f)^N=([\widetilde{X},\widetilde{Y}]\circ f)^N=(df([X,Y,]))^N=0$$
porque la diferencial va al tangente. 
\item Verifica: Si $X\in\Chi(O)$ con extensiones $\widetilde{X}_1,\widetilde{X}_2\in\Chi(\overline{M})$ y si $Y\in\Chi(O)$ con extensiones $\widetilde{Y}_1,\widetilde{Y}_2\in\Chi(\overline{M})$, entonces $\sigma$ no depende de las extensiones. Efectivamente, si hacemos la diferencia, la parte con signo menos se cancela y nos queda $\overline{\nabla}_{\widetilde{X}_1}\widetilde{Y}\circ f-\overline{\nabla}_{\widetilde{X}_2}\widetilde{Y}_2\circ f$. Sumamos y restamos un término con los índices cruzados $\overline{\nabla}_{\widetilde{X}_1}\widetilde{Y}_2$. Aplicando las propiedades de linealidad y aditividad de la conexión de Levi-Civita obtenemos
$$\overline{\nabla}_{\widetilde{X}_1}(\widetilde{Y}_1-\widetilde{Y}_2)\circ f+\overline{\nabla}_{\widetilde{X}_1+\widetilde{X}_2}\widetilde{Y}_2\circ f$$
Como son extensiones de $X$ y de $Y$, sobre $O$ valen lo mismo, luego ambos términos son 0. Así que $\overline{\nabla}_{\widetilde{X}_1}(\widetilde{Y}_1=\overline{\nabla}_{\widetilde{X}_2}(\widetilde{Y}_2$ en $f(O)$
\end{enumerate}
\end{defi}

Dado $p\in O$, $\sigma_p:T_pM\times T_pM\to T_pM^\perp$, $\sigma_p(u,v)=\sigma(U,V)(p)$, donde $U,V\in\Chi(O)$ tales que $U_p=u$ y $V_p=v$ es lo que llamaremos 2ª forma fundamental en $p$. $\sigma_p$ es claramente bilineal y simétrica. $\sigma_p$ está bien definida: dados $U_1,U_2\in\Chi(O)$ tales que $U_1(p)=u=U_2(p)$, entonces 
\[
\sigma(U_2,V)(p)-\sigma(U_2, V)(p)=(\overline{\nabla}_{\widetilde{U}_1}\widetilde{V}\circ f)^N(p)-(\overline{\nabla}_{\widetilde{U}_2}\widetilde{V}\circ f)^N(p)=(\overline{\nabla}_{\widetilde{U}_1-\widehat{U}_2}\widetilde{Y}\circ f)^N(p)=(\overline{\nabla}_{\widetilde{U}_1\circ f-\widetilde{U}_2\circ f}\widetilde{Y})^N(p)
\]
Como $(df)(u_1)-(df)(u_2)=(df)(u)-(df)u=0$, ambos vectores evaluados en $p$ coinciden y la diferencia es 0. Por simetría es suficiente probarlo para esta coordenada. 


\section{Endomorfismo de Weingarten}
Sea $f:M\to\overline{M}$ inmersión isométrica, $O\subseteq M$ abierto con $f|_O$ embebimiento, $\eta\in\Chi^\perp(f|_O)$ y la 2ª forma fundamental $\sigma$. Definimos en primer lugar $B_\eta:\Chi(O)\times\Chi(O)\to\CC^{\infty}(O)$ como $B_\eta(X,Y)=\gene{\sigma(X,Y),\eta}_{\overline{M}}:O\to\R$ (esta notación indica la métrica, similar al producto escalar) que actúa mediante $p\mapsto B_\eta(X,Y)(p)=\gene{\sigma(X,Y)(p),\eta_p}_{\overline{M}}\in\R$. Tenemos que $B_\eta$ es bilinial y simétrica, luego podemos considerar su enfomorfismo autoadjunto asociado $A_\eta:\Chi(O)\to \Chi(O)$, $X\mapsto A_\eta(X)=\gene{A_\eta(X),Y}_M=\gene{\sigma(X,Y),\eta}=B_\eta(X,Y)$ para todo $Y\in\Chi(O)$. A este endomorfismo es al que llamamos \emph{endomorfismo de Weingarten}. 

Obsérvese que $\gene{A_\eta(X),Y}=\gene{\sigma(X,Y),\eta}=\gene{\sigma(Y,X),\eta}=\gene{A_\eta(Y),X}$, por lo que es efectivamente autoadjunto. 

\begin{prop}
$A_\eta(X)=-(\overline{\nabla}_{\widetilde{X}}\widetilde{\eta})^t$ donde $\widetilde{\eta}\in\Chi(\overline{M})$ es la extensión dada por $\widetilde{\eta}\circ f=\eta$. 
\end{prop}
\begin{dem}
Debemos interpretar la igualdad como la conmutatividad del diagrama
\[
\begin{tikzcd}
O\arrow[r, "A_\eta(X)"]\arrow[d, "f"] & TO\arrow[d, "df"]\\
\overline{M}\arrow[r, "\overline{\nabla}_{\widetilde{X}}\widetilde{\eta}"'] & T\overline{M}
\end{tikzcd}
\]
que es lo que vamos a probar. Dado $Y\in\Chi(O)$, sea $\widetilde{Y}\in\Chi(\overline{M})$ tal que $(df)(Y)=\widetilde{Y}\circ f$. 
\[
\gene{A_\eta(X),Y}=\gene{\sigma(X,Y),\eta}=\gene{\overline{\nabla}_{\widetilde{X}}\widetilde{Y}\circ f-(df)(\nabla_XY),\eta}=\gene{\overline{\nabla}_{\widetilde{X}}\widetilde{Y}\circ f,\eta}
\]
por ser $\eta$ normal. Por otro lado,
\[
\gene{\widetilde{Y},\widetilde{\eta}}\circ f=\gene{\widetilde{Y}\circ f,\widetilde{\eta}\circ f}=\gene{(df)(Y),\eta}=0
\]
Por tanto $0=\widetilde{X}\gene{\widetilde{Y},\widetilde{\eta}}=\gene{\overline{\nabla}_{\widetilde{X}}\widetilde{Y},\eta}+\gene{\widetilde{Y}, \overline{\nabla}_{\widetilde{X}}\widetilde{\eta}}$ en $f(O)$. Entonces, $\gene{\overline{\nabla}_{\widetilde{X}}\widetilde{Y}\circ f,\widetilde{\eta}\circ f}=-\gene{\widetilde{Y},\overline{\nabla}_{\widetilde{X}}\widetilde{\eta}}\circ f$ en $O$. Así, $\gene{\overline{\nabla}_{\widetilde{X}}\widetilde{Y}\circ f,\eta}=-\gene{(df)(Y),\overline{\nabla}_{\widetilde{X}}\widetilde{\eta}\circ f}$. El primer término es al que habíamos llegado en el primer desarrollo, luego $
\gene{A_\eta(X),Y}=\gene{\overline{\nabla}_{\widetilde{X}}\widetilde{Y}\circ f,\eta}=-\gene{(df)(Y),\overline{\nabla}_{\widetilde{X}}\widetilde{\eta}\circ f}=-\gene{(df)(Y),(\overline{\nabla}_{\widetilde{X}}\widetilde{\eta}\circ f)^t}=
$

Como $f$ es isometría, podemos aplicar $df^{-1}$ y el producto escalar se mantiene:

$$
-\gene{(df)(Y),(\overline{\nabla}_{\widetilde{X}}\widetilde{\eta}\circ f)^t}=\gene{(df)(Y),-(\overline{\nabla}_{\widetilde{X}}\widetilde{\eta}\circ f)^t}=\gene{Y,(df)^{-1}((\overline{\nabla}_{\widetilde{X}}\widetilde{\eta}\circ f)^t)}
$$
como esta igualdad se tiene para todo $Y\in\Chi(O)$, deducimos la igualdad que queríamos demostrar. 
\end{dem}

A partir de la definición del endomorfismo de Weingarten pasamos a definirlo en cada punto $p\in O$ como $A_{\eta,p}:T_pM\to T_pM$, donde $v\mapsto A_{\eta,p}(v)=A_\eta(V)(p)$, siendo $V\in\Chi(O)$ tal que $V_p=v$. 

Esta aplicación está bien deinida: si $V_1,V_2\in\Chi(O)$ tales que $V_1(p)=V_2(p)=v$, entonces 
\[
(A_\eta(V_2)-A_\eta(V_2)_p=A_\eta(V_1-V_2)(p)=-(\overline{\nabla}_{\tilde{V}_1-\tilde{V}_2}\tilde{\eta})^t(p)=-(\overline{\nabla}_{(\tilde{V}_1-\tilde{V}_2)_p}\tilde{\eta})=0
\]

Tenemos que $A_{\eta,p}$ es un endomorfismo autoadjunto, esto es, $\gene{A_{\eta,p}(u),v}=\gene{A_{\eta,p}(v),y}$, lo cual implica que $A_{\eta,p}$ es diagonalizable por la simetría de su matriz. Por tanto, existe una base $\{e_1,\dots, e_n\}$ de $T_pM$ y existen números reales $k_1,\dots, k_n$ tales que $A_{\eta,p}(e_i)=k_ie_i$ para todo $i=1,\dots, n$. A los $k_i$ se los llama \emph{curvaturas principales} de $f$ en $p$ según la dirección de $\eta_p$. A los $e_i$ se los llama \emph{direcciones principales}. 

Como en el caso no evaluado, tenemos también $\gene{A_{\eta,p}(e_i),e_j}=\gene{\sigma_p(e_i,e_j),\eta_p}$. En este caso, el miembto de la izquierda no es más que $k_i\delta_{ij}$. 

\begin{ej}
Veamos el caso particular de codimensión 1, es decir, sea $f:M^n\to \overline{M}^{n+1}$ una inmersión isométrica. En general, $\dim(T_pM^\perp)=codim(M)$, por lo que en este caso es 1. Sea $\eta_p\in T_pM^\perp$ un generador. Si lo tomamos unitario solo hay dos posibilidades, dependiendo del signo. Así que no hay dependencia del campo normal para hallar las curvaturas principales. 

Supongamos ahora que $M$ y $\overline{M}$ son orientables, esto es, $\exists\eta:M\to T\overline{M}$ campo normal globalmente definido. Definimos la \emph{curvatura media} de $f$ en $p$ como $H_p=\frac{1}{n}tr(A_{\eta,p})=\frac{1}{n}\sum_i k_i$. Definimos también la \emph{curvatura de Gauss-Kronecker} de $f$ en $p$ como $K_p=\det(A_{\eta,p})=\prod_i k_i$.

Recordamos la definición de la 2ª forma fundamental $\sigma:\Chi(O)\times\Chi(O)\to\Chi^\perp(f|_O)$ como $\sigma(X,Y)=(\overline{\nabla}_{\tilde{X}}\tilde{Y})^N$. Para $p\in O$, esto induce $\sigma_p:T_pM\times T_pM\to T_pM^\perp$ mediante $\sigma_p(u,v)=\sigma(U,V)(p)$. Como $\dim T_pM^\perp=1$, podemos identidicar este espacio con $\R$ con generador $\eta_p$. Así, 
$$\sigma_p(u,v)=\gene{\sigma_p(u,v),\eta_p}\eta_p\equiv \gene{\sigma_p(u,v),\eta_p}=\gene{A_{\eta,p}(u),v}=\gene{A_\eta(U)(p),v}$$

Vimos anteriormente que $A_\eta(U)(p)=(df)^{-1}(-\overline{\nabla}_{\tilde{U}}\tilde{\eta}\circ f)^t(p)$, por lo que sustituyendo y aplicando $df$ (teniendo en cuenta que $f$ es isometría y que $df$ tiene imagen tangente) obtenemos
$$\gene{-\overline{\nabla}_{\tilde{U}}\tilde{\eta}\circ f)(p), (df)_p(v)}=\gene{(df)(-\nabla_{U}\eta\circ f)(p), (df)_p(v)}=\gene{-(\nabla_U\eta)(p),v}$$

En la primera igualdad se ha sudado que $\overline{\nabla}\tilde{Y}\circ f=df(\nabla_XY)$. Esto extiende lo que ocurría en superficies, pues si $S\subseteq\R^3$ es una superficie, $\sigma_p:T_pS\times T_pS\to \R$ estaba dada por $\sigma_p(u,v)=\gene{-(dN)_p(u),v}$. En $S$ tenemos entonces $\sigma_p(u,v)\equiv \gene{\sigma_p(u,v),\eta_p}=\gene{A_{\eta,p}(u),v}$, con lo que $K_p=\det(A_{\eta,p})=\det(\sigma_p)$, que era la curvatura de Gauss como superficie. Lo mismo ocurre con la curvatura media. 

Obsérvese que en la cadena de igualdades anterior hemos deducido en particular que $\A_{\eta,p}=-\nabla_U\eta_p$. 
\end{ej} 









\end{document}
