\documentclass[GSR.tex]{subfiles}

\begin{document}


%\hyphenation{equi-va-len-cia}\hyphenation{pro-pie-dad}\hyphenation{res-pec-ti-va-men-te}\hyphenation{sub-es-pa-cio}

\chapter{Subvariedades}

 
\section{Repaso de superficies}

\section{Subvariedades }
Sea $f:(M,g)\to(\overline{M},\overline{g})$ una inmersión isométrica, $\nabla:\Chi(M)\times\Chi(M)\to\Chi(M)$ la conexión de Levi-Civita de $M$ y $\overline{\nabla}:\Chi(\overline{M})\times\Chi(\overline{M})\to\Chi(\overline{M})$ la de $\overline{M}$. Vamos a ver qué relación existe entre ambas conexiones. Nos centramos en $U\subseteq M$ abierto tal que $f|_U:U\to\overline{M}$ sea un embebimiento. Si $X\in\Chi(U)$, existe $\overline{X}\in\Chi(M)$ tal que $\overline{X}\circ f=X$ en $U$. La relación que vamos a probar es $\nabla_X Y=(\overline{\nabla}_{\overline{X}}\overline{Y})^t$ (parte tangente), aunque en esta igualdad estamos haciendo algunas identificaciones.

Dado $\nabla_XY\in\Chi(U)$ y $p\in U$, $(\nabla_X Y)(p)\in T_pM$. Tomamos $(df)_p(\nabla_X)_p\in (df)_p(T_pM)$. Tomamos una extensión $\overline{\nabla}_{\tilde{X}}\tilde{Y}\in\Chi(\overline{M})$, de modo que si $q\in \overline{M}$, $(\overline{\nabla}_{\tilde{X}}\tilde{Y})(q)\in T_q\overline{M}$. En particular se tiene para $q=f(p)$, en cuyo caso $T_{f(p)}\overline{M}=T_pM\oplus T_pM^\perp$. Nos quedamos solo con la parte tangente $(\overline{\nabla}_{\tilde{X}}\tilde{Y})(f(p))^t\in(df)_p(T_pM)$. Formalmente tendremos $(df)_p(\nabla_XY(p))=(\overline{\nabla}_{\tilde{X}}\tilde{Y})(f(p))^t$, o equivalentemente, 
$\nabla_XY(p)=(df)^{-1}_p(\overline{\nabla}_{\tilde{X}}\tilde{Y})(f(p))^t$.

Para probar esta igualdad, vamos a ver que el miembro de la derecha define una conexión de Levi-Civita en $M$, de lo cual se desprenderá el resultado por unicidad. 

Definimos $T:\Chi(U)\times\Chi(U)\to\Chi(U)$ como $T(X,Y)=(df)^{-1}[(\overline{\nabla}_{\tilde{X}}\tilde{Y}\circ f)^t]$.

Probamos las cinco propiedades que definen la conexión de Levi-Civita:
\begin{enumerate}
\item $T(g_1X+g_2Y)=(df)^{-1}((\overline{\nabla}_{\widetilde{g_1X+g_2Y}}\tilde{Z}\circ f)^t)=$
$$(df)^{-1}(\overline{\nabla}_{\overline{g}_1\tilde{X}+\overline{g}_2\tilde{Y}}\tilde{Z}\circ f)^t=(df)^{-1}((\overline{g}_1\overline{\nabla}_{\tilde{X}}\tilde{Z}+\overline{g}_2\tilde{\nabla}_{\tilde{Y}}\tilde{Z})\circ f)^t $$
$$(df)^{-1}((\overline{g}_1\circ f)\overline{\nabla}_{\tilde{X}}\tilde{Z}+(\overline{g}_2\circ f)\tilde{\nabla}_{\tilde{Y}}\tilde{Z})^t=(df)^{-1}(g_1\overline{\nabla}_{\tilde{X}}\tilde{Z}+g_2\tilde{\nabla}_{\tilde{Y}}\tilde{Z})^t=$$
$$(df)^{-1}(g_1\overline{\nabla}_{\tilde{X}}\tilde{Z})^t+(df)^{-1}(g_2\overline{\nabla}_{\tilde{Y}}\tilde{Z})=g_1T(X,Z)+g_2T(Y,Z)$$

\item Igual que la anterior.
\item Para $g\in C^{\infty}(M)$, tomamos $\overline{g}=g\circ f^{-1}$. 
$$T(X,gY)=(df)^{-1}(\overline{\nabla}_{\tilde{X}}\tilde{Y}\circ f)^t=(df)^{-1}([\overline{g}\overline{\nabla}_{\tilde{X}}\tilde{Y}+\tilde{X}(\overline{g})\tilde{Y}]\circ f)^t=(df)^{-1}([g\overline{\nabla}_{\tilde{X}}\tilde{Y}]^t+[(\tilde{X}\circ f)(\tilde{Y}\circ  f)]^t)=$$
$$gT(X,Y)+(df)^{-1}([\tilde{X}(\overline{g})\circ f)(\tilde{Y}\circ f)]^t)$$
Ahora, 

$$(\tilde{X}(\overline{g})\circ f)(p)=(d\overline{g})_{f(p)}(\tilde{X}f(p))=(d\overline{g})_{f(p)}((df)_p(X_p))=$$
$$d(\overline{g}\circ f)_p(X_p)=(df)_p(X_p)=X_p(g)=X(g)(p)$$
Con esto, $\tilde{X}(\overline{g})\circ f=X(g)$ y $\tilde{Y}\circ g=(df)(Y)$. De modo que 
$$(df)^{-1}([\tilde{X}(\overline{g})\circ f)(\tilde{Y}\circ f)]^t)=X(g)(df)^{-1}(df)(Y)=X(g)Y$$
\item $T(X,Y)-T(Y,X)=(df)^{-1}[(\overline{\nabla}_{\tilde{X}}\tilde{Y}\circ f-\overline{\nabla}_{\tilde{Y}}\tilde{X}\circ f)^t]=$
$$(df)^{-1}((\overline{\nabla}_{\tilde{X}}\tilde{Y}-\overline{\nabla}_{\tilde{Y}}\tilde{X})\circ f)^t=(df)^{-1}([X,Y]\circ f)^t=$$
$$(df)^{-1}((df)[X,Y])=[X,Y]$$
\item  $\gene{T(X,Y),Z}+\gene{Y,T(X,Z)}=\gene{(df)^{-1}(\overline{\nabla}_{\tilde{X}}\tilde{Y}\circ f,Z}+\gene{Y,(df)^{-1}(\overline{\nabla}_{\tilde{X}}\tilde{Z}\circ f}=$
$$\gene{\overline{\nabla}_{\tilde{X}}\tilde{Y}\circ f,(df)(\tilde{Z})}+\gene{(df)(\tilde{Y}),\overline{\nabla}_{\tilde{X}}\tilde{Z}\circ f}=$$
$$\gene{\overline{\nabla}_{\tilde{X}}\tilde{Y}\circ f,\tilde{Z}\circ f}+\gene{\tilde{Y}\circ f,\overline{\nabla}_{\tilde{X}}\tilde{Z}\circ f}=$$
$$(\gene{\overline{\nabla}_{\tilde{X}}\tilde{Y},\tilde{Z}}+\gene{\tilde{Y},\overline{\nabla}_{\tilde{X}}\tilde{Z}})\circ f=\tilde{X}(\gene{\tilde{Y},\tilde{Z}})\circ f=X(\gene{Y,Z})$$
\end{enumerate}

Sea $f:(M^n,g)\to (M^m,\overline{g})$ una inmersión isométrica. Sea $O\subseteq M$ abierto tal que $f|_O$ es un embebimiento. Sean $X,Y\in\mathfrak{X}(O)$ y $\widetilde{X},\widetilde{Y}\in\mathfrak{X}(\overline{M})$ tales que $df(X)=\widetilde{X}\circ f$ y $df(Y)=\widetilde{Y}\circ f$.

Tenemos que para $p\in O$ y $f(p)\in\overline{M}$, $T_pM\oplus T_pM^\perp =T_{f(p)}M\ni (\overline{\nabla}_{\widetilde{X}}\widetilde{Y})(f(p))=(\overline{\nabla}_{\widetilde{X}}\widetilde{Y})(f(p))^t+(\overline{\nabla}_{\widetilde{X}}\widetilde{Y})(f(p))^N$. Al segundo término es a lo que llamamos \emph{2ª forma fundamental de $f|_O$}.

\begin{defi}
Sean $X,Y\in\Chi(O)$. $\sigma(X,Y)=\overline{\nabla}_{\widetilde{X}}\widetilde{Y}\circ f-(\overline{\nabla}_{\widetilde{X}}\widetilde{Y}\circ f)^t=\overline{\nabla}_{\widetilde{X}}\widetilde{Y}\circ f-\nabla_XY=(\overline{\nabla}_{\widetilde{X}}\widetilde{Y}\circ f)^N$. Para todo $p\in O$, $\sigma(X,Y)(p)=(\overline{\nabla}_{\widetilde{X}}\widetilde{Y})^N(f(p))\in T_pM^\perp$, luego $\sigma(X,Y)\in\Chi^\perp(f|_O)$ (esto es igual a lo que habíamos llamado 2ª forma fundamental). 

Vemos que $\sigma:\Chi(O)\times\Chi(O)\to\Chi^\perp(f|_O)$, $(X,Y)\mapsto\sigma(X,Y)=(\overline{\nabla}_{\widetilde{X}}\widetilde{Y}\circ f)-(df)(\nabla_X Y)$ 
\begin{enumerate}
\item Está bien definida.
\item Es bilineal. La aditividad es trivial. Sea $f_1\in\CC^{\infty}(O)$. Entonces , para $\overline{f}_1=f_1\circ f^{-1}$

$$
\sigma(f_1X,Y)=(\overline{\nabla}_{\overline{f}_1\widetilde{X}}\widetilde{Y}\circ f)-(df)(\nabla_{f_1X}Y)=(\overline{f}_1\overline{\nabla}_{\widetilde{X}}\widetilde{Y})\circ f-(df)(f_1\nabla_XY)=
$$
$$f_1(\overline{\nabla}_{\widetilde{X}}\widetilde{Y}\circ f)-f_1(df)(\nabla_XY)=f_\sigma(X,Y)$$

Ahora, en la otra componente
$$
\sigma(X,f_1Y)=\overline{\nabla}_{\widetilde{X}}(\overline{f}_1\widetilde{Y})\circ f-(df)(\nabla_X(f_1Y))=(\overline{f}_1\overline{\nabla}_{\widetilde{X}}\widetilde{Y} +\widetilde{X}(\overline{f}_1)\widetilde{Y})\circ f-(df)(f_1\nabla_XY+X(f_1)Y)=
$$
$$
f_1(\overline{\nabla}_{\widetilde{X}}\widetilde{Y}\circ f)+\widetilde{X}(\overline{f}_1)\circ f(\widetilde{Y}\circ f)-f_1(df)(\nabla_XY)-X(f_1)df(Y)
$$
Además $\widetilde{X}(\overline{f}_1)\circ f=X(f_1)$ y $\widetilde{Y}\circ f=(df)(Y)$, luego cancelando nos queda $f_1\sigma(X,Y)$. 
\item Es simétrica. Efectivamente
$$\sigma(X,Y)-\sigma(Y,X)=(\overline{\nabla}_{\widetilde{X}}\widetilde{Y}\circ f)^N-(\overline{\nabla}_{\widetilde{Y}}\widetilde{X}\circ f)^N=([\widetilde{X},\widetilde{Y}]\circ f)^N=(df([X,Y,]))^N=0$$
porque la diferencial va al tangente. 
\item Verifica: Si $X\in\Chi(O)$ con extensiones $\widetilde{X}_1,\widetilde{X}_2\in\Chi(\overline{M})$ y si $Y\in\Chi(O)$ con extensiones $\widetilde{Y}_1,\widetilde{Y}_2\in\Chi(\overline{M})$, entonces $\sigma$ no depende de las extensiones. Efectivamente, si hacemos la diferencia, la parte con signo menos se cancela y nos queda $\overline{\nabla}_{\widetilde{X}_1}\widetilde{Y}\circ f-\overline{\nabla}_{\widetilde{X}_2}\widetilde{Y}_2\circ f$. Sumamos y restamos un término con los índices cruzados $\overline{\nabla}_{\widetilde{X}_1}\widetilde{Y}_2$. Aplicando las propiedades de linealidad y aditividad de la conexión de Levi-Civita obtenemos
$$\overline{\nabla}_{\widetilde{X}_1}(\widetilde{Y}_1-\widetilde{Y}_2)\circ f+\overline{\nabla}_{\widetilde{X}_1+\widetilde{X}_2}\widetilde{Y}_2\circ f$$
Como son extensiones de $X$ y de $Y$, sobre $O$ valen lo mismo, luego ambos términos son 0. Así que $\overline{\nabla}_{\widetilde{X}_1}(\widetilde{Y}_1=\overline{\nabla}_{\widetilde{X}_2}(\widetilde{Y}_2$ en $f(O)$
\end{enumerate}
\end{defi}

Dado $p\in O$, $\sigma_p:T_pM\times T_pM\to T_pM^\perp$, $\sigma_p(u,v)=\sigma(U,V)(p)$, donde $U,V\in\Chi(O)$ tales que $U_p=u$ y $V_p=v$ es lo que llamaremos 2ª forma fundamental en $p$. $\sigma_p$ es claramente bilineal y simétrica. $\sigma_p$ está bien definida: dados $U_1,U_2\in\Chi(O)$ tales que $U_1(p)=u=U_2(p)$, entonces 
\[
\sigma(U_2,V)(p)-\sigma(U_2, V)(p)=(\overline{\nabla}_{\widetilde{U}_1}\widetilde{V}\circ f)^N(p)-(\overline{\nabla}_{\widetilde{U}_2}\widetilde{V}\circ f)^N(p)=(\overline{\nabla}_{\widetilde{U}_1-\widehat{U}_2}\widetilde{Y}\circ f)^N(p)=(\overline{\nabla}_{\widetilde{U}_1\circ f-\widetilde{U}_2\circ f}\widetilde{Y})^N(p)
\]
Como $(df)(u_1)-(df)(u_2)=(df)(u)-(df)u=0$, ambos vectores evaluados en $p$ coinciden y la diferencia es 0. Por simetría es suficiente probarlo para esta coordenada. 


\section{Endomorfismo de Weingarten}
Sea $f:M\to\overline{M}$ inmersión isométrica, $O\subseteq M$ abierto con $f|_O$ embebimiento, $\eta\in\Chi^\perp(f|_O)$ y la 2ª forma fundamental $\sigma$. Definimos en primer lugar $B_\eta:\Chi(O)\times\Chi(O)\to\CC^{\infty}(O)$ como $B_\eta(X,Y)=\gene{\sigma(X,Y),\eta}_{\overline{M}}:O\to\R$ (esta notación indica la métrica, similar al producto escalar) que actúa mediante $p\mapsto B_\eta(X,Y)(p)=\gene{\sigma(X,Y)(p),\eta_p}_{\overline{M}}\in\R$. Tenemos que $B_\eta$ es bilinial y simétrica, luego podemos considerar su enfomorfismo autoadjunto asociado $A_\eta:\Chi(O)\to \Chi(O)$, $X\mapsto A_\eta(X)=\gene{A_\eta(X),Y}_M=\gene{\sigma(X,Y),\eta}=B_\eta(X,Y)$ para todo $Y\in\Chi(O)$. A este endomorfismo es al que llamamos \emph{endomorfismo de Weingarten}. 

Obsérvese que $\gene{A_\eta(X),Y}=\gene{\sigma(X,Y),\eta}=\gene{\sigma(Y,X),\eta}=\gene{A_\eta(Y),X}$, por lo que es efectivamente autoadjunto. 

\begin{prop}
$A_\eta(X)=-(\overline{\nabla}_{\widetilde{X}}\widetilde{\eta})^t$ donde $\widetilde{\eta}\in\Chi(\overline{M})$ es la extensión dada por $\widetilde{\eta}\circ f=\eta$. 
\end{prop}
\begin{dem}
Debemos interpretar la igualdad como la conmutatividad del diagrama
\[
\begin{tikzcd}
O\arrow[r, "A_\eta(X)"]\arrow[d, "f"] & TO\arrow[d, "df"]\\
\overline{M}\arrow[r, "\overline{\nabla}_{\widetilde{X}}\widetilde{\eta}"'] & T\overline{M}
\end{tikzcd}
\]
que es lo que vamos a probar. Dado $Y\in\Chi(O)$, sea $\widetilde{Y}\in\Chi(\overline{M})$ tal que $(df)(Y)=\widetilde{Y}\circ f$. 
\[
\gene{A_\eta(X),Y}=\gene{\sigma(X,Y),\eta}=\gene{\overline{\nabla}_{\widetilde{X}}\widetilde{Y}\circ f-(df)(\nabla_XY),\eta}=\gene{\overline{\nabla}_{\widetilde{X}}\widetilde{Y}\circ f,\eta}
\]
por ser $\eta$ normal. Por otro lado,
\[
\gene{\widetilde{Y},\widetilde{\eta}}\circ f=\gene{\widetilde{Y}\circ f,\widetilde{\eta}\circ f}=\gene{(df)(Y),\eta}=0
\]
Por tanto $0=\widetilde{X}\gene{\widetilde{Y},\widetilde{\eta}}=\gene{\overline{\nabla}_{\widetilde{X}}\widetilde{Y},\eta}+\gene{\widetilde{Y}, \overline{\nabla}_{\widetilde{X}}\widetilde{\eta}}$ en $f(O)$. Entonces, $\gene{\overline{\nabla}_{\widetilde{X}}\widetilde{Y}\circ f,\widetilde{\eta}\circ f}=-\gene{\widetilde{Y},\overline{\nabla}_{\widetilde{X}}\widetilde{\eta}}\circ f$ en $O$. Así, $\gene{\overline{\nabla}_{\widetilde{X}}\widetilde{Y}\circ f,\eta}=-\gene{(df)(Y),\overline{\nabla}_{\widetilde{X}}\widetilde{\eta}\circ f}$. El primer término es al que habíamos llegado en el primer desarrollo, luego $
\gene{A_\eta(X),Y}=\gene{\overline{\nabla}_{\widetilde{X}}\widetilde{Y}\circ f,\eta}=-\gene{(df)(Y),\overline{\nabla}_{\widetilde{X}}\widetilde{\eta}\circ f}=-\gene{(df)(Y),(\overline{\nabla}_{\widetilde{X}}\widetilde{\eta}\circ f)^t}=
$

Como $f$ es isometría, podemos aplicar $df^{-1}$ y el producto escalar se mantiene:

$$
-\gene{(df)(Y),(\overline{\nabla}_{\widetilde{X}}\widetilde{\eta}\circ f)^t}=\gene{(df)(Y),-(\overline{\nabla}_{\widetilde{X}}\widetilde{\eta}\circ f)^t}=\gene{Y,(df)^{-1}((\overline{\nabla}_{\widetilde{X}}\widetilde{\eta}\circ f)^t)}
$$
como esta igualdad se tiene para todo $Y\in\Chi(O)$, deducimos la igualdad que queríamos demostrar. 
\end{dem}

A partir de la definición del endomorfismo de Weingarten pasamos a definirlo en cada punto $p\in O$ como $A_{\eta,p}:T_pM\to T_pM$, donde $v\mapsto A_{\eta,p}(v)=A_\eta(V)(p)$, siendo $V\in\Chi(O)$ tal que $V_p=v$. 

Esta aplicación está bien deinida: si $V_1,V_2\in\Chi(O)$ tales que $V_1(p)=V_2(p)=v$, entonces 
\[
(A_\eta(V_2)-A_\eta(V_2)_p=A_\eta(V_1-V_2)(p)=-(\overline{\nabla}_{\tilde{V}_1-\tilde{V}_2}\tilde{\eta})^t(p)=-(\overline{\nabla}_{(\tilde{V}_1-\tilde{V}_2)_p}\tilde{\eta})=0
\]

Tenemos que $A_{\eta,p}$ es un endomorfismo autoadjunto, esto es, $\gene{A_{\eta,p}(u),v}=\gene{A_{\eta,p}(v),y}$, lo cual implica que $A_{\eta,p}$ es diagonalizable por la simetría de su matriz. Por tanto, existe una base $\{e_1,\dots, e_n\}$ de $T_pM$ y existen números reales $k_1,\dots, k_n$ tales que $A_{\eta,p}(e_i)=k_ie_i$ para todo $i=1,\dots, n$. A los $k_i$ se los llama \emph{curvaturas principales} de $f$ en $p$ según la dirección de $\eta_p$. A los $e_i$ se los llama \emph{direcciones principales}. 

Como en el caso no evaluado, tenemos también $\gene{A_{\eta,p}(e_i),e_j}=\gene{\sigma_p(e_i,e_j),\eta_p}$. En este caso, el miembto de la izquierda no es más que $k_i\delta_{ij}$. 

\begin{ej}
Veamos el caso particular de codimensión 1, es decir, sea $f:M^n\to \overline{M}^{n+1}$ una inmersión isométrica. En general, $\dim(T_pM^\perp)=codim(M)$, por lo que en este caso es 1. Sea $\eta_p\in T_pM^\perp$ un generador. Si lo tomamos unitario solo hay dos posibilidades, dependiendo del signo. Así que no hay dependencia del campo normal para hallar las curvaturas principales. 

Supongamos ahora que $M$ y $\overline{M}$ son orientables, esto es, $\exists\eta:M\to T\overline{M}$ campo normal globalmente definido. Definimos la \emph{curvatura media} de $f$ en $p$ como $H_p=\frac{1}{n}tr(A_{\eta,p})=\frac{1}{n}\sum_i k_i$. Definimos también la \emph{curvatura de Gauss-Kronecker} de $f$ en $p$ como $K_p=\det(A_{\eta,p})=\prod_i k_i$.

Recordamos la definición de la 2ª forma fundamental $\sigma:\Chi(O)\times\Chi(O)\to\Chi^\perp(f|_O)$ como $\sigma(X,Y)=(\overline{\nabla}_{\tilde{X}}\tilde{Y})^N$. Para $p\in O$, esto induce $\sigma_p:T_pM\times T_pM\to T_pM^\perp$ mediante $\sigma_p(u,v)=\sigma(U,V)(p)$. Como $\dim T_pM^\perp=1$, podemos identidicar este espacio con $\R$ con generador $\eta_p$. Así, 
$$\sigma_p(u,v)=\gene{\sigma_p(u,v),\eta_p}\eta_p\equiv \gene{\sigma_p(u,v),\eta_p}=\gene{A_{\eta,p}(u),v}=\gene{A_\eta(U)(p),v}$$

Vimos anteriormente que $A_\eta(U)(p)=(df)^{-1}(-\overline{\nabla}_{\tilde{U}}\tilde{\eta}\circ f)^t(p)$, por lo que sustituyendo y aplicando $df$ (teniendo en cuenta que $f$ es isometría y que $df$ tiene imagen tangente) obtenemos
$$\gene{-\overline{\nabla}_{\tilde{U}}\tilde{\eta}\circ f)(p), (df)_p(v)}=\gene{(df)(-\nabla_{U}\eta\circ f)(p), (df)_p(v)}=\gene{-(\nabla_U\eta)(p),v}$$

En la primera igualdad se ha sudado que $\overline{\nabla}\tilde{Y}\circ f=df(\nabla_XY)$. Esto extiende lo que ocurría en superficies, pues si $S\subseteq\R^3$ es una superficie, $\sigma_p:T_pS\times T_pS\to \R$ estaba dada por $\sigma_p(u,v)=\gene{-(dN)_p(u),v}$. En $S$ tenemos entonces $\sigma_p(u,v)\equiv \gene{\sigma_p(u,v),\eta_p}=\gene{A_{\eta,p}(u),v}$, con lo que $K_p=\det(A_{\eta,p})=\det(\sigma_p)$, que era la curvatura de Gauss como superficie. Lo mismo ocurre con la curvatura media. 

Obsérvese que en la cadena de igualdades anterior hemos deducido en particular que $A_{\eta,p}(U)=-\nabla_U\eta_p$. 
\end{ej} 


\section{Curvatura seccional}
Sea $(M,g)$ una variedad rimaniana y $p\in M$ donde el espacio tangente es $T_pM$ con dimensión $n$. Sean $\Pi\subseteq T_pM$ un plano (subespacio 2-dimensional). Definimos $K(p,\Pi)$ como el número real que mide la curvatura de la 2-variedad formada por las geodésicas que pasan por $p$ y son tangentes a $M$. Sea $B_\Pi=\{u,v\}$ una base ortonormal de $M$. Entonces $K(p,\Pi)=R_p(u,v,v,u)=\gene{R(U,V)V,U}(p)=\gene{\nabla_U\nabla_VV-\nabla_V\nabla_UV-\nabla_{[U,V]}V,U}(p)$ donde $U,V\in\Chi(M)$ tales que $U_p=u$ y $V_p=v$. 

Sea $f:M^n\to\overline{M}^n$ Una inmersión isométrica. Entonces $(df)(T_pM)\subseteq T_{f(p)}\overline{M}$. Podemos ver $\Pi$ como un plano contenido en $T_{f(p)}\overline{M}$. Vamos a dar la relación entre $K(p,\Pi)$ y $\overline{K}(f(p),\Pi)$ (curvatura seccional en $\overline{M}$). 

\begin{teorema}[de Gauss]
$K(p,\Pi)-\overline{K}(f(p),\Pi)=\gene{\sigma_p(u,u),\sigma_p(v,v)}-|\sigma_p(u,v)|^2$. 
\end{teorema}
\begin{dem}
Sean $X,Y\in\Chi(M)$ tales que $X_p=u$, $Y_p=v$. 
\[
K(p,\Pi)=\gene{\nabla_X\nabla_YY-\nabla_Y\nabla_XY-\nabla_{[X,Y]}Y,X}_p=R_p(u,v,vu)=Rp(v,u,u,v)=\gene{\nabla_Y\nabla_X-\nabla_X\nabla_YX,Y}_p
\]
(Los que tienen abajo los corchetes se anulan ya pa siempre)
\[
\overline{K}(f(p),\Pi)=\gene{\tilde{\nabla}_{\tilde{Y}}\tilde{\nabla}_{\tilde{X}}\tilde{X}-\tilde{\nabla}_{\tilde{X}}\tilde{\nabla}_{\tilde{Y}}\tilde{X},\tilde{Y}}_{f(p)}
\]
para $\tilde{X},\tilde{Y}\in\Chi(\overline{M})$ extensiones. Tenemos
\[
\gene{(\tilde{\nabla}_{[\tilde{X},\tilde{Y}}\tilde{X})^t,\tilde{Y}}_{f(p)}=\gene{\nabla_{[X,Y]}X,Y}_p
\]
Así que
\[
K(p,\Pi)-K(f(p),\Pi)=\gene{\nabla_Y\nabla_X X-\nabla_X\nabla_YX,Y}-\gene{\tilde{\nabla}_{\tilde{Y}}\tilde{\nabla}_{\tilde{X}}\tilde{X}-\tilde{\nabla}_{\tilde{X}}\tilde{\nabla}_{\tilde{Y}}\tilde{X},\tilde{Y}}
\]
Por otro lado
\[
\tilde{\nabla}_{\tilde{X}}\tilde{X}=(\tilde{\nabla}_{\tilde{X}}\tilde{X})^t+(\tilde{\nabla}_{\tilde{X}}\tilde{X})^N-\nabla_XX+\sigma(X,X)
\]
luego
\[
\tilde{\nabla}_{\tilde{Y}}\tilde{\nabla}_{\tilde{X}}\tilde{X}=\tilde{\nabla}_{\tilde{Y}}(\nabla_XX+\sigma(X,X))=\tilde{\nabla}_{\tilde{Y}}\nabla_XX+\tilde{\nabla}_{\tilde{Y}}\sigma(X,X)
\]
Sea $\{E_1,\dots, E_{m-n}\}$ una base ortonormal de $\Chi^\perp(f(M))$ de modo que $\sigma(X,X)=\sum_iH_i(X,X)E_i$. Así, la última ecuación continuaría con
\[
\tilde{\nabla}_{\tilde{Y}}\nabla_XX+\tilde{\nabla}_{\tilde{Y}}(\sum_iH_i(X,X)E_i)=
\]
\[
\tilde{\nabla}_{\tilde{Y}}\nabla_XX+\sum_iH_i(X,X)\tilde{\nabla}_{\tilde{Y}}E_i+\sum_i\tilde{Y}(H_i(X,X))E_i
\]
Entonces, haciendo cuentas llegamos a que
\[
\gene{\tilde{\nabla}_{\tilde{Y}}\tilde{\nabla}_{\tilde{X}},\tilde{Y}}_{f(p)}=\gene{\tilde{\nabla}_{\tilde{Y}}\nabla_XX,\tilde{Y}}_{f(p)}+\sum_iH_i(X,X)\gene{\tilde{\nabla}_{\tilde{Y}}E_i,\tilde{Y}}+\sum_i\tilde{Y}(H_i(X,X))\gene{E_i,\tilde{Y}}
\]
El último sumando es 0 así que nos queda
\[
\gene{\nabla_Y\nabla_X,Y}_p-\sum_iH_i(X,X)H_i(Y,Y)
\]
Y esto último por las cuentas que no me ha dado tiempo a copiar es igual que si el último sumando lo cambiamos por $\sum_iH_i(X,Y)^2$. 

Restando nos queda
\[
\gene{\nabla_Y\nabla_XX-\nabla_X\nabla_YX,Y}-\sum H_i(X,X)H_i(Y,Y)+\sum_i H_i(X,Y)^2
\]
Volviendo a la resta de curvaturas que habíamos hecho obtenemos
\[
\sum_i H_i(X,X)H_i(Y,Y)-\sum_i H_i(X,Y)^2
\]

Ahora
\[
\gene{\sigma(X,X),\sigma(Y,Y)}=\gene{\sum_i H_i(X,X)E_i,\sum_jH_j(Y,Y)E_j}=\sum_{ij}H_i(X,X)H_j(Y,Y)\gene{E_i,E_j}=\sum_iH_i(X,X)
\]
Con lo que en la resta anterior obtenemos el resultado.

\end{dem}


\begin{ej}
En el caso de codimensión uno tenemos 1 inmersión inmersión isométrica $f:M^n\to\overline{M}^{n+1}$. Elegimos $B=\{e_1,\dots, e_n\}$ una base de $T_pM$ con cada $e_i$ la dirección principal asociada a $k_i$ y como base de $T_pM^\perp$ elegimos $\{\eta_p\}$. Sean $\Pi=\gene{e_i,e_j}$ un plano de $T_pM$. Por el teorema anterior,
\[
K(p,\Pi)-\overline{K}(f(p),\Pi)=\gene{\sigma_p(e_i,e_i),\sigma_p(e_j,e_j)}-|\sigma_p(e_i,e_j)|^2=\gene{\gene{\sigma_p(e_i,e_i),\eta_p}\eta_p,\gene{\sigma_p(e_j,e_j),\eta_p}\eta_p}=
\] 
\[
\gene{\sigma_p(e_i,e_i),\eta_p}\gene{\sigma_p(e_j,e_j),\eta_p}\gene{\eta_p,\eta_p}=\gene{A_{\eta_p}(e_i),e_i}\gene{A_{\eta_p}(e_j),e_j}=\gene{k_ie_i,e_i}\gene{k_je_j,e_j}=k_ik_j
\]
Hemos omitido $|\sigma_p(e_i,e_j)|^2$ porque haciendo el desarrollo se anula. Así que si $S\subseteq\R^3$ es una superficie y $f:S\to\R^3$ es una inmersión isométrica, para $p\in S$, como $\dim(T_pM)=2$, necesariamente $\Pi=T_pM$. Además, como la variedad ambiente es $\R^3$, $\overline{K}=0$, de modo que $K(p,T_pM)=k_1k_2$, que es justamente la curvatura de Gauss para $S$. Tenemos entonces que la curvatura de Gauss de una superficie es invariante pro isometrías. 

Consideremos $S=S^n$ y $f$ la inclusión en $\R^{n+1}$. Tenemos que para cualquier $\Pi\subseteq T_pM$,  $K(p,\Pi)=\gene{\sigma_p(u,u),\sigma_p(v,v)}-|\sigma_p(u,v)|^2$. Usando las cuentas de más arriba esto es igual a $\gene{A_{\eta_p}(u),u}\gene{A_{\eta_p}(v),v}-\gene{A_{\eta_p}(u),v)}$. Aplicamos la expresión que tenemos para el endomorfismo de Weingarten
\[
\gene{-(\nabla_u\eta)^t,y)}\gene{-(\nabla_v\eta)^r,v}-\gene{-(\nabla_u\eta)^t,v}
\]
Como $u,v$ son tangentes, no hace falta tomar parte tangente. Tomando $\eta_p=p$ como generador, esto nos da
\[
\gene{-u,u}\gene{-v,v}-{-u,v}=(-1)(-1)-0=1
\]
Algo que ya sabíamos, que la curvatura seccional de $S^n$ es 1. 
\end{ej}


\section{Ecuaciones fundamentales}

\subsection{Ecuación de Gauss}

\[
R(X,Y,Z,T)=\overline{R}(\tilde{X},\tilde{Y},\tilde{Z},\tilde{T})-\gene{\sigma(X,Z),\sigma(Y,T)}+\gene{\sigma(X,T),\sigma(Y,Z)}
\]
para todo $X,Y,Z,T\in\Chi(M)$, donde $\overline{R}$ es el tensor de curvatura de la variedad ambiente.

Tenemos 
$$\overline{R}(\tilde{X},\tilde{Y})\tilde{Z}=\overline{\nabla}_{\tilde{X}}\overline{\nabla}_{\tilde{Y}}\tilde{Z}-\overline{\nabla}_{\tilde{Y}}\overline{\nabla}_{\tilde{X}}\tilde{Z}-\overline{\nabla}_{[\tilde{X},\tilde{Y}}\tilde{Z}=$$
\begin{gather*}
overline{\nabla}_{\tilde{X}}(\nabla_XZ+\sigma(Y,Z))-\overline{\nabla}_{\tilde{Y}}(\nabla_XZ+\sigma(X,Y))-\nabla_{[X,Y]}Z-\sigma([X,Y],Z)=\nabla_X\nabla_YZ+\sigma(X,\nabla_YZ)+\nabla^\perp_X\sigma(Y,Z)-A_{\sigma(Y,Z)}X\\
-\nabla_Y\nabla_XZ-\sigma(Y,\nabla_XZ)-\nabla_Y^\perp\sigma(X,Z)+A_{\sigma(X,Z)}Y\\
-\nabla_{[X,Y]}Z-\sigma([X,Y],Z)=
\end{gather*}
\begin{equation}\label{A}
R(X,Y)Z+\sigma(X,\nabla_YZ)+\nabla_X^\perp\sigma(Y,Z)-A_{\sigma(Y,Z)}X-\sigma(Y,\nabla_XZ)-\nabla_Y^\perp\sigma(X,Z)+A_{\sigma(X,Z)}Y-\sigma([X,Y],Z)=\overline{R}(\tilde{X},\tilde{Y})\tilde{Z}
\end{equation}

\[
\overline{R}(\tilde{X},\tilde{Y},\tilde{Z},\tilde{T})=\gene{\overline{R}(\tilde{X},\tilde{Y})\tilde{Z},\tilde{T}}=\gene{R(X,Y)Z,T}+0-\gene{A_{\sigma(Y,Z)}X,T}+\gene{A_{\sigma(X,Z)}Y,T}=\ast
\]

Por la definición del endomorfismo de Weingarten tenemos
\[
\gene{A_{\sigma(Y,Z)}X,T}=\gene{\sigma(X,T),\sigma(Y,Z)}
\]
y análogo cambiando $X$ por $Y$. Así que en la igualdad que quedó abierta antes sigue como
\[
R(X,Y,Z,T)-\gene{\sigma(X,T),\sigma(Y,Z)}+\gene{\sigma(Y,T),\sigma(X,Z)}
\]

\subsection{Ecuación de Ricci}

Para $X,Y\in\Chi(M)$ y $\eta,\xi\in\Chi^\perp(f)$
\[
\gene{\overline{R}(\tilde{X},\tilde{Y})\tilde{\eta},\tilde{\xi}}=\gene{R^\perp(X,Y)\eta,\xi}-\gene{[A_\eta,A_\xi](X),Y}
\]
Usamos notación $[A_\eta,A_\xi](X)=A_\eta(A_\xi(X))-A_\xi(A_\eta(X))\in\Chi(M)$.

\begin{gather*}
\overline{R}(\tilde{X},\tilde{Y})\tilde{\eta}=\overline{\nabla}_{\tilde{X}}\overline{\nabla}_{\tilde{Y}}\tilde{\eta}-\overline{\nabla}_{[\tilde{X},\tilde{Y}]}\tilde{\eta}=\\
\overline{\nabla}_{\tilde{X}}(\nabla_Y^\perp\eta-A_\eta(Y))-\overline{\nabla}_{\tilde{Y}}(\nabla^\perp_{[X,Y]}\eta+A_{\eta}([X,Y])=\\
\nabla^\perp_X\nabla_Y^\perp\eta-A_{\nabla^\perp_Y\eta}(X)-\nabla_XA_\eta(Y)-\sigma(X,A_\eta(Y))-\\
-\nabla_Y^\perp\nabla_X^\perp\eta+A_{\nabla_X^\perp\eta}(Y)+\nabla_YA_\eta(X)+\sigma(Y,A_\eta(X))-\\
-\nabla^\perp_{[X,Y]}\eta+A_\eta([X,Y])=\\
R^\perp(X,Y)\eta-A_{\nabla^\perp_Y\eta}(X)-\nabla_XA_\eta(Y)-\sigma(X,A_\eta(Y))+A_{\nabla^\perp_X\eta}Y+\nabla_YA_\eta(X)+\sigma(Y,A_\eta(X))+A_\eta([X,Y])
\end{gather*}

Teniendo en cuenta que $\tilde{\xi}$ es normal
\[
\gene{\overline{R}(\tilde{X},\tilde{Y})\tilde{\eta},\tilde{\xi}}=\gene{R^\perp(X,Y)\eta,\xi}-\gene{\sigma(X,A_\eta(Y)),\xi}+\gene{\sigma(Y,A_\eta(X)),\xi}
\]
Ahora
\[
\gene{\sigma(X,A_\eta(Y)),\xi}=\gene{A_\xi(X),A_\eta(Y)}
\]
y análogo en el otro sumando, así que podemos escribir lo anterior como
\[
\gene{R^\perp(X,Y)\eta,\xi}-\gene{A_\eta A_\xi(X)-A_\xi A_\eta(X),Y}=\gene{R^\perp(X,Y)\eta,\xi}-\gene{[A_\eta,A_\xi](X),Y}
\]

\subsection{Ecuación de Codazzi}
Definimos $\sigma:\Chi(M)\times\Chi(M)\times\Chi^\perp(f)\to C^\infty(M)$ como $\sigma(X,Y,\eta)=\gene{\sigma(X,Y),\eta}_{\overline{M}}$. Definimos $(\overline{\nabla}_{\tilde{X}}\sigma)(Y,Z,\eta)=X(\sigma(Y,Z,\eta))-\sigma(\nabla_XY,Z,\eta)-\sigma(Y,\nabla_XZ,\eta)-\sigma(Y,Z,\nabla_X^\perp\eta)$. 

Obsérvese que $X(\sigma(X,Y,\eta))=\gene{\nabla_X^\perp \sigma(X,Y),\eta}+\gene{\sigma(X,Y),\nabla_X^\perp\eta}$. Así que la fórmula de la segunda definición se puede reescribir como 
\[
(\overline{\nabla}_{\tilde{X}}\sigma)(Y,Z,\eta)=\gene{\nabla_X^\perp \sigma(X,Y),\eta}-\sigma(Y,\nabla_XZ,\eta)-\sigma(Y,Z,\nabla_X^\perp\eta)
\]

La ecuación de Codazzi es
\[
\gene{\overline{R}(\tilde{X},\tilde{Y})\tilde{Z},\tilde{\eta}}=(\overline{\nabla}_{\tilde{X}}\sigma)(Y,Z,\eta)-(\overline{\nabla}_{\tilde{Y}}\sigma)(X,Z,\eta)
\]

Usando \ref{A} y cancelando términos vemos que
\begin{gather*}
\gene{\overline{R}(\tilde{X},\tilde{Y})\tilde{Z},\tilde{\eta}}=\gene{\sigma(X,\nabla_YZ),\eta}+\gene{\nabla_X^\perp\sigma(Y,Z),\eta}-\gene{\sigma(Y,\nabla_XZ),\eta}-\\
-\gene{\nabla_Y^\perp\sigma(X,Z),\eta}-\gene{\sigma([X,Y],X),\eta}
\end{gather*}
Fijémonos en el último término. 
\[
\sigma([X,Y],Z)=\sigma(\nabla_XY-\nabla_YX,Z)
\]
luego
\[
\gene{\sigma([X,Y],Z),\eta}=\gene{\sigma(\nabla_XY,Z),\eta}-\gene{\sigma(\nabla_YX,Z),\eta}
\]
De modo que sustituyendo obtenemos la igualdad buscada. 

\section{•}

En superficies tenemos el teorema de Bonnet, según el cual la primera y segunda forma fundamental determinan una superficie en $\R^3$ única salvo movimiento rígido. Esto es, dadas $E,F,G,e,f,g:V\subseteq\R^2\to\R^3$ diferenciables tales que $E,G>0, EG-F^2>0$ y además verifican la ecuación de Gauss y la ecuación de Mainardi-Codazzi, entonces para too $q\i V$, existe $U\subseteq V$ entorno de $q$ y existe $\chi:U\to\chi(U)$ una parametrización de forma que $\chi(U)$ es una superficie regular con primera forma fundamental determinada por $E,F,G$ y segunda forma funamental determinada por $e,f,g$. 

Lo que buscamos es generalizar este resultado a varieades de Riemann. 

\begin{teorema}[K. Tenenblat, \emph{On isometric immersions of Riemannian manifolds} (1971)]
Sea $M^{n}$ una varieda riemanniana simplemente conexa. Sea $O^k$ un fibrado vectorial métrico ($\forall p\in M$, $O^k_p$ es un espacio vectorial de dimensión $k$). Sean $\{s_p:T_pM\times T_pM\to O_p^k\}_{p\in M}$ bilineales y simétricas. Sea $\nabla^{O^k}:TM\times O^k\to O^k$ una conexión afín. Entonces se satisfacen las ecuaciones de Gauss, Ricci y Codazzi para una variedad ambiente $\mathcal{Q}_c^{n+k}$ y existe $f:M\to \mathcal{Q}_c^{n+k}$ inmersión isométrica tal que $\sigma_p^f\equiv s_p, \nabla^\perp\equiv\nabla^{O_k}, T_pM^\perp\equiv O_p^k$. Además tenemos unicidad salvo isometría del espacio ambiente. 
\end{teorema}

\end{document}
