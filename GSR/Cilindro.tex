\documentclass[twoside, 11pt]{article}
\usepackage{titlesec}

%\setcounter{secnumdepth}{0}
%\titleformat{\section}[block]{\Large\bfseries\filcenter}{}{1em}{}
%\titleformat{\subsection}[hang]{\large\bfseries}{}{1em}{}
%%\usepackage{../estilo-apuntes}
%\AtBeginDocument{%
%  \renewcommand\tablename{Tabla}
%}
%\newcommand{\yields}{\overset{\Pi,I^*}{\rightarrow}}
%\newcommand{\yieldsone}{\overset{\Pi,I}{\rightarrow}}
%\newcommand{\yieldsn}{\overset{\Pi,I^n}{\rightarrow}}
%


%------------------------------------------------------------------------
\usepackage[utf8x]{inputenc}
\usepackage[spanish,es-nodecimaldot]{babel}
\usepackage{amsmath,amssymb,amsthm,amsfonts}
\usepackage{enumerate}
\usepackage{mathrsfs}
\usepackage{bm}
\usepackage{array,multirow}
\usepackage{fancyhdr}
\usepackage{titlesec}
\usepackage{floatrow}
\usepackage{makeidx}
\usepackage[tocflat]{tocstyle}
\usetocstyle{standard}
\usepackage{subfiles}
\usepackage{color}
\usepackage{hyperref}
\hypersetup{colorlinks=true,citecolor=red, linkcolor=blue}
\usepackage[]{graphicx}
\usepackage{pgf,tikz}
\usepackage{tikz-cd}
\usetikzlibrary{babel}
\usetikzlibrary{arrows}
\usetikzlibrary{cd}

\renewcommand{\baselinestretch}{1,1}
\setlength{\oddsidemargin}{0.25in}
\setlength{\evensidemargin}{0.25in}
\setlength{\textwidth}{6in}
\setlength{\topmargin}{0.1in}
\setlength{\headheight}{0.1in}
\setlength{\headsep}{0.1in}
\setlength{\textheight}{8in}
\setlength{\footskip}{0.75in}

\theoremstyle{definition}

\newtheorem{teorema}{Teorema}[section]
\newtheorem{defi}[teorema]{Definición}
\newtheorem{coro}[teorema]{Corolario}
\newtheorem{lemma}[teorema]{Lema}
\newtheorem{ej}[teorema]{Ejemplo}
\newtheorem{ejs}[teorema]{Ejemplos}
\newtheorem{observacion}[teorema]{Observación}
\newtheorem{observaciones}[teorema]{Observaciones}
\newtheorem{prop}[teorema]{Proposición}
\newtheorem{propi}[teorema]{Propiedades}
\newtheorem{nota}[teorema]{Nota}
\newtheorem{notas}[teorema]{Notas}
\newtheorem*{dem}{Demostración}
\newtheorem{ejer}[teorema]{Ejercicio}
\newtheorem{consec}[teorema]{Consecuencia}
\newtheorem{consecs}[teorema]{Consecuencias}
\newtheorem{reco}[teorema]{Recordatorio}

\SetUnicodeOption{mathletters}
\SetUnicodeOption{autogenerated}
\PrerenderUnicode{áéíóú}
\DeclareMathOperator{\cardinal}{card}

\providecommand{\card}[1]{\cardinal\left(#1\right)}

\providecommand{\abs}[1]{\left|#1\right|}
\providecommand{\sen}[1]{sen #1}
\providecommand{\norm}[1]{\left\lVert#1\right\rVert}
\providecommand{\ninf}[1]{\norm{#1}_\infty}
\providecommand{\numn}[1]{\norm{#1}_1}
\providecommand{\gabs}[1]{\left|{#1}\right|}
\newcommand{\bor}[1]{\mathcal{B}(#1)}
\newcommand{\A}{\mathbb{A}}
\newcommand{\calA}{\mathcal{A}}
\newcommand{\calF}{\mathcal{F}}
\newcommand{\calG}{\mathcal{G}}
\newcommand{\Ank}{\A^n_k}
\newcommand{\Amk}{\A^m_k}
\newcommand{\R}{\mathbb{R}}
\newcommand{\Z}{\mathbb{Z}}
\newcommand{\F}{\mathbb{F}}
\newcommand{\N}{\mathbb{N}}
\newcommand{\PP}{\mathbb{P}}
\newcommand{\Pnk}{\PP^n_k}
\newcommand{\Pmk}{\PP^m_k}
\newcommand{\Q}{\mathbb{Q}}
\newcommand{\C}{\mathbb{C}}
\newcommand{\CC}{\mathcal{C}}
\newcommand{\K}{\mathbb{K}}
\newcommand{\KK}{{\mathcal K}}
\newcommand{\OO}{{\mathcal O}}
\newcommand{\X}{\mathbb{X}}
\newcommand{\Y}{\mathbb{Y}}
\newcommand{\Va}{\mathcal{V}_{a}}
\newcommand{\Vp}{\mathcal{V}_{p}}
\newcommand{\V}{\mathcal{V}}
\newcommand{\I}{\mathcal{I}}
\newcommand{\m}{{\mathfrak{m}}}
\newcommand{\p}{{\mathfrak{p}}}
\newcommand{\Tau}{\mathcal{T}}
\newcommand{\parcial}[2]{\frac{\partial #1}{\partial #2}}
\newcommand{\dparcial}[2]{\dfrac{\partial #1}{\partial #2}}
\DeclareMathOperator{\Ima}{Im}
\DeclareMathOperator{\Rea}{Re}
\DeclareMathOperator{\coker}{coker}

%--------------------------------------------------------------







\rhead{Geometría Semi-Riemanniana (Máster Universitario en Matemáticas)}
\lhead{Curso 2018/2019}

\begin{document}
\begin{titlepage}
	\centering
	{\huge\bfseries Geometría Semi-Riemanniana \par}
	\vspace{1cm}
	{\Huge\bfseries El cilindro \par}
	\vspace{1cm}
	{\Large Javier Aguilar Martín\par}
	\vspace{1cm}
	{\large \today\par}
	\vspace{1cm}

\begin{abstract}
\normalsize
Este trabajo consiste en desarrollar un ejemplo en el que se repasan los principales conceptos sobre Variedades Diferenciables y Geometría Riemanniana, concretamente el cilindro. 
\end{abstract}

	\vfill
	{\small Esta obra está licenciada bajo la Licencia Creative Commons Atribución 3.0 España. Para ver una copia de esta licencia, visite \url{http://creativecommons.org/licenses/by/3.0/es/} o envíe una carta a Creative Commons, PO Box 1866, Mountain View, CA 94042, USA.

\bigskip

Usted es libre de:
\begin{itemize}
 \item copiar, distribuir y comunicar públicamente la obra
\item hacer obras derivadas
\end{itemize}

Bajo las condiciones siguientes:
\begin{itemize}
 \item {\bf Reconocimiento:} Debe reconocer los créditos de la obra maestra especificada por el autor o el licenciador (pero no de una manera que sugiera que tiene su apoyo o apoyan el uso que hace de su obra).
 \item {\bf No comercial:} No puede utilizar esta obra para fines comerciales.
\item {\bf Compartir bajo la misma licencia:} Si altera o transforma esta obra, o genera una obra derivada, sólo puede distribuir la obra generada bajo una licencia idéntica a ésta.
\end{itemize}}

	
\end{titlepage}

\tableofcontents

\newpage

\section{Dos estructuras de variedad diferenciable compatibles}

Definimos el cilindro de radio $r$ como el subespacio euclídeo $C=\{(x,y,z)\in\R^3\mid x^2+y^2=r^2\}$. En este trabajo nos vamos a quedar en el caso $r=1$ por sencillez, pero todos los resultados son análogos para cualquier $r>0$. Vamos a ver que podemos dotar al cilindro de al menos dos estructuras diferenciables compatibles. Será útil recordar que hay una parametrización del cilindro $p:(0,2\pi)\times\R\to C$ dada por $p(\theta,t)=(\cos\theta,\sin\theta, t)$ y que podemos cubrir el cilindro por completo añadiendo otra parametrización igual cambiando el intervalo $(0,2\pi)$ por $(-\pi,\pi)$. 

\subsection{Primera estructura}
%Como sabemos por los cursos del Grado y además es fácil comprobar, el cilindro $C$ está parametrizado por $\varphi^{-1}:(-\pi,\pi)\times\R\to C$, donde $\varphi^{-1}(\theta,t)=(\cos\theta,\sin\theta, t)$. Para cubrir todo el cilindro necesitamos otra parametrización más, $\psi^{-1}:(0,2\pi)\times\R\to C$ definida de la misma forma que $\varphi^{-1}$. Por tanto, para dotar al cilindro de estructura de variedad diferenciable basta tomar las inversas de estas parametrizaciones. Denotamos $r_\varphi$ y $r_\psi$ respectivamente a las rectas verticales que dejan sin cubrir las parametrizaciones $\varphi^{-1}$ y $\psi^{-1}$. Sea entonces $\varphi:C\setminus r_\varphi\to(-\pi,\pi)\times\R$ dada por $\varphi(x,y,z)=(\arccos x-\pi, z)$ y $\psi:C\setminus r_\psi\to(0,2\pi)\times\R$ dada por $\psi(x,y,z)=(\arccos x, z)$. Las cartas están claramente relacionadas, ya que el cambio de cartas es simplemente el desfase de ángulo entre una y otra. 

Vamos a darle primero la estructura de variedad diferenciable como variedad producto $S^1\times\R$, por lo que bastará tomar las cartas de $S^1$ y añadirle la identidad en la última coordenada. Tenemos entonces cuatro cartas $\varphi_i$, $i=1,2,3,4$ definidas mediante
\[
\varphi_1:C_1\times\R\to(0,\pi)\times\R; \varphi(x,y,z)=(\arccos(x),z)
\]
\[
\varphi_2:C_2\times\R\to\left(-\frac{\pi}{2},\frac{\pi}{2}\right)\times\R; \varphi(x,y,z)=(\arcsin(y),z)
\]
\[
\varphi_3:C_3\times\R\to(-\pi,0)\times\R; \varphi(x,y,z)=(\arccos(x)-\pi,z)
\]
\[
\varphi_2:C_4\times\R\to\left(\frac{\pi}{2},\frac{3\pi}{2}\right)\times\R; \varphi(x,y,z)=(\arcsin(y)+\pi,z)
\]

donde $C_1$ es la semicircunferencia abierta superior, $C_2$ la derecha, $C_3$ la inferior y $C_4$ la izquierda. Es sencillo probar que las cartas efectivamente están relacionadas.

%HACER DIBUJO

\subsection{Segunda estructura}


%Obsérvese que la estructura de variedad diferenciable anterior que le hemos dado al cilindro coincide salvo elección de dominio con las estructuras que tiene como variedad producto $S^1\times \R$, como superficie de revolución generada por la recta $\alpha(t)=(1,0,t)$ y como superficie reglada generada por la curva $\alpha(\theta)=(\cos\theta,\sin\theta, 0)$ y el vector $(0,0,1)$. Así que podríamos dar cualquiera de estas estructuras y sería trivialmente compatible con la anterior. De modo que vamos a dar otra diferente que no sea tan trivial.

Consideremos la aplicación $\psi^{-1}:\R^2 \setminus\{0\}\to C$ dada por $\psi^{-1}(x,y)=\left(\frac{x}{\sqrt{x^2+y^2}},\frac{y}{\sqrt{x^2+y^2}}, \log\sqrt{x^2+y^2} \right)$, consistente en transformar las semirrectas que parten del origen de $\R^2$ en las líneas verticales del cilindro. Esta aplicación es claramente diferenciable como aplicación $\R^2\to\R^3$, y vamos a ver que tiene una inversa diferenciable, que será la carta que formará el nuevo atlas diferenciable. Basta tomar $\psi:C\to\R^2\setminus\{0\}$ definida como $\psi(x,y,z)=(xe^z,ye^z,e^z)$. 

\subsection{Compatibilidad}

Comprobamos que las cartas $(C, \psi)$ y $(C_i\times\R, \varphi_i)$ están relacionadas. Basta hacerlo para $i=1$, pues el resto son análogos. Los cambios de cartas

\[
\varphi_1\circ \psi^{-1}: \psi(C_1\times\R)\to \varphi_1(C_1\times\R)
\]
y 
\[
\psi\circ\varphi_1^{-1} : \varphi_1(C_1\times\R)\to \psi(C_1\times\R)
\]
tienen coordenadas definidas como composicion de funciones diferenciables en los dominios correspondientes, por lo que son diferenciables. 
%Notar que x/raiz <= 1

\section{Ejemplos de aplicaciones y funciones diferenciables}

\subsection{Aplicaciones diferenciables}
Damos dos ejemplos de aplicaciones diferenciables del cilindro en sí mismo y otra que vaya de otra variedad en el cilindro. Para una vamos a utilizar las coordenadas implícitas del cilindro y para otra la parametrización vista en la primera sección. 


\begin{enumerate}


\item La primera es la que ``le da la vuelta'' al cilindro, es decir, $f:C\to C$ descrita como $f(x,y,z)=(x,y,-z)$. Esta aplicación se puede describir como $Id_{S^1}\times -Id_{\R}:S^1\times\R\to S^1\times\R$. Como cada componente es claramente diferenciable, $f$ es diferenciable. 

\item El segundo ejemplo será una rotación de ángulo $\alpha$, la cual viene dada por la aplicación $g:C\to C$, $g(\cos\theta,\sin\theta, t)=(\cos(\theta+\alpha), \sin(\theta+\alpha), t)$. Se comprueba que al hacer la composición $\psi\circ g\circ \psi^{-1}$ el resultado es diferenciable por ser las componentes producto y composición de funciones diferenciables. 

\item Para esta última aplicación sea $K$ el cono (sin vértice) cuya ecuación implícita es $$x^2+y^2=z^2$$ para $z\neq 0$. Damos la aplicación $h:K\to C$ definida por $x\mapsto \frac{x}{z}, y\mapsto \frac{y}{z}, z\mapsto z$. Es claro que la aplicación está bien dedinida y la imagen está contenida en el cilindro. Ahora, para que tenga sentido hablar de diferenciabilidad, convertimos el cono en una variedad diferenciable mediante las parametrizaciones $\widetilde{\psi}_1^{-1}:(0,2\pi)\times\R^*\to K$ y $\widetilde{\psi}_2^{-1}:(-\pi,\pi)\times\R^*\to K$ definidas ambas por la correspondencia $(s,t)\mapsto (t\cos s, t\sin s, t)$. Vemos que, en el dominio correspondiente, la composición $\psi\circ h\circ \widetilde{\psi}_i^{-1}$ tiene para $i=1,2$  la expresión $(\cos(s)e^t, \sin(s)e^t, e^t)$, que es claramente diferenciable. 

Esta aplicación es la proyección del cono sobre el cilindro. 

%K de Konus en alemán %\R^* son las unidades de \R, o sea todos menos el 0

\end{enumerate}

\subsection{Funciones diferenciables}

Damos ahora también dos ejemplos de funciones diferenciables. 

\begin{enumerate}


\item
 La primera será la proyección sobre la tercera coordenada, es decir, $f:C\to\R$ definida como $f(x,y,z)=z$, que es trivialmente diferenciable. 

\item
 El siguiente ejemplo será $g:C\to\R$ dada por $g(x,y,z)=x+y+z$. Es trivial probar que es diferenciable usando la primera estructura diferenciable. 

\end{enumerate}


\section{Ejemplos de curvas diferenciables y vectores tangentes}

Vamos a dar dos ejemplos de curvas que luego veremos que son geodésicas y uno que se verá que no es geodésica, y calcularemos vectores tangentes a estas curvas.  QUIZÁ DIBUJARLOS

\begin{enumerate}

\item
 La curva $\gamma(t)=(\cos t, \sin t, t)$ con $t\in\R$ es una curva diferenciable sobre el cilindro. Esta curva tiene el vector tangente en cada punto $(-\sin t,\cos t, 1)$. Por ejemplo, para $t=0$, es decir, en el punto $(1,0,0)$, obtenemos el vector $(0,1,1)$. 

\item
 La curva $\beta(t)=(1, 0, t)$ para $t\in\R$ es una curva diferenciable sobre el cilindro. Esta curva tiene como vector tangente el vector $(1,0,1)$ en todo punto.

\item La curva $\sigma(t)=(\cos t, \sin t, t^2)$, $t\in\R$, es una curva diferenciable sobre el cilindro. En este caso la curva tiene el vector tangente asociado $(-\sin t, \cos t, 2t)$, que para $t=\pi$, esto es, en el punto $(1,0,\pi^2)$, nos da el vector $(0, 1, 2\pi)$. 
\end{enumerate}



\section{El cilindro es subvariedad regular de $\R^3$ con la inclusión}

Sea $i:C\hookrightarrow \R^3$ la inclusión. Tal como hemos definido $C$, tiene la topología relativa de $\R^3$, por lo que en virtud de la proposición 4.2.2 de la asignatura Variedades Diferenciables, es suficiente probar que $(C,i)$ es subvariedad de $\R^3$. Para ello probaremos que $i$ es una inmersión inyectiva. Que es inyectiva es evidente, y para que sea inmersión necesitamos que su diferencial lo sea, así que calculamos la matriz de la aplicación $i_{*p}$ para un punto $p\in C$ genérico, que coincide con la matriz jacobiana de la aplicación $i\circ \varphi^{-1}$. Por tanto, obtenemos la matriz
\[
\begin{pmatrix}
-\sin\theta & 0 \\
\cos\theta & 0\\
0 & 1
\end{pmatrix}
\]
Esta matriz representa una aplicación lineal $\R^2\to\R^3$ y tiene rango máximo, por lo que efectivamente es inyectiva, con lo que tenemos el resultado. 


\section{Métrica Riemanniana sobre el cilindro}




\section{Ejemplos de geodésicas}




\section{Curvatura escalar del cilindro}

\end{document}
