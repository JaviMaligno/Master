	\documentclass[twoside]{article}
\usepackage{../estilo-ejercicios}
\renewcommand{\baselinestretch}{1,3}
%--------------------------------------------------------
\begin{document}

\title{Ejercicios de la segunda parte de Geometría Semi-Riemanniana}
\author{Javier Aguilar Martín}
\maketitle


\begin{ejercicio}{1}
 Sea $f : M → \overline{M}$ una inmersión isométrica, donde $\overline{M}$ es una variedad riemanniana
con curvatura seccional constante.
Probar que, en tal caso, la ecuación de Codazzi queda
\[
(\overline{∇}_{\widetilde{X}} σ)(Y, Z, η) = (\overline{∇}_{\widetilde{Y}} σ)(X, Z, η),\ X, Y, Z ∈ \mathfrak{X}(M), η ∈ \mathfrak{X}^⊥(f),
\]
y que la ecuación de Ricci implica que
\[
R^⊥(X, Y )η = σ(X, A_η(Y )) − σ(Y, A_η(X)),\ X, Y ∈ \mathfrak{X}(M), η ∈ \mathfrak{X}^⊥(f).
\]
\end{ejercicio}
\begin{solucion}
 La ecuación de Codazzi en general tiene la expresión 
 \[
\gene{\overline{R}(\widetilde{X},\widetilde{Y})\widetilde{Z},\widetilde{\eta}}=(\overline{\nabla}_{\widetilde{X}}\sigma)(Y,Z,\eta)-(\overline{\nabla}_{\widetilde{Y}}\sigma)(X,Z,\eta)
\]
Como $\overline{M}$ tiene curvatura seccional constante $k$, tenemos la expresión 
\[
\overline{R}(\widetilde{X}, \widetilde{Y} )\widetilde{Z} = k(\gene{\widetilde{Y},\widetilde{Z}}\widetilde{X} - \gene{\widetilde{X},\widetilde{Z}}\widetilde{Y} ),
\]
la cual implica que $\overline{R}_p(\widetilde{X}_p, \widetilde{Y}_p )\widetilde{Z}_p$ es tangente para todo $p\in\overline{M}$, de modo que el producto escalar $\gene{\overline{R}(\widetilde{X},\widetilde{Y})\widetilde{Z},\widetilde{\eta}}=0$, de donde se deduce la primera igualdad del enunciado. 

Probamos ahora la segunda igualdad. Por tener $\overline{M}$ curvatura seccional constante, tenemos
\[
\gene{\overline{R}(\widetilde{X},\widetilde{Y})\widetilde{Z},\widetilde{\eta}}=\gene{k(\gene{\widetilde{Y},\widetilde{Z}}\widetilde{X} - \gene{\widetilde{X},\widetilde{Z}}\widetilde{Y} ),\widetilde{\eta}}=k(\gene{\gene{\widetilde{Y},\widetilde{Z}}\widetilde{X},\eta} - \gene{\gene{\widetilde{X},\widetilde{Z}}\widetilde{Y} ),\widetilde{\eta}})=
\]
\[
k\gene{\widetilde{Y},\widetilde{Z}}\gene{\widetilde{X},\widetilde{\eta}}-k\gene{\widetilde{X},\widetilde{Z}}\gene{\widetilde{Y},\widetilde{\eta}}
\]
ya que los campos $X$ y $Y$ son ortogonales a $\eta$. Vamos a probar ahora que
\[
R^\perp(X,Y)\eta=\sigma(X,A_\eta(Y))-\sigma(Y,A_\eta)
\]
usando la ecuación de Ricci, es decir, 
\[
\gene{\overline{R}(\widetilde{X},\widetilde{Y})\widetilde{\eta},\widetilde{\xi}}=\gene{R^\perp(X,Y)\eta,\xi}-\gene{[A_\eta,A_\xi](X),Y}
\]
para todo $\xi\in\mathfrak{X}^\perp(f)$. Veamos que el $\gene{R^\perp(X,Y)\eta,\xi}=0$
\[
\gene{\overline{R}(\widetilde{X},\widetilde{Y})\widetilde{\eta},\widetilde{\xi}}=k\gene{\widetilde{Y},\widetilde{\eta}}\gene{\widetilde{X},\widetilde{\xi}}-k\gene{\widetilde{X},\widetilde{\eta}}\gene{\widetilde{Y},\widetilde{\xi}}=0
\]
para todo $\xi\in\mathfrak{X}^\perp(f)$. Además, recordemos que por definición 
\[
[A_\eta,A_\xi](X)=A_\eta(A_\xi(X))-A_\xi(A_\eta(X)).
\]
Además, por ser el endomorfismo de Weingarte autoadjunto
\[
\gene{A_\eta(A_\xi(X)),Y}=\gene{A_\xi(X),A_\eta(Y)}=\gene{\sigma(X,A_\eta(Y)),\xi}
\]
y análogamente intercambiando los papeles de $X$ e $Y$. Con esto, llegamos a que para todo $\xi\in\mathfrak{X}^\perp(f)$
\[
\gene{R^\perp(X,Y)\eta,\xi}=\gene{\sigma(X,A_\eta(Y)),\xi}-\gene{\sigma(Y,A_\eta(X)),\xi}
\]
Usando la linealidad de la métrica 
\[
\gene{R^\perp(X,Y)\eta-\sigma(X,A_\eta(Y))+\sigma(Y,A_\eta(X)),\xi}=0
\]
para todo $\xi\in\mathfrak{X}^\perp(f)$. Basta entonces observar que $R^\perp(X,Y)\eta-\sigma(X,A_\eta(Y))+\sigma(Y,A_\eta(X))\in\mathfrak{X}^\perp(f)$. Como al hacer el producto escalar por sí mismo sale 0, obtenemos entonces 
\[
R^\perp(X,Y)\eta-\sigma(X,A_\eta(Y))+\sigma(Y,A_\eta(X))=0
\]
de donde se deduce la igualdad que buscábamos. 

\end{solucion}

\begin{ejercicio}{2}
Sea $f : M → \overline{M}$ una inmersión isométrica, donde $\overline{M}$ es una variedad riemanniana
con curvatura seccional constante. Supongamos además que $\dim(M) = \dim(\overline{M}) − 1$
(es decir, que $M$ es una hipersuperficie de $\overline{M}$).
Probar, usando la ecuación de Codazzi, que
\[
A_η([X, Y ]) = ∇_X(A_η(Y)) − ∇_Y (A_η(X)),\ X, Y ∈ \mathfrak{X}(M), η ∈ \mathfrak{X}^⊥(f).
\]
\end{ejercicio}
\begin{solucion}
Por el ejercicio anterior, la ecuación de Codazzi queda 
\[
(\overline{∇}_{\widetilde{X}} σ)(Y, Z, η) = (\overline{∇}_{\widetilde{Y}} σ)(X, Z, η).
\]
Como vimos en clase, en el caso de codimensión 1 tenemos que $A_\eta(X)=-\nabla_X \eta$, por lo que $\nabla^\perp\eta=0$. Con esto, obtenemos
\[
\overline{\nabla}_X\sigma(Y,Z,\eta)=X(\gene{A_\eta(Y),Z}-\gene{A_\eta(\nabla_XY),Z}-\gene{A_\eta(Y),\nabla_XZ}=
\]
\[
\gene{\nabla_X(A_\eta(Y)),Z}-\gene{A_\eta(\nabla_XY),Z}
\]
Por lo tanto, la ecuació de Codazzi se puede escribir como 
\[
A_η([X, Y ]) = ∇_X(A_η(Y)) − ∇_Y (A_η(X))
\]
tal como queríamos demostrar. 
\end{solucion}

\begin{ejercicio}{3}
Sea $f : M → \overline{M}$ una inmersión isométrica, y supongamos que $\dim(M) = \dim(\overline{M})−1$
(es decir, $M$ es una hipersuperficie de $\overline{M}$).
Comprobar que la ecuación de Ricci, en este caso, no da ninguna información.

\end{ejercicio}
\begin{solucion}
\end{solucion}


\begin{ejercicio}{4}
Sea $(M, g)$ una variedad riemanniana arbitraria.
Sea $X ∈ \mathfrak{X}(M)$. Diremos que $X$ es un \emph{campo de Killing} en $M$ si
\[
\gene{∇_Y X, Z} + \gene{Y, ∇_X Z} = 0,\text{ para cualesquiera }Y, Z ∈ \mathfrak{X}(M).
\]
Sea ahora $f : M → \overline{M}$ una inmersión isométrica, y sea $\overline{X} ∈ \mathfrak{X}(\overline{M})$ un campo de
Killing en $\overline{M}$. Probar formalmente que
\[
X = (df)^{−1}(\overline{X} \circ f) : M → TM,
\]
definido por
\[
X_p = (df)^{−1}_p(\overline{X}_{f(p)}) ∈ T_pM,
\]
es un campo de Killing en M.
\end{ejercicio}
\begin{solucion}
\url{https://math.stackexchange.com/questions/2900274/fm-rightarrow-n-an-isometry-show-that-x-is-a-killing-field-on-m-if-and}
isometría implica difeomorfismo afín (3.3.3 TFG), o sea, que hacer $df$ fuera es como hacerlo al mismo tiempo abajo y dentro
\end{solucion}


\end{document}
