	\documentclass[twoside]{article}
\usepackage{../estilo-ejercicios}
\renewcommand{\baselinestretch}{1,3}
%--------------------------------------------------------
\begin{document}

\title{Ejercicios de la segunda parte de Geometría Semi-Riemanniana}
\author{Javier Aguilar Martín}
\maketitle


\begin{ejercicio}{1}
 Sea $f : M → \overline{M}$ una inmersión isométrica, donde $\overline{M}$ es una variedad riemanniana
con curvatura seccional constante.
Probar que, en tal caso, la ecuación de Codazzi queda
\[
(\overline{∇}_{\widetilde{X}} σ)(Y, Z, η) = (\overline{∇}_{\widetilde{Y}} σ)(X, Z, η),\ X, Y, Z ∈ \mathfrak{X}(M), η ∈ \mathfrak{X}^⊥(f),
\]
y que la ecuación de Ricci implica que
\[
R^⊥(X, Y )η = σ(X, A_η(Y )) − σ(Y, A_η(X)),\ X, Y ∈ \mathfrak{X}(M), η ∈ \mathfrak{X}^⊥(f).
\]
\end{ejercicio}
\begin{solucion}
 La ecuación de Codazzi en general tiene la expresión 
 \[
\gene{\overline{R}(\widetilde{X},\widetilde{Y})\widetilde{Z},\widetilde{\eta}}=(\overline{\nabla}_{\widetilde{X}}\sigma)(Y,Z,\eta)-(\overline{\nabla}_{\widetilde{Y}}\sigma)(X,Z,\eta)
\]
Como $\overline{M}$ tiene curvatura seccional constante $k$, tenemos la expresión 
\[
\overline{R}(\widetilde{X}, \widetilde{Y} )\widetilde{Z} = k(\gene{\widetilde{Y},\widetilde{Z}}\widetilde{X} - \gene{\widetilde{X},\widetilde{Z}}\widetilde{Y} ),
\]
la cual implica que $\overline{R}_p(\widetilde{X}_p, \widetilde{Y}_p )\widetilde{Z}_p$ es tangente para todo $p\in\overline{M}$, de modo que el producto escalar $\gene{\overline{R}(\widetilde{X},\widetilde{Y})\widetilde{Z},\widetilde{\eta}}=0$, de donde se deduce la igualdad del enunciado. 

LO SIGUIENTE NO TENGO NI IDEA
\end{solucion}

\begin{ejercicio}{2}
Sea $f : M → \overline{M}$ una inmersión isométrica, donde $\overline{M}$ es una variedad riemanniana
con curvatura seccional constante. Supongamos además que $\dim(M) = \dim(\overline{M}) − 1$
(es decir, que $M$ es una hipersuperficie de $\overline{M}$).
Probar, usando la ecuación de Codazzi, que
\[
A_η([X, Y ]) = ∇_X(A_η(Y)) − ∇_Y (A_η(X)),\ X, Y ∈ \mathfrak{X}(M), η ∈ \mathfrak{X}^⊥(f).
\]
\end{ejercicio}
\begin{solucion}
PRUEBA CON LAS DEFINICIONES
\end{solucion}

\begin{ejercicio}{3}
Sea $f : M → \overline{M}$ una inmersión isométrica, y supongamos que $\dim(M) = \dim(\overline{M})−1$
(es decir, $M$ es una hipersuperficie de $\overline{M}$).
Comprobar que la ecuación de Ricci, en este caso, no da ninguna información.

\end{ejercicio}
\begin{solucion}
\end{solucion}


\begin{ejercicio}{4}
Sea $(M, g)$ una variedad riemanniana arbitraria.
Sea $X ∈ \mathfrak{X}(M)$. Diremos que $X$ es un \emph{campo de Killing} en $M$ si
\[
\gene{∇_Y X, Z} + \gene{Y, ∇_X Z} = 0,\text{ para cualesquiera }Y, Z ∈ \mathfrak{X}(M).
\]
Sea ahora $f : M → \overline{M}$ una inmersión isométrica, y sea $\overline{X} ∈ \mathfrak{X}(\overline{M})$ un campo de
Killing en $\overline{M}$. Probar formalmente que
\[
X = (df)^{−1}(\overline{X} \circ f) : M → TM,
\]
definido por
\[
X_p = (df)^{−1}_p(\overline{X}_{f(p)}) ∈ T_pM,
\]
es un campo de Killing en M.
\end{ejercicio}
\begin{solucion}
\end{solucion}


\end{document}
