\documentclass[MyPE.tex]{subfiles}
\setcounter{theorem}{1}
\begin{document}

\chapter{Estimación núcleo}
\section{Concepto de probabilidad}
\begin{theorem}
Sea $K:\R\to\R$ una función medible tal que es acotada, integrable y satisface
$$
|x||K(x)|\to 0 \quad |x|\to \infty
$$
Sea $g:\R\to\R$ una función medible e integrable. Sea además
$$
g_n(x) = \frac{1}{h}\int K(z/h)g(x-z)dz
$$
donde $0<h=h_n \to 0$. Entonces se tiene
\begin{itemize}
\item Si $g$ es continua en $x$ 
$$
g_n(x)\to g(x)\int K(z)dz
$$
\item Si $g$ es uniformemente continua
$$
\sup_x \abs{g_n(x)-g(x)\int K(z)dz}\to 0
$$
\end{itemize}
\end{theorem} 

\begin{proof}
\begin{align*}
g_n(x)-g(x)\int K(z)dz &=  \frac{1}{h}\int K(z/h)g(x-z)dz-g(x)\int K(z)dz \\
&=
  \frac{1}{h}\int K(z/h)(g(x-z)-g(x))dz 
\end{align*}
Podemos separar la integral en los dominios $|z|<\delta$ y $|z|\geq \delta$. No precisamos quién es $\delta$ sino que se escogerá posteriormente de manera conveniente. A la primera la denominamos $I_1$ y a la segunda $I_2$. Tenemos que
$$
\abs{I_1} \leq \sup_{|z|\leq \delta} |g(x-z)g(z)|\frac{1}{h} \int_{|z|<\delta} |K(z/h)|dz
$$
\newpage
Dado que $g$ es continua en x, $\forall \varepsilon >0$ $\exists \delta=\delta(\varepsilon,x)$ tal que $|z-x|<\delta$ implica $|g(x)-g(z)| < \varepsilon$. Si $g$ es uniformemente continua entonces $\delta = \delta(\varepsilon)$. Tenemos entonces que
$$
\abs{I_1} \leq \varepsilon \frac{1}{h}  \int_{|z|<\delta} |K(z/h)|dz \leq \varepsilon \int|K(z)|dz
$$
$$
\abs{I_2} \leq \frac{1}{h} \int_{|z|\geq \delta} |K(z/h)||g(x-z)|dz + \frac{1}{h} \int_{|z|\geq \delta} |K(z/h)||g(x)|dz = I_3 + I_4 
$$
Como $h \to 0$, $\delta/h\to \infty$, luego para $n$ suficientemente grande podemos acotar el resto de una integral covergente, de manera que
$$
I_4 = |g(x)|\frac{1}{h} \int_{|z|\geq \delta/h} K(z)dz < \varepsilon \abs{g(x)}
$$
Finalmente
$$
I_3 = \frac{1}{h} \int_{|z|\geq \delta} \frac{|z|}{|z|}|K(z/h)||g(x-z)|dz \leq \sup_{|z|\geq \delta} \frac{|z|}{|h|} |K(z/h)| \frac{1}{\delta} \int|g(z)|dz
$$

\end{proof}
\begin{coro}
En nuestro caso podemos aplicar el teorema anterior domando $K$ como la función núcleo y $g$ al estimador del núcleo, es decir,
$$
\hat{f}(x) = \frac{1}{nh}\sum_{i=1}^n K\left(\frac{x-X_i}{h}\right)
$$
En las condiciones del teorema anterior
$$
E[\hat{f}(x)] = \frac{1}{h}E\left[K\left(\frac{x-X}{h}\right)\right] = \int \frac{1}{h} K\left(\frac{x-y}{h}\right) f(y) dy \to f(x) \int K(z)dz = f(x)$$
Si además $f$ es uniformemente continua se verifica la segunda parte del teorema.
\end{coro}
\begin{coro}
Si denotamos para cada $x,n$ fijos $Z_{i,n} = K((x-X_i)h^{-1})h^{-1}$, entonces $\hat{f}(x) = \frac{1}{n}\sum_{i=1}^n Z_{i,n}$. Para cada $n$ fijo, las $Z_{i,n}$ son i.i.d., aunque para $n$ distintos no lo son. Además, del corolario anterior se deduce que
$$
E[Z_{i,n}] = \int\frac{1}{h} K\left(\frac{x-y}{h}\right) f(y) dy \to f(x) \int K(z)dz = f(x)
$$
Si queremos estimar la varianza de esta variable, lo más sencillo es calcular el momento de segundo orden de la misma forma. Tan solo es necesario ver que $K^2(\cdot)$ está en las condiciones de la $K(\cdot)$ del Teorema 1 (lo cuál es trivial). Por tanto
$$
hE[Z_{i,n}^2] = \int\frac{1}{h} K^2\left(\frac{x-y}{h}\right) f(y) dy \to f(x) \int K^2(z)dz
$$
Por tanto, para $n$ suficientemente grande podemos aproximar
$$
Var[Z_{i,n}] \approx \frac{1}{h}f(x) \int K^2(z)dz - f(x)^2
$$
\end{coro}
\newpage
\begin{theorem}
Sea $K:\R\to\R$ en las condiciones del Teorema 2, sea $0<h = h_n \to 0$ tal que $hn \to \infty$ y sea ${f}$ continua en $x$. Entonces 
$$
MSE\{\hat{f}(x)\}\to 0
$$
\end{theorem}
\begin{proof}
Sabemos que
$$
MSE(\hat{f}(x)) = Var[\hat{f}(x)] + \left(E[\hat{f}(x)]-f(x)\right)^2
$$
Como consecuencia del Corolario 1, el término que corresponde al sesgo tiende a $0$, pues $E[\hat{f}(x)]\to f(x)$. Además, por el  Corolario 2
$$
Var[\hat{f}(x)] \approx \frac{f(x)}{hn}\left(\int K^2(z)dz - hf(x)^2\right)  \approx \frac{f(x)}{hn}\int K^2(z)dz 
$$
Como $hf(x) \to 0$ y $f(x)\int K^2(z)dz$ es finito y $hn\to \infty$ tenemos que $Var[\hat{f}(x)] \to 0$ y el resultado. 
\end{proof}
\begin{coro}
En las condiciones del teorema anterior
$$
\hat{f}(x)\overset{p}{\longrightarrow}f(x)
$$
\end{coro}
\begin{proof}
Tenemos que probar que para cada $\varepsilon>0$ fijo
$$
\lim_{n\to\infty}P\left[\abs{\hat{f}(x)-f(x)}>\varepsilon\right]=0
$$
Para esto, aplicamos la desigualdad de Chebysev, que establece
$$
P[|X-\mu|\geq \varepsilon ]\leq \frac{Var[X]^2}{\varepsilon}
$$
En nuestro caso, esto se traduce en que
$$
P\left[\abs{\hat{f}(x)-f(x)}>\varepsilon\right] \leq \frac{Var[\hat{f}(x)]}{\varepsilon^2} \approx \frac{MSE(\hat{f}(x))}{\varepsilon^2} \to 0 
$$
En clase hemos utilizado en mi opinión innecesariamente el $MSE$ en la desigualdad, aunque al fin y al cabo para poder aplicar Chebyshev es necesario que el estimador sea asintóticamente insesgado y para que funcione la demostración que la varianza del estimador converja a $0$, que es equivalente a ser consistentemente insesgado.
\end{proof}
\begin{defi}
Diremos que un conjunto de variables $\{X_{i,n} \mid 1\leq i \leq n\}_{n\geq 1}$ satisface la \textbf{condición de Lindeberg} si
$$
\lim_{n\to \infty} \frac{1}{s_n^2}\sum_{k=1}^n E[(X_{k,n}-\mu_{k,n})^2 1_{\{|X_{i,n}-\mu_{k,n}|>\varepsilon s_n\}}]=0
$$
donde $s_n^2 = \sum_{k=1}^n Var[X_{k,n}]$.
\end{defi}
\newpage
\begin{defi}
Diremos que un conjunto de variables $\{X_{i,n} \mid 1\leq i \leq n\}_{n\geq 1}$ satisface la \textbf{condición de Liapunov} si $\exists \delta >0$  tal que
$$
\frac{\sum_{i=1}^n E[|X_{i,n}^{2+\delta}|]}{s_n^{2+\delta}} \to 0
$$
donde $s_n^2 = \sum_{k=1}^n Var[X_{k,n}]$.
\end{defi}
\begin{prop}
La condición de Liapunov implica la condición de Lindeberg.
\end{prop}
\begin{prop}
Sea array triangula $\{X_{i,n}\mid 1\leq i \leq n\}_{n\geq 1}$ de manera que para cada $n$ fijo las variables del conjunto son independientes (variables independientes por filas), que verifica $\forall n \geq 1$
$$
E[X_{i,n}] = 0, \; \leq i \leq n 
$$
$$
0< E[X_{i,n}^2] = \sigma_{i,n}^2 < \infty, \; 1\leq i \leq n
$$
y que además satisfacen la condición de Lindeberg. Entonces 
$$
\frac{\sum_{i=1}^n X_{i,n}}{s_n} \to N(0,1)
$$
donde $s_n^2 = \sum_{i=1}^n Var[X_{i,n}]$.
\end{prop}
\setcounter{theorem}{4}
\begin{theorem}
Sea $K:\R\to\R$ una función medible, acotada, integrable, simétrica tal que
$$
|x||K(x)|\to 0 \quad |x|\to \infty 
$$
$$
\int K(x)dx = 1 \qquad A=\int x^2 K(x)dx <\infty
$$
Sea $h=h_n \to 0$ tal que $nh\to \infty$. Si $f$ es continua en $x$ entonces 
$$
\sqrt{nh}(\hat{f}(x)-E[\hat{f}(x)])\overset{d}{\longrightarrow}N(0,\sigma^2)
$$
donde $\sigma^2 = f(x)\int K^2(u)du$.
\end{theorem}
\begin{proof}
Sabemos que $\hat{f}(x)$ es la media aritmética de unas ciertas variables $Z_{i,n}$ i.i.d. para cada $n$ fijo. Aunque no podemos usar el TCL tradicional, podemos usar el anterior. Sea $X_{i,n} = Z_{i,n} - E[z_{i,n}]$.  Por construcción las $X_{i,n}$ tienen media $0$ y varianza aproximadamente (para $n$ suficentemente grande) $ f(x)\int K^2(u)du/h  = \sigma^2/h$. Veamos que se satisface la condición de Liapunov 
\begin{align*}
E[|X_{i,n}^{2+\delta}|] &= \int \frac{1}{h^{2+\delta}}\abs{K\left(\frac{x-y}{h}\right) - h E[z_{i,n}]}^{2+\delta} f(y)dy \\
&= 
\int \frac{1}{h^{2+\delta}}|K(u) - h E[z_{i,n}]|^{2+\delta} f(x+uh)du 
\end{align*}
Se puede probar que la función integrando está en las condiciones del Teorema de Convergencia Dominada, de donde obtenemos junto con el Teorema 2
$$
h^{1+\delta}E[|X_{i,n}^{2+\delta}|] \to f(x)\int |K(u)|^{2+\delta}du < \infty
$$
Se tiene pues
$$
\frac{\sum_{i=1}^n E[|X_{i,n}^{2+\delta}|]}{s_n^{2+\delta}}  \approx \frac{n}{h^{1+\delta}} \frac{f(x)\int {|k(u)|^{2+\delta} du}}{(\frac{n}{h}f(x)\int k^2(u)du )^{1+\delta/2}} =
M \frac{n}{n^{1+\delta/2}} \frac{h^{1+\delta/2}}{h^{1+\delta}} = M \frac{1}{(nh)^{\delta/2}} \to 0
$$
Quedando probada la condición de Liapunov, deducimos
$$
\frac{\sum_{i=1}^n Z_{i,n}-E[Z_{i,n}]}{\left(n\sigma^2/h\right)^{½}} \to N(0,1)
$$
Dividiendo arriba y abajo por $n$ tenemos 
$$
\sqrt{nh} \frac{\hat{f}(x)-E[\hat{f}(x)]}{\sigma} \to N(0,1)
$$
\end{proof}
\begin{nota}
La varianza muestral no es un estimador robusto.
\end{nota}
\begin{example}
Probar que 
$$
\frac{1}{n}\sum_{i=1}^n \hat{f}(X_i)
$$
es un estimador sesgado de $\int \hat{f(x)}f(x)dx$.
\end{example}
\begin{prop}
Se tiene que
$$
E\left[\frac{1}{n}\sum_{i=1}^n \hat{f}_{-i}(X_i)\right] = E\left[\int \hat{f}(x)f(x)dx \right]
$$

\end{prop}
\begin{proof}
\begin{align*}
E\left[\int \hat{f}(x)f(x)dx \right] &= E\left[\int \frac{1}{h}K\left(\frac{x-X_i}{h}\right)f(x)dx \right]\\
&=\int \int   \frac{1}{h}K\left(\frac{x-y}{h}\right)f(x)f(y)dxdy
\end{align*}
Si desarrollamos el otro lado
\begin{align*}
E\left[\frac{1}{n}\sum_{i=1}^n \hat{f}_{-i}(X_i)\right] &= E\left[\hat{f}_{-1}(X_1)\right]\\
&= E\left[\frac{1}{h}K\left(\frac{X_1-X_2}{h}\right)\right]\\
&=\int \int   \frac{1}{h}K\left(\frac{x-y}{h}\right)f(x)f(y)dxdy
\end{align*}
\end{proof}

\begin{prop}
\end{prop}
\begin{proof}
Sumar y restar $Y_i$ y devidir en dos integrales, de manera que una queda $0$ por ser simétrica y la otra $Y_i$ por hipótesis de que $L$ integra con valor $1$.
\begin{align*}
\hat{m}(x) &= \int y \frac{\hat{f}(x,y)}{\hat{f}_X(x)}dy\\
&=\frac{1}{\hat{f}_X(x)}\frac{1}{nh}\sum_{i=1}^n K\left(\frac{x-X_i}{h}\right)\int y\frac{1}{g}L\left(\frac{y-Y_i}{g}\right)\\
&=\frac{1}{\hat{f}_X(x)}\frac{1}{nh}\sum_{i=1}^n K\left(\frac{x-X_i}{h}\right)\left(\int (y-Y_i)\frac{1}{g}L\left(\frac{y-Y_i}{g}\right)dy + \int Y_i\frac{1}{g}L\left(\frac{y-Y_i}{g}\right)dy \right)\\
&=\sum_{i=1}^n \frac{K\left(\frac{x-X_i}{h}\right)}{\sum_{j=1}^n K\left(\frac{x-X_j}{h}\right)}Y_i
\end{align*}
\end{proof}

\begin{theorem}
Supongamos que
\begin{itemize}
\item $X_1,\dotsc,X_n$ tiene fdd $f$ con soporte $[0,1]$, $f$ continuamente diferenciable.
\item $m$ es dos veces continuamente diferenciable en $(0,1)$ y que $v$ es continua.
\item $K$ es simétrico y acotado con soporte $[-1,1]$.
\item $h\to 0$ y $nh^2 \to \infty$.
\item $x$ es un punto interior del soporte, con $f(x)>0$.
\end{itemize}
Entonces
$$
E[\hat{m}(x;1,h)\mid X_1,\dotsc X_n]-m(x) = \frac{1}{2}h^2\mu_2(K)m''(x)+o_p(h^2)
$$
donde 
$$
\mu_2(K) = \int x^2 K(x)dx
$$
\end{theorem}
\begin{proof} 
Sabemos que
\begin{align*}
m(x_1) &= m(x)+m'(x)(x_1-x)+\frac{1}{2}m''(x)(x_1-x)^2+r_1 \\
&=
\begin{pmatrix}
1 & x_1-x
\end{pmatrix}
\begin{pmatrix}
m(x)\\
m'(x)
\end{pmatrix}+\frac{1}{2}m''(x)(x_1-x)^2+r_1
\end{align*}
donde $r_1 = O((x_1-x)^2)$. Por tanto
\begin{align*}
E[\hat{m}(x;1,h)\mid X_1,\dotsc X_n] &= (1,0)(X'WX)^{-1}X'WE[Y|X]\\
&= 
(1,0)(X'WX)^{-1}X'W
X\begin{pmatrix}
m(x)\\
m'(x)
\end{pmatrix}+\frac{1}{2}m''(x)
\begin{pmatrix}
(x_1-x)^2\\
\vdots \\
(x_n-x)^2
\end{pmatrix}+r_1)\\
&=m(x) + \frac{1}{2}m'(x)(1,0)(X'WX)^{-1}X'W\begin{pmatrix}
(x_1-x)^2\\
\vdots \\
(x_n-x)^2
\end{pmatrix} + O(\text{menor orden})
\end{align*}
Sea el vector de las $(x_i-x)^2$. Si $s_l (x,h)=\frac{1}{n}\sum_{i=1}^n(x_i-x)^lK_h(x-x_i)$. Entonces
$$
\frac{1}{n}X'WX = \begin{pmatrix}
s_0(x,h) &s_1(x,h) \\
s_1(x,h) & s_2(x,h)
\end{pmatrix} \qquad \frac{1}{n}X'WV = \begin{pmatrix}
s_2(x,h)\\
s_3(x,h) 
\end{pmatrix}
$$
Se tiene además
$$
s_l(x,h) = \begin{cases}
f(x)\mu_{l+1}(K)h^l+o_p(h^l) & \text{$l$ par}\\
f'(x)\mu_{l+1}(K)h^{l+1}+o_p(h^{l+1})& \text{$l$ impar}
\end{cases}
$$
Como nuestra $K$ es función de densidad, $\mu_0(K)=1$. Como $K$ es simétrico, $\mu_1(K)=0$. Sustituyendo esto en la expresión anterior y esta a su vez en la anterior obtenemos 
$$
E[\hat{m}(x;1,h)\mid X_1,\dotsc X_n]-m(x)
$$
\end{proof}
\end{document}
