	\documentclass[twoside]{article}
\usepackage{../../estilo-ejercicios}
\setcounter{section}{0}
\newtheorem{defin}{Definición}[section]
\renewcommand{\baselinestretch}{1,4}
%--------------------------------------------------------
\begin{document}

\title{Éléments de Géométrie Algébrique, Remarque 2.3.6}
\author{Javier Aguilar Martín}
\maketitle

%\varprojlim
%\varinjlim
\section{Pasos previos}

\subsection{Límite proyectivo}

\begin{defin}[EGA, 0-1.1.9]
En una categoría $\CC$, dado un conjunto parcialmente ordenado $I$, se define un \emph{sistema proyectivo} como una familia $(A_\alpha)_{\alpha\in I}$ de objetos de $\CC$ y para cada par $(\alpha,\beta)$ tal que $\alpha\leq\beta$ un morfismo $u_{\alpha\beta}:A_\beta\to A_\alpha$ cumpliendo $u_{\alpha\gamma}=u_{\alpha\beta}\circ u_{\beta\gamma}$ para $\alpha\leq\beta\leq\gamma$. Un \emph{límite proyectivo} de un sistema proyectivo está formado por un objeto $B\in\CC$ (denotado $\varprojlim A_\alpha$), y para cada $\alpha\in I$, un morfismo $u_\alpha:B\to A_\alpha$ tales que: 
\begin{enumerate}
\item $u_\alpha=u_{\alpha\beta}u_{\beta}$ para $\alpha\leq\beta$;
\item para todo objeto $X$ de $C$ y toda familia $(v_\alpha)_{\alpha\in I}$ de morfismos $v_\alpha:X\to A_{\alpha}$ verificando $v_\alpha=u_{\alpha\beta}v_{\beta}$ para $\alpha\leq\beta$, existe un único morfismo $v:X\to B$ (denotado $\varprojlim v_\alpha$) tal que $v_\alpha=u_\alpha v$ para todo $\alpha\in I$
\[
\begin{tikzcd}
X\arrow[drrr, bend left=20, "v_\gamma"']\arrow[drr, bend left=15, "v_\beta"']\arrow[dr, bend left=10, "v_\alpha"']\arrow[dd, dashed, bend right, "\varprojlim v_\alpha"']& & & & \\
\cdots &A_\alpha\arrow[l]  & A_\beta\arrow[l, "u_{\alpha\beta}"']  & \arrow[l, "u_{\beta\gamma}"']A_\gamma &\cdots\arrow[l]\\
\varprojlim A_\alpha\arrow[urrr, bend left=-20, "u_\gamma"']\arrow[urr, bend left=-15, "u_\beta"']\arrow[ur, bend left=-10, "u_\alpha"']&  & & & 
\end{tikzcd}
\]
\end{enumerate}
Decimos que $\CC$ \emph{admite límites proyectivos} si siempre existen límites proyectivos de sistemas proyectivos en $\CC$. 
\end{defin}

Se prueba fácilmente que el límite proyectivo es único salvo isomorfismo único. SI ME DA TIEMPO PROBARLO

%EN TOP ES INTERSECCIÓN CON TOPOLOGÍA INICIAL (LA QUE CON MENOS ABIERTOS HACE CONTINUAS A TODAS LAS FLECHAS AL SISTEMA PROJECTIVO) POR EL EJERCICIO 1.12 DE HARTSHORNE EL LÍMITE INVERSO DE HACES ES HAZ
%EL LÍMITE INVERSO EN ESPACIOS TOPOLÓGICOS POR SI ME HACE FALTA, DE TODOS MODOS EN SET SEGURAMENTE SÍ ME HAGA FALTA\url{https://math.stackexchange.com/questions/711334/inverse-limit-of-an-inverse-system-of-topological-spaces}

\begin{ej}[$\Set$ admite límites proyectivos]\label{set}
Sea $\{A_i\}_{i\in I}$ un sistema proyectivo de conjuntos con aplicaciones $f_{ji}:A_j\to A_{i}$. Construimos $A=\{(a_i)_{i\in I}\in \prod_i A_i\mid f_{ji}(a_j)=a_i, i\leq j\}$ y afirmamos que $A=\varprojlim A_i$. Es claro que existen aplicaciones $f_i:A \to A_i$ tales que $a_i\mapsto a_i$ compatibles con el sistema proyectivo por definición.

Sea $X$ otro conjunto con aplicaciones $g_i:X\to A_i$ compatibles con el sistema inverso. Sea $x\in X$ con imagen $g_i(x)=a_i\in A_i$. Definimos entonces $g:X\to A$ como $g(x)=(g_i(x))_{i\in I}$, que está bien definida porque los $g_i$ son compatibles con el sistema proyectivo y claramente conmuta con las aplicaciones del sistema. Supongamos ahora que existe otra aplicación $h:X\to A$ compatible con el sistema inverso, en particular $f_jh=g_j$ para todo $j\in I$. Cada $f_jh(x)$ son las coordenadas de $h(x)$ en $A$, que son justamente $g_j(x)=a_j$, luego en realidad $h=g$. 

Esta misma construcción con estructura adicional sirve para probar la existencia de límites proyectivos en categoría como $\Top$ o $\mathrm{\mathbf{Ab}}$.
\end{ej}

\subsection{Haces con valores en una categoría}

\begin{defin}[EGA, 0-3.1.1]\label{axioma}
Dado un espacio topológico $X$, denotamos por $\OO(X)$ a la categoría cuyos objetos son los abiertos de $X$ y donde, para dos abiertos $U$ y $V$, $\Hom(U,V)=\emptyset$ si $U\not\subseteq V$ y si $U\subseteq V$ tiene un solo elemento asociado a la inclusión. Podemos decir entonces, que un \emph{prehaz sobre $X$ con valores en $\CC$} es un functor contravariante $\calF:\OO(X)\to\CC$. Diremos que un prehaz $\calF$ sobre $X$ con valores en $\CC$ es un \emph{haz} si verifica el axioma siguiente:

\begin{verse}
Para todo objeto $T\in\CC$, $U\mapsto\Hom(T,\calF(U))$ es un haz de conjuntos. %EN EL SENTIDO USUAL
\end{verse}

Explícitamente, esta condición significa que para todo recubrimiento $(U_\alpha)$ de un abierto $U$ de $X$ por abiertos contenidos en $U$, si denotamos $\rho_\alpha:\calF(U)\to\calF(U_\alpha)$ y $\rho_{\alpha\beta}:\calF(U_\alpha)\to\calF(U_\alpha\cap U_\beta)$ a los morfismos de restricción, entonces la aplicación que a todo $f\in\Hom(T,\calF(U))$ le hace corresponder la familia $(\rho_\alpha\circ f)\in \prod_\alpha\Hom(T,\calF(U_\alpha))$ es una biyección entre $\Hom(T,\calF(U))$ y las $(f_\alpha)$ tales que $\rho_{\alpha\beta}\circ f_\alpha=\rho_{\beta\alpha}\circ f_\beta$ para toda pareja de índices $(\alpha,\beta)$. 
\end{defin}

\begin{observacion}\label{ecualizador}
Cuando en la categoría $\CC$ existen los productos, la condición anterior se expresa de la manera siguiente: sean
\[
r=(\rho_\alpha):\calF(U)\to\prod_\alpha\calF(U_\alpha)
\]
y sean
\[
r':\prod_\alpha\calF(U_\alpha)\to\prod_{\alpha,\beta}\calF(U_\alpha\cap U_\beta)\text{ tal que }\pi_{\alpha\beta}\circ r'=\rho_{\alpha\beta}\circ \pi_{\alpha}
\]
\[
r'':\prod_\alpha\calF(U_\alpha)\to\prod_{\alpha,\beta}\calF(U_\alpha\cap U_\beta)\text{ tal que }\pi_{\alpha\beta}\circ r''=\rho_{\alpha\beta}\circ \pi_{\beta}
\]
donde $\pi_{\alpha\beta}$ es la proyección $\prod_{\alpha,\beta}\calF(U_\alpha\cap U_\beta)\to \calF(U_\alpha\cap U_\beta)$. Entonces podemos decir que $\calF$ es un haz si el diagrama
\[
\begin{tikzcd}
\calF(U)\arrow[r, "r"]&\prod_\alpha\calF(U_\alpha)\arrow[r, shift left, "r'"]\arrow[r, shift right, "r''"'] &\prod_{\alpha,\beta}\calF(U_\alpha\cap U_\beta)
\end{tikzcd}
\]
es exacto, es decir, que $r$ es un monomorfismo y $\calF(U)=Eq(r',r'')$. %r es monomorfismo en el caso de las restricciones por la condición de haz
\end{observacion}


\begin{prop}[EGA, 0-3.2.6]\label{limite}
Sea $\CC$ una categoría que admite límites proyectivos y $X$ un espacio topológico. La categoría de haces sobre $X$ con valores en $\CC$ admite límites proyectivos. 
\end{prop}
\begin{dem}
Este resultado es una generalización del ejercicio II.1.12 de Hartshorne, con idéntica prueba.

Sea $\{\calF_i\}$ un sistema inverso de haces en $X$ y consideramos el prehaz $U\mapsto\varprojlim\calF_i(U)$, que está bien definido porque $\CC$ admite límites proyectivos. Probamos que este prehaz, al que denotamos $\calF$, es un haz y que tiene la propiedad universal de un límite proyectivo en la categoría de haces. 

Sea $U=\bigcup_{k\in I} U_k$ y secciones $s_k\in\calF(U_k)$ tales que $s_k|_{U_k\cap U_l}=s_l|_{U_k\cap U_l}$, entonces sus imágenes $s_k^i\in \calF_i(U)$ también son compatibles y admiten una extensión única $s^i\in\calF_i(U)$. Si $u_{i'i}:\calF_{i'}\to\calF_i$, entonces tenemos que $(u_{i'i})(s_{i'})_P=(u_{i'i})_P(s_{i'})_P=(s_i)_P$ puesto que $u_{i'i}|_{U_k}(s^{i'}_k)=s^i_k$ para todo $k$ y para $P\in U_k$. Deducimos entonces que los $s^i$ son compatibles con las aplicaciones que definen el sistema proyectivo, por lo que obtenemos un elemento $s\in\calF(U)$ que se restringe a $s_k$ para cada $U_k$. 

Supongamos que $f_i:\calG\to\calF_i$ es una familia de morfismos compatible con el sistema proyectivo. Entonces, para cada abierto $U$ tenemos aplicaciones $\calG(U)\to\calF_i(U)$ satisfaciendo la condición de límite proyectivo, tenemos un morfismo único $\calG(U)\to\calF(U)$, que es lo mismo que tener un morfismo único $\calG\to\calF$. 

\end{dem}

\subsection{Prehaces sobre una base de abiertos}

\begin{ej}[EGA, 0-3.2.1]\label{prehaztop}%RECUPERAR LA NOCIÓN HABITUAL DE PREHAZ A PARTIR DE LA CATEGÓRICA, GROTHENDIECK HACE HINCAPIÉ EN EL CASO DE X NOETHERIANO
Sea $X$ un espacio topológico y $\BB$ una base de abiertos para la topología de $X$. Dotamos a esta familia de estructura de categoría considerando un morfismo por cada relación de inclusión. Consideramos entonces un prehaz $\calF:\BB^{op}\to\CC$ con valores en una categoría $\CC$. Podemos asociarle un prehaz con valores en $\CC$ en el sentido habitual, $U\mapsto \calF'(U)$ tomando para todo abierto $U$, $\calF'(U)=\varprojlim\calF(V)$, donde $V$ recorre el conjunto $\BB$ parcialmente ordenado por la inclusión para $V\subseteq U$. Sean $U,U'$ abiertos con $U\subseteq U'$ y denotamos $\rho_{VW}:\calF(V)\to\calF(W)$ las restricciones dadas por $\calF$ para $W\subseteq V$ abiertos básicos.

Si consideramos el sistema inductivo de abiertos básico para $U$, $(U_\alpha)$ y el sistema inductivo de abiertos básicos para $U'$, $(U'_\beta)$, como $U\subseteq U'$, para todo $U_\alpha$ existe $\beta$ tal que $U_\alpha=U'_\beta$, lo que nos da diagramas de inclusiones
\[
\begin{tikzcd}
U_\alpha\arrow[rr, hookrightarrow]\arrow[ddr, hookrightarrow]\arrow[dr, hookrightarrow]& & U_\beta\arrow[dl, hookrightarrow]\arrow[ddl,hookrightarrow]\\
& U\arrow[d, hookrightarrow]& \\
& U' &
\end{tikzcd}
\]
Lo que, en el sistema proyectivo asociado $(\calF(U_\alpha))$, nos da diagramas conmutativos de la forma
\[
\begin{tikzcd}
\calF(U_\alpha)& & \calF(U_\beta)\arrow[ll, "\rho_{U_\beta U_\alpha}"']\\
& \calF'(U)=\varprojlim\calF(U_\alpha)\arrow[ur]\arrow[ul]& \\
& \calF'(U')=\varprojlim\calF(U'_\alpha)\arrow[u, dashed, "\rho'_{U'U}"]\arrow[uul,bend left=10] \arrow[uur,bend right=10]&
\end{tikzcd}
\]
Donde $\rho'_{U'U}$ existe y es único por definición del límite proyectivo, así que definimos de esa forma las restricciones del prehaz $\calF'$.%A ESTO QUE HE HECHO YO LE VEO MÁS SENTIDO, NO SÉ SI ES ESO LO QUE QUERÍA DECIR GROTHENDIECK

%Definimos entonces $\rho'_{U'U}$ como el límite proyectivo (para $V\subseteq U$) de los morfimos canónicos $\calF(U)\to\calF(V)$, o lo que es lo mismo, el único morfismo $\calF'(U')\to\calF(U)$ que compuesto con los morfismos canónicos $\calF'(U)\to\calF(V)$ da los morfismos canónicos $\calF'(U')\to\calF(V)$. La verificación de la transitividad de $\rho'_{U'U}$ es inmediata. Además, si $U\in\BB$, el morfismo canónico $\calF'(U)\to\calF(U)$ es isomorfismo, permitiéndonos identificar estos dos objetos.
\end{ej}
\begin{prop}[EGA, 0-3.2.2]\label{haztop}
Para que el prehaz $\calF'$ definido en el ejemplo anterior, es necesaria y suficiente la siguiente condición sobre $\calF$:

\begin{verse}


Para todo recubrimiento $(U_{\alpha})$ de $U\in\BB$ por conjuntos $U_\alpha\in\BB$ contenidos en $U$, y por todo objeto $T\in\CC$, la aplicación que a todo $f\in\Hom(T,\calF(U))$ le hace corresponder la familia $\rho_{UU_\alpha}\circ f\in\Hom(T,\calF(U_\alpha))$ es una biyección de $\Hom(T,\calF(U))$ en el conjunto de $(f_\alpha)$ tales que $\rho_{U_\alpha V}\circ f_\alpha=\rho_{U_\beta V}\circ f_\beta$ para toda pareja de índices $(\alpha,\beta)$ y todo $V\in B$ contenido en $U_\alpha\cap U_\beta$. 
\end{verse}
\end{prop}
\begin{dem}
Que es necesario es evidente por la definición \ref{axioma} y porque se necesita la compatibilidad con el sistema proyectivo. %CON COMPATIBILIDAD ME REFIERO A QUE ESTÁN COGIÉNDOSE LOS ABIERTOS BÁSICOS. ES NECESARIO QUE F SEA HAZ PORQUE LÍMITE DE PREHACES QUE NO SEA HAZ NO PUEDE DAR HAZ, SINO PODRÍAMOS EXTENDER CON LOS MORFISMOS DEL SISTEMA. 





 %Que es necesario es evidente porque para que sea un haz necesitamos que para toda familia $(f_\alpha)$ de modo que dos cuales quiera $f_\alpha$ y $f_\beta$ coincidan en la restricción al dominio común exista una única $f$ que la extienda. Como la propiedad se cumple para todos los abiertos básicos contenidos en la intersección, se traslada al límite. 

Para probar que es suficiente, consideramos en primer lugar una segunda base $\BB'$ de la topología de $X$ contenida en $\BB$ y probamos que si $\calF''$ es el prehaz definido a partir de la familia $(\calF(V))_{V\in\BB'}$, entonces $\calF''$ es canónicamente isomorfo a $\calF'$. %SE PRUEBA QUE NO DEPENDE DE LA BASE
En efecto, el límite proyectivo (para $V\in\BB'$, $V\subseteq U$) de los morfismos canónicos $\calF'(U)\to \calF(V)$ es un morfismo $\calF'(U)\to\calF''(U)$ para todo abierto $U$. Si $U\in\BB$, este morfismo es un isomorfismo, ya que por hipótesis QUÉ HIPÓTESIS los morfismos canónicos $\calF'(U)\to\calF(V)$ para $V\in\BB'$ ($V\subseteq U$) factorizan como $\calF''(U)\to\calF(U)\to\calF(V)$, y es inmediato ver que la composición de los morfismos $\calF(U)\to\calF''(U)$ y $\calF''(U)\to\calF(U)$ así definidos son identidades. COMPROBAR De esta forma, para todo abierto $U$, los morfismos $\calF''(U)\to\calF''(W)=\calF(W)$ para $W\in\BB$ y $W\subseteq U$ verifican las condiciones que caracterizan al límite proyectivo de los $\calF(W)$ COMPROBARLO, lo que demuestra nuestra afirmación por la unicidad salvo isomorfismo único del límite proyectivo.

Ahora, sea $U$ un abierto cualquiera de $X$, $(U_\alpha)$ un recubrimiento de $U$ por abiertos contenidos en $U$ y sea $\BB'$ la subfamilia de $\BB$ formada por los conjuntos de $\BB$ contenidos en al menos un $U_\alpha$. Es claro que $\BB'$ es entonces una base de la topología de $U$. Por lo tanto $\calF'(U)$ (respectivamente $\calF'(U_\alpha)$) es límite proyectivo de los $\calF(V)$ para $V\in\BB'$ y $V\subseteq U$ (respectivamente $V\subseteq U_\alpha$). El axioma de la definición \ref{axioma} se verifica inmediatamente por la definición de límite proyectivo COMPROBAR


 \end{dem}
 
 




\section{Remarque 2.3.6}


 \subsection{Caso de espacios anillados}

\begin{remarque}{4.5.6}\label{prehaz}
Llamamos \emph{prehaz} sobre una categoría $\CC$ a un functor $F:\CC^{op}\to\Set$. Denotamos por $\Top_{\Ring}|_S$ a la categoría de espacios anillados sobre un espacio anillado fijo $S$, esto es la categoría cuyos objetos son morfismos en $\Top_{\Ring}$ de la forma $X\to S$ (a los que llamamos $S$-espacios) y los morfismos son morfismos $X\to Y$ (llamados $S$-morfismos) en $\Top_{\Ring}$ que hacen conmutativo el diagrama
\[
\begin{tikzcd}
X\arrow[r]\arrow[dr] & Y\arrow[d]\\
 & S
\end{tikzcd}
\]

Un prehaz 
\begin{equation}\label{representable}
F:(\Top_{\Ring}|_S)^{op}\to\Set
\end{equation}
 nos da para todo $S$-espacio $X$ un prehaz de conjuntos $U\to F(U)$ sobre $X$, ya que los abiertos de $X$ tienen una estructura inducida de $S$ espacio. Cuando para todo $S$-spacio $X$ este prehaz es un haz, decimos que $F$ es un \emph{haz sobre la categoría} $\Top_{\Ring}|_S$, noción que se puede generalizar a cualquier categoría usando la topología de Grothendieck de una categoría, aunque no la necesitaremos para nuestros propósitos. Estos functores forman una subcategoría plena de $\HOM((\Top_{\Ring}|_S)^{op},\Set)$ denotada $\Sh_{\Top_{\Ring}|_S}$. Si añadimos a esto la proposición \ref{limite} y el ejemplo \ref{set}, resulta que $\Sh_{\Top_{\Ring}|_S}$ admite límites proyectivos de sistemas proyectivos de functores que son haces sobre un mismo espacio $X$  %COGE COMO ESPACIO EL LÍMITE INVERSO COMO DIGO ARRIBA Y EL HAZ TAMBIÉN LÍMITE INVERSO. %NO SÉ SI ME HARÁ FALTA QUE EXISTAN PARA ESPACIOS DISTINTOS

%RECUERDA LA DEFINICIÓN DE FUNCTOR REPRESENTABLE, EN ESTE CASO EL FUNCTOR ES REPRESENTABLE PORQUE DE HECHO ES UN ELEMENTO DE HOM(A,-)

Un functor representable como \ref{representable} es siempre un haz: es claro que en efecto el prehaz $U\mapsto\Hom_S(U,Z)$ sobre un espacio anillado $X$ es un haz, ya que para todo recubrimiento abierto $(U_\alpha)$ de un abierto $U\subseteq X$, dar un $S$-morfismo de $U\to Z$ equivale a dar una familia de $S$-morfismos $f_\alpha:U_\alpha\to Z$ tales que para toda pareja de índices $(\alpha,\beta)$ las restricciones de $f_\alpha$ y $f_\beta$ a $U_\alpha\cap U_\beta$ coincidan.

Por el lema de Yoneda podemos decir además que tenemos un functor plenamente fiel $Z\to h_Z$ de la categoría $\Top_{\Ring}|_S$ a la categoría $\Sh_{\Top_{\Ring}|_S}$, que nos permite identificar la primera categoría con una subcategoría plena de la segunda %COMO UN FUNCTOR REPRESENTABLE ES SIEMPRE UN HAZ, LA IMAGEN ESENCIAL DE FUNCTOR H_Z CAE EN SH


\end{remarque}

\begin{remarque}{4.5.7}\label{localmente}
%LA ACLARACIÓN ES PARA LA PARTE DEL FINAL DE LA NOTA ANTERIOR DONDE SE RECUBRE EL ESPACIO ANILLADO, ES UNA SIMPLE COMPROBACIÓN PORQUE EN CADA PUNTO VA A SEGUIR SIENDO UN ANILLO LOCAL LA FIBRA Y EL MORFISMO EN CADA ANILLO LOCAL VA A SEGUIR SIENDO LOCAL
Los resultados anterioes son válidos también en la categoría de espacios localmente anillados: si un espacio anillado $X$ se obtiene mediante pegamiento espacios localmente anillados $X_i$, entonces $X$ es localmente anillado; y si un morfismo de espacio anillados $X\to S$ (donde $S$ es localmente anillado) restringe a cada $X_i$ como un morfismo de espacios localmente anillados, entonces es un morfimo de espacios localmente anillados. 
\end{remarque}



\subsection{Remarque 2.3.6}

Notamos en primer lugar que \ref{prehaz} es válida sustituyendo la categoría de espacios anillados sobre $S$ por la categoría $\Sch|_{S}$ de esquemas sobre $S$ (donde $S$ es un esquema cualquiera) gracias a \ref{localmente}, ya que un esquema es en particular un espacio localmente anillado que puede ser obtenido por pegamiento de esquemas afines. Por tanto también es válida para la categoría $\Sch$ de todos los esquemas, que es equivalente a $\Sch_{\spec(\Z)}$. %TENDRÍA QUE DECIR QUE PEGANDO EL RECUBRIMIENTO AFÍN SALE EL ESQUEMA

Consideremos por otra parte la subcategoría plena $\mathrm{\mathbf{Aff}}$ de $\Sch$ formada por los esquemas afines. Si $G:\Aff^{op}\to\Set$ es un prehaz de categorías, para todo esquema afín $X$ podemos considerar la aplicación $U\mapsto G(U)$ para todo abierto afín $U$ de $X$. Como estos abiertos forman una base $\BB_X$ de la topología de $X$ (por ejemplo tomando los de la forma $D(f)$ para $f\in A$, donde $U=\spec(A)$). La aplicación $U\mapsto G(U)$ es un prehaz sobre $\BB_X$ con valores en $\Set$, viendo $\BB_X$ como una categoría con las inclusiones de abiertos como morfismos (\ref{prehaztop}). Cuando, para todo esquema afín $X$, $U\mapsto G(U)$ es un haz sobre $\BB_X$ en el sentido de \ref{haztop} diremos entonces que $G$ es un \emph{haz} sobre la categoría $\Aff$. Los functores con esta propiedad forman una subcategoría plena de la categoría $\HOM(\Aff^{op},\Set)$ denotada $\Sh_{\Aff}$. 

Con esto, dado un para todo prehaz $F:\Sch^{op}\to \Set$, podemos considerar su restricción $F|_{\Aff^{op}}:\Aff^{op}\to\Set$. Es claro que
\begin{equation}\label{functor}
F\mapsto F|_{\Aff^{op}}
\end{equation}
es un functor de $\HOM(\Sch^{op},\Set)$ en $\HOM(\Aff^{op},\Set)$. Vamos a ver que la restricción de este functor a la subcategoría plena $\Sh_{\Sch}$ (\ref{prehaz}) de $\HOM(\Sch^{op},\Set)$ define de hecho una equivalencia de categorías 
\begin{equation}\label{equivalencia}
\Sh_{\Sch}\cong\Sh_{\Aff}
\end{equation}
Es claro que la imagen de $\Sh_{\Sch}$ por \ref{functor} es una subcategoría de $\Sh_{\Aff}$ en virtud de 0-3.2.2 Y COMPROBAR QUE SE CUMPLE

Mostramos en primer lugar que la restricción de \ref{functor} a $\Sh_{\Aff}$ es plenamente fiel. Esto resulta por un lado de que si $F\in \Sh_{\Sch}$ y $X$ es un esquema, dar $F(V)$ para los abiertos afines de $X$ determina enteramente los $F(U)$ para todo abierto $U$ de $X$ mediante $F(U)=\varprojlim F(U_\alpha)$, donde $(U_\alpha)$ es el conjunto parcialmente ordenado de los abiertos afines contenidos en $U$ (\ref{prehaztop}); TENGO QUE PENSAR PARA QUÉ ES EL PASO ANTERIOR
 por otro lado, si $\varphi:F\to G$ es una transformación natural para $F,G\in \Sh_{\Sch}$, dar aplicaciones $\varphi(U_\alpha):F(U_\alpha)\to G(U_\alpha)$ determina enteramente $\varphi(U):F(U)\to G(U)$, siendo esta aplicación el límite proyectivo del sistema proyectivo de aplicaciones $\varphi(U_\alpha).$%POR DEFINICIÓN DE TRANSFORMACIÓN NATURAL SE CONSIGUE EL SISTEMA PROYECTIVO A PARTIR DEL DE LOS ABIERTOS 
 
 Falta ver que todo functor $G\in\Sh_{\Aff}$ es de la forma $F|_{\Aff^{op}}$ para un $F\in \Sh_{\Sch}$. Para ello, definimos para todo esquema $X$, $F(X)$ como el límite proyectivo de $G(U_\alpha)$ donde $(U_\alpha)$ es el sistema proyectivo de abiertos afines de $X$; y para todo morfimo de esquemas $u:X\to Y$ definimos $F(u)$ como el límite proyectivo de las aplicaciones $G(u_{\alpha\beta}):G(V_\beta)\to G(U_\alpha)$, siendo $(V_\beta)$ el sitema proyectivo de abiertos afines de $Y$, y las parejas $(\alpha,\beta)$ son solamente aquellas tales que $u(U_\alpha)\subseteq V_\beta$ y $u_{\alpha\beta}:U_\alpha\to V_\beta$ es la restricción de $u$. Verificar que $F$ es un functor es inmediato INTENTARLO. Finalmente, el hecho de que $F$ sea un haz se deduce del hecho de que los abiertos afines de $X$ forman una base para la topología de $X$ y de \ref{haztop} COMPROBAR ESTO ÚLTIMO.
 
 
 Vimos que la categoría $\Aff^{op}$ es canónicamente equivalente a la categoría $\mathrm{\mathbf{Ring}}$ de anillos. Esta equivalencia define también una equivalencia de categorías $\HOM(\Aff^{op},\Set)\cong \HOM(\mathrm{\mathbf{Ring}},\Set)$. Diremos por abuso de lenguaje que que un functor (covariante) $\mathrm{\mathbf{Ring}}\to\Set$ es un \emph{haz} si su imagen por la equivalencia anterior es un haz sobre $\Aff$. Notamos por $\Sh_{\mathrm{\mathbf{Ring}}}$ a la categoría de estos haces. Teniendo en cuenta que, para todo anillo $A$, el conjunto de abiertos afines $D(f)=\spec(A_f)$ forma una base de la topología de $\spec(A)$, podemos además, en virtud de \ref{haztop} y la observación \ref{ecualizador}, expresar que el functor $F:\mathrm{\mathbf{Ring}}\to\Set$ pertenece a $\Sh_{\mathrm{\mathbf{Ring}}}$ diciendo que, para toda familia $(f_i)_{i\in I}$ de elementos de $A$ tales que $\sum_i \gene{f_i}=A$, el diagrama de conjuntos
 \[F(A)\to\prod_i F(A_{f_i})\rightrightarrows F(A_{f_if_j})\]
es exacto. Este diagrama proviene del diagrama exacto
\[
D(f_i)\cap D(f_j)=D(f_if_j)\rightrightarrows \prod_i D(f_i)\to \spec(A).
\] 
Recordemos también que $\bigcup_i D(f_i)=\spec(A)$ implica que $\sum_i\gene{f_i}=A$ a partir de que $V(\sum_i I_i)=\bigcap_i V(I_i)$ por el lema 2.1(b) de Hartshorne. Podemos además fijarnos en este enunciado en las familias finitas $(f_i)_{i\in I}$ que engendran $A$. %SIEMPRE EXISTE ALGUNA PORQUE CON EL 1 SE GENERA
Así, hemos caracterizado la categoría $\Sh_{\mathrm{\mathbf{Ring}}}$ sin utilizar la noción de esquema. 


Recordemos que cada esquema $X$ define un functor contravariante $h_X:Y\to\Hom(Y,X)$ de $\Sch$ en $\Set$, que hemos visto en \ref{representable} que es un haz sobre $\Sch$. Sabemos además que $X\mapsto h_X$ es un functor plenamente fiel de $\Sch$ en $\Sh_{\Sch}$. En virtud de la equivalencia \ref{equivalencia} y de la definición de $\Sh_{\mathrm{\mathbf{Ring}}}$, podemos entonces identificar canónicamente la categoría $\Sch$ con una subcategoría plena de la categoría $\Sh_{\mathrm{\mathbf{Ring}}}$, siendo cada esquema $X$ identificado con un haz $A\mapsto \Hom(\spec(A),X)$ sobre $\mathrm{\mathbf{Ring}}$. De todos modos, debemos tener en cuenta que tener en cuenta que hay functores de $\Sh_{\mathrm{\mathbf{Ring}}}$ que no son isomorfos a haces provinientes de esquemas. 
 
 BUSCAR EJEMPLO PARA LA ÚLTIMA FRASE
\end{document}
