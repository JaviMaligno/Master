	\documentclass[twoside]{article}
\usepackage{../../estilo-ejercicios}
\renewcommand{\baselinestretch}{1,4}
%--------------------------------------------------------
\begin{document}

\title{Éléments de Géométrie Algébrique, Remarque 2.3.6}
\author{Javier Aguilar Martín}
\maketitle

%\varprojlim
%\varinjlim
\section{Pasos previos}

PROBABLEMENTE PONDRÉ EN NEGRITA LAS CATEGORÍAS, PARA ESO TENGO QUE METERLO COMO mathrm{mathbf{Aff}} NO COMO MATHOPERATOR, TAMBIÉN HACERLO CON HOM
\begin{defi}[EGA, 0-1.1.9]
En una categoría $\CC$, dado un conjunto parcialmente ordenado $I$, se define un \emph{sistema proyectivo} como una familia $(A_\alpha)_{\alpha\in I}$ de objetos de $\CC$ y para cada par $(\alpha,\beta)$ tal que $\alpha\leq\beta$ un morfismo $u_{\alpha\beta}:A_\beta\to A_\alpha$ cumpliendo $u_{\alpha\gamma}=u_{\alpha\beta}\circ u_{\beta\gamma}$ para $\alpha\leq\beta\leq\gamma$. Un \emph{límite proyectivo} de un sistema proyectivo está formado por un objeto $B\in\CC$ (denotado $\varprojlim A_\alpha$), y para cada $\alpha\in I$, un morfismo $u_\alpha:B\to A_\alpha$ tales que: 
\begin{enumerate}
\item $u_\alpha=u_{\alpha\beta}u_{\beta}$ para $\alpha\leq\beta$;
\item para todo objeto $X$ de $C$ y toda familia $(v_\alpha)_{\alpha\in I}$ de morfismos $v_\alpha:X\to A_{\alpha}$ verificando $v_\alpha=u_{\alpha\beta}v_{\beta}$ para $\alpha\leq\beta$, existe un único morfismo $v:X\to B$ (denotado $\varprojlim v_\alpha$) tal que $v_\alpha=u_\alpha v$ para todo $\alpha\in I$
\[
\begin{tikzcd}
& & & & X\arrow[dlll, bend left=-20, "v_\alpha"]\arrow[dll, bend left=-15, "v_\beta"]\arrow[dl, bend left=-10, "v_\gamma"]\arrow[dd, dashed, bend left, "\varprojlim v_\alpha"]\\
\cdots &A_\alpha\arrow[l]  & A_\beta\arrow[l, "u_{\alpha\beta}"', near start]  & \arrow[l, "u_{\beta\gamma}"', near start]A_\gamma &\cdots\arrow[l]\\
&  & & & \varprojlim A_\alpha\arrow[ulll, bend left=20, "u_\alpha"]\arrow[ull, bend left=15, "u_\beta"]\arrow[ul, bend left=10, "u_\gamma"]
\end{tikzcd}
\]
\end{enumerate}
Decimos que $\CC$ \emph{admite límites proyectivos} si siempre existen límites proyectivos de sistemas proyectivos en $\CC$. 
\end{defi}

%EN TOP ES INTERSECCIÓN CON TOPOLOGÍA INICIAL (LA QUE CON MENOS ABIERTOS HACE CONTINUAS A TODAS LAS FLECHAS AL SISTEMA PROJECTIVO) POR EL EJERCICIO 1.12 DE HARTSHORNE EL LÍMITE INVERSO DE HACES ES HAZ
EL LÍMITE INVERSO EN ESPACIOS TOPOLÓGICOS POR SI ME HACE FALTA, DE TODOS MODOS EN SET SEGURAMENTE SÍ ME HAGA FALTA\url{https://math.stackexchange.com/questions/711334/inverse-limit-of-an-inverse-system-of-topological-spaces}

\begin{ej}[$\mathrm{Set}$ admite límites proyectivos]\label{set}
Sea $\{A_i\}_{i\in I}$ un sistema proyectivo de conjuntos con aplicaciones $f_{ji}:A_j\to A_{i}$. Construimos $A=\{(a_i)_{i\in I}\in \prod_i A_i\mid f_{ji}(a_j)=a_i, i\leq j\}$ y afirmamos que $A=\varprojlim A_i$. Es claro que existen aplicaciones $f_i:A \to A_i$ tales que $a_i\mapsto a_i$ compatibles con el sistema proyectivo por definición.

Sea $X$ otro conjunto con aplicaciones $g_i:X\to A_i$ compatibles con el sistema inverso. Sea $x\in X$ con imagen $g_i(x)=a_i\in A_i$. Definimos entonces $g:X\to A$ como $g(x)=(g_i(x))_{i\in I}$, que está bien definida porque los $g_i$ son compatibles con el sistema proyectivo y claramente conmuta con las aplicaciones del sistema. Supongamos ahora que existe otra aplicación $h:X\to A$ compatible con el sistema inverso, en particular $f_jh=g_j$ para todo $j\in I$. Cada $f_jh(x)$ son las coordenadas de $h(x)$ en $A$, que son justamente $g_j(x)=a_j$, luego en realidad $h=g$. 

Esta misma construcción con estructura adicional sirve para probar la existencia de límites proyectivos en categoría como $\Top$ o $\mathrm{\mathbf{Ab}}$.
\end{ej}

\begin{prop}[EGA, 0-3.2.6]\label{limite}
Sea $\CC$ una categoría que admite límites proyectivos y $X$ un espacio topológico. La categoría de haces sobre $X$ con valores en $\CC$ admite límites proyectivos. 
\end{prop}
\begin{dem}
Este resultado es una generalización del ejercicio II.1.12 de Hartshorne, con idéntica prueba.

Sea $\{\calF_i\}$ un sistema inverso de haces en $X$ y consideramos el prehaz $U\mapsto\varprojlim\calF_i(U)$, que está bien definido porque $\CC$ admite límites proyectivos. Probamos que este prehaz, al que denotamos $\calF$, es un haz y que tiene la propiedad universal de un límite proyectivo en la categoría de haces. 

Sea $U=\bigcup_{k\in I} U_k$ y secciones $s_k\in\calF(U_k)$ tales que $s_k|_{U_k\cap U_l}=s_l|_{U_k\cap U_l}$, entonces sus imágenes $s_k^i\in \calF_i(U)$ también son compatibles y admiten una extensión única $s^i\in\calF_i(U)$. Si $u_{i'i}:\calF_{i'}\to\calF_i$, entonces tenemos que $(u_{i'i})(s_{i'})_P=(u_{i'i})_P(s_{i'})_P=(s_i)_P$ puesto que $u_{i'i}|_{U_k}(s^{i'}_k)=s^i_k$ para todo $k$ y para $P\in U_k$. Deducimos entonces que los $s^i$ son compatibles con las aplicaciones que definen el sistema proyectivo, por lo que obtenemos un elemento $s\in\calF(U)$ que se restringe a $s_k$ para cada $U_k$. 

Supongamos que $f_i:\calG\to\calF_i$ es una familia de morfismos compatible con el sistema proyectivo. Entonces, para cada abierto $U$ tenemos aplicaciones $\calG(U)\to\calF_i(U)$ satisfaciendo la condición de límite proyectivo, tenemos un morfismo único $\calG(U)\to\calF(U)$, que es lo mismo que tener un morfismo único $\calG\to\calF$. 

\end{dem}



\begin{remarque}{4.5.6}\label{prehaz}
Llamamos \emph{prehaz} sobre una categoría $\CC$ a un functor $F:\CC^{op}\to\mathrm{Set}$. Denotamos por $\mathrm{Top}_{Ring}|_S$ a la categoría de espacios anillados sobre un espacio anillado fijo $S$, esto es la categoría cuyos objetos son morfismos en $\mathrm{Top}_{Ring}$ de la forma $X\to S$ (a los que llamamos $S$-espacios) y los morfismos son morfismos $X\to Y$ (llamados $S$-morfismos) en $\mathrm{Top}_{Ring}$ que hacen conmutativo el diagrama
\[
\begin{tikzcd}
X\arrow[r]\arrow[dr] & Y\arrow[d]\\
 & S
\end{tikzcd}
\]

Un prehaz 
\begin{equation}\label{representable}
F:(\mathrm{Top}_{Ring}|_S)^{op}\to\mathrm{Set}
\end{equation}
 nos da para todo $S$-espacio $X$ un prehaz de conjuntos $U\to F(U)$ sobre $X$, ya que los abiertos de $X$ tienen una estructura inducida de $S$ espacio. Cuando para todo $S$-spacio $X$ este prehaz es un haz, decimos que $F$ es un \emph{haz sobre la categoría} $\mathrm{Top}_{Ring}|_S$, noción que se puede generalizar a cualquier categoría usando la topología de Grothendieck de una categoría, aunque no la necesitaremos para nuestros propósitos. Estos functores forman una subcategoría plena de $\Hom((\mathrm{Top}_{Ring}|_S)^{op},\mathrm{Set})$ denotada $\Sh_{\mathrm{Top}_{Ring}|_S}$. Si añadimos a esto la proposición \ref{limite} y el ejemplo \ref{set}, resulta que $\Sh_{\mathrm{Top}_{Ring}|_S}$ admite límites proyectivos de sistemas proyectivos de functores que son haces sobre un mismo espacio $X$  %COGE COMO ESPACIO EL LÍMITE INVERSO COMO DIGO ARRIBA Y EL HAZ TAMBIÉN LÍMITE INVERSO. %NO SÉ SI ME HARÁ FALTA QUE EXISTAN PARA ESPACIOS DISTINTOS

%RECUERDA LA DEFINICIÓN DE FUNCTOR REPRESENTABLE, EN ESTE CASO EL FUNCTOR ES REPRESENTABLE PORQUE DE HECHO ES UN ELEMENTO DE HOM(A,-)

Un functor representable como \ref{representable} es siempre un haz: es claro que en efecto el prehaz $U\mapsto\Hom_S(U,Z)$ sobre un espacio anillado $X$ es un haz, ya que para todo recubrimiento abierto $(U_\alpha)$ de un abierto $U\subseteq X$, dar un $S$-morfismo de $U\to Z$ equivale a dar una familia de $S$-morfismos $f_\alpha:U_\alpha\to Z$ tales que para toda pareja de índices $(\alpha,\beta)$ las restricciones de $f_\alpha$ y $f_\beta$ a $U_\alpha\cap U_\beta$ coincidan.

Por el lema de Yoneda podemos decir además que tenemos un functor plenamente fiel $Z\to h_Z$ de la categoría $\mathrm{Top}_{Ring}|_S$ a la categoría $\Sh_{\mathrm{Top}_{Ring}|_S}$, que nos permite identificar la primera categoría con una subcategoría plena de la segunda %COMO UN FUNCTOR REPRESENTABLE ES SIEMPRE UN HAZ, LA IMAGEN ESENCIAL DE FUNCTOR H_Z CAE EN SH


\end{remarque}

\begin{remarque}{4.5.7}\label{localmente}
%LA ACLARACIÓN ES PARA LA PARTE DEL FINAL DE LA NOTA ANTERIOR DONDE SE RECUBRE EL ESPACIO ANILLADO, ES UNA SIMPLE COMPROBACIÓN PORQUE EN CADA PUNTO VA A SEGUIR SIENDO UN ANILLO LOCAL LA FIBRA Y EL MORFISMO EN CADA ANILLO LOCAL VA A SEGUIR SIENDO LOCAL
Los resultados anterioes son válidos también en la categoría de espacios localmente anillados: si un espacio anillado $X$ se obtiene mediante unión espacios localmente anillados $X_i$, entonces $X$ es localmente anillado; y si un morfismo de espacio anillados $X\to S$ (donde $S$ es localmente anillado) restringe a cada $X_i$ como un morfismo de espacios localmente anillados, entonces es un morfimo de espacios localmente anillados. 
\end{remarque}

\begin{ej}[EGA, 0-3.2.1]\label{prehaztop}%RECUPERAR LA NOCIÓN HABITUAL DE PREHAZ A PARTIR DE LA CATEGÓRICA, GROTHENDIECK HACE HINCAPIÉ EN EL CASO DE X NOETHERIANO
Sea $X$ un espacio topológico y $\BB$ una base de abiertos para la topología de $X$. Dotamos a esta familia de estructura de categoría considerando un morfismo por cada relación de inclusión. Consideramos entonces un prehaz $\calF:\BB^{op}\to\C$ con valores en una categoría $\C$. Podemos asociarle un prehaz con valores en $\CC$ en el sentido habitual, $U\mapsto \calF'(U)$ tomando para todo abierto $U$, $\calF'(U)=\varprojlim\calF(V)$, donde $V$ recorre el conjunto $\BB$ ordenado por la inclusión para $V\subseteq U$. Sean $U,U'$ abiertos con $U\subseteq U'$ y denotamos $\rho_{VW}:\calF(V)\to\calF(W)$ las restricciones dadas por $\calF$ para $W\subseteq V$ abiertos básicos. ESTAS COSAS LAS TENGO QUE HACER Y COMPROBAR CON UN DIAGRAMA PARA VER QUE LAS COSAS VAN BIEN Definimos entonces $\rho'_{U'U}$ como el límite proyectivo (para $V\subseteq U$) de los morfimos canónicos $\calF(U)\to\calF(V)$, o lo que es lo mismo, el único morfismo $\calF'(U')\to\calF(U)$ que compuesto con los morfismos canónicos $\calF'(U)\to\calF(V)$ da los morfismos canónicos $\calF'(U')\to\calF(V)$. La verificación de la transitividad de $\rho'_{U'U}$ es inmediata. Además, si $U\in\BB$, el morfismo canónico $\calF'(U)\to\calF(U)$ es isomorfismo, permitiéndonos identificar estos dos objetos.
\end{ej}
\begin{prop}[EGA, 0-3.2.2]\label{haztop}
Para que el prehaz $\calF'$ definido en el ejemplo anterior, es necesaria y suficiente la siguiente condición sobre $\calF$:

Para todo recubrimiento $(U_{\alpha})$ de $U\in\BB$ por conjuntos $U_\alpha\in\BB$ contenidos en $U$, y por todo objeto $T\in\CC$, la aplicación que a todo $f\in\Hom(T,\calF(U))$ le hace corresponder la familia $\rho_{UU_\alpha}\circ f\in\Hom(T,\calF(U_\alpha))$ es una biyección de $\Hom(T,\calF(U))$ en el conjunto de $(f_\alpha)$ tales que $\rho_{U_\alpha V}f_\alpha=\rho_{U_\beta V}f_\beta$ para toda pareja de índices $(\alpha,\beta)$ y todo $V\in B$ contenido en $U_\alpha\cap U_\beta$. 
\end{prop}
\begin{dem}
 \end{dem}

\section{Remarque 3.2.6}

Notamos en primer lugar que \ref{prehaz} es válida sustituyendo la categoría de espacios anillados sobre $S$ por la categoría $\Sch|_{S}$ de esquemas sobre $S$ (donde $S$ es un esquema cualquiera) gracias a \ref{localmente}, ya que un esquema es en particular un espacio localmente anillado. Por tanto también es válida para la categoría $\Sch$ de todos los esquemas, que es equivalente a $\Sch_{\spec(\Z)}$.

Consideremos por otra parte la subcategoría plena $\mathrm{\mathbf{Aff}}$ de $\Sch$ formada por los esquemas afines. Si $G:\Aff^{op}\to\Set$ es un prehaz de categorías, para todo esquema afín $X$ podemos considerar la aplicación $U\mapsto G(U)$ para todo abierto afín $U$ de $X$. Como estos abiertos forman una base $\BB_X$ de la topología de $X$ (por ejemplo tomando los de la forma $D(f)$ para $f\in A$, donde $U=\spec(A)$). La aplicación $U\mapsto G(U)$ es un prehaz sobre $\BB_X$ con valores en $\Set$, viendo $\BB_X$ como una categoría con las inclusiones de abiertos como morfismos (\ref{prehaztop}). Cuando, para todo esquema afín $X$, $U\mapsto G(U)$ es un haz sobre $\BB_X$ en el sentido de \ref{haztop} diremos entonces que $G$ es un \emph{haz} sobre la categoría $\Aff$. Los functores con esta propiedad forman una subcategoría plena de la categoría $\Hom(\Aff^{op},\Set)$ denotada $\Sh_{\Aff}$. 

Con esto, dado un para todo prehaz $F:\Sch^{op}\to \Set$, podemos considerar su restricción $F|_{\Aff^{op}}:\Aff^{op}\to\Set$. Es claro que
\begin{equation}\label{functor}
F\mapsto F|_{\Aff^{op}}
\end{equation}
es un functor de $\Hom(\Sch^{op},\Set)$ en $\Hom(\Aff^{op},\Set)$. Vamos a ver que la restricción de este functor a la subcategoría plena $\Sh_{\Sch}$ (\ref{prehaz}) de $\Hom(\Sch^{op},\Set)$ define de hecho una equivalencia de categorías 
\[\Sh_{\Sch}\cong\Sh_{\Aff}\]
Es claro que la imagen de $\Sh_{\Sch}$ por \ref{functor} es una subcategoría de $\Sh_{\Aff}$ en virtud de 0-3.2.2 Y COMPROBAR QUE SE CUMPLE

\end{document}
