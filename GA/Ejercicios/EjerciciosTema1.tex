	\documentclass[twoside]{article}
\usepackage{../../estilo-ejercicios}

%--------------------------------------------------------
\begin{document}

\title{Ejercicios de Algebraic Geometry, Capítulo 2, Sección 1}
\author{Javier Aguilar Martín}
\maketitle


%\begin{ejercicio}{1.1}
%Sea $A$ un grupo abeliano, definimos el prehaz constante asociado a $A$ en el espacio topológico $X$ como el prehaz $U\mapsto A$ para todo $U\neq\emptyset$ con la identidad como restricciones. Probar que el haz constante $\calA$ definido en el texto es el haz asociado a este prehaz. 
%\end{ejercicio}
%\begin{solucion}
%
%\end{solucion}
%
%\newpage
%
%
%\begin{ejercicio}{1.2}
%
%\end{ejercicio}
%\begin{solucion}
%
%\end{solucion}
%
%\newpage
%
%\begin{ejercicio}{1.3}
%
%
%\end{ejercicio}
%\begin{solucion}
%\end{solucion}
%
%\newpage

\begin{ejercicio}{1.5}
Probar que un morfismo de haces es un isomorfismo si y solo si es inyectivo y sobreyectivo.

\end{ejercicio}
\begin{solucion}
Sea $\varphi:\calF\to\calG$ un isomorfismo de haces sobre un espacio topológico $X$ con inverso $\psi:\calG\to\calF$. Esto significa que los homomorfismos inducidos para cualquier abierto $U\subseteq X$, $\psi(U)\circ\varphi(U)=(\psi\circ\varphi)(U):\calF(U)\to\calF(U)$ y $\varphi(U)\circ\psi(U)=(\varphi\circ\psi)(U):\calG(U)\to\calG(U)$ son las correspondientes identidades. Con lo cual, tanto $\varphi(U)$ y $\psi(U)$ son isomorfismos de grupos abelianos, implicando como sabemos, que son inyectivos. Esto ya implica por definición que $\varphi$ es inyectivo, puesto que $\ker\varphi(U)=0$ para todo $U$. 

Para probar la sobreyectividad usamos que $\varphi$ es isomorfismo si y solo si $\varphi_p:\calF_p\to\calG_p$ es isomorfismo para todo $P\in X$. Como $\varphi_P$ es un isomorfismo de grupos abelianos, podemos decir que en particular es sobreyectivo para todo $P$. Esto es equivalente a afirmar que $\varphi$ es un morfismo de haces sobreyectivo gracias a la segunda parte de la demostración de la proposición 1.1. 
\end{solucion}

%\newpage

\begin{ejercicio}{1.6}\
\begin{enumerate}
\item[(a)]
Sea $\calF'$ un subhaz del haz $\calF$. Probar que la aplicación natural de $\calF$ al haz cociente $\calF/\calF'$ es sobreyectiva y tiene núcleo $\calF'$. Por tanto, hay una sucesión exacta 
\[
0\to \calF'\to\calF\to \calF/\calF'\to 0.
\]

\item[(b)] Recíprocamente, si $0\to \calF'\to\calF\to \calF''\to 0$ es una sucesión exacta, probar que $\calF'$ es isomorfo a un subhaz de $\calF$, y que $\calF''$ es isomorfo al cociente de $\calF$ por este subhaz.
\end{enumerate}
\end{ejercicio}
\begin{solucion}\
\begin{enumerate}[(a)]
\item Probamos la sobreyectividad viendo que la aplicación inducida en las fibras es sobreyectiva. Tenemos que en cada punto $P\in X$, la fibra del cociente $\calF/\calF'$ es $(\calF/\calF')_P=\calF_P/\calF'_P$, por tanto, a nivel de fibras tenemos la aplicación cociente $\calF_P\to \calF_P/\calF'_P$ que sabemos que es sobreyectiva para todo $P$ por ser las aplicaciones cociente de grupos abelianos. Así que $\varphi$ es sobreyectiva.


\item
\end{enumerate}
\end{solucion}

%\newpage
%
%\begin{ejercicio}{1.7}
%
%\end{ejercicio}
%\begin{solucion}
%
%\end{solucion}
%
%\newpage
%
%\begin{ejercicio}{1.9}
%
%\end{ejercicio}
%\begin{solucion}
%
%
%
%
%
%\end{solucion}
%
%\newpage
%
%\begin{ejercicio}{1.10}
%
%\end{ejercicio}
%\begin{solucion}
%
%\end{solucion}
%
%\newpage
%
%\begin{ejercicio}{1.11}
%
%\end{ejercicio}
%\begin{solucion}
%
%
%
%\end{solucion}
%
%\newpage
%
%\begin{ejercicio}{1.12}
%
%\end{ejercicio}
%\begin{solucion}
%
%\end{solucion}

\end{document}
