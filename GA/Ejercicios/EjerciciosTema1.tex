	\documentclass[twoside]{article}
\usepackage{../../estilo-ejercicios}

%--------------------------------------------------------
\begin{document}

\title{Ejercicios de Algebraic Geometry, Capítulo 2, Sección 1}
\author{Javier Aguilar Martín}
\maketitle


\begin{ejercicio}{1.1}
Sea $A$ un grupo abeliano, definimos el prehaz constante asociado a $A$ en el espacio topológico $X$ como el prehaz $U\mapsto A$ para todo $U\neq\emptyset$ con la identidad como restricciones. Probar que el haz constante $\calA$ definido en el texto es el haz asociado a este prehaz. 
\end{ejercicio}
\begin{solucion}

\end{solucion}

\newpage


\begin{ejercicio}{1.2}

\end{ejercicio}
\begin{solucion}

\end{solucion}

\newpage

\begin{ejercicio}{1.3}


\end{ejercicio}
\begin{solucion}



\end{solucion}

\newpage

\begin{ejercicio}{1.6}

\end{ejercicio}
\begin{solucion}

\end{solucion}

\newpage

\begin{ejercicio}{1.7}

\end{ejercicio}
\begin{solucion}

\end{solucion}

\newpage

\begin{ejercicio}{1.9}

\end{ejercicio}
\begin{solucion}





\end{solucion}

\newpage

\begin{ejercicio}{1.10}

\end{ejercicio}
\begin{solucion}

\end{solucion}

\newpage

\begin{ejercicio}{1.11}

\end{ejercicio}
\begin{solucion}



\end{solucion}

\newpage

\begin{ejercicio}{1.12}

\end{ejercicio}
\begin{solucion}

\end{solucion}

\end{document}
