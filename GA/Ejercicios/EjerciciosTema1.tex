	\documentclass[twoside]{article}
\usepackage{../../estilo-ejercicios}

%--------------------------------------------------------
\begin{document}

\title{Ejercicios de Algebraic Geometry, Capítulo 2, Sección 1}
\author{Javier Aguilar Martín}
\maketitle


\begin{ejercicio}{1.1}
Sea $A$ un grupo abeliano, definimos el prehaz constante asociado a $A$ en el espacio topológico $X$ como el prehaz $U\mapsto A$ para todo $U\neq\emptyset$ con la identidad como restricciones. Probar que el haz constante $\calA$ definido en el texto es el haz asociado a este prehaz. 
\end{ejercicio}
\begin{solucion}

\end{solucion}

\newpage


\begin{ejercicio}{1.2}

\end{ejercicio}
\begin{solucion}

\end{solucion}

\newpage

\begin{ejercicio}{1.3}


\end{ejercicio}
\begin{solucion}
\end{solucion}

\newpage

\begin{ejercicio}{1.5}
Probar que un morfismo de haces es un isomorfismo si y solo si es inyectivo y sobreyectivo.

\end{ejercicio}
\begin{solucion}

\end{solucion}

\newpage

\begin{ejercicio}{1.6}\
\begin{enumerate}
\item[(a)]
Sea $\calF'$ un subhaz del haz $\calF$. Probar que la aplicación natural de $\calF$ al haz cociente $\calF/\calF'$ es sobreyectiva y tiene núcleo $\calF'$. Por tanto, hay una sucesión exacta 
\[
0\to \calF'\to\calF\to \calF/\calF'\to 0.
\]

\item[(b)] Recíprocamente, si $0\to \calF'\to\calF\to \calF''\to 0$ es una sucesión exacta, probar que $\calF'$ es isomorfo a un subhaz de $\calF$, y que $\calF''$ es isomorfo al cociente de $\calF$ por este subhaz.
\end{enumerate}
\end{ejercicio}
\begin{solucion}

\end{solucion}

\newpage

\begin{ejercicio}{1.7}

\end{ejercicio}
\begin{solucion}

\end{solucion}

\newpage

\begin{ejercicio}{1.9}

\end{ejercicio}
\begin{solucion}





\end{solucion}

\newpage

\begin{ejercicio}{1.10}

\end{ejercicio}
\begin{solucion}

\end{solucion}

\newpage

\begin{ejercicio}{1.11}

\end{ejercicio}
\begin{solucion}



\end{solucion}

\newpage

\begin{ejercicio}{1.12}

\end{ejercicio}
\begin{solucion}

\end{solucion}

\end{document}
