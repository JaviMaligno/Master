\documentclass[GA.tex]{subfiles}

\begin{document}


%\hyphenation{equi-va-len-cia}\hyphenation{pro-pie-dad}\hyphenation{res-pec-ti-va-men-te}\hyphenation{sub-es-pa-cio}

\chapter{Esquemas en grupos (group schemes)}
Rereferencias: Alebgraic Geometry II, Abelian Varietis (Mumford), Abelian Varieties (Edihoven van der Geer y Moonen)

\section{El functor de los puntos}
Los puntos de un esquema $X$ normalmente son los puntos del espacio topológico subyacente, pero hay otra noción de ``puntos con valores en un esquema $T$'', que es el functor $X(T)=h_X(T)=\Hom_{\Sch}(T,X)$. 

\begin{ej}
Sea $X=\spec(\Z[x,y]/\gene{x^2+y^2-1})$ y $A$ un anillo. Calculamos $X(\spec(A))=\Hom_{\Sch}(\spec(A),X) \cong\Hom_{\Ring}(\Z[x,y]/\gene{x^2+y^2-1},A)=\{a,b\in A\mid a^2+b^2=1\}$. En este caso vemos que el resultado son los puntos de una circunferencia. En general veremos que esto es lo que tiene estructura de grupo. 
\end{ej}
\begin{ej}
Sea $X=\spec(Z[x_1,\dots, x_4]/\gene{x_1x_4-x_2x_3-1})$. Para un anillo $A$, haciendo lo mismo que en el ejemplo anterior obtenemos el conunto $\{(a_1,\dots, a_4)\in A^4\mid a_1a_4-a_2a_3=1\}$, que son las matrices cuadradas de orden 2 con determinante 1, es decir, $SL_2(A)$, que tiene claramente estructura de grupo, la cual depende de la suma y el producto de $A$ y de el esquema $X$ que hemos elegido. 
\end{ej}

\begin{ej}
Si $X=\spec(B)$ y $T$ es un esquema cualquiera. Entonces $X(T)=\Hom_{\Sch}(T,X)=\Hom_{\Sch}(T,\spec(B))\cong\Hom_{\Ring}(B,\OO_T(T))$ (Hartshorne 2.4 maybe) que tiene estructura de grupo heredada de $\OO_T(T)$. 
\end{ej}

Fijaremos en este capítulo un esquema $S$ como base. De la definición de producto fibrado se sigue que para cualquier esquema $Z$, $\Hom_{\Sch|_S}(Z,X\times_S Y)\cong\Hom_{\Sch|_S}(Z,X)\times \Hom_{\Sch|_S}(Z,Y)$. 

\begin{observacion}
Sean $X\xrightarrow{f}S, Y\xrightarrow{g}S$, $X'\xrightarrow{f'}S, Y'\xrightarrow{g'}S$ esquemas sobre $S$. Sean $\alpha:X\to X'$ y $\beta:Y\to Y'$ morfismos de $S$-esquemas. Entonces existe un único $\alpha\times_S\beta\in\Hom(X\times_SY,X'\times_SY')$ haciendo conmutativo el diagrama 

\[
\begin{tikzcd}
X\arrow[d, "\alpha"] & X\times_SY\arrow[l, "p_1"]\arrow[r, "p_2"]\arrow[d, dashed, "\alpha\times_S\beta"] & Y\arrow[d, "\beta"]\\
X' &X'\times_SY'\arrow[l, "p_1'"]\arrow[r, "p_2'"] & Y'
\end{tikzcd}
\]
\end{observacion}

\begin{defi}
Un esquema en grupos sobre $S$ es una tupla $(G,\mu,\tau,\varepsilon)$ donde $G\to S$ es un $S$-esquema, $\mu:G\times_SG\to G$ es la \emph{ley de producto}, $\tau:G\to G$ es el \emph{inverso} y $\varepsilon:S\to G$ es el \emph{elemento neutro} (todos ellos son morfismos de $S$-esquemas). A estos datos se les piden las siguientes condiciones:
\begin{enumerate}
\item Asociatividad:
\[
\begin{tikzcd}
& G\times_S(G\times_S G)\cong (G\times_SG)\times_S G\arrow[rd, "Id_G\times\mu"]\arrow[ld, "\mu\times Id_G"] &\\
G\times_S G \arrow[r, "\mu"] & G & G\times_SG\arrow[l, "\mu"]
\end{tikzcd}
\]
\item Elemento neutro a izquierda y derecha (escribimos el diagrama conmutativo para la derecha, es análogo por la izquierda)
\[
\begin{tikzcd}
G\times_S S \arrow[r, "Id_G\times\varepsilon"]\arrow[dr, "p_1"] & G\times_S G\arrow[d, "\mu"]\\
& G
\end{tikzcd}
\]
\item Inverso a izquierda y derecha (solo escribimos el de la derecha) 
\[
\begin{tikzcd}
G\arrow[r, "\Delta"]\arrow[d] & G\times_S G\arrow[r, "Id\times \tau"] & G\times_S G\arrow[d, "\mu"]\\
S\arrow[rr, "\varepsilon"] & & G
\end{tikzcd}
\]

$G$ será un esquema en grupos conmutativos si además se cumple
\[
\begin{tikzcd}
G\times_S G\arrow[r, "\sigma"] \arrow[dr, "\mu"] & G\times_SG\arrow[d, "\mu"]\\
& G
\end{tikzcd}
\]
donde $\sigma$ está definido mediante
\[
\begin{tikzcd}
G & G\times_SG\arrow[l, "p_2"]\arrow[r, "p_1"]\arrow[d, "\sigma"] & G\\
& G\times_S G\arrow[ul, "p_1"]\arrow[ur, "p_2"] & 
\end{tikzcd}
\]
\end{enumerate}

Existe una noción de esquema en grupos abeliano más fuerte que lo que acabamos de ver para generalizar a las variedades abelianas .
\end{defi}

\begin{ej}
Esquema en grupo aditivo $G_a$ sobre $\Z$. Sea $G_a=\spec(Z[X])$. Dar $\mu:\spec(\Z[X])\times_S\spec(\Z[X])\to\spec(\Z[X])$ es equivalente a dar $\mu^*:\Z[X]\to \Z[X]\otimes\Z[X]$, que en este caso lo vamos a definir mediante $x\mapsto x\times 1+1\otimes x$. Definimos $\tau:G_a\to G_a$ mediante $\tau^*:\Z[X]\to\Z[X]$ como $X\mapsto -X$. Por último $\varepsilon:\Z\to G_a$ está dado por $\varepsilon^*:\Z[X]\to\Z$, $X\mapsto 0$. 

Verificamos ahora las condiciones de esquema en grupo mediante la conmutatividad de los diagramas duales en anillos. Basta comprobar la conmutatividad para el elemento $X$. Tenemos que $X\xrightarrow{\mu^*}X\otimes 1+1\otimes X\xrightarrow{\mu^*\times Id} \mu^*(X)\otimes 1+\mu^*(1)\otimes X=(X\otimes 1+1\otimes X)\otimes X+(1\otimes 1)\otimes X=X\otimes 1\otimes 1+1\otimes X\otimes 1+1\otimes 1\otimes X$. Por otro lado, $\mu^*(X)\xrightarrow{Id\times\mu^*} X\otimes\mu^*(1)+1\otimes\mu^*(X)=X\otimes(1\otimes 1)+1\otimes(X\otimes 1+1\otimes X)=X\otimes 1\otimes 1+ 1\otimes X\otimes 1+1\otimes 1\otimes X$, con lo que se verifica la asociatividad. 

Para el elemento neutro recordamos que $\Z[X]\to\Z[X]\otimes_\Z[X]$ por la derecha está dado por $a\mapsto a\otimes 1$. Entonces $Id\times\varepsilon^*(\mu^*(X))=X\otimes \varepsilon^*(1)+1\otimes\varepsilon^*(X)=X\otimes 1+0=X\otimes 1$, por lo que se verifica la propiedad del elemento neutro (es análoga por la izquierda).

Por último tenemos que vemos el inverso. Vemos que $\Delta^*:\Z[X]\otimes\Z[X]\to \Z[X]$ está dado por $a\otimes b=a+b$, puesto que por el diagrama se debe cumplir $a\otimes 1\mapsto a$ y $1\otimes b\mapsto b$. Esto se observa dualizando el diagrama
\[
\begin{tikzcd}
& G\arrow[ld, "Id"] \arrow[rd, "Id"]\arrow[d, "\Delta"] & \\
G & G\times_\Z G\arrow[l]\arrow[r] & G
\end{tikzcd}
\]
Por tanto $\tau^*\mu^*(X)=X\otimes\tau^*(1)+1\otimes\tau^*(X)=X\otimes 1+1\otimes(-X)\mapsto X-X=0=i(\varepsilon(X))$. 

Vamos a ver que además se verifica la conmutatividad. En efecto, $\sigma\mu^*(X)=\sigma(X\otimes 1+1\otimes X)=1\otimes X+X\otimes 1=X\otimes 1+1\otimes \sigma=\mu^*(X)$. 
\end{ej}

\begin{ej}
Grupo multipliativo sobre $\Z$. Sea $G_m=\spec(\Z[X,Y]/\gene{XY-1})=\spec(\Z[X,X^{-1}])$. Las propiedades de este esquema se prueban de forma análoga a las de anterior ejemplo para $\mu^*:\Z[X,X^{-1}]\to\Z[X,X^{-1}]\otimes \Z[X,X^{-1}]: X\mapsto X\otimes X$; $\tau^*:\Z[X,X^{-1}]\to \Z[X,X^{-1}]: X\mapsto X^{-1}$; $\varepsilon^*:\Z[X,X^{-1}]\to\Z: X\mapsto 1$. 
\end{ej}


En general se puede definir un objeto grupo $X$ en una categoría $\CC$ que admita productos finitos y tenga objeto terminal $S$ dando  $\mu:X\times X\to X$, $\varepsilon:S\to X$ y $\tau:X\to X$ cumpliendo las propiedades análogas. 

\begin{prop}
Sea $G$ un esquema en grupos (conmutativo) sobre $S$. Sea $T\to S$ un $S$-esquema. Entonces $G(T)=\Hom_{\Sch|_S}(T,G)$ está dotado de estructura de grupo (conmutativo) mediante la siguiente ley: para todo $f,g\in\Hom_{\Sch|_S}(T,G)$ se define $f\cdot g=\mu\circ h\in\Hom_{\Sch|_S}(T,G)$ donde $h$ satisface el diagrama conmutativo
\[
\begin{tikzcd}
 & T\arrow[dl, "f"']\arrow[dr, "g"]\arrow[d, "h"] & \\
 G & G\times_S G\arrow[r] \arrow[l] & G
\end{tikzcd}
\]
\end{prop}
\begin{dem}
Existencia de elemento neutro: 

Podemos conectar el diagrama
\[
\begin{tikzcd}
 & T\arrow[dl, "\pi_T"']\arrow[dr, "f"]\arrow[d] & \\
 S & S\times_S G\arrow[r] \arrow[l] & G
\end{tikzcd}
\]
con el del elemento neutro (por la izquierda) para obtener
\[
\begin{tikzcd}
 & T\arrow[dl, "\varepsilon\circ\pi_T"']\arrow[dr, "f"] \arrow[d]& \\
G & \arrow[l]\arrow[r]\arrow[d, "\mu"]G\times_S G & G\\
& G &  
\end{tikzcd}
\] 
La flecha $T\to G$ resultante de este diagrama es justamente $(\varepsilon\circ\pi_T)\cdot f$. Además tenemos que esto es igual a $f$ (continuando $f$ con la identidad). Así que $\varepsilon\circ\pi_T$ es neutro por la izquierda. Combinando con el diagrama del neutro por la derecha se obtiene que es neutro por la derecha. 

Existencia de elemento inverso:

Ahora usamos el diagrama
\[
\begin{tikzcd}
& T\arrow[d, "f"] &\\
& G\arrow[dl, "Id"]\arrow[dr, "Id"] \arrow[d, "\Delta"]& \\
G\arrow[d, "Id"] & \arrow[l]\arrow[r]G\times_S G\arrow[d, "Id\times\tau"]& G\arrow[d, "Id"]\\
G & \arrow[l]\arrow[r]G\times_S G\arrow[d,"\mu"]& G\\
& G &
\end{tikzcd}
\]
Por definición, $T\to G$ es el producto $f\cdot (\tau\circ f)$. Por otro lado tenemos la flecha que surge de componer $T\xrightarrow{f}G\xrightarrow{\pi_G}S\xrightarrow{\varepsilon}G$.  Afirmamos que $T\to G\to S$ es $\pi_T$. Esto se sigue de que $\pi_G\circ f$ es un morfismo de $S$-esquemas de $T$ en $S$, por lo que tiene que ser $\pi_T$ (solo hay un morfismo de $S$-esquemas hacia $S$, porque debe conmutar con la identidad). Por lo tanto $f\cdot (\tau\circ f)=\varepsilon\circ\pi_t$ por conmutatividad, por lo que hemos encontrado el inverso. 

Asociatividad:

Sean $f_1,f_2,f_3:T\to G$. 
\[
\begin{tikzcd}
 &  & T \arrow[rd, "f_3"] \arrow[d] \arrow[ld] \arrow[d] \arrow[d, "f_2"'] \arrow[dd, bend left] \arrow[lld, "f_1"'] &  \\
G \arrow[rdd, "Id", bend right] & G\times_SG \arrow[l] \arrow[dd, "\mu", bend right] & G & G \arrow[dd, "Id"] \\
 &  & (G\times_S G)\times_SG \arrow[lu, "pr_1"'] \arrow[d, "\mu\times Id"] \arrow[ru, "pr_2"'] &  \\
 & G & G\times_S G \arrow[d, "\mu"] \arrow[l] \arrow[r] & G \\
 &  & G & 
\end{tikzcd}
\]


La flecha $T\to G$ es $(f_1\cdot f_2)\cdot f_3$. Usando la otra parte del diagrama de asociatividad se obtiene $f_1\cdot (f_2\cdot f_3)$, que por la conmutatividad de los diagramas nos da la propiedad asociativa. 

Conmutatividad si el esquema en grupo es conmutativo:

Sean $f_1,f_2:T\to G$. 
\[
\begin{tikzcd}
 & T \arrow[rd, "f_2"] \arrow[d] \arrow[ld, "f_1"'] \arrow[d] &  \\
G & G\times_S G \arrow[l] \arrow[r] \arrow[d, "\sigma"] \arrow[ld, "pr_1"'] \arrow[rd, "pr_2"] & G \\
G & G\times_S G \arrow[l, "pr_1"] \arrow[r, "pr_2"] \arrow[d, "\mu"] & G \\
 & G & 
\end{tikzcd}
\]
Y tenemos el mismo diagrama intercambiando $pr_1$ y $pr_2$. Teniendo en cuenta que $\sigma\mu=\mu$ y eliminando la parte central del diagrama anterior obtenemos
 \[
\begin{tikzcd}
 & T \arrow[rd, "f_2"] \arrow[d] \arrow[ld, "f_1"'] \arrow[d] &  \\
G & G\times_S G \arrow[l, "pr_1"'] \arrow[r, "pr_2"] \arrow[d, "\mu"] & G \\
 & G & 
\end{tikzcd}
\]
que nos define $f_1\cdot f_2$, y el mismo diagrama intercambiando $pr_1$ y $pr_2$, que es realmente el mismo diagrama girado, luego $f_2\cdot f_1=f_1\cdot f_2$. 
\end{dem}


\begin{ej}
Veamos qué pinta tiene este producto en el ejemplo del grupo $G_a=\spec(\Z[X])$. Sea $T\to\Z$ un esquema. Recordemos que $\Hom_{\Sch_\Z}(T,G_a)\cong\Hom_{\Ring}(\Z[X],\OO_T(T))$. Entonces $h$ se corresponde con $f^*\otimes g^*:\Z[X]\otimes\Z[X]\to\OO_T(T)$. Además, como cada morfismo $\Z[X]\to\OO_T(T)$ está caracterizado por la imagen de $X$, el conjunto de tales morfismos es precisamente $\OO_T(T)$. 

Ahora, $\mu^*(X)=X\otimes 1+1\otimes X\mapsto f^*(X)\cdot 1+1\cdot g^*(X)=a+b$ si asociamos $f$ con un elemento $a\in\OO_T(T)$ y $g$ con otro elemento $b$. Así que $G_a$ induce la suma en $\OO_T(T)$, de lo cual viene el nombre de grupo aditivo. 
\end{ej}

\begin{ej}
Hagamos lo mismo en el grupo multiplicativo. Para $T$ un esquema, obtenemos $\Hom_{\Sch_\Z}(T,G_m)\cong(\OO_T(T))^\times$ puesto que $X$ tiene que mapearse a un elemento invertible. Se comprueba que $G_m$ induce el producto en el grupo de las unidades. 
\end{ej}


A lo que hemos llamado $G(T)$ es lo que usualmente denotábamos $h_G(T)=\Hom_{\Sch}(T,G)$, que en general era un functor $\Sch_S\to\Set$, pero al tener la imagen estructura de grupos, podemos factorizarlo como $\Sch_S\to\Grp\to\Set$ donde el último functor es el functor olvido. Para comprobar esto tenemos que comprobar que $h_G$ lleva morfismos de $S$-esquemas en homomorfismos de grupos, esto es, si $\phi:T'\to T$ es un morfismo de $S$-esquemas y $h_G(\phi)(f)=f\circ\phi:T'\to G$ para $f:T\to G$, entonces $h_G(\phi)(f_1\cdot f_2)=h_G(f_1)\cdot h_G(\phi)(f_2)$, o equivalentemente $(f_1\cdot f_2)\circ\phi=(f_1\circ\phi)\cdot (f_1\circ\phi)$.  Para ello usamos el diagrama 
\[
\begin{tikzcd}
 & T' \arrow[d, "\phi"] &  \\
 & T \arrow[rd, "f_2"] \arrow[d] \arrow[ld, "f_1"'] \arrow[d] &  \\
G & G\times_S G \arrow[l, "pr_1"'] \arrow[r, "pr_2"] \arrow[d, "\mu"] & G \\
 & G & 
\end{tikzcd}
\]
que por conmutatividad es el mismo que
\[
\begin{tikzcd}
 & T' \arrow[rd, "f_2\circ\phi"] \arrow[d] \arrow[ld, "f_1\circ\phi"'] \arrow[d] &  \\
G & G\times_S G \arrow[l, "pr_1"'] \arrow[r, "pr_2"] \arrow[d, "\mu"] & G \\
 & G & 
\end{tikzcd}
\]
de donde se deduce la igualdad.

\begin{ej}
Volvemos al ejemplo $\spec(\Z[x_1,x_2,x_3,x_4]/\gene{x_1x_4-x_2x_3-2})=\spec(B)$ y consideramos $T\to\spec(\Z)$ un esquema, entonces $\Hom_{\Sch}(T, \spec(B))\cong \Hom_{\Ring}(B,\OO_T(T))\cong SL_2(\OO_T(T))$. Esto nos da $h_{\spec(B)}:\Sch\to\Grp$, aunque hay que ver que esta correspondencia es realmente un functor. Si $\phi:T'\to T$ es un morfismo de esquemas, esto nos da $\OO_T(T)\to\OO_{T'}(T')$. Esto nos daría un morfismo $SL_2(\OO_T(T))\to SL_2(\OO_{T'}(T'))$ y hay que ver que es justamente el morfismo inducido por la correspondencia.
%\O da la O de MO 
\[
\begin{tikzcd}
\Hom_{\Sch}(T,\spec(B))\arrow[r, "h(\phi)"]\arrow[d, "\cong"] & \Hom_{\Sch}(T',\spec(B))\arrow[d, "\cong"]\\
\Hom(B,\OO_T(T))\arrow[r, "\phi(T)\circ-"]\arrow[d, "\cong"] & \Hom(B,\OO_{T'}(T'))\arrow[d,"\cong"]\\
SL_2(\OO_T(T))\arrow[r] & SL_2(\OO_{T'}(T'))
\end{tikzcd}
\]
Si tenemos una aplicación $x_1\mapsto a, x_2\mapsto b, x_3\mapsto c, x_4\mapsto d$ formando estos coeficientes una matriz de $SL_2(\OO_T(T))$, en $SL_2(\OO_{T'}(T'))$ tendríamos la aplicación $x_1\mapsto \phi(T)(a), x_2\mapsto\phi(T)(b),x_3\mapsto\phi(T)(c),x_4\mapsto\phi(T)(d)$, por lo que efectivamente es el morifsmo inducido. 


Veamos cómo funciona $\mu:\spec(B)\times_\Z\spec(B)\to\spec(B)$, que viene inducido por $\mu^*:B\to B\otimes_\Z B$. Usando el producto de matrices definimos $\mu^*(x_1)=x_1\otimes x_1+x_2\otimes x_3$, $\mu^*(x_2)=x_1\otimes x_2+x_2\otimes x_4$ y así sucesivamente (sustituir el producto habitual por el tensorial en cada entrada de la matriz). De igual modo, $\tau$ viene inducida por $\tau^*:B\to B$ que se define usando la fórmula para la matriz inversa. También $\varepsilon$ estaría inducida por $\varepsilon^*:B\to \Z$, que está definida por la aplicación que envía cualquier matriz a la matriz identidad, es decir, $x_1,x_3\mapsto 1$, $x_2,x_4\mapsto 0$. 
\end{ej}



\begin{prop}
Sea $\tilde{h}_G:\Sch_S\to\Grp$ un levantamiento de $h_G:\Sch_S\to\Set$ (una factorización como arriba). Entonces existen $\mu,\tau,\varepsilon$ morfismos de $\Sch_S$ tales que $(G,\mu,\tau,\varepsilon)$ es un esquema en grupos sobre $S$ de mod oque para todo $T\to S$, $\tilde{h}_G(T)$ tiene estructura de grupo heredada de $\mu,\tau,\varepsilon$. 
\end{prop}

Recordemos que por el lema de Yoneda $h_*:\CC\to\Nat(\CC^{op}, \Set)$ es plenamente fiel, es decir, para cualesquiera $X,Y\in\CC$ se tiene una biyección $\Hom_\CC(X,Y)\cong \Nat(h_X,h_Y)$

\begin{dem}
Existencia de $\mu$:

Para todo $T\in\Sch_S$, $\Hom_{\Sch_S}(T,G)$ es un grupo, y tenemos $\mu_T:h_G(T)\times h_G(T)\to h_G(T)$, y además $h_G(T)\times h_G(T)\cong\Hom_{\Sch_S}(T,G\times_SG)=h_{G\times_S G}(T)$ por la definición del producto de morfismos. Es fácil ver que $h_{G\times_S G}$ es natural con respecto a morfismos $\alpha:T_1\to T_2$ y el producto $\mu_T$. Como $\tilde{h}_G$ es un functor en $\Grp$, para todo $f_1,f_2\in h_G(T_2)$, $\tilde{h}_G(\mu_{T_2}(f_1,f_2))=\mu_{T_1}(\tilde{h}_G(\alpha)(f_1),\tilde{h}_G(\alpha)(f_2))$. Por Yoneda $\mu$ se corresponde con un morfismo de $S$-esquemas $\mu:G\times_SG\to G$. 

Existencia de $\varepsilon$:


Para todo $T\to S$, $h_G(T)$ es un grupo con elemento neutro $e_T:\{*\}=\Hom_{\Sch_S}(T,S)\to\Hom_{Sch_S}(T,G)$, que envía el únivo morfismo del origen a $e_T$, y esto es claramente un homomorfismo de grupos. Definimos la transformación natural $\varepsilon:\{\varepsilon_T: T\in\Sch_S\}$. Por la definición de $\varepsilon_T$, es trivial comprobar la naturalidad de $\varepsilon$ con respecto a $\alpha:T_1\to T_2$ usando que $h_G$ es un morfismo de grupos. 

Existencia de $\tau$:

Para todo $T\to S$, $\tau_T:h_G(T)\to h_G(T)$ está definida como $f\mapsto f^{-1}$ (inverso del grupo). Obsérvese que esto no es un homomorfismo en general si el grupo no es conmutativo. Comprobamos la naturalidad para $\alpha:T_1\to T_2$. Sea $f:T_2\to G$, hay que ver que $h_G(\alpha)(\tau_{T_2}(f))=\tau_{T_1}(h_G(\alpha)(f))$, o equivalentemente $\tau_{T_2}(f)\circ\alpha=\tau_{T_1}(f\circ\alpha)$. Recordemos que $\tau_{T_1}(f\circ\alpha)$ está caracterizado por $\tau_{T_1}(f\circ\alpha)\cdot (f\circ\alpha)e_{T_1}$ y lo mismo por la izquierda. Así que vemos que el otro miembro de la igualdad satisface las mismas relaciones. 

\[
\tau_{T_2}(f\circ\alpha)\cdot (f\circ\alpha)=h_G(\alpha)(\tau_{T_2}(f))\cdot h_G(\alpha)(f)=h_G(\alpha)(\tau_{T_2}(f)\cdot f)=h_G(\alpha)(e_{T_2})=e_{T_1}
\]
donde en la segunda igualdad hemos usando que $h_G(\alpha)$ es homomorfismo de grupos. Análogamente a la izquierda. Esta conmutatividad nos da el $\tau$ buscado via lema de Yoneda.

Ahora hay que ver que los morfismos $\mu,\tau,\varepsilon$ satisfacen los diagramas conmutativos de los grupos. Vamos a probar uno y los demás se dejan como ejercicio
\[
\begin{tikzcd}
G\times_S S\arrow[r, "Id_G\times\varepsilon"]\arrow[dr, "pr_1"] & G\times_S G\arrow[d,"\mu"]\\
& G
\end{tikzcd}
\]
Hay que ver $\mu_T\circ(Id_G\times\varepsilon_t)=pr_1$. En este caso no para el diagrama escrito arriba sino para $\varepsilon_T:\Hom(T,S)\to\Hom(T,G)$, $Id_G:\Hom(T,G)\to\Hom(T,G)$ y $pr_1:\Hom(T,G\times_S G)\to \Hom(T,G)$. Esto se reduce al hecho de que $\Hom_{\Sch}(T,G)$ con $\mu_T,e_T,\tau_T$ es un grupo. 
\end{dem}

\section{Cambio de base}

Sea $S$ un esquema y $X$ un esquema en grupos sobre $S$. Sea $T\to S$. Queremos dotar a $X\times_S T\to T$ de estructura de esquemas en grupos sobre $T$. Para todo $Y\to T$ queremos dotar a $\Hom_{\Sch_T}(Y,X\times_S T)$ de estructura de grupo. Para ello buscamos una biyección $\Hom_{\Sch_T}(Y,X\times_S T)\cong\Hom_{\Sch_S}(Y,X)$. La prueba es parecida a que $\Hom$ preserva productos, pero tenemos que observar que hay un cambio de base. 

Necesitamos que conmute el diagrama
\[
\begin{tikzcd}
 & Y \arrow[d, "f"] \arrow[ldd] &  \\
 & X\times_S T \arrow[d] \arrow[ld, "pr_1"] \arrow[rd] &  \\
X \arrow[r] & S & T \arrow[l]
\end{tikzcd}
\]

donde $f$ es un morfismo de $T$-esquemas y los dos triángulos inferiores son conmutativos. Entonces asociamos $f\mapsto pr_1\circ f$ y obtenemos el diagrama conmutativo
\[
\begin{tikzcd}
 & Y \arrow[d] \arrow[ld, "f'"'] \arrow[rd, "{\pi_{Y,T}}"] &  \\
X & X\times_S T \arrow[l] \arrow[r] & T
\end{tikzcd}
\]
Haciendo $f'\mapsto f$, como $f$ es un morfismo de esquemas sobre $T$, obtennemos $\Hom_{\Sch_T}(Y,X\times_S T)$ es un grupo inducido por la ley de esquemas en grupos de $X$. 

Faltaría ver que para $\phi:Y_1\to Y_2$, $h_{X\times_S T}(\phi):\Hom_{\Sch_T}(Y_2,X\times_S T)\to\Hom_{\Sch_T}(Y_1,X\times_S T)$ es un morfismo de grupos. 
Llamamos $\Lambda:\Hom_{\Sch_S}(Y,X)\to\Hom_{\Sch_T}(Y,X\times_S T)$. Sean $f_1,f_2\in\Hom_{\Sch_T}(Y,X\times_S T)$. Queremos probar que $h_{X\times_S T}(f_1\cdot f_2)=h_{X\times_S T}(f_1)\cdot h_{X\times_S T}(f_2)$. Sean $f_1',f_2'\in\Hom_S(Y,X)$ y $\Lambda(f_i')=f_i$. Por definición de la estructura de grupo en $\Hom_T(Y,X\times_S T)$,$\Lambda$ es un homomorfismo de grupos, así que lo que hay que probar es que $h_{X\times_S T}(\Lambda(f_1'\cdot f_2'))=h_{X\times_S T}(\Lambda(f_1'))\cdot=h_{X\times_S T}(\Lambda(f_2'))$. Tenemos que $h_{X\times_S T}(\Lambda(f'))=\Lambda(h_X(f'))$: $\Lambda(h_X(f'))=\Lambda(f'\circ\phi)=h_{X\times_S T}(\Lambda(f'))=\Lambda(f')\circ\phi$. El diagrama siguiente nos da $\Lambda(f')\circ\phi$
\[
\begin{tikzcd}
 & Y_1 \arrow[d] &  \\
 & Y_2 \arrow[ld, "f'"'] \arrow[rd] \arrow[d, "\Lambda(f')"] &  \\
X & X\times_ST \arrow[l] \arrow[r] & T
\end{tikzcd}
\]
y el de a continuación nos da $\Lambda(f'\circ\phi)$
\[
\begin{tikzcd}
Y_1 \arrow[d, "\phi"'] \arrow[rdd, "\Lambda(f'\circ\phi)"] \arrow[rrdd, bend left] &  &  \\
Y_2 \arrow[d, "f'"'] &  &  \\
X & X\times_S T \arrow[l] \arrow[r] & T
\end{tikzcd}
\]

Las proyecciones de la derecha coinciden porque solo hay un morfismo $T$-esquemas $Y\to T$. Como $X$ es un esquema en grupos, $h_X$ s un morfismo de grupos. Hemos dado entonces un functor $\tilde{h}_{X\times_S T}:\Sch_T\to\Grp$ que levanta a $h_{X\times_S T}:\Sch_T\to\Set$. 




\begin{ej}[Esquema en grupos constante]
Sea $G$ un grupo. Nos preguntamos si existe un $S$-esquema $X$ en grupos tal que $X(T)\cong G$ para todo $T\to S$, o al menos para algún $T\to S$. Definimos $S^G$ como la unión disjunta de copias de $S$ indexadas por $G$, $S^G=\coprod_{g\in G}S_g$, donde $S_g=S$ para todo $g$. Si $U\subseteq S^G$ es un abierto, $\OO_{S^G}(U)=\prod_{g\in G}\OO_S(U)$. Sea $T\to S$ un esquema. $\Hom_{\Sch_S}(T,S^G)=\Hom(T,\coprod_{g\in G}S_g)$. Para todo $g\in G$, sea $U_g=f^{-1}(S_g)\subseteq T$ abierto. Tenemos $T=\coprod_{g\in G} U_g$, por lo que $f:T\to\coprod_{g\in G}S_g$ se corresponde con la familia de restricciones $f|_{U_g}:U_g\to S$, cada uno de ellos siendo el morfismo estructural $\pi_{U_g,S}$ como $S$-esquema. Para otro morfismo $g$ definimos la misma familia pero llamando $V_h$ a los abiertos. Definimos $h=f\cdot g$ como el morfismo estructural $h|_{U_g\cap V_h}:U_g\cap V_h\to S_{g\cdot h}=S$.

Nótese que si $T$ es conexo, solo hay un $U_x$, porque el complementario es abierto y por tanto tendríamos $T$ como unión de abiertos y cerrados disjutnos. En este caso, solo hay un $h$ y la estructura de grupo es claramente la de $G$. En el caso de que $T$ no sea conexo esto no podrá ser así nunca porque cada componente conexo nos dará una copia de $G$, en concreto $\Hom_S(T,X)=\coprod_i G_i$, donde el índice $i$ recorre las componentes conexas de $T$ y $G_i=G$ para todo $i$. En particular, tenemos una biyección entre $\Hom_S(T,X)$ y las funciones localmente constantes $T\to G$. 
\end{ej}

\section{Morfismos de esquemas en grupos}
Tenenemos dos definiciones equivalentes, una desde el punto de vista de los morfismos de esquemas y otro desde el functor de los puntos. 

\begin{defi}
Un morfismo de $S$-esquemas $\alpha:G\to G'$ donde $G,G'$ son esquemas en grupos sobre $S$ se dice morfismo de esquemas en grupos si el siguiente diagrama es conmutativo.
\[
\begin{tikzcd}
G\times_S G \arrow[r, "\alpha\times\alpha"] \arrow[d, "\mu_G"'] & G'\times_S G' \arrow[d, "\mu_{G'}"] \\
G \arrow[r, "\alpha"] & G'
\end{tikzcd}
\]
\end{defi}

\begin{lemma}
Para todo $T\to S$, $\alpha$ induce un morfismo de grupos $\Hom(T,G)\xrightarrow{\alpha(T)}\Hom(T,G')$ asignando $f\mapsto \alpha\circ f$. 
\end{lemma}
\begin{proof}
Para todo $f_1,f_2\in G(T)$, $\alpha(T)(f_1\cdot f_2)=\alpha\circ(f_1\cdot f_2)=(\alpha(T)\cdot f_1)\cdot(\alpha(T)\cdot f_2)=(\alpha\circ f_1)\cdot(\alpha\circ f_2)$ por el diagrama simplicado del producto
\[
\begin{tikzcd}
 & T \arrow[ld, "f_1"'] \arrow[rd, "f_2"] \arrow[d] \arrow[dd, bend left] &  \\
G \arrow[d, "\alpha"'] & G'\times_S G' \arrow[ld] \arrow[rd] \arrow[d, "\mu_{G'}"'] & G \arrow[d, "\alpha"] \\
G' & G' & G'
\end{tikzcd}
\]
donde la flecha curva es $(\alpha\circ f_1)\cdot (\alpha\circ f_2)$. Recíprocamente, si damos una familia $\{\alpha(T):\Hom_S(T,G)\to\Hom_S(T,G')\}$ de morfismos de grupos tales que si $\phi:T_1\to T_2$ el diagrama siguiente es conmutativo 
\[
\begin{tikzcd}
{\Hom_S(T_2,G)} \arrow[r] \arrow[d, "\phi\circ-"] \arrow[r, "\alpha(T_2)"] & {\Hom_S(T_2,G')} \arrow[d, "\phi\circ-"] \\
{\Hom_S(T_1,G)} \arrow[r, "\alpha(T_1)"] & {\Hom_S(T_1,G')}
\end{tikzcd}
\]
Se sigue del lema de Yoneda que esta familia proviene de un $\alpha:G\to G'$ morfismo de esquemas en grupos sobre $S$.
\end{proof}


\begin{ej}
Sea $G$ un esquema en grupos conmutativo sobre $S$ y $n\in\N$. Definimos $[n]:G\to G$ como la multiplicación por $n$, es decir, para todo $T\to S$, $[n](T):\Hom_S(T,G)\to\Hom_S(T,G)$ definida como $f\mapsto f\cdots f$ ($n$ veces). Se verifica trivialmente que es un homomorfismo de gurpos. Se verifica fácilmente la conmutatividad del diagrama
\[
\begin{tikzcd}
{\Hom_S(T_2,G)} \arrow[r] \arrow[d, "\phi\circ-"] \arrow[r, "\alpha(T_2)"] & {\Hom_S(T_2,G)} \arrow[d, "\phi\circ-"] \\
{\Hom_S(T_1,G)} \arrow[r, "\alpha(T_1)"] & {\Hom_S(T_1,G)}
\end{tikzcd}
\]

Se puede definir también directamente sobre $G$ inductivamente. $[1]:G\to G$ se define como la itendidad y $[n]:G\to G$ se define como la composición 
\[
G\xrightarrow{\Delta}G\times_SG\xrightarrow{[1]\times[n-1]}G\times_S G\xrightarrow{\mu_G}G
\]
y se comprueba que da lo mismo que con la anterior definición. 
\end{ej}

\begin{ej}
Sea $k$ un cuerpo de característica $p>0$ y $G_{a,k}=\spec(k[x])\to\spec(k)$. Tenemos el producto inducido por $x\mapsto 1\otimes x+x\otimes 1$. Definimos el morfismo de Frobenius $F_p$ definido por $G_{a,k}(T)\cong\OO_T(T)\to \OO_T(T)\cong G_{a,k}$, $a\mapsto a^p$, que es un morfismo de grupos y $\{F_p(T)\}$ define un morfismo de esquemas en grupos sobre $\spec(k)$.  
\end{ej}

\subsection{Núcleo de un morfismo de esquemas en grupos}
Sea $\alpha:G\to G'$ un morfismo de esquemas en grupos sobre $S$. Para todo $T\to S$, $\alpha(T):G(T)\to G'(T)$ es un morfismo de grupos. Sea $K_T=\ker(\alpha(T))$. Consideramos el functor $\Sch_S\to\Grp$ dado por $T\mapsto K_T$. Nos preguntamos si existe $Z\to S$ tal que $\Hom_S(T,Z)\cong K_T$ para todo $T\to S$. Tenemos que $K_T=\ker(\alpha(T))=\{f\in\Hom_S(T,G')\mid \alpha(T)(f)=\alpha\circ f=\varepsilon_{G'}\circ\pi_T\}$. Es decir, $f\in K_T$ si y solo conmuta el diagrama
\[
\begin{tikzcd}
T \arrow[r] \arrow[d, "\pi_T"'] \arrow[r, "f"] & G \arrow[d, "\alpha"] \\
S \arrow[r, "\varepsilon_{G'}"] & G'
\end{tikzcd}
\]
Si definimos $Z=S\times_{G'}G$ podemos colocarlo en el interior del diagrama con las proyecciones naturales de modo que obtenemos $f':T\to Z$ (las estructuras de $G'$-esquemas están dadas por las flechas del diagrama). Así que hay una correspondencia biyectiva entre las $f$ que hacen conmutativo el diagrama y $\Hom_S(T,S\times_{G'}\times G\}$. Esto significa que $S\times_{G'}G$ representa al functor $T\to K_T$. Por tanto hacemos la siguiente definición.

\begin{defi}
Se define $\ker\alpha=S\times_{G'}G$. 
\end{defi}
\begin{nota}
Por el ejercicio 3.11 de Hartshorne, si $\alpha:X\to Y$ es una inmersión cerrada, $h:X'\to X$ morfismo de esquemas, y definimos $\alpha'$ mediante 
\[
\begin{tikzcd}
X'\times_X Y\arrow[r] \arrow[d, "\alpha'"] & Y\arrow[d, "\alpha"]\\
X'\arrow[r, "h"] & X
\end{tikzcd}
\]
entonces $\alpha'$ también es una inmersión cerrada. Por tanto, si $\varepsilon_{G'}$ es una inmersión cerrada, también lo es $S\times_{G'}G\to G$, por lo que el núcleo se puede ver en este caso como un subesquema cerrado de $G$ (obsérvese que esto es independiente de $\alpha$). 
\end{nota}

\begin{prop}
Sea $G$ un esquema en grupos sobre $S$. Son equivalentes:
\begin{enumerate}
\item $G\to S$ es separado.
\item $\varepsilon_G:S\to G$ es una inmersión cerrada. 
\end{enumerate}
\end{prop}
La prueba consiste en comparar $\Delta$ y $\varepsilon_{G'}$ usando 3.11 de Hartshorne. 
\begin{ej}
Sea $G_m$, $[n]:G_m\to G_m$. Para todo $T\to S$, $\ker[n](T)=\{f:T\to \spec(\Z[x,x^{-1}])\mid f^n=id\}=\{t\in\OO_T(T)^\times\mid t^n=1\}$. Equivalentemente, $\ker[n]=\spec(\Z[x,x^{-1}]/\gene{x^n-1})$. 
\end{ej}

\subsection{Cocientes}

\begin{ej}
Calculamos el conúcleo de $[2]:G_m\to G_m$, siendo $G_m=\spec(\Z[x,x^{-1}])\to\spec(\Z)$. A nivel de anillos, la operación es simplemente elevar al cuadrado, luego $[2](T):\OO_T(T)^\times\to\OO_T(T)^\times$ es $t\mapsto t^2$. $\coker[2](T)=G_m(T)/[2](T)(G_m(T))=\OO_T(T)^\times/(\OO_T(T)^\times)^2$. Querríamos representar el functor $T\mapsto \coker[2](T)$, pero esto no es posible. Sea $T=\spec(\Q)$. Tenemos que $\coker[2](\spec(\Q))\neq\{1\}$, pero si pasamos a la clausura algebraica $\overline{\Q}$, sí obtenemos un conúcleo trivial. Para $Z$ separable, usando EGA IV tenemos que $\Q\hookrightarrow \overline{\Q}$ por ser inyectivo entre dominios de integrar, la aplicación $\Hom(\spec(\Q),Z)\to\Hom(\spec(\overline{Q}),Z)$ sería inyectivo, pero es un homomorfismo $\{1\}\neq Z(\Q)\to Z(\overline{\Q})=1$. El caso general se sale del nivel de este universo. 

Podemos ver que $[2]$ es sobre entre los espacios topológicos, porque $[2](\p)=\phi^{-1}(\p)=\{a\in\Z[x,x^{-1}]\mid a^2\in\p\}=\p$. Recordamos que si $f:X\to Y$ con $Y$ reducido, $f(X)\subseteq Y$ dotado dela estructura reducida es la imagen esquemática de $f$. Tomando $X=Y=G_m$, que son reducidos, $[2](G_m)=G_m$, por lo que la imagen esquemática de $[2]$ es $G_m$. Querríamos que $\coker[2]=G_m/G_m=1$. Pero $\coker([2](\spec\Q))\neq 1$. 
\end{ej}

Este ejemplo muestra la dificultad para definir los cocientes. La construcción de los cocientes requiere uso de las topologías de Grothendieck, por lo que pasamos a definirlos directamente mediante una propiedad unviersal. 

\begin{defi}
Sea $\phi:H\to G$ un morfismo de esquemas en grupos sobre $S$. Un par $(Q,\eta)$ con $Q\to S$ un esquema y $\eta:G\to Q$ es un \emph{cociente} si el siguiente diagrama es conmutativo
\[
\begin{tikzcd}
G\times_S H \arrow[r, "Id_G\times\phi"] \arrow[d, "pr_1"'] & G\times_S G \arrow[r, "\mu"] & G \arrow[d, "\eta"] \\
G \arrow[rr] &  & G
\end{tikzcd}
\]
y para todo para $(Y,f)$ satisfaciendo las mismas propiedades existe un único morfismo que hace conmutar el diagrama siguiente
\[
\begin{tikzcd}
G\times_S H \arrow[r, "Id_G\times\phi"] \arrow[dd, "pr_1"'] & G\times_S G \arrow[r, "\mu"] & G \arrow[dd, "f"] \arrow[ld, "\eta"] \\
 & Q \arrow[rd, dashed] &  \\
G \arrow[ru, "\eta"] \arrow[rr] &  & Y
\end{tikzcd}
\]
Si para todo $T\to S$ $\phi(H(T))<G(T)$ es normal, $Q$ es un esquema en grupos sobre $S$, $\eta$ es un morfismo de esquemas en grupos y $\ker\eta\cong\phi(H)$. 
\end{defi}
\end{document}