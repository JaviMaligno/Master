\documentclass[GA.tex]{subfiles}

\begin{document}


%\hyphenation{equi-va-len-cia}\hyphenation{pro-pie-dad}\hyphenation{res-pec-ti-va-men-te}\hyphenation{sub-es-pa-cio}

\chapter{Esquemas en grupos (group schemes)}
Rereferencias: Alebgraic Geometry II, Abelian Varietis (Mumford), Abelian Varieties (Edihoven van der Geer y Moonen)

\section{El functor de los puntos}
Los puntos de un esquema $X$ normalmente son los puntos del espacio topológico subyacente, pero hay otra noción de ``puntos con valores en un esquema $T$'', que es el functor $X(T)=h_X(T)=\Hom_{\Sch}(T,X)$. 

\begin{ej}
Sea $X=\spec(\Z[x,y]/\gene{x^2+y^2-1})$ y $A$ un anillo. Calculamos $X(\spec(A))=\Hom_{\Sch}(\spec(A),X) \cong\Hom_{\Ring}(\Z[x,y]/\gene{x^2+y^2-1},A)=\{a,b\in A\mid a^2+b^2=1\}$. En este caso vemos que el resultado son los puntos de una circunferencia. En general veremos que esto es lo que tiene estructura de grupo. 
\end{ej}
\begin{ej}
Sea $X=\spec(Z[x_1,\dots, x_4]/\gene{x_1x_4-x_2x_3-1})$. Para un anillo $A$, haciendo lo mismo que en el ejemplo anterior obtenemos el conunto $\{(a_1,\dots, a_4)\in A^4\mid a_1a_4-a_2a_3=1\}$, que son las matrices cuadradas de orden 2 con determinante 1, es decir, $SL_2(A)$, que tiene claramente estructura de grupo, la cual depende de la suma y el producto de $A$ y de el esquema $X$ que hemos elegido. 
\end{ej}

\begin{ej}
Si $X=\spec(B)$ y $T$ es un esquema cualquiera. Entonces $X(T)=\Hom_{\Sch}(T,X)=\Hom_{\Sch}(T,\spec(B))\cong\Hom_{\Ring}(B,\OO_T(T))$ (Hartshorne 2.4 maybe) que tiene estructura de grupo heredada de $\OO_T(T)$. 
\end{ej}

Fijaremos en este capítulo un esquema $S$ como base. De la definición de producto fibrado se sigue que para cualquier esquema $Z$, $\Hom_{\Sch|_S}(Z,X\times_S Y)\cong\Hom_{\Sch|_S}(Z,X)\times \Hom_{\Sch|_S}(Z,Y)$. 

\begin{observacion}
Sean $X\xrightarrow{f}S, Y\xrightarrow{g}S$, $X'\xrightarrow{f'}S, Y'\xrightarrow{g'}S$ esquemas sobre $S$. Sean $\alpha:X\to X'$ y $\beta:Y\to Y'$ morfismos de $S$-esquemas. Entonces existe un único $\alpha\times_S\beta\in\Hom(X\times_SY,X'\times_SY')$ haciendo conmutativo el diagrama 

\[
\begin{tikzcd}
X\arrow[d, "\alpha"] & X\times_SY\arrow[l, "p_1"]\arrow[r, "p_2"]\arrow[d, dashed, "\alpha\times_S\beta"] & Y\arrow[d, "\beta"]\\
X' &X'\times_SY'\arrow[l, "p_1'"]\arrow[r, "p_2'"] & Y'
\end{tikzcd}
\]
\end{observacion}

\begin{defi}
Un esquema en grupos sobre $S$ es una tupla $(G,\mu,\tau,\varepsilon)$ donde $G\to S$ es un $S$-esquema, $\mu:G\times_SG\to G$ es la \emph{ley de producto}, $\tau:G\to G$ es el \emph{inverso} y $\varepsilon:S\to G$ es el \emph{elemento neutro} (todos ellos son morfismos de $S$-esquemas). A estos datos se les piden las siguientes condiciones:
\begin{enumerate}
\item Asociatividad:
\[
\begin{tikzcd}
& G\times_S(G\times_S G)\cong (G\times_SG)\times_S G\arrow[rd, "Id_G\times\mu"]\arrow[ld, "\mu\times Id_G"] &\\
G\times_S G \arrow[r, "\mu"] & G & G\times_SG\arrow[l, "\mu"]
\end{tikzcd}
\]
\item Elemento neutro a izquierda y derecha (escribimos el diagrama conmutativo para la derecha, es análogo por la izquierda)
\[
\begin{tikzcd}
G\times_S S \arrow[r, "Id_G\times\varepsilon"]\arrow[dr, "p_1"] & G\times_S G\arrow[d, "\mu"]\\
& G
\end{tikzcd}
\]
\item Inverso a izquierda y derecha (solo escribimos el de la derecha) 
\[
\begin{tikzcd}
G\arrow[r, "\Delta"]\arrow[d] & G\times_S G\arrow[r, "Id\times \tau"] & G\times_S G\arrow[d, "\mu"]\\
S\arrow[rr, "\varepsilon"] & & G
\end{tikzcd}
\]

$G$ será un esquema en grupos conmutativos si además se cumple
\[
\begin{tikzcd}
G\times_S G\arrow[r, "\sigma"] \arrow[dr, "\mu"] & G\times_SG\arrow[d, "\mu"]\\
& G
\end{tikzcd}
\]
donde $\sigma$ está definido mediante
\[
\begin{tikzcd}
G & G\times_SG\arrow[l, "p_2"]\arrow[r, "p_1"]\arrow[d, "\sigma"] & G\\
& G\times_S G\arrow[ul, "p_1"]\arrow[ur, "p_2"] & 
\end{tikzcd}
\]
\end{enumerate}

Existe una noción de esquema en grupos abeliano más fuerte que lo que acabamos de ver para generalizar a las variedades abelianas .
\end{defi}

\begin{ej}
Esquema en grupo aditivo $G_a$ sobre $\Z$. Sea $G_a=\spec(Z[X])$. Dar $\mu:\spec(\Z[X])\times_S\spec(\Z[X])\to\spec(\Z[X])$ es equivalente a dar $\mu^*:\Z[X]\to \Z[X]\otimes\Z[X]$, que en este caso lo vamos a definir mediante $x\mapsto x\times 1+1\otimes x$. Definimos $\tau:G_a\to G_a$ mediante $\tau^*:\Z[X]\to\Z[X]$ como $x\mapsto -x$. Por último $\varepsilon:\Z\to G_a$ está dado por $\varepsilon^*:\Z[X]\to\Z$, $x\mapsto 0$. 

Verificamos ahora las condiciones de esquema en grupo mediante la conmutatividad de los diagramas duales en anillos. Basta comprobar la conmutatividad para el elemento $x$. Tenemos que $x\xrightarrow{\mu^*}x\otimes 1+1\otimes x\xrightarrow{\mu^*\times Id} \mu^*(x)\otimes 1+\mu^*(1)\otimes x=(x\otimes 1+1\otimes x)\otimes x+(1\otimes 1)\otimes x=x\otimes 1\otimes 1+1\otimes x\otimes 1+1\otimes 1\otimes x$. Por otro lado, $\mu^*(x)\xrightarrow{Id\times\mu^*} x\otimes\mu^*(1)+1\otimes\mu^*(x)=x\otimes(1\otimes 1)+1\otimes(x\otimes 1+1\otimes x)=x\otimes 1\otimes 1+ 1\otimes x\otimes 1+1\otimes 1\otimes x$, con lo que se verifica la asociatividad. 

Para el elemento neutro recordamos que $\Z[X]\to\Z[X]\otimes_\Z\Z$ por la derecha está dado por $a\mapsto a\otimes 1$. Entonces $Id\times\varepsilon^*(\mu^*(x))=x\otimes \varepsilon^*(1)+1\otimes\varepsilon^*(x)=x\otimes 1+0=x\otimes 1$, por lo que se verifica la propiedad del elemento neutro (es análoga por la izquierda).

Por último tenemos que vemos el inverso. Vemos que $\Delta^*:\Z[X]\otimes\Z\Z[X]\to \Z[X]$ está dado por $a\otimes b=a+b$, puesto que por el diagrama se debe cumplir $a\otimes 1\mapsto a$ y $1\otimes b\mapsto b$. Esto se observa dualizando el diagrama
\[
\begin{tikzcd}
& G\arrow[ld, "Id"] \arrow[rd, "Id"]\arrow[d, "\Delta"] & \\
G & G\times_\Z G\arrow[l]\arrow[r] & G
\end{tikzcd}
\]
Por tanto $\tau^*\mu^*(x)=x\otimes\tau^*(1)+1\otimes\tau^*(x)=x\otimes 1+1\otimes(-x)\mapsto x-x=0=i(\varepsilon(x))$. 

Vamos a ver que además se verifica la conmutatividad. En efecto, $\sigma\mu^*(x)=\sigma(x\otimes 1+1\otimes x)=1\otimes x+x\otimes 1=x\otimes 1+1\otimes \sigma=\mu^*(x)$. 
\end{ej}

\begin{ej}
Grupo multipliativo sobre $\Z$. Sea $G_m=\spec(\Z[X,Y]/\gene{XY-1})=\spec(\Z[X,X^{-1}])$. Las propiedades de este esquema se prueban de forma análoga a las de anterior ejemplo para $\mu^*:\Z[X,X^{-1}]\to\Z[X,X^{-1}]\otimes \Z[X,X^{-1}]: x\mapsto x\otimes x$; $\tau^*:\Z[X,X^{-1}]\to \Z[X,X^{-1}]: x\mapsto x^{-1}$; $\varepsilon^*:\Z[X,X^{-1}]\to\Z: x\mapsto 1$. LAS X PONERLAS TODAS IGUAL AQUÍ Y ANTES
\end{ej}


En general se puede definir un objeto grupo $X$ en una categoría $\CC$ que admita productos finitos y tenga objeto terminal $S$ dando  $\mu:X\times X\to X$, $\varepsilon:S\to X$ y $\tau:X\to X$ cumpliendo las propiedades análogas. 

\begin{prop}
Sea $G$ un esquema en grupos (conmutativo) sobre $S$. Sea $T\to S$ un $S$-esquema. Entonces $G(T)=\Hom_{\Sch|_S}(T,G)$ está dotado de estructura de grupo (conmutativo) mediante la siguiente ley: para todo $f,g\in\Hom_{\Sch|_S}(T,G)$ se define $f\cdot g=\mu\circ h\in\Hom_{\Sch|_S}(T,G)$ donde $h$ satisface el diagrama conmutativo
\[
\begin{tikzcd}
 & T\arrow[dl, "f"]\arrow[dr, "g"]\arrow[d, "h"] & \\
 G & G\times_S G\arrow[r] \arrow[l] & G
\end{tikzcd}
\]
\end{prop}
\begin{dem}
Existencia de elemento neutro: 

Podemos conectar el diagrama
\[
\begin{tikzcd}
 & T\arrow[dl, "\pi_T"']\arrow[dr, "f"]\arrow[d] & \\
 S & S\times_S G\arrow[r] \arrow[l] & G
\end{tikzcd}
\]
con el del elemento neutro (por la izquierda) para obtener
\[
\begin{tikzcd}
 & T\arrow[dl, "\varepsilon\circ\pi_T"']\arrow[dr, "f"] \arrow[d]& \\
G & \arrow[l]\arrow[r]\arrow[d, "\mu"]G\times_S G & G\\
& G &  
\end{tikzcd}
\] 
La flecha $T\to G$ resultante de este diagrama es justamente $(\varepsilon\circ\pi_T)\cdot f$. Además tenemos que esto es igual a $f$ (continuando $f$ con la identidad). Así que $\varepsilon\circ\pi_T$ es neutro por la izquierda. Combinando con el diagrama del neutro por la derecha se obtiene que es neutro por la derecha. 

Existencia de elemento inverso:

Ahora usamos el diagrama
\[
\begin{tikzcd}
& T\arrow[d, "f"] &\\
& G\arrow[dl, "Id"]\arrow[dr, "Id"] \arrow[d, "\Delta"]& \\
G\arrow[d, "Id"] & \arrow[l]\arrow[r]G\times_S G\arrow[d, "Id\times\tau"]& G\arrow[d, "Id"]\\
G & \arrow[l]\arrow[r]G\times_S G\arrow[d,"\mu"]& G\\
& G &
\end{tikzcd}
\]
Por definición, $T\to G$ es el producto $f\cdot (\tau\circ f)$. Por otro lado tenemos la flecha que surge de componer $T\xrightarrow{f}G\xrightarrow{\pi_G}S\xrightarrow{\varepsilon}G$.  Afirmamos que $T\to G\to S$ es $\pi_T$. Esto se sigue de que $\pi_G\circ f$ es un morfismo de $S$-esquemas de $T$ en $S$, por lo que tiene que ser $\pi_T$ (solo hay un morfismo de $S$-esquemas hacia $S$, porque debe conmutar con la identidad). Por lo tanto $f\cdot (\tau\circ f)=\varepsilon\circ\pi_t$ por conmutatividad, por lo que hemos encontrado el inverso. 

Asociatividad:

Sean $f_1,f_2,f_3:T\to G$. 
\[
\begin{tikzcd}
 &  & T \arrow[rd, "f_3"] \arrow[d] \arrow[ld] \arrow[d] \arrow[d, "f_2"'] \arrow[dd, bend left] \arrow[lld, "f_1"'] &  \\
G \arrow[rdd, "Id", bend right] & G\times_SG \arrow[l] \arrow[dd, "\mu", bend right] & G & G \arrow[dd, "Id"] \\
 &  & (G\times_S G)\times_SG \arrow[lu, "pr_1"'] \arrow[d, "\mu\times Id"] \arrow[ru, "pr_2"'] &  \\
 & G & G\times_S G \arrow[d, "\mu"] \arrow[l] \arrow[r] & G \\
 &  & G & 
\end{tikzcd}
\]


La flecha $T\to G$ es $(f_1\cdot f_2)\cdot f_3$. Usando la otra parte del diagrama de asociatividad se obtiene $f_1\cdot (f_2\cdot f_3)$, que por la conmutatividad de los diagramas nos da la propiedad asociativa. 

Conmutatividad si el esquema en grupo es conmutativo:

Sean $f_1,f_2:T\to G$. 
\[
\begin{tikzcd}
 & T \arrow[rd, "f_2"] \arrow[d] \arrow[ld, "f_1"'] \arrow[d] &  \\
G & G\times_S G \arrow[l] \arrow[r] \arrow[d, "\sigma"] \arrow[ld, "pr_1"'] \arrow[rd, "pr_2"] & G \\
G & G\times_S G \arrow[l, "pr_1"] \arrow[r, "pr_2"] \arrow[d, "\mu"] & G \\
 & G & 
\end{tikzcd}
\]
Y tenemos el mismo diagrama intercambiando $pr_1$ y $pr_2$. Teniendo en cuenta que $\sigma\mu=\mu$ y eliminando la parte central del diagrama anterior obtenemos
 \[
\begin{tikzcd}
 & T \arrow[rd, "f_2"] \arrow[d] \arrow[ld, "f_1"'] \arrow[d] &  \\
G & G\times_S G \arrow[l, "pr_1"'] \arrow[r, "pr_2"] \arrow[d, "\mu"] & G \\
 & G & 
\end{tikzcd}
\]
que nos define $f_1\cdot f_2$, y el mismo diagrama intercambiando $pr_1$ y $pr_2$, que es realmente el mismo diagrama girado, luego $f_2\cdot f_1=f_1\cdot f_2$. 
\end{dem}


\begin{ej}
Veamos qué pinta tiene este producto en el ejemplo del grupo $G_a=\spec(\Z[X])$. Sea $T\to\Z$ un esquema. Recordemos que $\Hom_{\Sch_\Z}(T,G_a)\cong\Hom_{\Ring}(\Z[x],\OO_T(T))$. Entonces $h$ se corresponde con $f^*\otimes g^*:\Z[X]\otimes\Z[X]\to\OO_T(T)$. Además, como cada morfismo $\Z[X]\to\OO_T(T)$ está caracterizado por la imagen de $x$, el conjunto de tales morfismos es precisamente $\OO_T(T)$. 

Ahora, $\mu^*(x)=x\otimes 1+1\otimes x\mapsto f^*(x)\cdot 1+1\cdot g^*(x)=a+b$ si asociamos $f$ con un elemento $a\in\OO_T(T)$ y $g$ con otro elemento $b$. Así que $G_a$ induce la suma en $\OO_T(T)$, de lo cual viene el nombre de grupo aditivo. 
\end{ej}

\begin{ej}
Hagamos lo mismo en el grupo multiplicativo. Para $T$ un esquema, obtenemos $\Hom_{\Sch_\Z}(T,G_m)\cong(\OO_T(T))^\times$ puesto que $x$ tiene que mapearse a un elemento invertible. Se comprueba que $G_m$ induce el producto en el grupo de las unidades. 
\end{ej}


A lo que hemos llamado $G(T)$ es lo que usualmente denotábamos $h_G(T)=\Hom_{\Sch}(T,G)$, que en general era un functor $\Sch_S\to\Set$, pero al tener la imagen estructura de grupos, podemos factorizarlo como $\Sch_S\to\Grp\to\Set$ donde el último functor es el functor olvido. Para comprobar esto tenemos que comprobar que $h_G$ lleva morfismos de $S$-esquemas en homomorfismos de grupos, esto es, si $\phi:T'\to T$ es un morfismo de $S$-esquemas y $h_G(\phi)(f)=f\circ\phi:T'\to G$ para $f:T\to G$, entonces $h_G(\phi)(f_1\cdot f_2)=h_G(f_1)\cdot h_G(\phi)(f_2)$, o equivalentemente $(f_1\cdot f_2)\circ\phi=(f_1\circ\phi)\cdot (f_1\circ\phi)$.  Para ello usamos el diagrama 
\[
\begin{tikzcd}
 & T' \arrow[d, "\phi"] &  \\
 & T \arrow[rd, "f_2"] \arrow[d] \arrow[ld, "f_1"'] \arrow[d] &  \\
G & G\times_S G \arrow[l, "pr_1"'] \arrow[r, "pr_2"] \arrow[d, "\mu"] & G \\
 & G & 
\end{tikzcd}
\]
que por conmutatividad es el mismo que
\[
\begin{tikzcd}
 & T' \arrow[rd, "f_2\circ\phi"] \arrow[d] \arrow[ld, "f_1\circ\phi"'] \arrow[d] &  \\
G & G\times_S G \arrow[l, "pr_1"'] \arrow[r, "pr_2"] \arrow[d, "\mu"] & G \\
 & G & 
\end{tikzcd}
\]
de donde se deduce la igualdad.

\begin{ej}
Volvemos al ejemplo $\spec(\Z[x_1,x_2,x_3,x_4]/\gene{x_1x_4-x_2x_3-2})=\spec(B)$ y consideramos $T\to\spec(\Z)$ un esquema, entonces $\Hom_{\Sch}(T, \spec(B))\cong \Hom_{\Ring}(B,\OO_T(T))\cong SL_2(\O_T(T))$. Esto nos da $h_{\spec(B)}:\Sch\to\Grp$, aunque hay que ver que esta correspondencia es realmente un functor. Si $\phi:T'\to T$ es un morfismo de esquemas, esto nos da $\OO_T(T)\to\OO_{T'}(T')$. Esto nos daría un morfismo $SL_2(\OO_T(T))\to SL_2(\OO_{T'}(T'))$ y hay que ver que es justamente el morfismo inducido por la correspondencia.

\[
\begin{tikzcd}
\Hom_{\Sch}(T,\spec(B))\arrow[r, "h(\phi)"]\arrow[d, "\cong"] & \Hom_{\Sch}(T',\spec(B))\arrow[d, "\cong"]\\
\Hom(B,\OO_T(T))\arrow[r, "\phi(T)\circ-"]\arrow[d, "\cong"] & \Hom(B,\OO_{T'}(T'))\arrow[d,"\cong"]\\
SL_2(\OO_T(T))\arrow[r] & SL_2(\OO_{T'}(T'))
\end{tikzcd}
\]
Si tenemos una aplicación $x_1\mapsto a, x_2\mapsto b, x_3\mapsto c, x_4\mapsto d$ formando estos coeficientes una matriz de $SL_2(\OO_T(T))$, en $SL_2(\OO_{T'}(T'))$ tendríamos la aplicación $x_1\mapsto \phi(T)(a), x_2\mapsto\phi(T)(b),x_3\mapsto\phi(T)(c),x_4\mapsto\phi(T)(d)$, por lo que efectivamente es el morifsmo inducido. 


Veamos cómo funciona $\mu:\spec(B)\times_\Z\spec(B)\to\spec(B)$, que viene inducido por $\mu^*:B\to B\otimes_\Z B$. Usando el producto de matrices definimos $\mu^*(x_1)=x_1\otimes x_1+x_2\otimes x_3$, $\mu^*(x_2)=x_1\otimes x_2+x_2\otimes x_4$ y así sucesivamente (sustituir el producto habitual por el tensorial en cada entrada de la matriz). De igual modo, $\tau$ viene inducida por $\tau^*:B\to B$ que se define usando la fórmula para la matriz inversa. También $\varepsilon$ estaría inducida por $\varepsilon^*:B\to \Z$, que está definida por la aplicación que envía cualquier matriz a la matriz identidad, es decir, $x_1,x_3\mapsto 1$, $x_2,x_4\mapsto 0$. 

\section{Cambio de base}

Sea $S$ un esquema y $G$ un esquema en grupos sobre $S$. Sea $S'\to S$. Queremos dotar a $G\times_S S'\to S'$ de estructura de esquemas en grupos sobre $S'$. Queremos encontrar un functor $h_{G\times_S S'}:\Sch_{S'}\to\Set$ que factorice a través de $\Grp$. Sea $T\to S'$ un esquema. Se tiene $\Hom_{S'}(T, G\times_S S')\cong\Hom_S(T,G)$ ya que si $f':T\to G\times_S S'$, $f=pr_1\circ f'$, tenemos
\[
\begin{tikzcd}
&T\arrow[dl, "f"]\arrow[d, "f'"]&\\
G& G\times_S S'\arrow[l, "pr_1"]\arrow[r]& S'
\end{tikzcd}
\]
EL PRÓXIMO DÍA LO EXPLICA MEJOR, PARCE QUE COMO HOM PRESERVA COPRODUCTOS PODRÍA SALIR, PERO HAY LÍO CON LAS PRIMAS

\end{ej}























\end{document}