\documentclass[GA.tex]{subfiles}

\begin{document}


%\hyphenation{equi-va-len-cia}\hyphenation{pro-pie-dad}\hyphenation{res-pec-ti-va-men-te}\hyphenation{sub-es-pa-cio}

\chapter{Haces}
En inglés sheaf (sheaves), en francés faisceau (faisceaux).

\section{Introducción}

\begin{ej}
Sea $\R^n$ con la topología euclídea $\Tau$. Consideremos el anillo de funciones continuas en un abierto $U$ de $\R^n$ en $\R$, $C^0(U,\R)$. Recordemos que $C^0(\emptyset)$ tiene un elemento, que denotaremos como $0$, por el hecho de que los conjuntos de aplicaciones continuas forman un grupo abeliano. 

Si tenemos una inclusión de abiertos $V\subseteq U$, podemos considerar la restricción $C^0(U,\R)\overset{\rho_{UV}}{\longrightarrow} C^0(V,\R)$, definida como $\rho_{UV}(f)=f|_V$. Además, la restricción de la suma es la suma de restricciones, por lo que $\rho_{UV}$ es un homomorfismo de grupos abelianos. 

Si tenemos 3 abiertos $W\subseteq V\subseteq U$ se cumple $\rho_{UW}=\rho_{VW}\circ\rho_{UV}$, es decir, no importa restringir directamente o hacer una restricción intermedia. Otra propiedad trivial pero remarcable es que $\rho_{UU}=Id_U$. 


Si $U=\bigcup_{i\in I} U_i$ es un recubrimiento abierto de $U$ y $f_i\in C^0(U_i,\R)$ $\forall i\in I$ verificando que $f_i|_{U_i\cap U_j}=f_j|_{U_i\cap U_j}$ $\forall i,j\in I$, entonces existe una única función continua $f\in C^0(U,\R)$ tal que $f|_{U_i}=f_i$ $\forall i$. Esto es, si tenemos funciones en cada abierto que coinciden en la intersección, entonces podemos extenderlas a la unión. La continuidad de $f$ se obtiene del hecho de que la continuidad es una propiedad local. 

Todo esto vale también si consideramos $C^n(U,\R)\forall n\geq 0, C^\infty(U,\R), C^w(U,\R)$.
\end{ej}

\begin{ej}
Ahora vamos a considerar las funciones constantes $Cte(U,\R)\cong \R$. En este caso, lo explicado en el ejemplo anterior ya no es válido. Si consideramos un abierto donde la función vale 2 y otro disjunto en el que la función vale 3 (naturalmente su intersección es vacía), no existe una función constante que valga 2 en un abierto y 3 en otra. Esto muestra que ser constante no es una propiedad local. Si exigiéramos que fuera localmente constante entonces sí se verificaría.
\end{ej}

Estos ejemplos nos dan la noción de prehaz (que no verifica necesariamente la condición local) y de haz (que sí la verifica).

\section{Haces}

\begin{defi}
Sea $X$ un espacio topológico. Un \emph{prehaz de grupos abelianos} $\mathcal{F}$ en $X$ consiste en los siguientes datos:
\begin{enumerate}
\item Para cada abierto $U\subseteq X$ un grupo abeliano $\mathcal{F}(U)$ asociado
\item Para todo inclusión $V\subseteq U$ de abiertos de $X$, un homomorfismo de grupos abelianos $\rho_{UV}:\mathcal{F}(U)\to\mathcal{F}(V)$
\end{enumerate}
sujeto a las condiciones:
\begin{enumerate}
\item $\mathcal{F}(\emptyset)=0$
\item $\rho_{UU}=Id_{\mathcal{F}(U)}$
\item Si $W\subseteq V\subseteq U$ son tres abiertos, entonces $\rho_{UW}=\rho_{VW}\circ \rho_{UV}$.
\end{enumerate}

\end{defi}

\begin{ej}
Si $G$  es un grupo abeliano fijo, entonces $\mathcal{F}(U)=G$ si $U\neq\emptyset$, $\mathcal{F}(\emptyset)=\{0\}$ y $\rho_{UV}=Id_G$ si $V\neq \emptyset$ es un prehaz. En este caso podríamos interpretar $G$ como las funciones constantes de $U$ en $G$, en cuyo caso las restricciones son auténticas restricciones.
\end{ej}

Esta definición se puede reescribir en lenguaje de categorías. Dado un espacio topológico $X$, se define la categoría $\mathfrak{Top}(X)$ cuyos objetos son los abiertos de $X$ y donde los únicos morfismos son las inclusiones. Así, un prehaz de grupos abelianos no es más que que un functor contravariante entre la categoría $\mathfrak{Top}(X)$ y la categoría $\mathfrak{Ab}$ de grupos abelianos. De igual manera se puede definir un prehaz de anillos, de conjuntos o con valores en cualquier categoría fijada.

En adelante, si $\calF$ es un prehaz sobre $X$, nos referiremos a $\calF(U)$ como las \emph{secciones} del prehaz $\calF$ sobre el conjunto $U$ y algunas veces usaremos la notación $\Gamma(U,\calF)$ para denotar al grupo $\calF(U)$. Llamaremos a las aplicaciones $\rho_{UV}$ \emph{restricciones}, y a menudo escribiremos $s|_V$ en lugar de $\rho_{UV}(s)$ para $s\in\calF(U)$. 

\begin{defi}
Un prehaz en un espacio topológico $X$ es un \emph{haz} si satisface las siguientes condiciones:
\begin{enumerate}
\item Si $U$ es un abierto, si $\{V_i\}$ es un recubrimiento abierto de $U$, y si $s\in\calF(U)$ es un elemento tal que $s|_{V_i}=0$ para todo $i$, entonces $s=0$. 
\item Si $U$ es abierto, si $\{V_i\}$ es un recubrimiento abierto de $U$, y si tenemos elementos $s_i\in\calF(U)$ para cada $i$ con la propiedad de que para cada $i,j$, $s_i|_{U_i\cap U_j}=s_j|_{U_i\cap U_j}$, entonces existe un elemento $s\in\calF(U)$ tal que $s|_{V_i}=s_i$ para todo $i$ (obsérveseque la condición anterior implica que este $s$ es único). 
\end{enumerate}
\end{defi}



\begin{ej}
Sea $X$ un espacio topológico y $A$ un grupo abeliano. Definimos el \emph{haz constante} $\calA$ en $X$ determinado por $A$ como sigue. Dotamos a $A$ de la topología discreta, y para un abierto $U\subseteq X$, sea $\calA(U)$ el grupo de funciones continuas de $U$ en $A$ (que son localmente constantes). Entonces, con las restricciones de funciones usuales, obtenemos un haz. Nótese que para todo abierto conexo $U$, $\calA(U)\cong A$, de ahí que se llame constante. Si $U$ es un abierto cuyas componentes conexas sean abiertas (siempre en un espacio topológico localmente conexo), entonces $\calA(U)$ es isomorfo a una suma directa de tantas copias de $A$ como componentes conexas tenga $U$. 
\end{ej}

Sea $(I,\prec)$ un orden parcial dirigido, es decir $\forall i,j\in I$, $\exists k\in I$ tal que $i,j\prec k$ (todo par de elementos tiene una cota superior). En un espacio topológico $X$, dado $p\in X$, definimos el conjunto $I=\{U\subseteq X\mid p\in U\}$ dirigido por la relación de la inclusión.

Sea $(I,\prec)$ un conjunto dirigido, y para cada $i\in I$ asociamos un grupo abeliano $M_i$, de modo que para $i\prec j$ se tenga un homomorfismo $\rho_{ij}:M_i\to M_j$ verificando para $i\prec j\prec k$, $\rho_{ik}=\rho_{jk}\circ\rho_{ij}$. A esto lo llamaremos sistema inductivo de grupos abelianos. Por ejemplo, consideremos $(\N,\leq)$. Entonces un sistema inductivo de grupos abelianos sería simplemente una cadena
\[
M_1\to M_2\to M_3\to\cdots
\]

Si $(M_i)_{i\in I}$ es un sistema inductivo de grupos abelianos, su límite inductivo o directo, $\underset{i\in I}{\underset{\rightarrow}{\lim}}M_i$ es otro grupo $M$ junto con una sucesión de homomorfismos de grupso $\alpha_i:M_i\to M$ tales que:
\begin{enumerate}
\item El siguiente diagramma conmuta para todo $i,j$
\[
\begin{tikzcd}
M_i\arrow[r, "\alpha_i"]\arrow[d, "\rho_{ij}"'] & M\\
M_j\arrow[ur, "\alpha_j"'] &
\end{tikzcd}
\]
\item $M$ es ``lo más pequeño posible'', es deicr, para cualquier otro grupo abeliano $N$ que verifique la conmutatividad anterior para morfismos $\beta_i$ y $\beta_j$, entonces existe un único homomorfismo $\Gamma:M\to N$ a través del cual factorizan los anteriores, es decir, $\beta_i=\gamma\circ\alpha_i$. 
\end{enumerate} 

Estas definiciones son válidas para cualquier categoría, obviamente. Podemos ver un ejemplo en el caso de conjuntos. 

\begin{ej}
Sea $I=(\N, \leq)$ y sea $M_n=(-n,n)\subseteq\R$. Tenemos un sistema inductivo dado por las inclusiones obvias. El límite inductivo es claramente la unión de todos ellos, es decir $\R$. 
\end{ej}

Para probar que el límite directo existe en el caso de grupos abelianos, dados $(M_i,\rho_{ij})$, consideramos $\bigoplus_{k\in I}M_k$ cocientado por las relaciones que nos dan la conmutatividad del diagrama, $\sigma_i=\sigma_j\circ \rho_{ij}$. Esto nos da el límite directo.


\begin{defi}
\emph{Fibra (Stalk) de un prehaz}. Sea $X$ un espacio topológico, $\calF$ un prehaz de grupos abelianos en $X$, $p\in X$. Definimos el stalk de $\calF$ como
\[
\calF_p=\underset{p\in U}{\underset{\rightarrow}{\lim}} \calF(U).
\]
\end{defi}

Podemos pensar de otra forma el límite directo además de la que hemos definido antes. Consideremos $\{(U,s)\mid U$ es entorno abierto de $p$ y $s\in\calF(U)\}$. Definimos sobre este conjunto la relación de equivalencia $(U,s)\sim (V,t)$ si $\exists w\in U\cap V$ entorno abierto de $p$ tal que $s|_W=t|_W$. Este conjunto tiene además de estructura de grupo abeliano haciendo suma donde dos funciones estén definidas. 

\begin{ejs}
\begin{enumerate}
\item Un ejemplo sencillo es considerar funciones que vayan a $\R$. En ese caso las clases de equivalencia serían los gérmenes de funciones en $p$. 

\item Sea $X=\C$ y $\calF(U)$ el grupo de funciones holomorfas (o analíticas) $f:U\to\C$. Claramente las funciones de una clase tendrían el mismo valor en $p$, pero no solo eso, sino también la misma serie de Taylor porque también coincidirían sus derivadas. Por tanto, el límite directo aquí serían los posibles desarrollos de Taylor en $p$. En el caso analítico, por el principio de extensión analítica, las funciones de una clase deben ser iguales en la componente conexa de $p$ en la intersección de los dominios. 
\end{enumerate}
\end{ejs}


\begin{defi}
\emph{Morfismo de prehaces}. Sea $X$ un espacio topológico, $\calF$ y $\mathcal{G}$ prehaces de grupos abelianos. Entonces $\varphi:\calF\to\mathcal{G}$ es un morfismo de prehaces si para cada $U\subseteq X$, existe un homorfismo de grupos abelianos $\varphi(U):\calF(U)\to\mathcal{G}(U)$ que haga conmutar el diagrama siguiente, donde $V\subseteq U$:
\[
\begin{tikzcd}
\calF(U)\arrow[r, "\varphi(U)"]\arrow[d, "\rho_{UV}"'] & \calG(U)\arrow[d, "\rho'_{UV}"]\\
\calF(V)\arrow[r, "\varphi(V)"] & \calG(V)
\end{tikzcd}
\]
La misma definición es válida para morfismo de haces. Un isomorfismo es un morfismo con inversa a derecha e izquierda. 
\end{defi}

\begin{ej}
Sea $X=\R$ y el haz $\mathcal{E}=\{f:U\to \R\mid f\in C^{\infty}\}$. La derivada es un morfismo de este haz en sí mismo. Lo que la condición de ser morfismo representa es que da lo mismo derivar y restringuir que hacerlo en orden contrario.
\end{ej}


\begin{prop}
Sea $\varphi:\calF\to\calG$ un morfismo de haces en un espacio topológico $X$. Entonces $\varphi$ es un isomorfismo si y solo si la aplicación inducida en la fibra $\varphi_p:\calF_p\to\calG_p$ es un isomorfismo para todo $p\in X$.
\end{prop}

El morfismo inducido $\varphi_p:\calF_p\to \mathcal{G}_p$ viene definido por $\varphi_p(\overline{(U,s)})=\overline{(U, \varphi(U)(s))}$. Además se verifica $(\varphi\circ\psi)_p=\varphi_p\circ \psi_p$. 


\begin{defi}[Proposición]
\emph{Haz asociado a un prehaz}. Sea $X$ un espacio topológico y $\calF$ un prehaz de grupos abelianos a $X$. Existe un haz $\calF^+$ dotado de un morfismo de prehaces $\theta:\calF\to\calF^+$ con la siguiente propiedad universal:

Para todo haz $\mathcal{G}$ y para todo homomorfismo de prehaces $\varphi:\calF\to\mathcal{G}$ el siguiente diagrama conmuta
\[
\begin{tikzcd}
\calF\arrow[r,"\theta"] \arrow[dr, "\varphi"]& \calF^+\arrow[d, "\exists!\psi"]\\
& \mathcal{G}
\end{tikzcd}
\]

\end{defi}

Si $\calF$ ya es un haz, entonces claramente $\calF^+=\calF$. Se tiene además para todo $p\in X$, que $\theta_p:\calF_p\to\calF^+_p$ es un isomorfismo. 

\begin{defi}
\emph{Núcleo ($\ker$), conúcleo ($\coker$) e imagen}. Dado un morfismo de prehaces $\varphi:\calF\to \mathcal{G}$,  $(\ker\varphi)(U)=\ker\varphi(U)$. Se tiene que $\ker\varphi\subseteq \calF$. Además, si $\calF$ es un haz, su núcleo también lo es.  De la misma forma se definen el conúcleo y la imagen, pero estos no son en general haces. Por ello los denotamos $\Ima^0$ y $\coker^0$ hasta que podamos hacer una definición mejor.
\end{defi}

\begin{ej}\label{ej1}
$X=\C\setminus\{0\}$ y $\OO_X$ el haz de las funciones holomorfas en $X$. Consideramos el morfismo de haces $\frac{d}{dz}:\OO_X\to\OO_X$. El $\ker$ es el haz constante, es decir, el haz de las funciones localmente constantes. 

Consideremos ahora una bola abierta $U$ situada en el primer cuadrante y el haz anterior de $\OO_X(U)$ en sí mismo. Para cualquier función holomorfa $f$ en $U$ podemos encontrar una preimagen fijando $p\in U$ y haciendo una integral definida $g(z)=\int_p^zf(\chi)d\chi$. Esto implica que $\Ima^0\frac{d}{dz}(U)=\Ima(\frac{d}{dz}:\OO_X(U)\to\OO_X(U))=\OO_X(U)$. 

Si ahora tomáramos un abierto que rodee al origen (una corona circular), por ejemplo $\OO_X$, entonces ya no podemos hacer esto, pues hay funciones como $\frac{1}{z}$ que no tienen primitiva (habría que eliminar una semirrecta para poder elegir una rama del logaritmo). Sin embargo, podemos elegir subconjuntos contráctiles que cubran la corona, de modo que en todos ellos existiría primitiva, pero no se tiene en la unión. Este es un caso en el que la imagen no es un haz. 
\end{ej}

\begin{defi}
Un \emph{subhaz} de un haz $\calF$ es un haz $\calF'$ tal que para todo abierto $U\subseteq X$, $\calF'(U)$ es un subgrupo de $\calF(U)$, y las restricciones de $\calF'$ son las inducidas por las de $\calF$. 

Decimos que un morfismo de haces $\varphi:\calF\to\calG$ es inyectivo si $\ker\varphi=0$. Por tanto, $\varphi$ es inyectivo si y solo si la aplicación inducida $\varphi(U):\calF(U)\to\calG(U)$ es inyectiva para todo abierto de $X$. Esto es equivalente a que $\varphi_p:\calF_p\to \calG_p$ es inyectiva (la implicación directa se cumple para prehaces, la inversa requiere que sean haces). 

Definimos la imagen de un morfismo de haces $\Ima\varphi$ como el haz asociado al prehaz imagen, es decir, $(\Ima^0\varphi)^+$. Por la propiedad universal del haz asociado, existe una aplicación natural $\Ima\varphi\to\calG$ que es de hecho inyectiva, y por tanto podemos identificar la imagen con un subhaz de $\calG$. 

Decimos que un morfismo de haces es sobreyectivo si $\Ima\varphi=\calG$.

Decimos que una sucesión $\cdots\to\calF^{i-1}\xrightarrow{\varphi^{i-1}}\calF^i\xrightarrow{\varphi^i}\calF^{i+1}\to\cdots$ es exacta si para todo $i$ se tiene $\ker\varphi^i=\Ima\varphi^{i-1}$. Por tanto, una sucesión $0\to\calF\xrightarrow{\varphi}\calG$ es exacta si y solo si $\varphi$ es inyectivo y $\calF\xrightarrow{\varphi}\calG\to 0$ es exacta si y solo si $\varphi$ es sobreyectivo. 

Sea ahora $\calF'$ un subhaz de $\calF$. Definimos el \emph{haz cociente} $\calF/\calF'$ como el haz asociado al prehaz $U\mapsto \calF(U)/\calF'(U)$. Se sigue que para todo punto $p$, la fibra $(\calF/\calF')_p$ es el cociente $\calF_p/\calF'_p$.

Se define el haz $\coker\varphi$ como el haz asociado al prehaz conúcleo de $\varphi$, esto es $(\coker^0\varphi)^+$. 
\end{defi}

\begin{nota}
Hemos visto que un morfismo de haces es inyectivo si y solo si lo son las aplicaciones en las secciones para todo abierto. Esto no es válido para la sobrectividad, pero sí podemos afirmarlo para la aplicación en las fibras para todo punto. En general, una sucesión de haces y morfismos es exacta si y solo si lo es en las fibras. 
\end{nota}

\begin{ej}
Volviendo al ejemplo \ref{ej1}, la imagen es localmente el prehaz imagen, ya que las clases son gérmenes, así que tendríamos que $\Ima \frac{d}{dz}=\OO_X(U)$. 
\end{ej}


\subsection{Operaciones entre haces}
Dados dos (pre)haces $\calF$ y $\calG$, consideramos el conjunto $\Hom(\calF,\calG)$, que es un grupo abeliano con la suma $(\varphi+\psi)(U)=\varphi(U)+\psi(U)$. 

\begin{defi}
Sea $f:X\to Y$ una aplicación continua. Si $\calF$ es un prehaz en $X$, definimos la \emph{imagen directa} de $f$, $f_*\calF$, como $(f_*\calF)(V)=\calF(f^{-1}(U))$ para todo abierto $V\subseteq Y$. Con las restricciones usuales la imagen directa es un prehaz. Si $\calF$ es un haz, entonces $f_*\calF$ es un haz. 

Sea $\calG$ un prehaz en $Y$. Definimos el prehaz $f^{-1}_0(\calG)(U)$. Para ello tomamos todos los abiertos $V$ de $Y$ que contengan a $f(U)$ y consideramos el límite inductivo en estos abiertos $\underset{\rightarrow}{\lim}\calG(V)$ (si $f$ es abierta, sería simplemente $\calG(f(U))$). Las restricciones son las usuales. Definimos el haz \emph{imagen inversa} de $f$ como $f^{-1}\calG=(f^{-1}_0(\calG))^+$. 
\end{defi}

\begin{prop}
Se verifica $(f^{-1}\calG)_p\cong \calG_{f(p)}$ para todo $p\in X$. No así para la imagen directa.
\end{prop}

Tanto $f_*$ como $f^{-1}$ definen funtores, el primero entre la categoría $\mathfrak{Ab}(X)$ de haces sobre $X$ en $\mathfrak{Ab}(Y)$ y el segundo a la inversa. 


\begin{ej}
Tenemos una sucesión exacta de haces 
\[
0\to \C_X \hookrightarrow \OO_X\xrightarrow{\frac{d}{dz}}\OO_X\to 0
\]
donde $\C_X$ denota las funciones localmente constantes en $X$. Para $U=\C-\{0\}$ tenemos
\[
0\to \C \hookrightarrow \OO_X(\C-\{0\})\xrightarrow{\frac{d}{dz}}\OO_X(\C-\{0\})\to H^1(\C-\{0\})
\]
donde $H^1(\C-\{0\})$ es la cohomología de haces (Hartshorne, capítulo III) y coincide en este caso con la cohomología singular del espacio. En este caso es un espacio vectorial de dimensión 1 generado por $\frac{1}{z}$. 
\end{ej}


\subsection{Ejercicios}


\begin{ejer}
El prehaz núcleo de un morfismo haces es un haz.
\end{ejer}

\begin{ejer}
$\Ima\varphi=\ker (\calG\to \coker\varphi)$. 
\end{ejer}

\begin{ejer}
Si $\calF$ es un haz, entonces $f_*\calF$ es un haz. 
\end{ejer}

\end{document}

