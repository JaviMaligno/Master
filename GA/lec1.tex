\documentclass[GA.tex]{subfiles}

\begin{document}


%\hyphenation{equi-va-len-cia}\hyphenation{pro-pie-dad}\hyphenation{res-pec-ti-va-men-te}\hyphenation{sub-es-pa-cio}

\chapter{Haces}
En inglés sheaf (sheaves), en francés faisceau (faisceaux).

\section{Introducción}

\begin{ej}
Sea $\R^n$ con la topología euclídea $\Tau$. Consideremos el aniñño de funciones continuas en un abierto $U$ de $\R^n$ en $\R$, $C^0(U,\R)$. Recordemos que $C^0(\emptyset)$ tiene un elemento, que denotaremos como $0$, por el hecho de que los conjuntos de aplicaciones continuas forman un grupo abeliano. 

Si tenemos una inclusión de abiertos $V\subseteq U$, podemos considerar la restricción $C^0(U,\R)\overset{\rho_{UV}}{\longrightarrow} C^0(V,\R)$, definida como $\rho_{UV}(f)=f|_V$. Además, la restricción de la suma es la suma de restricciones, por lo que $\rho_{UV}$ es un homomorfismo de grupos abelianos. 

Si tenemos 3 abiertos $W\subseteq V\subseteq U$ se cumple $\rho_{UW}=\rho_{VW}\circ\rho_{UV}$, es decir, no importa restringir directamente o hacer una restricción intermedia. Otra propiedad trivial pero remarcable es que $\rho_{UU}=Id_U$. 


Si $U=\bigcup_{i\in I} U_i$ es un recubrimiento abierto de $U$ y $f_i\in C^0(U_i,\R)$ $\forall i\in I$ verificando que $f_i|_{U_i\cap U_j}=f_j|_{U_i\cap U_j}$ $\forall i,j\in I$, entonces existe una única función continua $f\in C^0(U,\R)$ tal que $f|_{U_i}=f_i$ $\forall i$. Esto es, si tenemos funciones en cada abierto que coinciden en la intersección, entonces podemos extenderlas a la unión. La continuidad de $f$ se obtiene del hecho de que la continuidad es una propiedad local. 

Todo esto vale también si consideramos $C^n(U,\R)\forall n\geq 0, C^\infty(U,\R), C^w(U,\R)$.
\end{ej}

\begin{ej}
Ahora vamos a considerar las funciones constantes $Cte(U,\R)\cong \R$. En este caso, lo explicaco en el ejemplo anterior ya no es válido. Si consideramos un abierto donde la función vale 2 y otro en el que la función vale 3 (naturalmente su intersección es vacía), no existe una función constante que valga 2 en un abierto y 3 en otra. Esto muestra que ser constante no es una propiedad local. Si exigiéramos que fuera localmente constante entonces sí se verificaría.
\end{ej}

Estos ejemplos nos dan la noción de prehaz (que no verifica necesariamente la condición local) y de haz (que sí la verifica).

\section{Haces}

\begin{defi}
Sea $X$ un espacio topológico. Un \emph{prehaz de grupos abelianos} $\mathcal{F}$ en $X$ consiste en los siguientes datos:
\begin{enumerate}
\item A cada abierto $U\subseteq X$ un grupo abeliano $\mathcal{F}(U)$ asociado
\item Si $V\subseteq U$, entonces se le asocia $\rho_{UV}:\mathcal{F}(U)\to\mathcal{F}(V)$ homomorfismo de grupos abelianos
\end{enumerate}
sujeto a las condiciones:
\begin{enumerate}
\item $\mathcal{F}(\emptyset)=0$
\item $\rho_{UU}=Id_{\mathcal{F}(U)}$
\item Si $W\subseteq V\subseteq U$, entonces $\rho_{UW}=\rho{VW}\circ \rho{UV}$.
\end{enumerate}

\end{defi}

\begin{ej}
Si $G$  es un grupo abeliano fijo, entonces $\mathcal{F}(U)=G$ si $U\neq\emptyset$, $\mathcal{F}(\emptyset)=\{0\}$ y $\rho_{UV}=Id_G$ si $V\neq \emptyset$ es un prehaz.
\end{ej}


HACER LOS PRIMEROS EJERCICIOS DE LA SECCIÓN






\end{document}
