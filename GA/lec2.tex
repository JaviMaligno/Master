\documentclass[GA.tex]{subfiles}

\begin{document}


%\hyphenation{equi-va-len-cia}\hyphenation{pro-pie-dad}\hyphenation{res-pec-ti-va-men-te}\hyphenation{sub-es-pa-cio}

\chapter{Esquemas}

\section{Espectro y espacios localmente anillados}

\begin{defi}
Sea $A$ un anillo no trivial, es decir, que tiene ideales primos. Definimos $\mathrm{Spec}(A)$ como el conjunto de los ideales primos de $A$. 
\end{defi}

Por ejemplo, para $A=\Z$, $\mathrm{Spec}(\Z)=\{\{0\}\}\cup\{\Z_p\mid p$ primo$\}$. En este caso vemos que $\{0\}\subseteq\Z_p$ para todo $p$. 

Veamos ahora como ejemplo el caso $A=\C[x]$. En este caso, los ideales primos no nulos son los generados por polinomios irreducibles. Como $\C$ es algebraicamennte cerrado, son los de la forma $\gene{x-\alpha}$ con $\alpha\in\C$, así que podemos asociar este espectro a la recta afín compleja. De nuevo $\{0\}$ es primo y está contenido en todos los demás. 

Nótese que el ideal trivial solo es primo en dominios de integridad.

Dotamos a $\mathrm{Spec}(A)$ de una topología. Dado un ideal $I\subseteq A$, definimos $V(I)=\{\p\in\spec(A)\mid I\subseteq\p\}$. Se puede probar que $V(\sum_i I_i)=\bigcap_i V(I_i)$ y $V(I\cap J)=V(I)\cup V(J)$, $V(\{0\})=\spec(A)$ y $V(A)=\emptyset$, con lo que podemos definir la topología de Zariski sobre $\spec(A)$, que tiene como cerrados los conjuntos de la forma $V(I)$. Además se verifica que si $I\subseteq J$, entonces $V(J)\subseteq V(I)$. 

Si $f:A\to B$ es un homomorfismo de anillos, se define el functor contravariante $f^*:\spec(B)\to\spec(A)$ como $f^*(\p)=f^{-1}(\p)$ (el ideal contraído también denotado $\p^c$). Se comprueba fácilmente que $f^*$ es continua para la topología de Zariski y que cumple las propiedades para ser un functor contravariante. 

Vamos a definir \emph{el haz estructural en $\spec(A)$.} Para ello daremos varios pasos previos.

Denotamos $X=\spec(A)$ por simplicidad. Sea $f\in A$. Definimos $X_f=X\setminus V(\gene{f})=\{\p\in X\mid f\notin\p\}$. Sea $U\subseteq X$ un abierto, que por definición será de la forma $X\setminus V(I)$. Dado $\p\in U$, por definición $I\not\subseteq\p$, luego existe $f\in I$ con $f\notin\p$, así que $\p\in X_f\subseteq U$. Esto demuestra que los conjuntos $X_f$ forman una base de abiertos para $\spec(A)$. 

Recordemos que para $\p\in\spec(A)$, $S=A\setminus\p$ es un conjunto multiplicativamente cerrado (para más referencias consultar Atiyah-MacDonald, capítulo 3). Otro ejemplo importante de conjunto multiplicativamente cerrado es, para $f\in A$, el conjunto de las potencias de $f$, $\{1,f,f^2,\dots\}$. Es conveniente recordar también la definición del anillo localizado $S^{-1}A$ como el conjunto de pares $(a,s)\in A\times S$ de modo que $(a_1,s_1)\sim (a_2,s_2)$ si existe $t\in S$ tal que $t(a_1s_2-a_2s_1)=0$. Hay un homomorfismo evidente $h:A\to S^{-1}A$ dado por $h(a)=\frac{a}{1}$, que no es necesariamente inyectivo, pero verifica la siguiente propiedad universal: para todo anillo $B$ y para todo $\varphi:A\to B$ homomorfismo de anillos tal que $\varphi(s)$ es invertible para todo $s\in S$, existe un único homomorfismo $\psi$ que hace conmutar el siguiente diagrama
\[
\begin{tikzcd}
A\arrow[r, "h"]\arrow[dr, "\varphi"'] & S^{-1}A\arrow[d, dashed, "\exists!\psi"]\\
& B
\end{tikzcd}
\]

Se denota $A_\p$ al localizado de $A$ en $\p$, definido como $S^{-1}A$ para $S=A\setminus\p$. Se cumple que el ideal $\p A_\p$ es el ideal extendido de $\p$ por $h$. Este ideal está formado por las fracciones de la forma $\frac{a}{s}$ con $a\in\p$ y $s\notin\p$. Además este es el único ideal maximal, lo que convierte a $A_\p$ en un anillo local. 

\begin{ej}
En el anillo $\Z$ y con el ideal $\p=\Z_p$, $\Z_\p=\{\frac{a}{s}\mid a\in\Z, p\not\mid s\}\subset\Q$. Se denota también $\Z_p$. Sea $p\Z_p=\gene{\frac{p}{1}}$. Sea $\frac{a}{s}\in\Z_p$ que no esté en el ideal. Entonces $s$ no es divisible por $p$, así que el inverso $\frac{s}{a}$ está bien definido. Entonces, todo elemento que no esté en el ideal es unidad, con lo que el ideal es maximal. 
\end{ej}

Ahora podemos definir el haz que pretendíamos, $$\OO_X(U):=\{s\in\prod_{\p\in U}A_\p\mid\forall\p\in U\exists a,f\in A: \p\in X_f\subseteq U\land s_\q=\frac{a}{f}\in A_\q \}.$$ Aquí $\prod$ denota el producto cartesiano y $s_\p$ denota los elementos de este producto cartesiano. La segunda propiedad de la definición nos garantizará que el haz está definido localmente y por lo tanto será realmente un haz. Concretamente, se están tomando fracciones que localmente tienen una expresión común. Se demuestra además que $\OO_X(U)$ es un subanillo de $\prod_{\p\in U}A_\p$. Las restricciones tienen la siguiente forma: dados $V\subseteq U$ podemos utilizar la proyección $\prod_{\p\in U}A_\p\to\prod_{\p\in V}A_\p$ (que es homomorfismo de anillos) para generar un homomorfismo de anillos $\OO_X(U)\to\OO_X(V)$. 

\begin{nota}
Si $X$ es un espacio topológico y tenemos definido para cada $x\in X$ un anillo o grupo abeliano (etc) $A_x$, se puede definir el prehaz $\calF(U)=\prod_{x\in U} A_x$ con las proyecciones como restricciones (que son todas sobreyectivas). Este prehaz es un haz. Obsérvese que la topología en particular que tenga $X$ no juega un papel relevante, así que se puede tomar la topología discreta. Una vez que tenemos definido este haz para la topología discreta podemos llevarlo a cualquier topología. Tenemos que la identidad $X^{dis}\to X$ es continua, donde $X^{dis}$ es $X$ con la topología discreta. Entonces, $Id_*\calF(U)=\calF(U)$ por definición. 
\end{nota}

Dado un anillo $A$, hemos construido un espacio topológico $\spec(A)$ al que le hemos asociado un haz de anillos $\OO_{\spec(A)}$. Al par $(\spec{A}, \OO_{\spec(A)})$ se le llama \emph{espectro} de $A$. 


\begin{ej}
Vimos que $X=\spec(\Z)$ se puede identificar con el conjunto de primos naturales unión el 0. Los cerrados de la topología de Zariski son, además del total (obtenido como $V(0)$) y del vacío, los de la forma $V(n\Z)$. Los ideales que contienen a $n\Z$ son los ideales generados por los divisores de $n$, luego los cerrados propios son los conjuntos finitos de primos. Como hemos visto, $\overline{0}=\spec(\Z)$. A esto lo llamaremos punto genérico. Los abiertos no triviales serían entonces el 0 y los conjuntos de primos con complementario finito. Denotamos $U_{p_1,\dots, p_r}$ al abierto que tiene como complementario el conjuto de primos indicado.  $U_\emptyset=X$. 

Vamos a calcuclar $\OO_X(U_{p})$. Tenemos que calcular $\Z_{\gene{p}}=\{\frac{a}{s}\mid a\in\Z, p\not| s\}$ y $\Z_{\gene{0}}=\Q$.  $U_{p}=X\setminus V(p\Z)=\{0\}\cup\{p_i\mid p\neq p_i\}$. Entonces $\OO_X(U_p)=\{(s_q)_{q\in U_p}\mid s_q\in\Z_{\gene{p}}\}=\Z_p=\{\frac{a}{p^r}\mid a\in\Z\}$ (esto habría que demostrarlo). 

Si cogemos un abierto genérico $U=U_{p_1,\dots, p_r}$, el resultado es $\OO_X(U)=\Z_{p_1\cdots p_r}$. En el caso del total, no ponemos ningún primo en el denominador, así que saldría simplemente $\Z$. La restricción $\OO_X(X)=\Z\to \OO_X(U)=\Z_{p_1\cdots p_r}$ es simplemente el homomorfismo usual de un anillo en el localizado. 
\end{ej}

\begin{ej}
Sea $k$ un cuerpo algebraicamente cerrado. $\spec(k[x])=k\cup\{\gene{0}\}$. La topología de zariski es análoga a la del anterior ejemplo, los abiertos no triviales son subconjuntos de $k$ con complementario finito en $k\cup\{0\}$. Si se hace el cálculo, se obtiene $\OO_{\spec(k[x)]}(k[x])=k[x]$. Si se toma un abierto que no sea el total, $\OO_{\spec(k[x])}(\{\alpha_1,\dots,\alpha_r\}^c)=k[x]_{(x-\alpha_1)\cdots(x-\alpha_r)}$. 
\end{ej}

\begin{prop}
Sea $A$ un anillo y $(\spec(A),\OO)$ su espectro.
\begin{enumerate}[(a)]
\item Para todo $\p\in\spec(A)$, la fibra $\OO_\p$ es isomorfa al anillo local $A_\p$.
\item Para todo elemento $f\in A$, si denotamos $X_f=\spec(A)\setminus V(\gene{f})$, entonces $\OO(X_f)$ es isomorfo al anillo localizado $A_f$.
\item En particular, $\OO(\spec(A))\cong A$. 
\end{enumerate} 
\end{prop}

\section{Esquemas}

En geometría diferencial se suelen definir las variedades diferenciables a partir de cartas en un atlas. Se pueden definir también de otro modo más próximo al que nos interesa en nuestro estudio. Dado $U\subseteq\R^n$ abierto, definimos el haz $E_U$ de funciones $C^{\infty}$, de modo que $\mathcal{E}_U(V)$ son las funciones $f:V\to\R$ de clase $C^{\infty}$ para $V\subseteq U$.

 Así que podríamos definir variedad diferenciable como un espacio topológico expresado como unión de abiertos $X=\cup X_i$ y un haz $E_X$ de modo que $(X_i,\mathcal{E}_X|_{X_i})\cong (U_i,\mathcal{E}_{U_i})$ con $U_i\subseteq\R^n$. Este isomorfismo es en el sentido de que hay un homeomorfismo entre los espacios topológicos y un isomorfismo de haces en el segundo elemento del par. Este enfoque será el que usemos para definir los esquemas. 
 
 
 \begin{defi}
 Un \emph{espacio anillado} es un par $(X,\calA)$ donde $X$ es un espacio topológico y $\calA$ es un haz de anillos (comutativos con unidad) en $X$.
 \end{defi}
 
 \begin{ejs}\
 \begin{enumerate}
 \item Sea $A$ un anillo fijo y $\calA$ el haz de todas las funciones con valor en $A$. 
 \item Dado un anillo topológico $A$ (por ejemplo $\R$ o $\C$), definimos $\calA$ como el haz de las funciones continuas con valores en $A$.
 \item Sea $X$ una variedad diferenciable (real) y $\calA$ el haz de las funciones $C^{\infty}$ con valores en $\R$.
 \item $(\spec(A), \OO_{\spec(A)})$. 
 \end{enumerate}
 
 \end{ejs}
 
 \begin{defi}
 Sean $(X,\calA)$, $(Y,\mathcal{B})$ espacios anillados. Un morfimo entre ellos es un par $(f, f^{\sharp})$ donde $f:X\to Y$ es continua y $f^{\sharp}:\mathcal{B}\to f_*\calA$ es un morfismo de haces de anillos en $Y$. 
 
\end{defi} 


\begin{ej}
Sean $X,Y$ variedades diferenciables y $f:X\to Y$ una aplicación diferenciable. Consideramos el haz de las funciones diferenciables $\mathcal{E}_X$ tal que $\mathcal{E}(U)$ es el anillo de funciones diferenciables de $U$ en $\R$, y hacemos lo mismo para $Y$. Recordemos que $(f_*\mathcal{E}_X)(V)=\mathcal{E}_X(f^{-1}(V))$. Por otro lado tenemos $\mathcal{E}_Y(V)$, pero dada $\varphi\in \mathcal{E}_Y(V)$, podemos asociarle $\varphi\circ f|_{f^{-1}(V)}\in\mathcal{E}_X(f^{-1}(V))$. Esta es la definición de $f^{\sharp}$. 
\end{ej}

Si tenemos tres haces $(X,\calA)\xrightarrow{(f,f^{\sharp})}(Y,\mathcal{B})\xrightarrow{(g,g^{\sharp})}(Z,\mathcal{C})$, tendríamos una composición $(g,g^{\sharp})\circ(f, f^{\sharp})$. La primera coordenada de la composición es simplemente $g\circ f$, pero en la segunda tenemos que hacerlo con más cuidado. Tenemos $f^{\sharp}:\mathcal{B}\to f_*\calA$, $g^{\sharp}:\mathcal{C}\to g_*\mathcal{B}$. Podemos considerar $g_*(f^{\sharp}):g_*\mathcal{B}\to g_*(f_*\mathcal{A})=(g\circ f)_*(\calA)$. Así que la segunda componente de la composición es $g_*(f^\sharp)\circ g^{\sharp}$. 



\begin{defi}
Un espacio anillado $(X,\calA)$ se dice \emph{localmente anillado} si para todo $x\in X$, la fibra $\calA_x$ es un anillo local.
\end{defi}


\begin{ej}
Sea $(X,\mathcal{E}_X)$ una variedad diferenciable, $x\in X$. Entonces $\mathcal{E}_{X.x}$ consideramos los gérmenes que se anulan en el punto $x$, que forman claramente un ideal. Si un germen no se anula en $x$, entonces tampoco se anula en un entorno, por lo que es invertible multiplicativamente, o sea, es una unidad en el anillo. Así que las variedades diferenciables son espacios localmente anillados.

Tenemos además un homomorfismo de anillos $v_x:\mathcal{E}_{X,x}\to \R$, que es la evaluación del gérmen en $x$. Se tiene que $\ker v_x=m$, ideal maximal por el primer teorema de isomorfía (el homomorfismo es sobreyectivo porque se puede tomar el gérmen constante). Este ideal es justamente el que hemos mencionado antes. Se demuestra que $m/m^2$ tiene la misma dimensión que la variedad diferenciable, y es lo que se llama espacio cotangente (el dual de él es el tangente). 
\end{ej}

\begin{defi}
Si $(A, \m)$ y $(B,\mathfrak{n})$ son anillos locales. Un homomorfismo $\varphi:A\to B$ se dice \emph{local} si $\varphi^{-1}(\mathfrak{n})=\m$. 
\end{defi}

\begin{defi}
Sean $(X,\calA), (Y,\mathcal{B})$ espacios localmente anillados. Diremos que un morfismo $(f,f^{\sharp}):(X,\calA)\to (Y,\mathcal{B})$ de espacios anillados es local cuando para cada punto $x\in X$, $f^{\sharp}_x:\mathcal{B}_{f(x)}\to\calA_x$ sea un homomorfismo local. 
\end{defi}

\begin{defi}
Un \emph{esquema afín} es un espacio localmente anillado $(X,\OO_X)$ que es isomorfo (como espacio localmente anillado) al espectro de algún anillo. Un \emph{esquema} es un espacio localmente anillado $(X,\OO_X)$ tal que $X$ se puede recubrir por abiertos $\{U_i\}_{i\in I}$ tales que $(U_i,\OO_X|_{U_i})$ es un esquema afín. 
\end{defi}


Vamos a ver que la correspondencia $A\to\spec(A)$ es functorial, y que además este functor es plenamente fiel. De hecho es equivalencia de categorías si nos restringimos a los esquemas afines, que son la imagen de este functor. 
\begin{prop}
\begin{enumerate}[(a)]
\item Si $A$ es un anillo, $(\spec(A), \OO)$ es un espacio localmente anillado.
\item Si $\varphi:A\to B$ es un homomorfismo de anillos, $\varphi$ induce un morfismo natural de espacios localmente anillados
\[
\varphi^*=(f,f^\sharp):\spec(B)\to\spec(A)
\]
\item Si $A$ y $B$ son anillos, entonces cualquier morfismo de espacios localmente anillados $\spec(B)\to\spec(B)$ está inducido por un homomorfismo de anillos $A\to B$. 
\end{enumerate}
\end{prop}
\begin{dem}
Vamos a describir el morfismo $\varphi^*$. 

La aplicación $f:\spec(B)\to\spec(B)$ está definidad como $f(\p)=\varphi^{-1}(\p)$. Tenemos que probar que es continua, para lo cual, probaremos que $f^{-1}(V(I))=V(I^e)$, donde $I^e$ denota el ideal extendido. Si $\p\in f^{-1}(V(I))\Rightarrow f(\p)\in V(I)\Rightarrow \varphi^{-1}(\p)\supseteq I\Rightarrow \varphi^{-1}(\p)^e\supseteq I^e$. Como además $\p\in\varphi^{-1}(\p)^e$ (aunque no fuera primo), $\p\in V(I^e)$. Si ahora $\p\in V(I^e)$, $I^e\subseteq\p\Rightarrow I\subseteq \varphi^{-1}(I^e)\subseteq \varphi^{-1}(\p)$, luego $f(\p)=\varphi^{-1}(\p)\in V(I)$. Probar que $f$ es functorial para la composición es muy sencillo.

Definimos ahora $f^\sharp:\OO_{\spec(A)}\to f_*\OO_{\spec(B)}$. Consideramos la fibra $\OO_{\spec(A),\p}\cong A_{\p}$. Recordamos el ejercicio 1.8, que dice que $\Hom_X(f^{-1}\calF,\calG)\cong\Hom_Y(\calF, f_*\calG)$. Por esta propiedad podemos definir con más comodidad un morfismo $\tilde{f}^\sharp:f^{-1}\OO_{\spec(A)}\to \OO_{\spec(B)}$. Ahora las fibras que cogemos son, dado $\q\in\spec(B)$, $(f^{-1}\OO_{\spec(A)})_{\q}=\OO_{\spec(A), f(\q)}=\OO_{\spec(A), \varphi^{-1}(\q)}=A_{\varphi^{-1}(\q)}$. Por otro lado, $\OO_{\spec(B),\q}=B_{\q}$, por lo que basta definir un homomorfismo $A_{\varphi^{-1}(\q)}\to B_{\q}$ a partir de $\varphi:A\to B$. Por la propiedad universal de los anillos de fracciones, basta tomar la composición $A\to B\to B_{\q}$, tal que $a\in A\setminus\varphi^{-1}(\q)\mapsto\varphi(a)\in B\setminus\q\mapsto \frac{\varphi(a)}{1}$, que es una unidad. La propiedad universal nos que esta aplicación es única, y al precomponer con $A\to A_{\varphi^{-1}(\q)}$ el diagrama resultante es conmutativo. 
\[
\begin{tikzcd}
A\arrow[r,"\varphi"] \arrow[dr]\arrow[d] & B\arrow[d]\\
A_{\varphi^{-1}(\q)}\arrow[r, dashed, "\exists!"] & B_{\q}
\end{tikzcd}
\]
Faltaría probar que efectivamente el resultado es un morfimo de haces, lo cual se deja como ejercicio para el lector. 


\end{dem}

\begin{ej}[Fancy example of category equivalence]
Construimos la categoría $\CC'$ cuyos objetos son los números naturales y las flechas de $n\to m$ como las matrices $n\times m$ con coeficientes en $k$ (entendemos la matrices $n\times 0$ y $0\times m$ como un espacio de un punto. Hay una equivalencia de categorías evidente entre esta categoría y la de los $k$-espacios vectoriales de dimensión finito $\mathrm{Vec}_k$. Esto es, tenemos un functor $F:\CC\to\CC'$ que es esencialmente sobreyectivo: para todo $X\in Ob(\CC)$ existe $X'\in Ob(\CC)$ con $X\cong F(x')$; y además tiene que ser completamente fiel (fully faithful): esto es, que la aplicación $F_{X,Y}: \Hom(X,Y)\to \Hom(FX,FY)$ entre los conjuntos de homomorfismos es biyectiva para cada par de objetos $X$ e $Y$. Si el functor fuera contravariante, se ajustaría de manera evidente la definición.

También se puede definir en términos de la existencia un functor ``inverso''. 
\end{ej}

\begin{ejs}[Ejemplos de $\spec$]\
\begin{enumerate}
\item Sea $k$ un cuerpo.  $\spec(k)=(\{\{0\}\}, k)$. 
\item \emph{Anillos de valuación discreta (DVR)}.  Antes de dar la definición damos un ejemplo. Consideramos $\Z_p$, de modo que $\spec(\Z_p)=\{\{0\}, p\Z_p\}$, con $p\Z_p$ como único punto cerrado. Un anillo de valoración discreto es un dominio de integridad local que es dominio de ideales principales. $\Z_p\setminus\{0\}=\{\frac{m}{n}\mid m,n\in\Z, p\not| n\}$. Definimos $v_p:\Z_p\setminus\{0\}\to\N$ como $v_p(\frac{m}{n})=\max\{e\geq 0\mid p^e|m\}$, que es lo que se llama una valuación discreta. En todo anillo de evaluación discreta $A$, $\spec(A)$ es homeomorfo al caso $\Z_p$, es decir, de la forma $\{\{0\},\m\}$. En este caso
\[
\OO_{\spec(A)}(U)=\begin{cases}
\{0\} & U=\emptyset\\
Q(A) & U=\{\{0\}\}\\
A & U=\spec(A)
\end{cases}
\]
donde $Q(A)$ es el cuerpo de fracciones. 
\end{enumerate}
\end{ejs}

\begin{ej}
Para el cuerpo $\R$, se puede definir el espacio afín $\R^n$ con la topología de Zariski. Para $n=1$, el anillo de funciones regulares sería $k[x]$. Desde el punto de vista de teoría de esquemas, se define el espacio afín de dimensión uno como $\A_{\R}^1=\spec(\R[x])$, que vamos a ver que contiene mucha más información. Para comprobarlo, vemos los ideales primos de $\R[x]$, que son $\{0\}$ (punto denso), los ideales de la forma $\gene{x-\alpha}$ (que son los que están en correspondencia biunívoca con $\R$), y todos los demás ideales generados por un polinomio mónico irreducible (que son de la forma $x^2-2\alpha x+(\alpha^2+\beta^2)$ con $\alpha,\beta\in\R$, $\beta\neq 0$, es decir, que tenga como raíces $\alpha\pm\beta i$). Como conjunto, tenemos un punto especial, el conjunto $\R$ y los números complejos cocientados por la relación del conjugado. Los abiertos serían simplemente el complementario de un conjunto finito de puntos elegidos de $\R$ y el conjunto cociente de $\C$. 
\end{ej}

\begin{ej}[Pegamiento de esquemas]
Para pegar esquemas primero tenemos que saber cómo se pegan haces (se puede ver también en el ejercicio 1.22). Sea $X$ un espacio topológico y $\bigcup_{i\in I}U_i$ un recubrimiento abierto. Sea $\calF_i$ un haz en $U_i$ con las siguientes propiedades:
\begin{enumerate}
 \item
  Para cada para $(i,j)$, existe un isomorfismo $\varphi_{ij}:\calF_i|_{U_i\cap U_j}\to\calF_j|_{U_i\cap U_j}$. 
 \item $\varphi_{ii}=Id_{\calF_i}$.
 \item Denotando $U_{ijk}=U_i\cap U_j\cap U_k$, $\varphi_{jk}|_{U_{ijk}}\circ\varphi_{ij}|_{U_{ijk}}=\varphi_{ik}|_{U_{ikj}}$
 \end{enumerate}
 Entonces existe un haz $\calF$ en $X$ y existe un isomorfismo $\psi_i:\calF|_{U_i}\to\calF_i$ tal que conmuta
 \[
 \begin{tikzcd}
 \calF|_{U_{ij}}\arrow[r, "\psi_i|_{U_{ij}}"]\arrow[d, "\psi_j|_{U_{ij}}"'] &\calF_i|_{U_{ij}}\arrow[dl, "\varphi_{ij}"] \\
 \calF_j|_{U_{ij}} &
 \end{tikzcd}
 \]
 Este par es único salvo isomorfismo único.
 
 Vamos ahora al pegamiento de esquemas. Sean $X_1, X_2$ esquemas. Sean $U_i\subseteq X_i$ abiertos y sea $\varphi:(U_1,\OO_{X_1}|_{U_1})\to(U_2,\OO_{X_2}|_{U_2})$ un isomorfismo. 
 
\begin{lemma}
Si $X=\spec(A)$, todo abierto de $X$ es un esquema.
\end{lemma}
\begin{proof}
Sea $U\subseteq X$ abiertos. Dado $f\in A$, sabemos que los abiertos $X_f$ son una base de abiertos, luego $U=\cup X_{f_i}$. Tenemos $(U,\OO_X|_U)$, que a su vez restringe a $(X_{f_i}, \OO_X|_{X_{f_i}})$, que se puede ver que es isomorfo a $\spec(A_{f_i})$ usando resultados de álgebra conmutativa y la construcción hecha hasta ahora. Gracias a esto $\varphi$ es un isomorfismo de esquemas. Entonces existe un esquema $\widetilde{X}=\widetilde{X}_1\cup\widetilde{X}_2$, donde los subespacios escritos son abiertos del total, tal que $(\widetilde{X}_i,\OO_{\widetilde{X}_i}\cong(X_i,\OO_{X_i})$ mediante $\psi_i$. Este espacio cumple que $\psi_1^{-1}(U_1)=\psi^{-1}_2(U_2)$ en $X$ y además $\varphi\circ\psi_1|_V=\psi_2|_V$. Este espacio es el espacio de pegamiento usual en topología. 
\end{proof}
\end{ej}

\begin{ej}
$\spec(k[x]/\gene{x^2})=\{\gene{\overline{x}}\}$, ya que los primos de este anillo son los primos de $k[x]$ que contienen a $\gene{x^2}$, que solo hay uno.  Como espacio anillado tiene el haz que a cualquier abierto no vacío le asigna $k[x]/\gene{x^2}$, lo cual se puede ver notando que este anillo es de la forma $k[\varepsilon]=\{a+b\varepsilon\mid a,b\in k, \varepsilon^2=0\}$.
\end{ej}

\begin{ej}
Sea $(X,\mathcal{E})$ una variedad diferenciable y $\alpha:\spec(\R)=(\{*\},\R)\to (X,\mathcal{E})$ morfismo de espacios localmente anillados sobre $\R$ (es decir, de $\R$-álgebras). Necesariamente $\alpha(*)=p\in X$. $\alpha^{\sharp}:\mathcal{E}\to\alpha_*\R$, donde $\alpha_*\R(U)=\R(\alpha^{-1}(U))=0$ si $p\notin U$ y $\R$ en caso contrario. Para $p\in U$ tenemos
\[
\begin{tikzcd}
\mathcal{E}(U)\arrow[r, "\alpha^{\sharp}(U)"]\arrow[r, "\alpha^{\sharp}(U)"]\arrow[d]& \R\arrow[d, equals]\\
\mathcal{E}_p\arrow[r, "\alpha^{\sharp}_p"]\arrow[r] & \R
\end{tikzcd}
\]
Dado $\zeta\in \mathcal{E}(U)$ y su germen $\zeta(p)=c$, $\zeta=(\zeta-c)+c$, luego $\zeta_p=(\zeta_p-c)+c$, estando el término entre paréntesis en el ideal maximal $\m_p$ (los gérmenes que se anulan en $p$), luego $\alpha^{\sharp}(\zeta_p)=c$ por ser homomorfismo de $\R$-álgebras. 
\end{ej}

\begin{ej}
Siguiendo el ejemplo anterior, tomamos ahora $\alpha:\spec(\R[\varepsilon])\to (X,\mathcal{E})$. De nuevo $\alpha:\{*\}\to X$ tiene imagen $p\in X$ y $\alpha_*(\R[\varepsilon])$ es $\R[\varepsilon]$ si $p\in U$ y 0 en otro caso. También $\alpha^{\sharp}(U)$ será un homomorfismo de $\R$-álgebras local. 
\[
\begin{tikzcd}
\mathcal{E}(U)\arrow[r] \arrow[d] &\R[\varepsilon]\\
\R[\varepsilon]\arrow[ur, "\alpha^{\sharp}_p=\beta"]&
\end{tikzcd}
\]
Como $\beta$ es local, $\beta(\m_p)\subseteq\gene{\varepsilon}$ y $\beta(\m^2_p)\subseteq\gene{\varepsilon}^2=\{0\}$. Por otra parte tenemos la evaluación $ev_p:\mathcal{E}\to \R$ y $\pi:\R[\varepsilon]\to \R=\R[\varepsilon]/\gene{\varepsilon}$. Así que $\beta(\m_p)\subseteq\ker\pi$. Se tiene en realidad que $\ker\pi=\R\varepsilon$, puesto que $\R[\varepsilon]=\R\oplus\R\varepsilon$. Podemos definir entonces $\overline{\beta}:\m_p/\m^2_p\to\R\varepsilon$ que es $\R$ lineal, es decir, es un vector tangente. 
\end{ej}

\section{Ejercicios}
\begin{ejer}
Probar que el haz $\OO_X(U)$ es realmente un haz, describiendo las restricciones. 
\end{ejer}
\begin{ejer}
Probar que el prehaz $\calF(U)=\prod_{x\in U} A_x$ es un haz para $X$ con la topología discreta. 
\end{ejer}
\end{document}

