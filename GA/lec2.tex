\documentclass[GA.tex]{subfiles}

\begin{document}


%\hyphenation{equi-va-len-cia}\hyphenation{pro-pie-dad}\hyphenation{res-pec-ti-va-men-te}\hyphenation{sub-es-pa-cio}

\chapter{Esquemas}

\section{Espectro y espacios localmente anillados}

\begin{defi}
Sea $A$ un anillo no trivial, es decir, que tiene ideales primos. Definimos $\mathrm{Spec}(A)$ como el conjunto de los ideales primos de $A$. 
\end{defi}

Por ejemplo, para $A=\Z$, $\mathrm{Spec}(\Z)=\{\{0\}\}\cup\{\Z_p\mid p$ primo$\}$. En este caso vemos que $\{0\}\subseteq\Z_p$ para todo $p$. 

Veamos ahora como ejemplo el caso $A=\C[x]$. En este caso, los ideales primos no nulos son los generados por polinomios irreducibles. Como $\C$ es algebraicamennte cerrado, son los de la forma $\gene{x-\alpha}$ con $\alpha\in\C$, así que podemos asociar este espectro a la recta afín compleja. De nuevo $\{0\}$ es primo y está contenido en todos los demás. 

Nótese que el ideal trivial solo es primo en dominios de integridad.

Dotamos a $\mathrm{Spec}(A)$ de una topología. Dado un ideal $I\subseteq A$, definimos $V(I)=\{\p\in\spec(A)\mid I\subseteq\p\}$. Se puede probar que $V(\sum_i I_i)=\bigcap_i V(I_i)$ y $V(I\cap J)=V(I)\cup V(J)$, $V(\{0\})=\spec(A)$ y $V(A)=\emptyset$, con lo que podemos definir la topología de Zariski sobre $\spec(A)$, que tiene como cerrados los conjuntos de la forma $V(I)$. Además se verifica que si $I\subseteq J$, entonces $V(J)\subseteq V(I)$. 

Si $f:A\to B$ es un homomorfismo de anillos, se define el funtor contravariante $f^*:\spec(B)\to\spec(A)$ como $f^*(\p)=f^{-1}(\p)$ (el ideal contraído también denotado $\p^c$). Se comprueba fácilmente que $f^*$ es continua para la topología de Zariski y que cumple las propiedades para ser un funtor contravariante. 

Vamos a definir \emph{el haz estructural en $\spec(A)$.} Para ello daremos varios pasos previos.

Denotamos $X=\spec(A)$ por simplicidad. Sea $f\in A$. Definismo $X_f=X\setminus V(\gene{f})=\{\p\in X\mid f\notin\p\}$. Sea $U\subseteq X$ un abierto, que por definición será de la forma $X\setminus V(I)$. Dado $\p\in U$, por definición $I\not\subseteq\p$, luego existe $f\in I$ con $f\notin\p$, así que $\p\in X_f\subseteq U$. Esto demuestra que los conjuntos $X_f$ forman una base de abiertos para $\spec(A)$. 

Recordemos que para $\p\in\spec(A)$, $S=A\setminus\p$ es un conjunto multiplicativamente cerrado (para más referencias consultar Atiyah-MacDonald, capítulo 3). Otro ejemplo importante de conjunto multiplicativamente cerrado es, para $f\in A$, el conjunto de las potencias de $f$, $\{1,f,f^2,\dots\}$. Es conveniente recordar también la definición del anillo localizado $S^{-1}A$ ESCRIBIR LA DEFINICIÓN. Hay un homomorfismo evidente $h:A\to S^{-1}A$ dado por $h(a)=\frac{a}{1}$, que no es necesariamente inyectivo, pero verifica la siguiente propiedad universal: para todo anillo $B$ y para todo $\varphi:A\to B$ homomorfismo de anillos tal que $\varphi(s)$ es invertible para todo $s\in S$, existe un único homomorfismo $\psi$ que hace conmutar el siguiente diagrama
\[
\begin{tikzcd}
A\arrow[r, "h"]\arrow[dr, "\varphi"'] & S^{-1}A\arrow[d, dashed, "\exists!\psi"]\\
& B
\end{tikzcd}
\]

Se denota $A_\p$ al localizado de $A$ en $\p$, definido como $S^{-1}A$ para $S=A\setminus\p$. Se cumple que el ideal $\p A_\p$ es el ideal extendido de $\p$ por $h$. Este ideal está formado por las fracciones de la forma $\frac{a}{s}$ con $a\in\p$ y $s\notin\p$. Además este es el único ideal maximal, lo que convierte a $A_\p$ en un anillo local. 

\begin{ej}
En el anillo $\Z$ y con el ideal $\p=\Z_p$, $\Z_\p=\{\frac{a}{s}\mid a\in\Z, p\not\mid s\}\subset\Q$. Se denota también $\Z_p$. Sea $p\Z_p=\gene{\frac{p}{1}}$. Sea $\frac{a}{s}\in\Z_p$ que no esté en el ideal. Entonces $s$ no es divisible por $p$, así que el inverso $\frac{s}{a}$ está bien definido. Entonces, todo elemento que no esté en el ideal es unidad, con lo que el ideal es maximal. 
\end{ej}

Ahora podemos definir el haz que pretendíamos, $$\OO_X(U):=\{s\in\prod_{\p\in U}A_\p\mid\forall\p\in U\exists a,f\in A: \p\in X_f\subseteq U\land s_\q=\frac{a}{f}\in A_\q \}.$$ Aquí $\prod$ denota el producto cartesiano y $s_\p$ denota los elementos de este producto cartesiano. La segunda propiedad de la definición nos garantizará que el haz está definido localmente y por lo tanto será realmente un haz. Concretamente, se están tomando fracciones que localmente tienen una expresión común. Se demuestra además que $\OO_X(U)$ es un subanillo de $\prod_{\p\in U}A_\p$. Las restricciones tienen la siguiente forma: dados $V\subseteq U$ podemos utilizar la proyección $\prod_{\p\in U}A_\p\to\prod_{\p\in V}A_\p$ (que es homomorfismo de anillos) para generar un homomorfismo de anillos $\OO_X(U)\to\OO_X(V)$. 

\begin{nota}
Si $X$ es un espacio topológico y tenemos definido para cada $x\in X$ un anillo o grupo abeliano (etc) $A_x$, se puede definir el prehaz $\calF(U)=\prod_{x\in U} A_x$ con las proyecciones como restricciones (que son todas sobreyectivas). Este prehaz es un haz. Obsérvese que la topología en particular que tenga $X$ no juega un papel relevante, así que se puede tomar la topología discreta. Una vez que tenemos definido este haz para la topología discreta podemos llevarlo a cualquier topología. Tenemos que la identidad $X^{dis}\to X$ es continua, donde $X^{dis}$ es $X$ con la topología discreta. Entonces, $Id_*\calF(U)=\calF(U)$ por definición. 
\end{nota}

Dado un anillo $A$, hemos construido un espacio topológico $\spec(A)$ al que le hemos asociado un haz de anillos $\OO_{\spec(A)}$. Al par $(\spec{A}, \OO_{\spec(A)})$ se le llama \emph{espectro} de $A$. 


\begin{ej}
Vimos que $X=\spec(\Z)$ se puede identificar con el conjunto de primos naturales unión el 0. Los cerrados de la topología de Zariski son, además del total (obtenido como $V(0)$) y del vacío, los de la forma $V(n\Z)$. Los ideales que contienen a $n\Z$ son los ideales generados por los divisores de $n$, luego los cerrados propios son los conjuntos finitos de primos. Como hemos visto, $\overline{0}=\spec(\Z)$. A esto lo llamaremos punto genérico. Los abiertos no triviales serían entonces el 0 y los conjuntos de primos con complementario finito. Denotamos $U_{p_1,\dots, p_r}$ al abierto que tiene como complementario el conjuto de primos indicado.  $U_\emptyset=X$. 

Vamos a calcuclar $\OO_X(U_{p})$. Tenemos que calcular $\Z_{\gene{p}}=\{\frac{a}{s}\mid a\in\Z, p\not| s\}$ y $\Z_{\gene{0}}=\Q$.  $U_{p}=X\setminus V(p\Z)=\{0\}\cup\{p_i\mid p\neq p_i\}$. Entonces $\OO_X(U_p)=\{(s_q)_{q\in U_p}\mid s_q\in\Z_{\gene{p}}\}=\Z_p=\{\frac{a}{p^r}\mid a\in\Z\}$ (esto habría que demostrarlo). 

Si cogemos un abierto genérico $U=U_{p_1,\dots, p_r}$, el resultado es $\OO_X(U)=\Z_{p_1\cdots p_r}$. En el caso del total, no ponemos ningún primo en el denominador, así que saldría simplemente $\Z$. La restricción $\OO_X(X)=\Z\to \OO_X(U)=\Z_{p_1\cdots p_r}$ es simplemente el homomorfismo usual de un anillo en el localizado. 
\end{ej}

\begin{ej}
Sea $k$ un cuerpo algebraicamente cerrado. $\spec(k[x])=k\cup\{\gene{0}\}$. La topología de zariski es análoga a la del anterior ejemplo, los abiertos no triviales son subconjuntos de $k$ con complementario finito en $k\cup\{0\}$. Si se hace el cálculo, se obtiene $\OO_{\spec(k[x)]}(k[x])=k[x]$. Si se toma un abierto que no sea el total, $\OO_{\spec(k[x])}(\{\alpha_1,\dots,\alpha_r\}^c)=k[x]_{(x-\alpha_1)\cdots(x-\alpha_r)}$. 
\end{ej}

\begin{prop}
Sea $A$ un anillo y $(\spec(A),\OO)$ su espectro.
\begin{enumerate}[(a)]
\item Para todo $\p\in\spec(A)$, la fibra $\OO_\p$ es isomorfa al anillo local $A_\p$.
\item Para todo elemento $f\in A$, si denotamos $X_f=\spec(A)\setminus V(\gene{f})$, entonces $\OO(X_f)$ es isomorfo al anillo localizado $A_f$.
\item En particular, $\OO(\spec(A)\cong A$. 
\end{enumerate} 
\end{prop}

\section{Esquemas}

En geometría diferencial se suelen definir las variedades diferenciables a partir de cartas en un atlas. Se pueden definir también de otro modo más próximo al que nos interesa en nuestro estudio. Dado $U\subseteq\R^n$ abierto, definimos el haz $E_U$ de funciones $C^{\infty}$, de modo que $E_U(V)$ son las funciones $f:V\to\R$ de clase $C^{\infty}$ para $V\subseteq U$.

 Así que podríamos definir variedad diferenciable como un espacio topológico expresado como unión de abiertos $X=\cup X_i$ y un haz $E_X$ de modo que $(X_i,E_X|_{X_i})\cong (U_i,E_{U_i})$ con $U_i\subseteq\R^n$. Este isomorfismo es en el sentido de que hay un homeomorfismo entre los espacios topológicos y un isomorfismo de haces en el segundo elemento del par. Este enfoque será el que usemos para definir los esquemas. 
 
 
 \begin{defi}
 Un \emph{espacio anillado} es un par $(X,\calA)$ donde $X$ es un espacio topológico y $\calA$ es un haz de anillos (comutativos con unidad) en $X$.
 \end{defi}
 
 \begin{ejs}\
 \begin{enumerate}
 \item Sea $A$ un anillo fijo y $\calA$ el haz de todas las funciones con valor en $A$. 
 \item Dado un anillo topológico $A$ (por ejemplo $\R$ o $\C$), definimos $\calA$ como el haz de las funciones continuas con valores en $A$.
 \item Sea $X$ una variedad diferenciable (real) y $\calA$ el haz de las funciones $C^{\infty}$ con valores en $\R$.
 \item $(\spec(A), \OO_{\spec(A)})$. 
 \end{enumerate}
 
 \end{ejs}
 
 \begin{defi}
 Sean $(X,\calA)$, $(Y,\mathcal{B})$ espacios anillados. Un morfimo entre ellos es un par $(f, f^{\sharp})$ donde $f:X\to Y$ es continua y $f^{\sharp}:\mathcal{B}\to f_*\calA$ es un morfismo de haces de anillos en $Y$. 
 
\end{defi} 


\begin{ej}
Sean $X,Y$ variedades diferenciables y $f:X\to Y$ una aplicación diferenciable. Consideramos el haz de las funciones diferenciables $\mathcal{E}_X$ tal que $\mathcal{E}(U)$ es el anillo de funciones diferenciables de $U$ en $\R$, y hacemos lo mismo para $Y$. Recordemos que $(f_*\mathcal{E}_X)(V)=\mathcal{E}_X(f^{-1}(V))$. Por otro lado tenemos $\mathcal{E}_Y(V)$, pero dada $\varphi\in \mathcal{E}_Y(V)$, podemos asociarle $\varphi\circ f|_{f^{-1}(V)}\in\mathcal{E}_X(f^{-1}(V))$. Esta es la definición de $f^{\sharp}$. 
\end{ej}

Si tenemos tres haces $(X,\calA)\xrightarrow{(f,f^{\sharp})}(Y,\mathcal{B})\xrightarrow{(g,g^{\sharp})}(Z,\mathcal{C})$, tendríamos una composición $(g,g^{\sharp})\circ(f, f^{\sharp})$. La primera coordenada de la composición es simplemente $g\circ f$, pero en la segunda tenemos que hacerlo con más cuidado. Tenemos $f^{\sharp}:\mathcal{B}\to f_*\calA$, $g^{\sharp}:\mathcal{C}\to g_*\mathcal{B}$. Podemos considerar $g_*(f^{\sharp}):g_*\mathcal{B}\to g_*(f_*\mathcal{A})=(g\circ f)_*(\calA)$. Así que la segunda componente de la composición es $g_*(f^\sharp)\circ g^{\sharp}$. 



\begin{defi}
Un espacio anillado $(X,\calA)$ se dice \emph{localmente anillado} si para todo $x\in X$, la fibra $\calA_x$ es un anillo local.
\end{defi}


\begin{ej}
Sea $(X,\mathcal{E}_X))$ una variedad diferenciable, $x\in X$. Entonces $\mathcal{E}_{X.x}$ consideramos los gérmenes que se anulan en el punto $x$, que forman claramente un ideal. Si un germen no se anula en $x$, entonces tampoco se anula en un entorno, por lo que es invertible multiplicativamente, o sea, es una unidad en el anillo. Así que las variedades diferenciables son espacios localmente anillados.

Tenemos además un homomorfismo de anillos $v_x:\mathcal{E}_{X,x}\to \R$, que es la evaluación del gérmen en $x$. Se tiene que $\ker v_x=m$, ideal maximal por el primer teorema de isomorfía (el homomorfismo es sobreyectivo porque se puede tomar el gérmen constante). Este ideal es justamente el que hemos mencionado antes. Se demuestra que $m/m^2$ tiene la misma dimensión que la variedad diferenciable, y es lo que se llama espacio cotangente (el dual de él es el tangente). 
\end{ej}

\begin{defi}
Si $(A, \m)$ y $(B,\mathfrak{n})$ son anillos locales. Un homomorfismo $\varphi:A\to B$ se dice \emph{local} si $\varphi^{-1}(\mathfrak{n})=\m$. 
\end{defi}

\begin{defi}
Sean $(X,\calA), (Y,\mathcal{B})$ espacios localmente anillados. Diremos que un morfismo $(f,f^{\sharp}):(X,\calA)\to (Y,\mathcal{B})$ de espacios anillados es local cuando para cada punto $x\in X$, $f^{\sharp}_x:\mathcal{B}_{f(x)}\to\calA_x$ sea un homomorfismo local. 
\end{defi}

\begin{defi}
Un \emph{esquema afín} es un espacio localmente anillado $(X,\OO_X)$ que es isomorfo (como espacio localmente anillado) al espectro de algún anillo. Un \emph{esquema} es un espacio localmente anillado $(X,\OO_X)$ tal que $X$ se puede recubrir por abiertos $\{U_i\}_{i\in I}$ tales que $(U_i,\OO_X|_{U_i})$ es un esquema afín. 
\end{defi}



\section{Ejercicios}
\begin{ejer}
Probar que el haz $\OO_X(U)$ es realmente un haz, describiendo las restricciones. 
\end{ejer}
\begin{ejer}
Probar que el prehaz $\calF(U)=\prod_{x\in U} A_x$ es un haz para $X$ con la topología discreta. 
\end{ejer}
\end{document}

