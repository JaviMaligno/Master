\documentclass[GA.tex]{subfiles}

\begin{document}


%\hyphenation{equi-va-len-cia}\hyphenation{pro-pie-dad}\hyphenation{res-pec-ti-va-men-te}\hyphenation{sub-es-pa-cio}

\chapter{Morfismos propios y separados}

La noción de morfismo propio en esquemas es la habitual de espacios topológicos, es decir, la preimagen de un compacto es compacta. El concepto de morfismo separado sin embargo tendrá que ser refinado. En topología general una aplicación es separada si la preimagen de cada punto es un espacio Hausdorff, pero esta condición es demasiado fuerte en los espacios que atañen a la geometría algebraica. 


PONER DE PRELIMINARES LAS COSAS QUE APARECEN EN AGII 4. 
\section{Morfismos separados}


\begin{defi}
Sea $f:X\to Y$ un morfismo de esquemas. El morfismo \emph{diagonal} es el único morfismo $\Delta:X\to X\times_Y X$ cuya composición con ambas proyecciones $p_1,p_2:X\times_Y X\to X$ es la identidad $X\to X$. Decimos que el morfismo $f$ es \emph{separado} si $\Delta$ es una inmersión cerrada. También decimos que $X$ es \emph{separado} sobre $Y$. Un esquema $X$ es \emph{separado} si lo es sobre $\spec(\Z)$.  
\end{defi}

\begin{ej}
Sea $k$ un cuerpo y $X$ la recta afín con doble origen. Entonces $X$ no es separado sobre $k$. Si $U$ y $V$ son las dos rectas afines pegadas por el complementario del origen, entonces $U\cap V$ es justamente el complementario del origen y $U\times V$ es el producto de dos rectas afines. Claramente la imagen de $\Delta:U\cap V\to U\times V$ no es cerrada.\footnote{\url{https://math.stackexchange.com/questions/490674/hartshorne-example-ii-4-0-1}}\footnote{\url{https://math.stackexchange.com/questions/322188/how-to-compute-the-topological-space-of-fibered-product-of-schemes}}
\end{ej}

\begin{prop}
Si $f:X\to Y$ es un morfismo de esquemas afines, entonces $f$ es separado.
\end{prop}
\begin{dem}
Sea $X=\spec(A)$, $Y=\spec(B)$. Entonces $A$ es una $B$-álgebra y $X\times_Y X$ es también afín, dado por $\spec(A\otimes_B A)$. El morfismo diagonal $\Delta$ viene del homomorfismo diagonal $A\otimes_B A\to A$ dado por $a\otimes a'\mapsto aa'$. Este es un homomorfismo de anillos sobreyectivo, y por tanto $\Delta$ es una inmersión cerrada (ejercicio 3.12).
\end{dem}

\begin{coro}
Un morfismo $f:X\to Y$ es separado si y solo si la imagen del morfismo diagonal es un subconjunto cerrado de $X\times_Y X$.
\end{coro}
\begin{dem}
Una implicación es obvia, así que solo tenemos que probar que si $\Delta(X)$ es un subconjunto cerrado, entonces $\Delta:X\to X\times_Y X$ es una inmersión cerrada. En otras palabras, tenemos que comprobar que $\Delta:X\to \Delta(X)$ es un homeomorfismo y que el morfismo de haces $\OO_{X\times_Y X}\to\Delta_*\OO_X$ es sobreyectivo. Sea $p_1:X\times_Y X\to X$ la primera proyección. Como $p_1\circ\Delta=Id_X$, se sigue que $\Delta$ da un homeomorfismo sobre su imagen. Ver que el morfismo de haces $\OO_{X\times_Y X}\to\Delta_*\OO_X$ es sobreyectivo es una cuestión local. Para cualquier punto $P\in X$, sea $U$ un abierto afín con $P\in U$ lo bastante pequeño como para que $f(U)$ esté contenido en un abierto afín $V$ de $Y$. Entonces $U\times_V U$ is un entorno abierto afín de $\Delta(P)$, y por la proposición, $\Delta:U\to U\times_V U$ es una inmersión cerrada, por lo que la aplicación de haces es sobreyectiva en un entorno de $P$ por definición de inmersión cerrada, lo que completa la prueba.  
\end{dem}

\begin{teorema}[Criterio Valuativo de Separación]
Sea $f:X\to Y$ un morfismo de esquemas y supongamos que $X$ es noetheriano. Entonces $f$ es separado si y solo si la siguiente condición se cumple. Para cualquier cuerpo $K$ y para cualquier anillo de valuación\footnote{\url{https://en.wikipedia.org/wiki/Valuation_ring}} $R$ con cuerpo de fracciones $K$, sean $T=\spec(R)$, $U=\spec(K)$ y sea $i:U\to T$ el morfismo inducido por la inclusión $R\subseteq K$. Dado un morfismo $T\to Y$ y dado un morfismo $U\to X$ haciendo conmutativo el cuadrado del diagrama
\[
\begin{tikzcd}
U\arrow[r]\arrow[d, "i"'] & X\arrow[d, "f"]\\
T\arrow[r]\arrow[ur, dashed] & Y
\end{tikzcd}
\]
hay a lo sumo un morfismo $T\to X$ que hace conmutar todo el diagrama. 
\end{teorema}

Para la prueba de este teorema se necesitan dos lemas.

\begin{lemma}
Sea $R$ un anillo de valuación de un cuerpo $K$. Sea $T=\spec(R)$ y sea $U=\spec(K)$. Dar un morfismo de $U$ a un esquema $X$ es equivalente a dar un punto $x_1\in X$ y una inclusión de cuerpos $k(x_1)\subseteq K$. Dar un morfismo de $T$ a $X$ es equivalente a dar dos puntos $x_0,x_1\in X$, con $x_0$ una \emph{especialización } (ver ejercicio 3.17) de $x_1$, y una inclusión de cuerpos $k(x_1)\subseteq K$ tal que $R$ domina el anillo local $\OO$ de $x_0$ en el subesquema $Z=\overline{\{x_1\}}$ (clausura) de $X$ con su estructura reducida inducida. 
\end{lemma}



\begin{lemma}
Sea $f:X\to Y$ un morfismo quasi-compacto de esquemas (ver ejercicio 3.2). Entonces el subconjunto $f(X)$ de $Y$ es cerrado si y solo si es estable bajo especialización (ver ejercicio 3.17e). 
\end{lemma}


\begin{defi}
Un morfismo $f:X\to Y$ es \emph{propio} si es propio, de tipo finito y universalmente cerrado, esto es, para todo $Z\to Y$, $X\times_Y\to Z$ es cerrado.
\end{defi}

\begin{prop}
Sean $X$ e $Y$ espacio stopológicos y $f:X\to Y$ de tipo finito. Supongamos que $X$ es noetheriano. Entonces son equivalentes:
\begin{enumerate}
\item $f$ es propio.
\item Para todo anillo de valoración $R$ con $K=Frac(R)$, $T=\spec(R)$ y $U=\spec(K)$, existe un único morfismo en la diagonal del diagrama siguiente que hace conmutar los triangulos
\[
\begin{tikzcd}
U\arrow[r] \arrow[d, "i"]& X\arrow[d, "f"]\\
T\arrow[r]\arrow[ur, dashed] & Y
\end{tikzcd}
\]

\begin{ej}\
\begin{enumerate}
\item Los morfismos finitos son propios. Por resultados de álgebra conmutativa (Atiyah-MacDonald) existe un morfismo diagonal haciendo conmutar el diagrama
\[
\begin{tikzcd}
K & B\arrow[l]\arrow[dl, dashed]\\
R\arrow[u]\arrow[ur, dashed] & A\arrow[u]
\end{tikzcd}
\]
\item Sea $A$ un anillo y $B=A[X_0,\dots, X_n]$ un anillo graduado. Al esquema $\proj(B)$ se le conoce como el espacio proyectivo sobre $A$ y se lo denota $\mathbb{P}_A^n$. Hay un morfismo $\PP_A^n\to\spec(A)$ inducido por $A\to B$. Obsérvese que podemos hacer cambios de base. El ejemplo más claro se tiene con $A=\Z$. Podemos cambiar, por ejemplo, a $\Q$, dándonos un diagrama conmutativo
\[
\begin{tikzcd}
\PP_{\Z}^n\arrow[d] & \PP_\Q^n\arrow[l]\arrow[d]\\
\spec(\Z) & \spec(\Q)\arrow[l]
\end{tikzcd}
\]
Este morfismo además es propio. Para demostrarlo es suficiente demostrar que es propio el morfismo del caso $A=\Z$, ya que en el cambio de base $\PP_A^n=\PP_\Z^n\times_{\spec(A)}\spec(A)$, lo cual es equivalente a probar que $A[X_0,\dots, X_n]=A\otimes_\Z \Z[X_0,\dots,X_n]$. Este hecho aparece demostrado en el teorema 4.9 de Hartshorne TRATAR DE ENTENDERLA
\end{enumerate}
\end{ej}
\end{enumerate}
\end{prop}

\end{document}

