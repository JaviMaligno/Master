\documentclass[twoside]{article}
\usepackage{../estilo-ejercicios}
\newcommand{\x}{{\mathbf{x}}}
\newcommand{\y}{{\mathbf{y}}}
\usepackage[]{algorithm2e}
%--------------------------------------------------------
\begin{document}

\title{Teoría de Juegos}
\author{Rafael González López}
\maketitle

\begin{ejercicio}{1} Comprobar que se verifica la dualidad de los juegos de utilidad entre la versión optimista y pesimista del problema de la bancarrota, es decir, que se verifica
$$
\overline{v_{op}^{(E,d)}} = v^{(E,d)}
$$ 
\end{ejercicio}
\begin{solucion}
Escribamos las definiciones
$$
v_{op}^{(E,d)}(S) = \min\left\{\sum_{i\in S} d_i,E\right\}
\qquad
v^{(E,d)}(S) =\max\left\{0,E-\sum_{i\notin S} d_i\right\}
$$
Comprobamos directamente, utilizando que $d(N)>E$ y que $$a-\min(c,b) = \max(a-c,b-c)$$
Tenemos entonces
\begin{align*}
\overline{v_{op}^{(E,d)}}(S) &= v_{op}^{(E,d)}(N)-v_{op}^{(E,d)}(N\setminus S)\\
&= \min\left\{\sum_{i\in N} d_i,E\right\} -  \min\left\{\sum_{i\in N\setminus S} d_i,E\right\}\\
&= E -  \min\left\{\sum_{i\notin S} d_i,E\right\}\\
&= \max\left\{E-\sum_{i\notin S} d_i, E-E\right\}\\
&= \max\left\{0,E-\sum_{i\notin S} d_i\right\}\\
&= v^{(E,d)}(S)
\end{align*}

\end{solucion}

\newpage


\begin{ejercicio}{1} Comprobar que se verifica la dualidad de los juegos de utilidad entre la versión optimista y pesimista del problema de la bancarrota, es decir, que se verifica
$$
\overline{v_{op}^{(E,d)}} = v^{(E,d)}
$$ 
\end{ejercicio}
\begin{solucion}
Escribamos las definiciones
$$
v_{op}^{(E,d)}(S) = \min\left\{\sum_{i\in S} d_i,E\right\}
\qquad
v^{(E,d)}(S) =\max\left\{0,E-\sum_{i\notin S} d_i\right\}
$$
Comprobamos directamente, utilizando que $d(N)>E$ y que $$a-\min(c,b) = \max(a-c,b-c)$$
Tenemos entonces
\begin{align*}
\overline{v_{op}^{(E,d)}}(S) &= v_{op}^{(E,d)}(N)-v_{op}^{(E,d)}(N\setminus S)\\
&= \min\left\{\sum_{i\in N} d_i,E\right\} -  \min\left\{\sum_{i\in N\setminus S} d_i,E\right\}\\
&= E -  \min\left\{\sum_{i\notin S} d_i,E\right\}\\
&= \max\left\{E-\sum_{i\notin S} d_i, E-E\right\}\\
&= \max\left\{0,E-\sum_{i\notin S} d_i\right\}\\
&= v^{(E,d)}(S)
\end{align*}
\end{solucion}
\newpage


\begin{ejercicio}{1} Comprobar que se verifica la dualidad de los juegos de utilidad entre la versión optimista y pesimista del problema de la bancarrota, es decir, que se verifica
$$
\overline{v_{op}^{(E,d)}} = v^{(E,d)}
$$ 
\end{ejercicio}
\begin{solucion}
Escribamos las definiciones
$$
v_{op}^{(E,d)}(S) = \min\left\{\sum_{i\in S} d_i,E\right\}
\qquad
v^{(E,d)}(S) =\max\left\{0,E-\sum_{i\notin S} d_i\right\}
$$
Comprobamos directamente, utilizando que $d(N)>E$ y que $$a-\min(c,b) = \max(a-c,b-c)$$
Tenemos entonces
\begin{align*}
\overline{v_{op}^{(E,d)}}(S) &= v_{op}^{(E,d)}(N)-v_{op}^{(E,d)}(N\setminus S)\\
&= \min\left\{\sum_{i\in N} d_i,E\right\} -  \min\left\{\sum_{i\in N\setminus S} d_i,E\right\}\\
&= E -  \min\left\{\sum_{i\notin S} d_i,E\right\}\\
&= \max\left\{E-\sum_{i\notin S} d_i, E-E\right\}\\
&= \max\left\{0,E-\sum_{i\notin S} d_i\right\}\\
&= v^{(E,d)}(S)
\end{align*}

\end{solucion}

\newpage


\begin{ejercicio}{2}Consideremos el juego de la bancarrota.
\begin{enumerate}
\item ¿Es monótono? ¿Es superaditivo?
\item Supongamos que el capital total $E=50000$ euros y existen tres acreedores con deudas de $25000$, $20000$ y $40000$ euros. Construye el juego y calcula los dividendos.
\end{enumerate} 
\end{ejercicio}
\begin{solucion}
Vamos a considerar $v = v^{(E,d)}_{op}$.
\begin{enumerate}
\item Sea $S,T \subset N$ tales que $S\subset T$. Naturalmente se tiene que
$$\sum_{i \in S} d_i \leq \sum_{i \in T} d_i$$
Por lo que
\begin{align*}
v^{(E,d)}_{op}(S) &= \min\left\{\sum_{i\in S} d_i,E\right\}\\
&\leq \min\left\{\sum_{i\in T} d_i,E\right\} \\
&= v^{(E,d)}_{op}(T)
\end{align*}
Por lo que el juego es monótono.

Sean $S,T\subset N$ tales que $S\cap T = \emptyset$. Naturalmente se tiene que
$$
\sum_{i \in S\cup T} d_i = \sum_{i \in S} d_i + \sum_{i \in T} d_i
$$
Además, si $a,b,c\geq 0$ tenemos que $\min(a+b,c) \leq \min(a,c)+\min(b,c)$.  
\begin{align*}
v^{(E,d)}_{op}(S) + v^{(E,d)}_{op}(T) &= \min\left\{\sum_{i\in S} d_i,E\right\} + \min\left\{\sum_{i\in T} d_i,E\right\}\\
&\geq \min\left\{\sum_{i\in S} d_i + \sum_{i\in T} d_i,E\right\}\\
&= \min\left\{\sum_{i\in S\cup T} d_i,E\right\} \\
&= v^{(E,d)}_{op}(T\cup S)
\end{align*}
Por lo que el juego es subaditivo.
\item En el juego $N=\{1,2,3\}$, $d_1 = 25000$, $d_2 = 20000$ y $d_3 = 40000$; $E=50000$ y $v=v^{(E,d)}_{op}$. Calculamos los dividendos
\begin{align*}
v(\emptyset) &= 0\\
v(\{i\})& = d_i  \qquad \forall i = 1,2,3\\
v(\{1,2\}) &= 45000 \\ 
v(\{1,3\}) &= v(\{2,3\}) = v(N) = 50000
\end{align*}
\end{enumerate}
\end{solucion}
\newpage

\end{document}
