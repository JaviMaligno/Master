\documentclass[twoside]{article}
\usepackage{../estilo-ejercicios}
\newcommand{\x}{{\mathbf{x}}}
\newcommand{\y}{{\mathbf{y}}}
\usepackage[]{algorithm2e}
%--------------------------------------------------------
\begin{document}

\title{Teoría de Juegos}
\author{Rafael González López}
\maketitle

\begin{ejercicio}{1} Comprobar que se verifica la dualidad de los juegos de utilidad entre la versión optimista y pesimista del problema de la bancarrota, es decir, que se verifica
$$
\overline{v_{op}^{(E,d)}} = v^{(E,d)}
$$ 
\end{ejercicio}
\begin{solucion}
Escribamos las definiciones
$$
v_{op}^{(E,d)}(S) = \min\left\{\sum_{i\in S} d_i,E\right\}
\qquad
v^{(E,d)}(S) =\max\left\{0,E-\sum_{i\notin S} d_i\right\}
$$
Comprobamos directamente, utilizando que $d(N)>E$ y que $$a-\min(c,b) = \max(a-c,b-c)$$
Tenemos entonces
\begin{align*}
\overline{v_{op}^{(E,d)}}(S) &= v_{op}^{(E,d)}(N)-v_{op}^{(E,d)}(N\setminus S)\\
&= \min\left\{\sum_{i\in N} d_i,E\right\} -  \min\left\{\sum_{i\in N\setminus S} d_i,E\right\}\\
&= E -  \min\left\{\sum_{i\notin S} d_i,E\right\}\\
&= \max\left\{E-\sum_{i\notin S} d_i, E-E\right\}\\
&= \max\left\{0,E-\sum_{i\notin S} d_i\right\}\\
&= v^{(E,d)}(S)
\end{align*}

\end{solucion}
\newpage 


\begin{ejercicio}{2}Consideremos el juego de la bancarrota.
\begin{enumerate}
\item ¿Es monótono? ¿Es superaditivo?
\item Supongamos que el capital total $E=50000$ euros y existen tres acreedores con deudas de $25000$, $20000$ y $40000$ euros. Construye el juego y calcula los dividendos.
\end{enumerate} 
\end{ejercicio}
\begin{solucion}
Vamos a considerar $v = v^{(E,d)}$.
\begin{enumerate}
\item Sea $S,T \subset N$ tales que $S\subset T$. Naturalmente se tiene que $T^c \subset S^c$. Por tanto
\begin{gather*}
\sum_{i \notin S} d_i \geq \sum_{i \notin T} d_i \Rightarrow E -\sum_{i \notin S} d_i \leq E - \sum_{i \notin T} d_i
\end{gather*} 
Por lo que
\begin{align*}
v^{(E,d)}_{}(S) &= \max\left\{0,E-\sum_{i\notin S} d_i\right\}\\
&\leq \max\left\{0,E-\sum_{i\notin S} d_i\right\} \\
&= v^{(E,d)}(T)
\end{align*}
Por lo que el juego es monótono.

Sean $S,T\subset N$ tales que $S\cap T = \emptyset$. Naturalmente se tiene que
$$
\sum_{i \notin S\cup T} d_i \leq \sum_{i \notin S} d_i + \sum_{i \notin T} d_i \Rightarrow E - \sum_{i \notin S\cup T} d_i \geq \left(E - \sum_{i \notin S} d_i \right) + \left(E- \sum_{i \notin  T} d_i \right)
$$
Además, si $a,b,c\geq 0$ tenemos que $\max(a+b,c) \geq \max(a,c)+\max(b,c)$.  
\begin{align*}
v^{(E,d)}(S) + v^{(E,d)}(T) &=\max\left\{0,E-\sum_{i\notin S} d_i\right\} + \max\left\{0,E-\sum_{i\notin T} d_i\right\} \\
&\leq \max\left\{0,E-\sum_{i\notin T\cup S} d_i\right\} \\
&= v^{(E,d)}(T\cup S)
\end{align*}
Por lo que el juego es superaditivo.
\newpage
\item En el juego $N=\{1,2,3\}$, $d_1 = 25000$, $d_2 = 20000$ y $d_3 = 40000$; $E=50000$ y $v=v^{(E,d)}$. Calculamos los dividendos
\begin{align*}
v(\emptyset) &= v(1)=v(2) =0\\
v(3) &=  \max\{0,50000-45000\} = 5000\\
v(\{N\setminus \{i\}\})& = 50000- d_i  \qquad \forall i = 1,2,3\\
v(N) &= 50000
\end{align*}
\end{enumerate}
\end{solucion}
\newpage

\begin{ejercicio}{3}
Probar que el Core de $v$ $C(v)$ es un conjunto convexo.
\end{ejercicio}
\begin{solucion}
Recordemos la definición de $C(v)$
$$
C(v)=\{x\in \R^N\mid x(N)=v(N),\; x(S)\geq v(S) \,\forall S\subset N\}
$$
Sean $x,y \in C(v)$ y sea $\lambda \in [0,1]$, entonces
\begin{align*}
(\lambda x + (1-\lambda)y)(N) &= \lambda x (N) + (1-\lambda)y(N)\\
&= \lambda v(N) + (1-\lambda) v(N) \\
&= v(N)
\end{align*}
Además, sea $S\subset N$
\begin{align*}
(\lambda x + (1-\lambda)y)(S) & = \lambda x (S) + (1-\lambda)y(S)\\
&\geq  \lambda v(S) + (1-\lambda) v(S) \\
&\geq v(S)
\end{align*}
Por tanto, si $x,y\in C(v)$ entonces $\lambda x + (1-\lambda) y \in C(v)$, por lo que $C(v)$ es convexo.
\end{solucion}

\newpage
\begin{ejercicio}{4}
Una finca está valorada en por su propietario en 350 mil euros. Un
empresario le ofrece al propietario acondicionarla para polígono
industrial, de manera que podría venderse en 700 mil euros. Una
empresa constructora le ofrece parcelarla y urbanizarla para obtener
un beneficio de 775 mil euros. Se pide:
\begin{enumerate}
\item Representar la situación con un juego TU.
\item Esbozar su conjunto de imputaciones y su core.
\end{enumerate}
\end{ejercicio}
\begin{solucion}
Denotemos por $1$ al dueño de la finca, $2$ al empresario y $3$ a la empresa constructora.
\begin{enumerate}
\item El juego está representado por $N=\{1,2,3\}$ y $v(\cdot)$ donde 
\begin{gather*}
v(\emptyset) = 0 \qquad v(\{1\}) = 350k \qquad v(\{2\}) = v(\{3\}) = v(\{2,3\}) = 0\\
v(\{1,2\}) = 700k \qquad v(\{1,3\}) = v(N) = 775000 
\end{gather*}
\item Calculamos directamente
\begin{align*}
I(v) &= \{x \in \R^3 \mid x_1+x_2+x_3 = 775000,\;x_2,x_3\geq 0,\;x_1\geq 350k\}\\
&=\{x\in \R^3_+ \mid x_1+x_2+x_3 = 775000,\; x_1 \geq 350k\}
\end{align*}

\begin{figure}[h!]
\centering
\includegraphics[scale=0.5]{Iv}
\caption{Imputaciones de $v$.}
\end{figure}

Si denotamos
\begin{align*}
H(v) &= \{x \in \R^3_+ \mid x_2+x_3 \geq 0,\; x_1+x_3 \geq 775000,\;x_1+x_2 \geq 700000\}\\
&=\{x \in \R^3_+ \mid  x_1+x_3 \geq 775000,\;x_1+x_2 \geq 700000\}
\end{align*}
Entonces tenemos
\begin{align*}
C(v) &= I(v) \cap H(v)\\
&=\{x \in \R^3_+ \mid x_1+x_2+x_3 = 775000,\;x_1 \geq 350000,\; x_1+x_3 \geq 775000,\;x_1+x_2 \geq 700000\}\
\end{align*}
Naturalmente, si $x_1+x_3 \geq 775000$ y $x_1+x_2+x_3 = 775000$, dado que las variables son no negativas, tenemos que $x_2 = 0$. Por tanto, podemos escribir de manera compacta
$$
C(v)=\{x \in \R^3_+ \mid x_1+x_3 =775000,\; x_1 \geq 700000,\; x_2 = 0\}
$$
Lo cual puede expresarse de manera paramétrica como un segmento de línea:
$$
C(v) = \{(700000 + s, 0, 75000 - s) \in \R^3 \mid s \in [0,75000]\}
$$

\begin{figure}[h!]
\centering
\includegraphics[scale=0.45]{Cv}
\caption{Core de $v$. Representado $x_1$ frente a $x_3$.}
\end{figure}
\end{enumerate}
\end{solucion}

\newpage

\begin{ejercicio}{5}
Probar que $\phi(v)\in C(v)$ si $v$ es convexo.
\end{ejercicio}
\begin{solucion}
Recordemos las definiciones. Decimos que $v$ es convexo si $\forall S,T\subset N$
$$
v(S\cup T) + v(S\cap T) \geq v(S) + v(T) 
$$
Además, el core de $v$ es precisamente
$$
C(v) = \{x\in \R^{N}\mid x(N) = v(N), \; x(S) \geq v(S)\; \forall S \subset N\}
$$
Por otro lado, el valor de Shapley se puede escribirse como
$$
\phi(v) = \frac{1}{n!}\sum_{\theta \in \Theta^N}m^\theta(v)
$$
donde $m^{\theta}(v)$ se puede definir componente a componente como, si $\theta = (i_1,\dotsc,i,\dotsc,i_n)$,
$$
m^\theta(v)_i = v(S^\theta_i) - v(S^\theta_i\setminus\{i\}) \qquad S_i^\theta = \{i_1,\dotsc,i\} 
$$
Naturalmente, $|\Theta^N| = n!$, al ser el espacio de las permutaciones de $N$, por lo que $\phi(v)$ es la media de los $m^\theta(v)$. Sabemos por el Ejercicio 3 que $C(v)$ es convexo, luego basta probar que cada $m^\theta(v)\in C(v)$. Denotamos por $v(S_{i_0}^\theta) = 0$. Naturalmente, por la definición de los $m^\theta(v)_i$, tenemos que 
\begin{align*}
m^\theta(v)(N) &= \sum_{i=1}^N m^\theta(v)_i \\
&= \sum_{i=1}^N  v(S^\theta_i) - v(S^\theta_i\setminus\{i\}) \\
&= \sum_{j=1}^N  v(S^\theta_{i_j}) - v(S^\theta_{i_j}\setminus\{i_j\}) \\
&= \sum_{j=1}^N  v(S^\theta_{i_j}) - v(S^\theta_{i_{j-1}}) \\
&= v(S_{i_n}^\theta) \\
&= v(N) 
\end{align*}
\newpage
Finalmente, tenemos que comprobar que $m^\theta(v)(S) \geq v(S)$. Supongamos que $S=\{a_1,\dotsc,a_k\}$ con $a_1< \cdots < a_k$ en $\theta$. Sabemos que ser convexo es equivalente a
$$
v(T) - v(T\setminus\{i\}) \leq v(S) - v(S\setminus\{i\}) \qquad \forall T\subset S \subset N
$$
Ahora ben, tengamos en cuenta que por la forma en la que hemos ordenado los elementos de $S$, $\{a_1,\dotsc,a_j\}\subset S_{a_j}^\theta$. Por tanto,
$$
v(\{a_1,\dotsc,a_j\}) - v(\{a_1,\dotsc,a_{j-1}\}) \leq v(S_{a_j}^\theta) - v(S_{a_j}^\theta\setminus\{a_j\})
$$
Luego
\begin{align*}
m^\theta(v)(S) &= \sum_{i\in S}m^\theta(v)_i \\
&= \sum_{j=1}^k   v(S^\theta_{a_j}) - v(S^\theta_{a_j}\setminus\{a_j\}) \\
&\geq  \sum_{j=1}^k   v(\{a_1,\dotsc,a_j\}) - v(\{a_1,\dotsc,a_{j-1}\})\\
&= v(\{a_1,\dotsc,a_k\})\\
& = v(S)
\end{align*}

\end{solucion}
\newpage
\begin{ejercicio}{6}
Calcular el nucleolo para el juego de La herencia del tío Pepe. 
\end{ejercicio}
\begin{solucion}
Comencemos escribiendo los conceptos que definen el nucleolo. Definimos el exceso de $x$ para $S$ como $e(S,x) = v(S)-x(S)$. Consideremos entonces el vector de excesos $\lambda(x) \in \R^{2^n}$. Finalmente, definimos el nucleolo como
$$
\nu(v)=\{x\in I(v) \mid \lambda(x)\leq_L\lambda(y) \;\forall y \in I(v)\}
$$
Recordemos que en el juego del tío de Pepe
$$
I(v) = \{(a,b,1-a-b) \mid  0\leq a,b,1-a-b\leq 1\}
$$
Para cada $x\in I(v)$ tenemos que sin ordenar todavía de manera decreciente, el vector $\lambda(x)$ sería
$$
\lambda(x) =(-a,-b,a+b-1,1-a-b,b,a,0)
$$ 
Notemos que las tres primeras componentes siempre serán negativas o nulas, así como la última. Nos interesan principalmente $(a,b,1-a-b)$, que siempre son no negativas. Pensemos en la asignación $a=b=1/3$  que hace que la mayor componente de $\lambda(x)$ sea precisamente $1/3$. Obviamente, si $a>1/3$ o $b>1/3$, tendremos una componente mayor que $1/3$, luego estas imputaciones no pueden ser óptimas. Por tanto, sabemos que $a,b\leq 1/3$. Si $a<1/3$ y $b\leq 1/3$ entones $1-a-b> 1/3$, por lo que nuestra asignación $a=b=1/3$ sigue siendo menor. Análogamente para $a\leq 1/3$, $b<1/3$. Por tanto, se minimiza para precisamente $a=b=1/3$.

Notemos que en este juego $C(v)=\emptyset$. 
\end{solucion}
\newpage
\begin{ejercicio}{7}
Calcular el valor de Shapley para el juego del premio Nobel japonés y comprobar si pertenece al núcleo.
\end{ejercicio}

\begin{solucion}
Consideremos  los valores de $v^c$.
\begin{gather*}
v^c(\emptyset) = 0 \qquad v^c(\{i\}) = 0 \quad \forall i = 1,2,3 \qquad  v(\{1,2\}) = v(\{1,3\}) = 1300\\
v(\{2,3\}) = 1600 \qquad v(N)=2900
\end{gather*}
Ahora calculemos el valor de Shapley de $v^c$. 
\begin{align*}
\phi(v^c)_1& = \frac{2!0!}{3!}(v^c(\{1\})+ v^c(N) -v^c(\{2,3\})) + \frac{1!1!}{3!}\left((v^c(\{1,2\})-v^c(\{2\}) + v^c(\{1,3\})-v^c(\{3\}) \right)\\
&= \frac{1}{6}(5800-3200 + 1300 + 1300)\\& = \frac{2600}{3}\\
\phi(v^c)_2& = \frac{2!0!}{3!}(v^c(\{2\})+ v^c(N) -v^c(\{1,3\})) + \frac{1!1!}{3!}\left((v^c(\{1,2\})-v^c(\{1\}) + v^c(\{2,3\})-v^c(\{3\}) \right)\\
&= \frac{1}{6}(5800-2600 + 1300 + 1600)\\& = \frac{3050}{3}\\
\phi(v^c)_3& = \frac{2!0!}{3!}(v^c(\{3\})+ v^c(N) -v^c(\{1,3\})) + \frac{1!1!}{3!}\left((v^c(\{1,3\})-v^c(\{1\}) + v^c(\{2,3\})-v^c(\{2\}) \right)\\
&= \frac{1}{3}(5800-2600 + 1300 + 1600)\\& = \frac{3050}{3}
\end{align*}
Por tanto
$$
\phi(v^c) = \frac{1}{3}(2600,3050,3050)
$$
y el valor de Shapley de $c$ sabemos que verifica $\phi_i(c) = c(\{i\}) - \phi_i(v^c)$. Luego 
$$
\phi(c)= \frac{1}{3}(1600,1750,2050)
$$
Pasemos ahora a comprobar que está en el conjunto $\overline{C}(c)$.  Basta realizar sencillos cálculos para ver que, efectivamente, se verifican las siguientes desigualdades.
$$
x\in \overline{C}(x) \Leftrightarrow \begin{cases}
x_1+x_2+x_3 = 1800\\
x_1\leq 1400, \; x_2 \leq 1600, \; x_3 \leq 1700\\
x_1+x_2 \leq 1700,\, x_1+x_3 \leq 1800,\, x_2+x_3 \leq 1700
\end{cases}
$$
\end{solucion}
\newpage

\begin{ejercicio}{8}
Para el juego del parlamento de Amigolandia calcular su índice de Banzhaf $\beta(v)$. 
\end{ejercicio}
\begin{solucion}
Vamos a comenzar recordando las definiciones. En primer lugar, recordemos que la definición de índice de Banzha es 
$$
\beta_i(v) = \frac{|SW_i(v)|}{2^{n-1}}
$$
donde $SW_i(v)$ son los swings de $i$ en $v$, es decir, aquellos conjuntos $S$ que verifican la igualdad $v(S)-v(S\setminus\{i\}) = 1$. Por tanto, el problema se reduce a calcular dichos conjuntos. En nuestro caso $n-1 =4$ y ya sabemos que $\beta_5(v)=1/8$. Calculemos el resto.
\begin{itemize}
\item Para $i=1$, tenemos que cualquier subconjunto que tenga tres elementos y contenga a 1 va sumar más de 47, pero sin 1 la suma de dos nos da como máximo 40. Por tanto, los $\binom{4}{2} = 6$ están. Para los de dos, todos son válidos salvo $\{1,5\}$, luego añadimos $3$ más. Resta ver los de tamaño cuatro. Pero estos son todos válidos también salvo $\{1,2,3,4\}$ y $\{1,2,3,5\}$, luego añadimos $\binom{4}{3}-2 =2$. Por tanto, $\beta_1(v) = 11/16$.
\item Para $i=2$. De tamaño dos tenemos $\{1,2\}$. De tamaño tres $\{1,2,4\}$, $\{1,2,5\}$, $\{2,3,4\}$ y $\{2,3,5\}$. Con cuatro elementos tenemos solo $\{2,3,4,5\}$. Por tanto tenemos que $\beta_2(v)=5/16$.
\item Para $i=3$. De tamaño dos tenemos $\{1,3\}$. De tamaño tres $\{1,3,4\}$, $\{1,3,5\}$, $\{2,3,4\}$ y $\{2,3,5\}$. De tamaño cuatro tenemos $\{2,3,4,5\}$. Por tanto $\beta_3(v)= 5/16$.
\item Para $i=4$. De tamaño dos no hay. De tamaño tres tenemos $\{1,4,5\}$ y $\{2,3,4\}$. Con cuatro elementos no hay ninguno. Por tanto, $\beta_4(v) = 2/16$. \end{itemize}
\end{solucion}

\newpage

\begin{ejercicio}{9}
Encuentra la solución de Owen para el juego de la producción de cerveza.
\end{ejercicio}

\begin{solucion}
En primer lugar, vamos a determinar todos los elementos del juego. $N=\{1,2,3\}$ correspondientes a las tres cervezeras. $M=\{\text{cebada, lúpulo}\}$ correspondientes a los recursos. $R=\{\text{cerveza rubia, cerveza negra}\}$ el único producto. Además
$$
b^1 = (4,33) \quad b^2 =(6,39) \quad b^3 = (60,0) \qquad A = \begin{pmatrix}
4 & 5\\
6 & 2
\end{pmatrix} \qquad p = (68,52)
$$
Para calcular la solución de Owen consideramos el problema
\begin{align*}
\min\, & b(N)y\\
s.a.\,&A'y\geq p\\
& y\geq 0
\end{align*}
Sustituyendo los valores,
\begin{align*}
\min\, & 70y_1 + 72y_2\\
s.a.\,&
4y_1 + 5y_2\geq 68\\
&6y_1 + 2y_2 \geq 52\\
& y\geq 0
\end{align*}
Obtenemos la solución $y^* = (8,6)$. La solución de Owen está dada por
$$
x = (b^1y^*,b^2y^*,b^3y^*) = (230,282,480) 
$$
\end{solucion}
\newpage
\begin{ejercicio}{10}
Comprobar que en los juegos de asignación $v$ es superaditivo.
\end{ejercicio}
\begin{solucion}
La función $v$ es en este caso
$$
v(S,T) = \max\left\{\sum_{i\in S} a_{ib(i)} \mid b \in B(S,T)\right\}
$$
donde $B(S,T)$ es el conjunto de las asignaciones $b$ entre $S$ y $T$. Supongamos que $N= P\cup Q$. Sean $S_1 \cup T_1$ y $S_2 \cup T_2$ tales que $S_1,S_2 \subset P$, $T_1,T_2 \cup Q$, $S_1\cap S_2 = T_1 \cap T_2 = \emptyset$. Basta tener en cuenta que, dado que los $S_i$ y los $T_i$ son disjuntos, tenemos
$$
B(S_1,T_1)\cup B(S_2,T_2) \subset B(S_1\cup S_2, T_1\cup T_2) \qquad B(S_1,T_1)\cap B(S_2,T_2) = \emptyset
$$
Luego se sigue de inmediato
\begin{align*}
v(S_1\cup S_2, T_1\cup T_2) &= \max\left\{\sum_{i\in S_1 \cup S_2} a_{ib(i)} \mid b \in B(S_1\cup S_2, T_1\cup T_2)\right\}\\
&\geq  \max\left\{\sum_{i\in S_1 \cup S_2} a_{ib(i)} \mid b \in B(S_1,T_1)\cup B(S_2,T_2)\right\}\\
&=  \max\left\{\sum_{i\in S_1} a_{ib_1(i)} + \sum_{i \in S_2} a_{ib_2(i)} \mid b_1 \in B(S_1,T_1),\, b_2\in B(S_2,T_2)\right\}\\
&= \max\left\{\sum_{i\in S_1} a_{ib_1(i)} \mid b_1 \in B(S_1,T_1)\right\}+   \max\left\{\sum_{i \in S_2} a_{ib_2(i)} \mid b_2\in B(S_2,T_2)\right\}\\
&=v(S_1,T_1) + v(S_2,T_2)
\end{align*}

\end{solucion}
\end{document}
