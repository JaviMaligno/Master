\documentclass[twoside]{article}
\usepackage{../estilo-ejercicios}
\newcommand{\x}{{\mathbf{x}}}
\newcommand{\y}{{\mathbf{y}}}
\usepackage[]{algorithm2e}
%--------------------------------------------------------
\begin{document}

\title{Teoría de Juegos}
\author{Rafael González López}
\maketitle

\begin{ejercicio}{1} Comprobar que se verifica la dualidad de los juegos de utilidad entre la versión optimista y pesimista del problema de la bancarrota, es decir, que se verifica
$$
\overline{v_{op}^{(E,d)}} = v^{(E,d)}
$$ 
\end{ejercicio}
\begin{solucion}
Escribamos las definiciones
$$
v_{op}^{(E,d)}(S) = \min\left\{\sum_{i\in S} d_i,E\right\}
\qquad
v^{(E,d)}(S) =\max\left\{0,E-\sum_{i\notin S} d_i\right\}
$$
Comprobamos directamente, utilizando que $d(N)>E$ y que $$a-\min(c,b) = \max(a-c,b-c)$$
Tenemos entonces
\begin{align*}
\overline{v_{op}^{(E,d)}}(S) &= v_{op}^{(E,d)}(N)-v_{op}^{(E,d)}(N\setminus S)\\
&= \min\left\{\sum_{i\in N} d_i,E\right\} -  \min\left\{\sum_{i\in N\setminus S} d_i,E\right\}\\
&= E -  \min\left\{\sum_{i\notin S} d_i,E\right\}\\
&= \max\left\{E-\sum_{i\notin S} d_i, E-E\right\}\\
&= \max\left\{0,E-\sum_{i\notin S} d_i\right\}\\
&= v^{(E,d)}(S)
\end{align*}

\end{solucion}
\newpage 


\begin{ejercicio}{2}Consideremos el juego de la bancarrota.
\begin{enumerate}
\item ¿Es monótono? ¿Es superaditivo?
\item Supongamos que el capital total $E=50000$ euros y existen tres acreedores con deudas de $25000$, $20000$ y $40000$ euros. Construye el juego y calcula los dividendos.
\end{enumerate} 
\end{ejercicio}
\begin{solucion}
Vamos a considerar $v = v^{(E,d)}$.
\begin{enumerate}
\item Sea $S,T \subset N$ tales que $S\subset T$. Naturalmente se tiene que $T^c \subset S^c$. Por tanto
\begin{gather*}
\sum_{i \notin S} d_i \geq \sum_{i \notin T} d_i \Rightarrow E -\sum_{i \notin S} d_i \leq E - \sum_{i \notin T} d_i
\end{gather*} 
Por lo que
\begin{align*}
v^{(E,d)}_{}(S) &= \max\left\{0,E-\sum_{i\notin S} d_i\right\}\\
&\leq \max\left\{0,E-\sum_{i\notin S} d_i\right\} \\
&= v^{(E,d)}(T)
\end{align*}
Por lo que el juego es monótono.

Sean $S,T\subset N$ tales que $S\cap T = \emptyset$. Naturalmente se tiene que
$$
\sum_{i \notin S\cup T} d_i \leq \sum_{i \notin S} d_i + \sum_{i \notin T} d_i \Rightarrow E - \sum_{i \notin S\cup T} d_i \geq \left(E - \sum_{i \notin S} d_i \right) + \left(E- \sum_{i \notin  T} d_i \right)
$$
Además, si $a,b,c\geq 0$ tenemos que $\max(a+b,c) \geq \max(a,c)+\max(b,c)$.  
\begin{align*}
v^{(E,d)}(S) + v^{(E,d)}(T) &=\max\left\{0,E-\sum_{i\notin S} d_i\right\} + \max\left\{0,E-\sum_{i\notin T} d_i\right\} \\
&\leq \max\left\{0,E-\sum_{i\notin T\cup S} d_i\right\} \\
&= v^{(E,d)}(T\cup S)
\end{align*}
Por lo que el juego es superaditivo.
\newpage
\item En el juego $N=\{1,2,3\}$, $d_1 = 25000$, $d_2 = 20000$ y $d_3 = 40000$; $E=50000$ y $v=v^{(E,d)}$. Calculamos los dividendos
\begin{align*}
v(\emptyset) &= v(1)=v(2) =0\\
v(3) &=  \max\{0,50000-45000\} = 5000\\
v(\{N\setminus \{i\}\})& = 50000- d_i  \qquad \forall i = 1,2,3\\
v(N) &= 50000
\end{align*}
\end{enumerate}
\end{solucion}
\newpage

\begin{ejercicio}{3}
Probar que el Core de $v$ $C(v)$ es un conjunto convexo.
\end{ejercicio}
\begin{solucion}
Recordemos la definición de $C(v)$
$$
C(v)=\{x\in \R^N\mid x(N)=v(N),\; x(S)\geq v(S) \,\forall S\subset N\}
$$
Sean $x,y \in C(v)$ y sea $\lambda \in [0,1]$, entonces
\begin{align*}
(\lambda x + (1-\lambda)y)(N) &= \lambda x (N) + (1-\lambda)y(N)\\
&= \lambda v(N) + (1-\lambda) v(N) \\
&= v(N)
\end{align*}
Además, sea $S\subset N$
\begin{align*}
(\lambda x + (1-\lambda)y)(S) & = \lambda x (S) + (1-\lambda)y(S)\\
&\geq  \lambda v(S) + (1-\lambda) v(S) \\
&\geq v(S)
\end{align*}
Por tanto, si $x,y\in C(v)$ entonces $\lambda x + (1-\lambda) y \in C(v)$, por lo que $C(v)$ es convexo.
\end{solucion}

\newpage
\begin{ejercicio}{4}
Una finca está valorada en por su propietario en 350 mil euros. Un
empresario le ofrece al propietario acondicionarla para polígono
industrial, de manera que podría venderse en 700 mil euros. Una
empresa constructora le ofrece parcelarla y urbanizarla para obtener
un beneficio de 775 mil euros. Se pide:
\begin{enumerate}
\item Representar la situación con un juego TU.
\item Esbozar su conjunto de imputaciones y su core.
\end{enumerate}
\end{ejercicio}
\begin{solucion}
Denotemos por $1$ al dueño de la finca, $2$ al empresario y $3$ a la empresa constructora.
\begin{enumerate}
\item El juego está representado por $N=\{1,2,3\}$ y $v(\cdot)$ donde 
\begin{gather*}
v(\emptyset) = 0 \qquad v(\{1\}) = 350k \qquad v(\{2\}) = v(\{3\}) = v(\{2,3\}) = 0\\
v(\{1,2\}) = 700k \qquad v(\{1,3\}) = v(N) = 775000 
\end{gather*}
\item Calculamos directamente
\begin{align*}
I(v) &= \{x \in \R^3 \mid x_1+x_2+x_3 = 775000,\;x_2,x_3\geq 0,\;x_1\geq 350k\}\\
&=\{x\in \R^3_+ \mid x_1+x_2+x_3 = 775000,\; x_1 \geq 350k\}
\end{align*}

\begin{figure}[h!]
\centering
\includegraphics[scale=0.5]{Iv}
\caption{Imputaciones de $v$.}
\end{figure}

Si denotamos
\begin{align*}
H(v) &= \{x \in \R^3_+ \mid x_2+x_3 \geq 0,\; x_1+x_3 \geq 775000,\;x_1+x_2 \geq 700000\}\\
&=\{x \in \R^3_+ \mid  x_1+x_3 \geq 775000,\;x_1+x_2 \geq 700000\}
\end{align*}
Entonces tenemos
\begin{align*}
C(v) &= I(v) \cap H(v)\\
&=\{x \in \R^3_+ \mid x_1+x_2+x_3 = 775000,\;x_1 \geq 350000,\; x_1+x_3 \geq 775000,\;x_1+x_2 \geq 700000\}\
\end{align*}
Naturalmente, si $x_1+x_3 \geq 775000$ y $x_1+x_2+x_3 = 775000$, dado que las variables son no negativas, tenemos que $x_2 = 0$. Por tanto, podemos escribir de manera compacta
$$
C(v)=\{x \in \R^3_+ \mid x_1+x_3 =775000,\; x_1 \geq 700000,\; x_2 = 0\}
$$
Lo cual puede expresarse de manera paramétrica como un segmento de línea:
$$
C(v) = \{(700000 + s, 0, 75000 - s) \in \R^3 \mid s \in [0,75000]\}
$$

\begin{figure}[h!]
\centering
\includegraphics[scale=0.45]{Cv}
\caption{Core de $v$. Representado $x_1$ frente a $x_3$.}
\end{figure}
\end{enumerate}
\end{solucion}
\end{document}
