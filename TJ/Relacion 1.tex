\documentclass[twoside]{article}
\usepackage{../estilo-ejercicios}
\newcommand{\x}{{\mathbf{x}}}
\newcommand{\y}{{\mathbf{y}}}
\usepackage[]{algorithm2e}
%--------------------------------------------------------
\begin{document}

\title{Optimización}
\author{Rafael González López}
\maketitle

\begin{ejercicio}{18.1.1} Construir la forma estratégica del juego cuyo árbol está dado en la figura siguiente. El juego comienza en el vértice $0$ (un moviemiento aleatorio); cada uno de los tres jugadores tiene un conjunto de información que contiene dos vértices, con dos alternativas (denominadas $a$ y $b$; $c$ y $d$; y $e$ y $f$ , respectivamente), en cada vértice.
\begin{figure}[h!]
\centering
\includegraphics[scale=0.7]{juego1}
\end{figure}

\end{ejercicio}
\begin{solucion}
\begin{enumerate}
\item[]
\item Supongamos que el jugador 1 escoge $a$. 
\begin{enumerate}[i]
\item Supongamos que el jugador 2 escoge $c$
\begin{enumerate}
\item Supongamos que el jugador 3 escoge $e$. Entonces, el beneficio esperado es $(0,0,0)$.
\item Supongamos que el jugador 3 escoge $f$. Entonces, con $2/3$ el beneficio es $(0,0,0)$ y con $1/3$ el beneficio es $(9,-3,3)$, por lo que el beneficio esperado es $(3,-1,1)$.
\end{enumerate}
\item Supongamos que el jugador 2 escoge $d$.
\begin{enumerate}
\item Supongamos que el jugador 3 escoge $e$. Con $2/3$ tenemos $(0,0,0)$, con $1/3$ $(-3,3,9)$. Por tanto, el beneficio esperado es $(-1,1,3)$.
\item Supongamos que el jugador 3 escoge $f$. Entonces, con $1/3$ el beneficio es $(0,0,0)$, con $1/3$ el beneficio es $(0,9,-6)$ y con $1/3$ es $(9,-3,3)$; por lo que el beneficio esperado es $(3,2,-1)$.
\end{enumerate}
\end{enumerate}
\item Supongamos que el jugador 1 escoge $b$. 
\begin{enumerate}[i]
\item Supongamos que el jugador 2 escoge $c$
\begin{enumerate}
\item Supongamos que el jugador 3 escoge $e$. Entonces, con $1/3$ $(3,9,-3$ y con $2/3$ $(0,0,0)$, por lo que el beneficio esperado es $(1,3,-1)$.
\item Supongamos que el jugador 3 escoge $f$. Entonces, con $1/3$ $(3,9,-3)$, con $1/3$ $(0,0,0)$ y con $1/3$ $(-6,0,9)$, entonces el beneficio esperado es $(-1,3,2)$.
\end{enumerate}
\item Supongamos que el jugador 2 escoge $d$.
\begin{enumerate}
\item Supongamos que el jugador 3 escoge $e$. Entonces con $1/3$ $(9,-6,0)$, con $1/3$ $(-3,3,9)$ y con $1/3$ $(0,0,0)$. Por tanto, el beneficio esperado es $(2,-1,3)$.
\item Supongamos que el jugador 3 escoge $f$. Entonces con $1/3$ $(9,-6,0)$, con $1/3$ $(0,9,-6)$ y con $1/3$ $(-6,0,9)$. Por tanto, el beneficio esperado es $(1,1,1)$.
\end{enumerate}
\end{enumerate}
\end{enumerate}

\end{solucion}
\end{document}