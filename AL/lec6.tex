\documentclass[AL.tex]{subfiles}

\begin{document}
\chapter{Estructuras de datos básicas}
Una estructura de datos es una forma de guardar y organizar datos de forma que se facilite el acceso a ellos y las actualizaciones o modificaciones necesarias.

Para realizar ciertas operaciones hacen falta estructuras de datos que permitan hacerlas en tiempo constante (queries, modificación de operadores como insertar, borar, etc.)

Algunos ejemplos de estructura de datos son:
\begin{itemize}
\item Árboles binarios de búsqueda: heap (montículo).
\item Listas.
\item Grafos. 
\end{itemize}

\section{Heap}
Un heap es un árbol binario que cumple:
\begin{enumerate}
\item Las hojas están en los dos últimos niveles.
\item En el último nivel, las hojas están a la izquierda.
\item El atributo del padre es mayor o igual que el atributo del hijo (alternativamente, menor o igual). 
\end{enumerate}

\begin{ej}
DIBUJAR ÁRBOL CON LO COMENTADO

%Entre círculos los nodos. 
%               11  
%        7              10
%    3       2        9    8
%  1   1     2

Este árbol se almacena en un vector (en el ejemplo anterior $[11,7,10,3,2,9,8,1,1,2]$). Se puede acceder a los nodos mediante las siguientes reglas: $padre(i)=i/2$, $hijo-izda(i)=2i$, $hijo-dcha(i)=2i+1$.
\end{ej}


\subsection{Construcción de un heap}
Consideremos la lista de números $[2,5,1,8,3,4]$.  Construiríamos primero un árbol cumpliendo las dos primeras propiedads de heap. Para construirlo se colocan los números de arriba a abajo y de izquierda a derecha.  Dado un árbol binario con estas propiedades, recolocamos de abajo a arriba los nodos comparando los números que tiene cada hijo con el de su padre, intercambiándolos si es necesario. 

DIBUJARLO

En general, puede construirse el heap en tiempo $O(n)$. 

\subsection{Aplicación: Heap-Sort}
%Vamos a enunciar los pasos del algoritmo junto con la aplicación al ejemplo $[2,4,1,8,3,4]$.

\begin{enumerate}
\item Construir un heap.
\item Bucle: 
\begin{itemize}
\item[2.1] Tomar la raíz.
\item[2.2] Poner el último de la lista en la raíz.
\item[2.3] Hundir la raíz por el mayor de los dos hijos y obtener un heap de $n-1$ elementos. 
\end{itemize}
\end{enumerate}
Hundir un nodo para reconstruir el heap requiere $O(\log n)$ por la altura del árbol binario. Como hay $n$ nodos, el algoritmo tiene complejidad $O(n\log n)$, que es óptima para la ordenación. 





\section{Listas}

Dentro de las listas destacamos las \emph{multilistas} (matrices dispersas). Si tenemos por ejemplo una matriz $$\begin{bmatrix}
0 & 8 & 0 & 6\\
5 & 0 & 0 & 9\\
0 & 2 & 3 & 0
\end{bmatrix}$$
(en general dimensión $n\times m$) para ahorrarnos los ceros se guarda una lista con los elementos no nulos con un puntero para cada uno de ellos que apunta a la posición en la matriz. Así guarda las matrices MATLAB. 



\section{Pilas (stacks) y colas (queues)}
Siguen el principio FILO (First In Last Out). Se comportan como una pila de platos, en el que la inserción y el borrado de datos se empieza siempre por la misma posición (la parte superior de la pila). Esta estructura de datos tiene utilidad en la implementación de algoritmos de recursión.

Las colas siguen el principio FIFO (First In First Out) y se comportan como una cola de supermercado. Por un lado se insertan los datos y por otro se extraen. El algoritmo de Dijkstra usa colas. 

Tenemos también las dobles colas, donde la inserción y el borrado se puede hacer por ambos extremos de la cola. 

\section{Grafos}
Un grafo $G=(V,E)$ puede ser dirigido o no dirigido. En general, si $|V|\in O(n)$, $|E|\in O(n^2)$. Se pueden representar mediante una matriz de adyacencia. También se pueden representar con una \emph{lista de adyacencia}. Supongamos que tenemos la matriz de adyacencia 
$$\begin{pmatrix}
1 & 1 & 1\\
0 & 0 & 1\\
0 & 1 & 0
\end{pmatrix}$$
Para el vértice 1, tenemos la lista de adyacencia $[1,2,3]$, para el 2 la lista $[3]$ y para el 3 la lista $[2]$. El espacio consumido por la matriz de adyacencia es $|V|^2$ mientras que en el caso de listas es $|V|+|E|$. En la práctica el número de aristas suele ser menor que cuadrático, por lo que suele ser más útil esta representación. 

Ejemplos de uso de las listas de adyacencia son los algoritmos BFS y DFS. Estos algoritmos son cuadráticos porque recorren todas las aristas, aunque no los repiten. Como existen ciclos si y solo si existen aristas de cruce, se pueden usar estos algoritmos para encontrarlos. También para encontrar las componentes conexas.


\end{document}

