\documentclass[AL.tex]{subfiles}

\begin{document}


%\hyphenation{equi-va-len-cia}\hyphenation{pro-pie-dad}\hyphenation{res-pec-ti-va-men-te}\hyphenation{sub-es-pa-cio}

\chapter{Paradigmas algorítmicos}
\section{Técnicas algorítmicas}
Vamos a ver las siguientes técnicas algorítmicas.
\begin{itemize}
\item Incremental
\item Divide y vencerás
\item Programación Dinámica
\item Greedy
\item Recorridos en grafos
\item Aproximaciones
\item Heurísticos
\item etc.
\end{itemize}

\subsection{Algoritmo Incremental}
Consisten en un caso base que resuelve el problema para $[a_1,\dots, a_c]$ con $c$ pequeño y paso inductivo que extiende la solución de $[a_1,\dots, a_{i-1}]$ a $[a_1,\dots, a_i]$. 
\begin{ejs}
\begin{enumerate}
\item
 Al calcular el mínimo o el máximo de un conjunto de valores $[a_1,\dots, a_n]$. Se resuelve para $[a_1,a_2]$ (caso base) y después se pasa a $[a_1,a_2,a_3]$, etc. 
 
 \item Insertion-Sort (con dos bucles).
 
 \item Cálculo de la envolvente convexa en $\R^2$. Consideramos como caso base la envolutra convexa de $n=3$ puntos, que es simplemente el triángulo que determinan. Supongamos que tenemos calculada la envolvente convexa de un conjunto de $i-1$ puntos. Si añadimos un punto $p_i$ exterior, calculamos las tangentes superior e inferior al polígono convexo que pasan por $p_i$. HACER UN DIBUJO
 
 Para ello vamos a pensar la envoltura convexa como una lista ordenada de puntos. Usamos la primitiva $$Orient(p,q,r)=\begin{vmatrix}
 1 & p_x & q_x\\
 1 & q_x & p_y\\
 1 & r_x & r_y
 \end{vmatrix}$$
 Si $Orient(p,q,r)$, los puntos están orientados en sentido antihorario; si $Orient(p,q,r)$ están orientados en sentido horario; si $Orient(p,q,r)=0$, están alineados. Se cumple: si $Orient(p,q_j,q_{j-1})=Orient(p,q_j,q_{j+1})$, entonces $q_j$ es punto de tangencia. El cálculo del determinante es constante con respecto a $n$, es decir, $O(1)$. Basta entonces hacer los tests de tangencia para encontrar los puntos que son tangentes (escan de Graham). El algoritmo tiene la siguiente procedimiento:
 \begin{enumerate}
 \item Ordenar estos puntos por abcisa ($\Theta(n\log n)$)
 \item Aplicar test desde $n=3$ de forma incremental. ($O(n)$)
 \item Actualizar la envoltura convexa. 
 \end{enumerate}
 En total el algoritmo tiene complejidad $O(n\log n)$. 
\end{enumerate}
\end{ejs}



\end{document}
