\documentclass[AL.tex]{subfiles}

\begin{document}


%\hyphenation{equi-va-len-cia}\hyphenation{pro-pie-dad}\hyphenation{res-pec-ti-va-men-te}\hyphenation{sub-es-pa-cio}

\chapter{Análisis asintótico}

\section{Complejidad}

El análisis asintótico consiste en el estudio de la eficiencia de un algoritmo en el \emph{peor caso}. 

\begin{ej}
Supongamos que tenemos un inventario, que para poder leerlo necesitamos 10000 milisegundos (ms) y 10 ms en procesar una transacción (por ejemplo clasificar un libro). Entonces, el tiempo de $n$ transacciones será $T(n)=10000+10n$. Cuando $n>>0$, el término que domina es $10n$. 
\end{ej}

La notación que usamos es la llamada Big-O (o grande): si $n$ es el tamaño de la entrada, $T(n)$ es una función de complejidad temporal (running time) y supongamos que existe una función conocida $f(n)$, decimos que $T(n)\in O(f(n))$ si $T(n)\leq c f(n)$ para alguna constante $c$ a partir de un $n$ suficientemente grande (se dice que $T$ es de orden $f(n)$). $O(f(n))$ es el conjunto de funciones que cumplen la propiedad que acabamos de definir.

\begin{ej}
$T(n)=10^6n\in O(n)$. Podemos tomar $c=10^6$ y $n_0=1$, de modo que para todo $n\geq n_0$ se cumple la definición. Se observa con este ejemplo que la notación $O(\cdot)$ prescinde de las constantes.
\end{ej}
\begin{ej}
$T(n)=n\in O(n^k)$ para todo $k\geq 1$. $T(n)=n^3+n^2+n\in O(n^3)$, por ejemplo con $c=3$ y $n_0=1$.
\end{ej}

\begin{ej}
$n^2\notin O(n)$. 
\end{ej}

\begin{ej}
$O(\cdot)$ no da toda la información. El algoritmo de búsqueda binaria tiene una función de complejidad temporal $T(n)=\log n\in O(\log n)$. Pero este es el peor caso. En el mejor caso podría encontrar el elemento al primer paso, es decir, sería $O(1)$. Existe también el estudio del caso promedio, aunque no lo estudiaremos. 
\end{ej}

\begin{ej}
$e^{3n}\notin O (e^n)$ pues $e^{3n}=e^{2n}e^n$, luego no se puede minorar por una constante (se puede probar usando límites). $10^n\notin O(2^n)$ por una razón similar.
\end{ej}

\begin{ej}
Consideremos $T(n)=n\log n$ y $U(n)=100n$. Asintóticamente $U(n)$ es más rápido, pero el $n_0$ a partir del cual se cumple esto es muy grande para la mayoría de los casos prácticos, en los que $\log n\sim 50$. Así que en la práctica es preferible $T(n)$.
\end{ej}

Introducimos la notación $\Omega(f(n))$ para referirnos al conjunto de las funciones $T(n)$ tales que $T(n)\geq df(n)$ para todo $n\geq n_0$, $d\in\N$. Es claro que $\Omega$ es inverso de $O$, es decir, $T(n)\in O(f(n))\Leftrightarrow f(n)\in\Omega(T(n))$.

\begin{ej}
$2n\in\Omega(n)$. Se puede ver por la definición o por la equivalencia con $O$. De la misma forma, $n^2\in\Omega(3n^2+n\log n)$. 
\end{ej}

La notación $\Theta(f(n))$ se refiere al conjunto $O(f(n))\cap\Omega(f(n))$. En cierto sentido, $\Theta$ representa la complejidad ``óptima''.

\begin{ej}
$n^3\in\Theta(3n^3-n^2)$. 
\end{ej}

\begin{ej}
El algoritmo para calcular el máximo de $n$ valores tiene $T(n)\in\Theta(n)$. De hecho es lineal siempre, puesto que siempre hay que comprobar todos los elementos. 
\end{ej}

\begin{ej}
$T(n)=n(1+\sin n)\in O(n)$ y $T(n)\in\Omega(0)$ ya que $\sin n\in [-1,1]$. No puede estar en $\Theta(f(n))$, porque la gráfica de la función oscila entre $2n$ y 0. Y no puede ser otra función porque tiene que estar entre esas 2. 
\end{ej}

\begin{nota}
Asumimos el modelo de computación RAM, donde hay operaciones de coste constante $O(1)$: $+,-,\times, \div, \leq$, if... llamadas funciones primitivas. Hay también operaciones más complejas como bucles, subrutinas, etc. que sí dependerán de $n$. En la práctica esto no es exactamente así, pues no se tarda lo mismo en sumar dos números pequeños que dos números muy grandes. 
\end{nota}

\subsection{Complejidades habituales}

$O(1)\subset O(\log n)\subset O(n)\subset O(n\log n)\subset O(n^2)\subset O(2^n)\subset O(n!)\subset O(n^n)$

Normalmente se considerar que un algoritmo es eficiente cuando a lo sumo está en $O(n\log n)$. Si está en $O(n^k)$ para $k\geq 2$ se llama polinomial. A partir de $O(2^n)$ se considera inútil.


%\section{Ejercicios}
%
%\begin{ejer}[La menor caja contenedora]\
%
%Input: $n$ puntos en $\R^p$. 
%
%Output: la caja $p$-dimensional de mínimo volumen que contiene a todos los puntos. 
%
%Una caja es hipercubo que tiene lados paralelos a los ejes (isotéticos). 
%
%El ejercicio consiste en dar un algoritmo (preferiblemente no inútil) y su complejidad.
%
%%dbanez@us.es
%\end{ejer}

\section{Análisis de algoritmos}

\subsection{Problema 1}
Dado un conjunto de items, decidir si hay o no dos iguales. Existe un algoritmo sencillo que es cuadrático, ya que hay $\binom{n}{2}$ pares, y se trata de comparar los elementos de los pares. Otra posibilidad es ordenar la lista, lo cual ya nos detectaría si hay dos iguales ya sea en el propio ordenamiento o bien luego mirando dos seguidos. Así que podemos mejorar la eficiencia si encontrar una forma de ordenar de complejidad menor que cuadrática, el cual existe (en concreto $O(n\log n)$).

\subsection{Problema 2}
Dados $m$ hombres y $w$ mujeres, ver qué pareja $(w,m)$ es compatible, suponiendo que comprobar la compatibilidad es de complejidad constante. %Podría darse una descripción menos sexista, pero esta es la que ha dado el profesor. 
El algoritmo exhaustivo consistiría en realizar todas las parejas posibles y comprobar si son compatibles, que tiene orden $O(wm)$. Obsérvese que $w$ y $m$ son variables independientes, por lo que $O(wm)\neq O(w^2)$. 


A propósito de este último comentario, nos preguntamos si $\Theta(w^6+wm+m^3)=\Theta(w^6+m^3)$. Si $w>>m$, entonces $w^6+wm+m^3$ estará en la misma clase que $w^6+m^3$, que es $\Theta(w^6)$. Si $w<<m$ podemos hacer el razonamiento análogo, con lo que pertenecería a $\Theta(m^3)$. Así que la respuesta es afirmativa. Se puede probar haciendo el cociente y tomando límtie, que da 1 y por tanto efectivamente son del mismo orden. 

Sin embargo $\Theta(w^6+wm)\neq\Theta(w^6)$, porque si domina $m$ no se tiene la igualdad. 

\subsection{Problema 3}
Hay un arreglo ordenado por título de $n$ CD's. Existen $k$ que comienzan por ``The best of...''. Se trata de obtener todos los de este tipo. Hacemos búsqueda binaria. En el mejor caso llegas a un disco de ellos y avanzas y retrocedes para cogerlos todos, con lo que sería $O(k)$. En el peor caso hay $O(\log n)$ pasos y luego $O(k)$, o sea, $O(\log n +k)$. Este es un algoritmo \emph{output sensitive}, porque dependiendo del tamaño de la salida tiene una complejidad. 


\subsection{Problema 4}
Ordenar un conjunto $L$ de items. Vamos a ver el algoritmo Merge-Sort. 
\begin{enumerate}
\item Se divide $L$ en dos sublistas $L_1$ y $L_2$ de igual (lo más igual posible) tamaño. $T(n)=O(1)$
\item Se ordena cada sublista con llamadas recursivas. $T(n)=2T(\frac{n}{2})$
\item Se mezclan $L_1$ y $L_2$.  Para ello se comparan los mínimos y se elimina el que se ha elegido. $T(n)=O(n)$
\end{enumerate}

Algoritmo recursivo con 
\[
T(n)=\begin{cases}
O(1) & n\leq 2\\
2T(\frac{n}{2})+\Theta(n) & n>2
\end{cases}
\]
Tenemos que resolver la ecuación de recursión $T(n)=2T(\frac{n}{2})+\Theta(n)$. Para ello hacemos lo siguiente
\[
2T\left(\frac{n}{2}\right)+\Theta(n)=2(2T\left(\frac{n}{2^2}+\Theta\left(\frac{n}{2}\right)\right)+\Theta(n)=2^2T\left(\frac{n}{2^2}\right)+2\Theta(n)
\]
Seguimos aplicando esta estrategia y obtenemos $T(n)=2^kT(\frac{n}{2^k})+k\Theta(n)$. Cuando $n=2^k\Rightarrow k\in O(\log n)$ se termina, obteniendo $T(n)=nT(1)+O(\log n)\Theta(n)=O(n)+O(n\log n)=O(n\log n)$. 

Vemos ahora un teorema general que nos ayudará a resolver ecuaciones de recurrencia como la anterior.

\begin{teorema}[Teorema del maestro]

Sea $$T(n)=\begin{cases}
O(1) & n\leq n_0\\
aT(\frac{n}{b})+\Theta(f(n)) & n> n_0
\end{cases}$$ 
\begin{enumerate}
\item Si $f(n)\in O(n^{\log_ba-\varepsilon})$, $\varepsilon\in [0,1]$, entonces $T(n)\in\Theta(n^{\log_ba})$.
\item Si $f(n)\in \Theta(n^{\log_ba})$, entonces $T(n)\in\Theta(n^{\log_ba}\log n)$.
\item Si $f(n)\in\Omega(n^{\log_ba+\varepsilon})$, $\varepsilon\in[0,1]$, y $af(\frac{n}{b})\leq cf(n)$ para $c<1$ y $n$ suficientemenre grande, entonces $T(n)\in\Theta(f(n))$. 
\end{enumerate}

\end{teorema}
En el enunciado del teorema, $a$ representa el número de llamadas recursivas, $b$ las partes en las que se divide el conjunto y $f(n)$ es el tiempo necesario para ``procesado extra''. La prueba del resultado puede encontrarse en el libro de Cormen \emph{Introducton to Algorithms}. En dicha prueba se muestra además que el costo de todas las llamadas recursivas es $\Theta(n^{\log_ba})$ y el costo del procesado extra es $\Theta(f(n))$. En función de cuál de estos costos domine más, obtenemos cada uno de los casos del teorema. 

\begin{ejs}\
\begin{enumerate}
\item En el caso de Merge-Sort, $a=b=2$, con lo que $\log_22=1$. En este caso $f(n)=\Theta(n)$ (abuso de lenguaje), que coincide con el caso 2 del teorema. Así que $T(n)\in\Theta(n\log n)$, que fue lo que demostramos a mano. 

\item La búsqueda binaria tiene una recurrencia $T(n)=T(\frac{n}{2})+O(1)$, puesto que se divide la lista en 2 tras hacer una pregunta para decidir con qué mitad quedarnos. Así que $a=1$, $b=2$, con lo que $\log_21=0$ y $f(n)=O(1)\in O(n^0)$. Volvemos a estar en el caso 2 del teorema, con lo que $T(n)=\Theta(\log n)$, cosa que hay conocíamos. 

\item Resolvemos $T(n)=4T(\frac{n}{2})+n^3$. Claramente $a=4$, $b=2$ y $f(n)=n^3$, de donde $\log_24=2$. Como $n^3\in\Omega (n^{2+\varepsilon})$ para cualquier $\varepsilon\leq 1$ y $4(\frac{n^3}{8})=\frac{1}{2}n^3$, estamos en las hipótesis del caso 3 del teorema. Por ello concluimos que $T(n)\in\Theta(n^3)$. 

\item $T(n)=2T(\frac{n}{2})+n^2$. Tenemos aquí $a=2$, $b=2$ y $f(n)=n^2$. De aquí $\log_22=1$, y como $n^2\in\Omega(n^{1+\varepsilon})$ para cualquier $\varepsilon\leq 1$, esatmos en el tercer caso. Esto nos da $T(n)\in\Theta(n^2)$. 
\end{enumerate}
\end{ejs}

\begin{coro}[versión para $f(n)=\Theta(n^k)$]
En las condiciones del teorema
\begin{enumerate}
\item Si $k<\log_ba$, entones $T(n)\in\Theta(n^{\log_ba})$.
\item Si $k=\log_ba$, entonces $T(n)\in\Theta(n^{\log_ba}\log n)$.
\item Si $k>\log_ba$, entonces $T(n)\in\Theta(n^k)$. 
\end{enumerate}
\end{coro}

Atención, el teorema del maestro no resuelve todos los casos, como vemos a continuación.
\begin{ej}
Sea la ecuación de recurrencia $T(n)=2T(\frac{n}{2})+\Theta(n\log n)$. Aquí $a=b=2$ y $f(n)=n\log n$. Aquí $f$ no verifica ninguno de los dos casos. El más interesante de ver es el tercero. $n^{1+\varepsilon}=nn^{\varepsilon}$. Con lo que una $n$ se compensa con la de $n\log n$, luego tendríamos que comparar $n^{\varepsilon}$ con $\log n$. Se puede demostrar que $\frac{n^{\varepsilon}}{\log n}\to\infty$ cuando $n\to\infty$, por lo que efectivamente $n^{1+\varepsilon}\notin O(n\log n)$. Como $f$ no está en ninguno de los casos, el teorema del maestro no es aplicable. Sin embargo sí lo será el siguiente teorema.
\end{ej}

\begin{teorema}
Si $(n)=aT(\frac{n}{b})+\Theta(n^{\log_ba}\log^kn)$, entonces $T(n)\in\Theta(n^{\log_ba}\log^{k+1}n)$.
\end{teorema}

En el ejemplo anterior podemos aplicar este teorema para $k=1$, con lo que $T(n)\in\Theta(n\log^2n)$. 

\end{document}
