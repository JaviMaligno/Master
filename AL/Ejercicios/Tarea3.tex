	\documentclass[twoside]{article}
\usepackage{../../estilo-ejercicios}
\renewcommand{\baselinestretch}{1,3}
%--------------------------------------------------------
\begin{document}

\title{Tarea 3 de Algorítmica}
\author{Javier Aguilar Martín}
\maketitle


\begin{ejercicio}{Lower Bound for Binary Search}
Supongamos que hemos procesado $n$ elementos formando una lista ordenada. Probar que encontrar un elemento concreto entre ellos en el \emph{Comparison Model} requiere un tiempo $\Omega(\log n)$. 
\end{ejercicio}
\begin{solucion}

%tenemos cota superior porque el binary search de irte al medio y ver si coger la mitad superior o inferior es log(n)


\end{solucion}



\end{document}
