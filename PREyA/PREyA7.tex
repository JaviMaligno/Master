\documentclass[PREyA.tex]{subfiles}

\usepackage{tikz}
\usetikzlibrary{automata,positioning}
\providecommand{\abs}[1]{\left\lvert#1\right\rvert}
\begin{document}

\chapter{Procesos de renovación}
\section{Introducción}
\begin{defi}Seam $X_i$ v.a. independientes e idénticamente distribuidas.
\end{defi}
\begin{defi}
Consideremos $p_{ij}=P[X=x_i,Y=y_j]$. Entonces definimos la probablidad condicionada de $Y$ a $X=x_i$ como
$$
P[Y=y_i \mid X=x_i] = \frac{p_{ij}}{P[X=x_i]}
$$
Si consideramos la esperanza de $Y$ condicionado a $X=x_i$ 
$$
E[Y\mid X=x_i] = \sum_j y_j P[Y=y_j \mid X=x_i]
$$
\end{defi}
\begin{defi}
Definimos la esperanza de $Y$ condicionada a $X$ como
$$
E[Y\mid X]:= g(X)
$$
donde $g$ es la función
$$
g(x)=\begin{cases}
E[Y\mid X=x] & P[X=x]>0\\
0 & P[X=x]=0
\end{cases}
$$
\end{defi}
Los conceptos anteriores pueden definirse análogamente para el caso absolutamente continuo. En tal caso tendríamos una función de densidad $f(x,y)$.
\begin{defi}
Definimos la funcion de densidad de $Y$ a $X=x_i$ como
$$
f_{Y|X=x}(y)=\frac{f(x,y)}{f_X(x)}
$$
Si consideramos la esperanza de $Y$ condicionado a $X=x_i$ 
$$
E[Y\mid X=x_i] = \int y \frac{f(x,y)}{f_X(x)} dy
$$
\end{defi}
La definición de la esperanza de $Y$ condicionada a $X$ es exactamente la misma.
\subsection{Suma de v.a. independientes}
La función de probabilidad y de densidad (caso discreto y continuo) es simplemente la convolución. En el caso continuo
$$
f_Z(x)=f_X * f_Y (z) = \int f_X(z-y)f_Y(y)dy = \int f_X(x)f_Y(z-x)dx
$$
En el caso discreto
$$
P[Z=z] = \sum_k P[X=k]P[Y=z-k]

$$
\end{document}
