
\documentclass[twoside]{article}
\usepackage{../estilo-ejercicios}
\usepackage{enumerate}
\usepackage{mathtools}
\DeclarePairedDelimiter\floor{\lfloor}{\rfloor}
\newcommand{\x}{{\mathbf{x}}}
\newcommand{\y}{{\mathbf{y}}}
\usepackage{tikz}
\usetikzlibrary{automata,positioning}
%--------------------------------------------------------
\begin{document}

\title{Procesos Estocásticos. Aplicaciones}
\author{Rafael González López}
\maketitle
\begin{ejercicio}{1} En una cola $M/M/1$ en régimen estacionario, calcular la distribución de espera en cola, $W_{q}$.
\end{ejercicio}
\begin{solucion}
Sea $N$ el número de clientes que están en el sistema en un momento determinado. Si $N=0$ entonces no tenemos que esperar nada, luego $(W_q|N=0) \sim 0$ (distribución degenerada). En el caso $N = n >0$, tenemos que esperar que el primero termine el servicio ($\sim Exp(\mu)$), así como los servicios de los $n-1$ restantes. Por tanto, se tiene que $(W_q|N=n)\sim Ga(n,\mu)$. Recordemos que $P[N=n] = (1-\rho)\rho^n$. Tenemos entonces
\begin{align*}
P[W_q \leq w] &= \sum_{n=0}^\infty P[W_q \leq w \mid N=n]P[N=n]\\
&=(1-\rho) + \sum_{n=1}^\infty P[W_q \leq w \mid N=n](1-\rho)\rho^n\\
&=(1-\rho) + (1-\rho)\sum_{n=1}^\infty \rho^n \int_0^w \frac{\mu^n x^{n-1}e^{-\mu x}}{(n-1)!}dx
\end{align*}
Intercambiamos suma con integral y tengamos en cuenta que $\rho^n \mu^n = \lambda^n$.
\begin{align*}
P[W_q\leq w] &=(1-\rho) + (1-\rho) \int_0^w \sum_{n=1}^\infty \frac{\lambda^n x^{n-1}e^{-\mu x}}{(n-1)!}dx\\
&= (1-\rho) + (1-\rho)\int^w_0 \lambda e^{-\mu x}e^{\lambda x}dx\\
&= (1-\rho) + (1-\rho)\lambda \left[\frac{e^{(\lambda-\mu)x}}{\lambda-\mu}\right]_0^w\\
&= 1-\rho e^{(\lambda-\mu)w}
\end{align*}
\end{solucion}
\newpage


\begin{ejercicio}{2} 
Para la cola $M/M/1/K$ en régimen estacionario: 
\begin{enumerate}[(a)]
	\item Calcular la distribución de $N=$ número de clientes en el sistema.
	\item Calcular la función de densidad de $W=$ tiempo de espera en el sistema.
	\item ¿Se verifica la fórmula de Little $\mathbb{E}N= \lambda \mathbb{E}W$?
\end{enumerate}
\end{ejercicio}
\begin{solucion}
\begin{enumerate}[(a)]
\item[]
\item Este modelo es parecido al que hemos estudiado antes, con la salvedad de que en el anterior la capacidad del sistema era infinita y ahora solo puede tener hasta $K$ individuos. Aun así, podemos razonar en el cálculo de $\rho_n = P[N=n]$ de la misma forma que en el caso de capacidad infinita a través de las ecuaciones de flujo, obteniendo que
$$
\rho_n  = \rho^n \rho_0
$$
pero con la particularidad de que $n\leq K$. Si imponemos
$$
1 = \sum_{n=0}^K \rho_n  = \rho_0 \sum_{n=0}^K \rho^n = \begin{cases}
\rho_0 (K+1) & \rho= 1\\ \\
\rho_0 \dfrac{1-\rho^{K+1}}{1-\rho}& \rho\neq 1
\end{cases}
$$
por tanto
$$
\rho_0  = \begin{cases}
\dfrac{1}{K+1} & \rho = 1 \\ \\
\dfrac{1-\rho}{1-\rho^{K+1}} & \rho \neq 1
\end{cases}
$$
Por tanto, 
$$
P[N=n] = \begin{cases}
\dfrac{1}{K+1} & \rho = 1 \\ \\
\dfrac{\rho^n(1-\rho)}{1-\rho^{K+1}} & \rho \neq 1
\end{cases}
$$
\begin{itemize}
\item Si $\rho=1$ entonces, considerando siempre tiempos positivos, 
$$
F_N(t) = \sum_{n=0}^{\floor{t}} P[N=n] = \frac{\floor{t}+1}{K+1}1_{(t\leq K)}(t) + 1_{(t>K)}(t)
$$
\item Análogamente, si $\rho\neq 1$ entonces
$$
F_N(t) = \sum_{n=0}^{\floor{t}} P[N=n] = \frac{1-\rho^{\floor{t}+1}}{1-\rho^{K+1}}1_{(t\leq K)}(t) + 1_{(t>K)}(t) 
$$
\end{itemize}
\item Para calcular la densidad del tiempo de espera en el sistema también realizamos un análisis análogo al caso estudiado. Si $N=0$ entonces no hay nadie en el sistema y tan solo tenemos que esperar nuestro tiempo $(\sim Exp(\mu))$. Si $N=n$ entonces hay que esperar los $n$ tiempos de servicio y el nuestro, luego $(W\mid N=n) \sim Ga(n+1,\mu)$. La particularidad de este caso es que $0\leq n \leq K-1$, pues ya en $n\geq K$ no podemos acceder a la cola del servicio. Dado que $Exp(\mu) \sim Ga(1,\mu)$. Tengamos en cuenta que no podemos tomar exactamente la variable $N$, pues tenemos que suponer que $N\leq K-1$. Por tanto, pasamos a utilizar $N^*:=(N\mid N\leq K-1)$. Naturalmente, $P[N^*=n] = P[N=n]/(1-P[N=K])$.
\begin{itemize}
\item Por tanto, supongamos que $\rho=1$. En este caso $$P[N^*=n] = 1/(K+1)/(K/(K+1)) = 1/K$$
Calculemos la densidad
\begin{align*}
P[W \leq w] &= \sum_{n=0}^{K-1} P[W \leq w \mid N^*=n]P[N^*=n]\\
&= \frac{1}{K}\sum_{n=0}^{K-1} F_{Ga(n+1,\mu)}(t) 
\end{align*}
La función de densidad consiste simplemente en derivar lo anterior
\begin{align*}
f(w) &= \frac{1}{K}\left(\sum_{n=0}^{K-1} \frac{\mu^{n+1} w^{n}e^{-\mu x}}{(n+1)!}\right)
\end{align*}
\item Si $\rho\neq 1$, entonces
$$
P[N^*=n] = \frac{\rho^n(1-\rho)(1-\rho^{K+1})^{-1}}{1-\rho^K(1-\rho)(1-\rho^{K+1})^{-1}} = \frac{\rho^n(1-\rho)}{1-\rho^K}
$$
Análogamente al caso anterior
\begin{align*}
P[W \leq w] &= \sum_{n=0}^{K-1} P[W \leq w \mid N^*=n]P[N^*=n]\\
&= \sum_{n=0}^{K-1} \frac{\rho^n(1-\rho)}{1-\rho^K} F_{Ga(n+1,\mu)}(t)
\end{align*}
Por tanto
$$
f(w) = \sum_{n=0}^\infty  \frac{\rho^n(1-\rho)}{1-\rho^K} \frac{\mu^{n+1} w^{n}e^{-\mu x}}{(n+1)!}
$$
\end{itemize}
\item Calculemos las esperanzas. 
\begin{itemize}
\item Si $\rho = 1$,
\begin{align*}
\mathbb{E}N & = \sum_{n=0}^K n \frac{1}{K+1}  = \frac{1}{K+1}\sum_{n=1}^K n = \frac{1}{K+1}\frac{K(K+1)}{2} = \frac{K}{2}
\end{align*}
Para calcular la esperanza de $W$ basta tener en cuenta que si $X \sim Ga(a,b)$ entonces $E[X] = a/b$. Usando la linealidad de la integral, podemos escribir simplemente
\begin{align*}
\mathbb{E}W &= \frac{1}{K}\sum_{n=0}^{K-1} \frac{n+1}{\mu} = \frac{K+1}{2\mu} = \frac{1}{\lambda}\frac{K+1}{2}
\end{align*}
Usando que $\mu = \lambda$. Por tanto, no se cumple literalmente la Ley de Little. Sin embargo, si consideramos $\lambda^* = \lambda (1-\rho_K)$ entonces sí se verifica.
\item Si $\rho\neq 1$, entonces
\begin{align*}
\mathbb{E}N &= \frac{1-\rho}{1-\rho^{K+1}}\sum_{n=0}^K n \rho^n \\
& = \frac{1-\rho}{1-\rho^{K+1}}\rho \sum_{n=1}^K \frac{d}{d\rho}\left(\rho^n\right) \\
&=\frac{1-\rho}{1-\rho^{K+1}}\rho \frac{d}{d\rho}\left(\frac{\rho(1-\rho^K)}{1-\rho}\right)\\
&= \rho\frac{1-(K+1)\rho^K+K\rho^{K+1}}{(1-\rho)(1-\rho^{K+1})}
\end{align*}
Para $W$ consideremos, análogamente al caso anterior, podemos 
\begin{align*}
\mathbb{E}W &= \sum_{n=0}^{K-1} \frac{n+1}{\mu} \frac{\rho^n(1-\rho)}{1-\rho^K}\\
&= \frac{1-\rho}{\mu (1-\rho^{K})}\sum_{n=0}^{K-1} (n+1)\rho^n\\
&= \frac{1-\rho}{\mu(1-\rho^K)}\frac{1-(K+1)\rho^K-K\rho^{K+1}}{(1-\rho)^2}\\
&=\frac{1}{\mu}\frac{1-(K+1)\rho^K-K\rho^{K+1}}{(1-\rho^K)(1-\rho)}
\end{align*}
Nuevamente, no se verifica que $\mathbb{E}N = \lambda \mathbb{E}W$, sin embargo volviendo a tomar $\lambda^* = \lambda(1-\rho_K)$ sí que se verifica.
\end{itemize}
\end{enumerate}
\end{solucion}
\newpage
\begin{ejercicio}{3}
 Para la cola $M/M/c$, en régimen estacionario:
\begin{enumerate}[(a)]
\item Probar que
$$
C(c,a):= \mbox{probabilidad de que un cliente espere en cola } = \frac{ \frac{a^{c}}{c! (1- \rho)}}{  \sum_{n=0}^{c-1} \frac{a^{n}}{n!} + \frac{a^{c}}{c! (1-\rho)} }
$$ 
donde $a=\lambda / \mu$ y $\rho = a/c$.
\item Probar que el número medio de clientes en cola es:	
$$
\mathbb{E} N_{q} = \dfrac{ \rho C(c,a)}{1- \rho}
$$
\item Probar que la fdd del tiempo de espera en el sistema, $W$ es:
$$
f_{W}(t) = (1-C(c,a)) f_{Exp(\mu)} (t) + \displaystyle \sum_{n=c}^{\infty} \pi_{n} ( f_{Ga(n-c+1,c\mu)} * f_{Exp(\mu)})(t)
$$
donde $\pi_{n}$ es la probabilidad de que haya $n$ clientes en el sistema.
\item Probar que $\mathbb{E}W = \frac{1}{\mu} \left( 1+ \dfrac{C(c,a)}{c ( 1- \rho)} \right)$.
\item ¿Se verifica la fórmula de Little? 	
\end{enumerate}
	
\end{ejercicio}
\begin{solucion}
\begin{enumerate}[(a)]
\item[]
\item Sea $N$ la cantidad de clientes en el sistema $C(c,a)=P[N\geq c] = 1-P[N\leq c-1]$. Por tanto, vamos a calcular $P[N=n]$. Para ello, podemos utilizar también los diagramas de flujo. Podemos ver el diagrama cola $M/M/c$ como uno en el que  en los primeros $c$ estados, la tasa de entrada nuevos individuos es $\lambda$ y la de salida $i \mu$ con $1 \leq i \leq c$, y $c\mu$ para todos los demás estados, tal y como vimos en la primera parte de la asignatura. La recursión en $a$ es sencillamente
$$
a_n = \begin{cases}
\frac{a^n}{n!}a_0 & n\leq c\\
\frac{a^n}{c^{n-c}c!}a_0 & n\geq c
\end{cases}
$$
Para encontrar $a_0$ sumamos e imponemos que la suma sea $1$.
$$
a_0 = \frac{1}{\sum_{n=0}^\infty a_n/a_0} = \frac{1}{\sum_{n=0}^{c-1}\frac{a^n}{n!} + \sum_{n=c}\frac{a^n}{c^{n-c}c!}} = \frac{1}{\sum_{n=0}^{c-1}\frac{a^n}{n!} + \frac{a^c}{c!(1-\rho)}} 
$$
Por tanto
\begin{align*}
C(c,a)&= 1- \sum_{n=0}^{c-1} a_n\\
&= 1 - \sum_{n=0}^{c-1}\frac{\frac{a^n}{n!}}{\sum_{n=0}^{c-1}\frac{a^n}{n!} + \frac{a^c}{c!(1-\rho)}}\\
&=1 -\frac{ \sum_{n=0}^{c-1}\frac{a^n}{n!}}{\sum_{n=0}^{c-1}\frac{a^n}{n!} + \frac{a^c}{c!(1-\rho)}}\\
&=\frac{\frac{a^c}{c!(1-\rho)}}{\sum_{n=0}^{c-1}\frac{a^n}{n!} + \frac{a^c}{c!(1-\rho)}}
\end{align*}
\item Tengamos en cuenta que $N_q = 0$ si $N\leq c$ y $N_q = N-c$ si $N\geq c+1$. De estas igualdades podemos derivar directamente que
$$
P[N_q = n] = \begin{cases}
\sum_{n=0}^c \frac{a^n}{n!}a_0 & n=0\\
\frac{a^{n+c}}{c^nc!}a_0 & n>0
\end{cases}
$$
Vamos a calcular la esperanza
\begin{align*}
\mathbb{E}N_q &= \sum_{n=1}^\infty n P[N_q =n] \\
& = \sum_{n=1}^\infty n \frac{a^{n+c}}{c^n c!}a_0\\
&= a_0\frac{a^{c+1}}{(c-1)!}\sum_{n=1}^\infty n \left(\frac{a}{c}\right)^{n-1}\\
&= a_0\frac{a^{c+1}}{(c-1)!}\frac{c^2}{(c-a)^2}\\
&= a_0\frac{a^{c+1}}{c!(1-\rho)^2} \\
&= \rho\frac{C(c,a)}{1-\rho}
\end{align*}
\item Denotemos, para ser consistente con el enunciado $\pi_n = a_n$, que es como hemos estado denotando $P[N=n]$. Comencemos nuestro análisis. Si $N\leq c-1$, entonces tan solo tenemos que esperar nuestro tiempo en el sistema. Por tanto $(W\mid N = n) \sim Exp(\mu)$ para $n\leq c-1$. Si $n\geq c$ entonces los $n-c$ clientes que están en cola deben esperar (y 1 más por nosotros) que alguien salga de la cola, luego tenemos que sumar $n-c+1$ mínimo de los $c$ servidores con $Exp(\mu)$, además de nuestro propio tiempo en el servicio que es $Exp(\mu)$. Por tanto, $Ga(n+1-c,c\mu) + Exp(\mu)$. Tenemos
\begin{align*}
F_W(t) &= \sum_{n=0}^\infty P[W\leq w\mid N=n]P[N=n]\\
&=\sum_{n=0}^{c-1} \frac{a^n}{n!}a_0 F_{Exp(\mu)}(t) + \sum_{n=c}^\infty F_{Ga(n+1-c,c\mu)+Exp(\mu)}(t) \pi_n \\
&= (1-C(c,a))F_{Exp(\mu)}(t) +\sum_{n=c}^\infty F_{Ga(n+1-c,c\mu)+Exp(\mu)}(t) \pi_n 
\end{align*}
Derivamos y tenemos en cuenta que la densidad de la suma es la convolución de las densidades
$$
f_W(t) = (1-C(c,a))f_{Exp(\mu)}(t) + \sum_{n=c}^\infty \pi_n (f_{Ga(n+1-c,c\mu)}\ast f_{Exp(\mu)})(t)
$$
\item Vamos a intentar calcular la esperanza de $W$ a partir de la descripción de la densidad que hemos obtenido anteriormente. Usando la linealidad de la esperanza analizamos sumando a sumando. 
\begin{align*}
\int_0^\infty t f_{Exp(\mu)}(t)((1-C(c,a))dt & = ((1-C(c,a))\frac{1}{\mu} \\
\int_0^\infty t(f_{Ga(n+1-c,c\mu)}\ast f_{Exp(\mu)})(t)dt &= \frac{1}{\mu} + \frac{n+1-c}{c\mu}
\end{align*}
Por tanto, tenemos
\begin{align*}
\mathbb{E}W &= \frac{1}{\mu}\left( (1-C(c,a)) + \sum_{n=c}^\infty \pi_n \left(1+\frac{n+1-c}{c}\right)\right)\\
&=\frac{1}{\mu}\left( (1-C(c,a)) + a_0\sum_{n=c}^\infty \frac{a^n}{c^{n-c}c!}\left(1+\frac{n+1-c}{c}\right)\right)\\
&=\frac{1}{\mu}\left( (1-C(c,a))+ {C(c,a)} + a_0\sum_{n=c}^\infty \frac{a^n}{c^{n-c}c!}\frac{n+1-c}{c}\right)\\
&=\frac{1}{\mu}\left( 1 + \frac{a^c a_0}{cc!}\sum_{n=0}^\infty \left(\frac{a}{c}\right)^n (n+1)\right)\\
&= \frac{1}{\mu}\left( 1 + \frac{a^c a_0}{c c! (1-\rho)^2}\right)\\ 
&= \frac{1}{\mu}\left( 1+ \frac{C(c,a)}{c(1-\rho)}\right) 
\end{align*}
\item Para comprobar si se verifica, calculamos la $\mathbb{E}N$.
\begin{align*}
\mathbb{E}N &= \sum_{n=1}^\infty n a_n \\
&=  \sum_{n=1}^{c} \frac{a^n}{(n-1)!}a_0 + \sum_{n=c+1}^\infty n\frac{a^n}{c^{n-c}c!}a_0\\
&= a\sum_{n=0}^{c-1} \frac{a^n}{n!}a_0 + \frac{a^{c+1}}{c c!}\sum_{n=0}^\infty \frac{(n+c+1)a^n}{c^n}a_0\\
&= a(1-C(c,a)) + a \frac{a^c}{cc!}\frac{c+1-a}{(1-\rho)^2}a_0\\
&= a\left(1 - C(c,a) + \frac{c+1-a}{c(1-\rho)}C(c,a)\right)\\
&= \frac{\lambda}{\mu}\left(1 + \frac{C(c,a)}{c(1-\rho)}\right) = \lambda \mathbb{E}W
\end{align*}
Por tanto, se verifica.
\end{enumerate}
\end{solucion}
\end{document}