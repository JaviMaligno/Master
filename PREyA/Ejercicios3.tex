\documentclass[]{article}
\usepackage[utf8]{inputenc}
\usepackage{float}
\usepackage{pdfpages}
\usepackage[utf8]{inputenc}
\usepackage{graphicx}
\usepackage{enumerate}
\usepackage{parskip}
\usepackage{amsmath,amsthm,amsfonts,amssymb,fancyhdr,graphics}
\usepackage[spanish]{babel}
\usepackage{amssymb, amsmath, amsbsy} % simbolitos
\usepackage{upgreek} % para poner letras griegas sin cursiva
\usepackage{cancel} % para tachar
\usepackage{mathdots} % para el comando \iddots
\usepackage{mathrsfs} % para formato de letra
\usepackage{stackrel} % para el comando \stackbin
\theoremstyle{plain}
\usepackage{vmargin}
\usepackage{ragged2e}
\usepackage[spanish]{babel}
\usepackage[usenames,dvipsnames,svgnames,table]{}
\usepackage{ragged2e}
\usepackage[utf8]{inputenc}
\usepackage{graphicx}
\usepackage{multirow}
\usepackage{listings} %para incluir codigo Octave
\justifying %estilo justificado
\newtheorem{teorema}{Teorema}[section]
\newtheorem{definicion}{Definición}[section]
\newtheorem{proposicion}{Proposición}[section]
\newtheorem{ejercicio}{Ejercicio}[section]
\newtheorem{lema}{Lema}[section]
\newtheorem{coro}{Corolario}[section]
\newtheorem{ejemplo}{Ejemplo}[section]
\newenvironment{demostracion}{\begin{proof}[Demostraci\'on]}{\end{proof}}
\newenvironment{solucion}{\begin{proof}[Soluci\'on]}{\end{proof}}

\setpapersize{A4}
\setmargins{2.5cm}       % margen izquierdo
{1.5cm}                        % margen superior
{16.5cm}                      % anchura del texto
{23.42cm}                    % altura del texto
{10pt}                           % altura de los encabezados
{1cm}                           % espacio entre el texto y los encabezados
{0pt}                             % altura del pie de página
{1.5cm}                           % espacio entre el texto y el pie de página

\renewcommand{\baselinestretch}{1.5}
%opening

\title{EJERCICIOS TEORÍA DE COLAS \\ Procesos Estocásticos. Aplicaciones }

\author{Andrea Prieto García}
\date{}

\begin{document}

\maketitle

\textcolor{blue}{\textbf{EJERCICIO 1.}} En una cola $M/M/1$ en régimen estacionario, calcular la distribución de espera en cola, $W_{q}$.


\medskip
\textcolor{blue}{\textbf{EJERCICIO 2.}} Para la cola $M/M/1/K$ en régimen estacionario: 
\begin{enumerate}[(a)]
	\item Calcular la distribución de $N=$ número de clientes en el sistema.
	\item Calcular la función de densidad de $W=$ tiempo de espera en el sistema.
	\item ¿Se verifica la fórmula de Little $\mathbb{E}N= \lambda \mathbb{E}W$?
\end{enumerate}


\begin{solucion}
	Esta cola marcoviana es un proceso de nacimiento y muerte. La evolución de la cola se puede interpretar como competición de relojes.
	
	Hay una limitación de clientes en el sistema, como se puede observar
	\begin{enumerate}[(a)]
		\item Calculamos $\underline{\pi} = G \underline{\pi}$ sujeto a $\displaystyle \sum_{j=0}^{k} \pi_{j} =1$. Siempre hay régimen estacionario ya que el número de estados es finito y están comunicados. 
		
		\item Para calcular la distribución de $W$
		$$\begin{cases}
		\pi_{0} : \mbox{ si la cola está vacía } & \Rightarrow Exp(\mu) \\
		\pi_{n}: \mbox{ si hay }n \mbox{ clientes } & \Rightarrow Ga(n+1, \mu)
		\end{cases}$$
		Si hay $n$ clientes, para $n=1,...,k$ lo que tendría que esperar el cliente que llega sería: \newline
		1. El resto de servicio del cliente que está siendo servido $\sim Exp(\mu)$ \newline
		2. El tiempo de espera de los $n-1$ clientes restantes $\sim Ga(n-1, \mu)$
		 \newline
		3. su tiempo de servicio $\sim Exp(\mu)$ \newline 
		Luego 
	\end{enumerate}
	
	
\end{solucion}

\textcolor{blue}{\textbf{EJERCICIO 3.}} Para la cola $M/M/c$, en régimen estacionario:
\begin{enumerate}[(a)]
	\item Probar que
	$$ C(c,a):= \mbox{ probabilidad de que un cliente espere en cola } = \dfrac{ \dfrac{a^{c}}{c! (1- \rho)}}{ \displaystyle \sum_{n=0}^{c-1} \dfrac{a^{n}}{n!} + \dfrac{a^{c}}{c! (1-\rho)} }$$ donde $a=\lambda / \mu$ y $\rho = a/c$.
		\item Probar que el número medio de clientes en cola es:
	
	$$ \mathbb{E} N_{q} = \dfrac{ \rho C(c,a)}{1- \rho}  $$
	
	\item Probar que la fdd del tiempo de espera en el sistema, $W$ es:
	
	$$ f_{W}(t) = (1-C(c,a)) f_{Exp(\mu)} (t) + \displaystyle \sum_{n=c}^{\infty} \pi_{n} ( f_{Ga(n-c+1,\mu)} * f_{Exp(c\mu)})(t)   $$ donde $\pi_{n}$ es la probabilidad de que haya $n$ clientes en el sistema.
	
	\item Probar que $\mathbb{E}W = \frac{1}{\mu} \left( 1+ \dfrac{C(c,a)}{c ( 1- \rho)} \right)$.
	
	\item ¿Se verifica la fórmula de Little? 
		
	\end{enumerate}
	
	\begin{solucion}
		
		\begin{enumerate}[(a)]
			\item 
		\end{enumerate}
		
	\end{solucion}
	
	
	
	
	




\end{document}