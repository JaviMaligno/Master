	\documentclass[twoside]{article}
\usepackage{../../estilo-ejercicios}

%--------------------------------------------------------
\begin{document}

\title{Relación 1}
\author{Javier Aguilar Martín}
\maketitle


\begin{ejercicio}{1.1}
Prueba que todo grafo bipartito de orden $n \geq 3$ es también $r$-partito para $3\leq r \leq n$. ¿Es
cierto el recíproco?
\end{ejercicio}
\begin{solucion}
Sea $G$ un grafo en las condiciones del enunciado. Si $V=V_1\cup V_2$ es la descomposición de $V=V(G)$ como grafo bipartito, basta subdividir $V_1$ y $V_2$ en un total de $r$ subconjuntos disjuntos. Como $n\geq 3$, alguno de los subconjuntos tiene al menos 2 vértices, digamos $V_1$, por lo que podemos subdividirlo en $V_1^1$ y $V_2^1$. Si $n=3$ ya habríamos terminado. Si $n>3$, entonces un subconjunto de entre $V_1^1, V_2^1$ y $V_2$ tiene al menos dos elementos y podemos reiterar el proceso. De este modo, conseguimos que en cada paso sea $r$-partito para un $r$ cada vez mayor, y siempre acotado por $n$, lo cual prueba el resultado. 

El recíproco no es cierto. Por ejemplo en un grafo tripartito completo no podemos unir componentes porque tienen aristas entre ellas, así que no sería bipartito. 
\end{solucion}

\newpage


\begin{ejercicio}{1.2}
Sea $T$ un árbol de orden $p$ que contiene $p_i$ vértices de grado $i$, para cada $i \in\{1,\dots, p-1\}$. Demuestra que $p_1 =
\sum^{p-1}_{i=3} (i - 2)p_i + 2$.
\end{ejercicio}
\begin{solucion}
%Como $T$ tiene orden $p$, $p_{p-1}\in \{0,1\}$. Si $p_{p-1}=1$ entonces todos los demás vértices son hojas conectadas al vértice de grado $p-1$, es decir, $p_1=p-1$ y $p_i=0$ para $1<i<p-1$. Así, la ecuación resultante sería
%\[
%p-1=p-3+2
%\]
%que es trivialmente cierta. Supongamos entonces que $p_{p-1}=0$. 
La ecuación del enunciado se puede reescribir como $\sum_{i=1}^{p-1}(i-2)p_i+2=0$ ya que $p_1$ pasaría a tener coeficiente $-1=1-2$ y $p_2$ tendría coeficiente 0. Ahora la escribimos como
\[
\sum_{i=1}^{p-1} i p_i + 2 = \sum_{i=1}^{p-1}2p_i.
\]
O lo que es lo mismo
\[
\sum_{v\in V(G)}\delta(v)+2=2|V(G)|
\]
Por un lado tenemos que al ser un árbol $|V(G)|=|E(G)|+1$ y por otro tenemos que $\sum_{v\in V(G)}\delta(v)=|E(G)|$ por el lema del apretón de manos, de modo que la igualdad es cierta. 
\end{solucion}

\newpage

\begin{ejercicio}{1.3}

Sea $T$ un árbol de orden $p$ tal que todos sus vértices son de grado 1 o de grado 3. Prueba
que $T$ contiene exactamente $(p - 2)/2$ vértices de grado 3.
\end{ejercicio}
\begin{solucion}
Basta sustituir en la ecuación del ejercicio anterior, que nos daría $-p_1+p_3+2=0$ junto con la condición $p_1+p_3=p$. 
\end{solucion}

\newpage

\begin{ejercicio}{1.4}

Encuentra todos los árboles $T$ tales que su complementario $\overline{T}$ es también un árbol.
\end{ejercicio}
\begin{solucion}
Un árbol de $n$ vértices tiene $n-1$ aristas. Además, para un conjunto de $n$ vértices hay $\binom{n}{2}$ posibles aristas. Para que el complementario de un árbol sea también un árbol debería darse $\binom{n}{2}-(n-1)=(n-1)$, lo que nos da $n(n-1)=4(n-1)$. Esto se cumple para $n=1$ y $n=4$, luego tenemos el árbol trivial y los árboles de 4 vértices, que salvo isomorfismo son los dos que aparecen dibujados.

\begin{tikzpicture}[line cap=round,line join=round,>=triangle 45,x=1.0cm,y=1.0cm]
\clip(-3.2672,3.8856) rectangle (11.4528,5.592);
\draw [line width=2.pt] (0.,5.)-- (3.,5.);
\draw [line width=2.pt] (4.,5.)-- (6.,5.);
\draw [line width=2.pt] (5.,5.)-- (5.,4.);
\begin{scriptsize}
\draw [fill=black] (0.,5.) circle (2.5pt);
\draw [fill=black] (1.,5.) circle (2.5pt);
\draw [fill=black] (2.,5.) circle (2.5pt);
\draw [fill=black] (3.,5.) circle (2.5pt);
\draw [fill=black] (4.,5.) circle (2.5pt);
\draw [fill=black] (6.,5.) circle (2.5pt);
\draw [fill=black] (5.,5.) circle (2.5pt);
\draw [fill=black] (5.,4.) circle (2.5pt);
\end{scriptsize}
\end{tikzpicture}

\end{solucion}

\newpage

\begin{ejercicio}{1.5}

Demuestra que todo árbol $T$ tiene un número de hojas mayor o igual que el grado máximo
de $T$.
\end{ejercicio}
\begin{solucion}
Trivial: considerar el vértice de grado máximo, en el peor de los casos todas las hojas están conectadas a este vértice, en otros casos de los vértices adyacentes saldrán más hojas. 
\end{solucion}

\newpage

\begin{ejercicio}{1.6}
Prueba que si $G$ es un grafo conexo de orden $p$ tal que todo subgrafo suyo de tamaño
$p -1$ es un árbol ``spanning'', entonces $G$ es un árbol o un ciclo.
\end{ejercicio}
\begin{solucion}
Si $G$ no es un árbol ni un ciclo, entonces contiene un grafo formado por un ciclo y una arista  con un vértice de grado 1. A este tipo de grafo lo vamos a llamar $J_n$, con $n$ dependiendo el orden del ciclo. En el caso de que este sea todo el grado, podemos conseguir un subgrafo de tamaño $p-1$ que no es un árbol spanning, esto es, el ciclo que contiene. Si el grafo es mayor, basta aislar partes del grafo que sean $J_n$ hasta que quede como unión de un árbol o un ciclo y grafos $J_n$.
\end{solucion}

\newpage

\begin{ejercicio}{1.7}
Prueba o refuta el siguiente enunciado: Si $G$ es un grafo conexo tal que dos cualesquiera
de sus árboles ``spanning'' son isomorfos, entonces $G$ es un árbol o un ciclo.
\end{ejercicio}
\begin{solucion}
Razonamos de igual manera que en el ejercicio anterior. Es claro que $J_n$ tiene siempre dos árboles spanning no isomorfos, uno en el que tomamos la arista suelta y vamos rodeando el ciclo hasta dejar solo una arista fuera, y otra en la que empezando por la arista suelta añadimos aristas del ciclo a izquierda y derecha hasta que solo falte una por añadir. Cualquier grafo que lo contenga tendrá más de una clase de isomorfismo de árbol maximal, por lo que para que solo tenga un árbol spanning salvo isomorfismo tendrá que ser un árbol o un ciclo. 
\end{solucion}

\newpage

\begin{ejercicio}{1.8}
Obtén un árbol spanning por el algoritmo DFS del grafo de Grotzsch de la figura adjunta.
\end{ejercicio}
\begin{solucion}

Damos la lista de vértices que se va produciendo y luego añadimos un dibujo del árbol spanning. 
\[
[v_1], [v_1,v_2], [v_1,v_2,v_3], [v_1,v_2, v_3, v_4], [v_1,v_2, v_3, v_4, v_5], [v_1,v_2, v_3, v_4, v_6], [v_1,v_2, v_3, v_4, v_6, v_{11}], 
\]
\[
[v_1,v_2, v_3, v_4, v_6, v_{11}, v_7], [v_1,v_2, v_3, v_4, v_6, v_{11}], [v_1,v_2, v_3, v_4, v_6, v_{11}, v_8], [v_1,v_2, v_3, v_4, v_6, v_{11}]
\]
\[
[v_1,v_2, v_3, v_4, v_6, v_{11}], [v_1,v_2, v_3, v_4, v_6, v_{11}, v_9], [v_1,v_2, v_3, v_4, v_6, v_{11}], [v_1,v_2, v_3, v_4, v_6, v_{11}, v_{10}]
\]
A partir de entonces la lista se va vaciando sin volver a añadir ningún vértice. El resultado es el siguiente

\definecolor{qqqqff}{rgb}{0.,0.,1.}
\begin{tikzpicture}[line cap=round,line join=round,>=triangle 45,x=1.0cm,y=1.0cm]
\clip(-3.742666666666666,-0.7253333333333322) rectangle (8.470666666666665,5.28);
\draw[line width=2.pt] (0.,0.) -- (3.,0.) -- (3.9270509831248424,2.85316954888546) -- (1.5,4.616525305762879) -- (-0.9270509831248419,2.853169548885461) -- cycle;
\draw [line width=2.pt] (0.,0.)-- (3.,0.);
\draw [line width=2.pt] (3.,0.)-- (3.9270509831248424,2.85316954888546);
\draw [line width=2.pt] (3.9270509831248424,2.85316954888546)-- (1.5,4.616525305762879);
\draw [line width=2.pt] (1.5,4.616525305762879)-- (-0.9270509831248419,2.853169548885461);
\draw [line width=2.pt] (-0.9270509831248419,2.853169548885461)-- (0.,0.);
\draw [line width=2.pt] (-0.9270509831248419,2.853169548885461)-- (3.,0.);
\draw [line width=2.pt] (-0.9270509831248419,2.853169548885461)-- (3.9270509831248424,2.85316954888546);
\draw [line width=2.pt] (1.5,4.616525305762879)-- (0.,0.);
\draw [line width=2.pt] (1.5,4.616525305762879)-- (3.,0.);
\draw [line width=2.pt] (0.,0.)-- (3.9270509831248424,2.85316954888546);
\draw [line width=2.pt] (1.5,2.8531695488854605)-- (1.5,2.04);
\draw [line width=2.pt] (1.5,2.04)-- (0.7498492564294226,2.3077987118759378);
\draw [line width=2.pt] (1.0129721070609903,1.443660268553074)-- (1.5,2.04);
\draw [line width=2.pt] (1.5,2.04)-- (2.0449101167030137,1.4857141657331447);
\draw [line width=2.pt] (1.5,2.04)-- (2.2515228718441938,2.303575735277337);
\draw (1.313333333333333,5.194666666666663) node[anchor=north west] {$v_1$};
\draw (4.2253333333333325,3.104) node[anchor=north west] {$v_2$};
\draw (3.414666666666666,0.096) node[anchor=north west] {$v_3$};
\draw (-0.45733333333333326,-0.1706666666666659) node[anchor=north west] {$v_4$};
\draw (-1.396,3.232) node[anchor=north west] {$v_5$};
\draw (1.505333333333333,2.976) node[anchor=north west] {$v_6$};
\draw (2.252,2.432) node[anchor=north west] {$v_7$};
\draw (2.049333333333333,1.6106666666666658) node[anchor=north west] {$v_8$};
\draw (1.0146666666666664,1.568) node[anchor=north west] {$v_9$};
\draw (0.,2.432) node[anchor=north west] {$v_{10}$};
\draw (1.505333333333333,2.165333333333332) node[anchor=north west] {$v_{11}$};
\draw [line width=2.8pt,color=qqqqff] (1.5,4.616525305762879)-- (3.9270509831248424,2.85316954888546);
\draw [line width=2.8pt,color=qqqqff] (3.9270509831248424,2.85316954888546)-- (3.,0.);
\draw [line width=2.8pt,color=qqqqff] (3.,0.)-- (0.,0.);
\draw [line width=2.8pt,color=qqqqff] (0.,0.)-- (-0.9270509831248419,2.853169548885461);
\draw [line width=2.8pt,color=qqqqff] (-0.9270509831248419,2.853169548885461)-- (1.5,2.8531695488854605);
\draw [line width=2.8pt,color=qqqqff] (1.5,2.8531695488854605)-- (1.5,2.04);
\draw [line width=2.8pt,color=qqqqff] (1.5,2.04)-- (2.2515228718441938,2.303575735277337);
\draw [line width=2.8pt,color=qqqqff] (1.5,2.04)-- (2.0449101167030137,1.4857141657331447);
\draw [line width=2.8pt,color=qqqqff] (1.5,2.04)-- (1.0129721070609903,1.443660268553074);
\draw [line width=2.8pt,color=qqqqff] (1.5,2.04)-- (0.7498492564294226,2.3077987118759378);
\begin{scriptsize}
\draw [fill=black] (0.,0.) circle (2.0pt);
\draw [fill=black] (3.,0.) circle (2.5pt);
\draw [fill=black] (3.9270509831248424,2.85316954888546) circle (2.0pt);
\draw [fill=black] (1.5,4.616525305762879) circle (2.0pt);
\draw [fill=black] (-0.9270509831248419,2.853169548885461) circle (2.0pt);
\draw [fill=black] (0.7498492564294226,2.3077987118759378) circle (2.5pt);
\draw [fill=black] (1.5,2.8531695488854605) circle (2.5pt);
\draw [fill=black] (2.2515228718441938,2.303575735277337) circle (2.5pt);
\draw [fill=black] (2.0449101167030137,1.4857141657331447) circle (2.5pt);
\draw [fill=black] (1.0129721070609903,1.443660268553074) circle (2.5pt);
\draw [fill=black] (1.5,2.04) circle (2.5pt);
\end{scriptsize}
\end{tikzpicture}



\end{solucion}
\newpage

\begin{ejercicio}{1.9}
Obtén un árbol spanning por el algoritmo DFS del grafo de Petersen de la figura adjunta.
\end{ejercicio}
\begin{solucion}\

\definecolor{qqqqff}{rgb}{0.,0.,1.}
\begin{tikzpicture}[line cap=round,line join=round,>=triangle 45,x=1.0cm,y=1.0cm]
\clip(-3.796,-0.928) rectangle (8.417333333333332,5.077333333333329);
\draw[line width=2.pt] (0.,0.) -- (3.,0.) -- (3.9270509831248424,2.85316954888546) -- (1.5,4.616525305762879) -- (-0.9270509831248419,2.853169548885461) -- cycle;
\draw [line width=2.pt] (0.,0.)-- (3.,0.);
\draw [line width=2.pt] (3.,0.)-- (3.9270509831248424,2.85316954888546);
\draw [line width=2.pt] (3.9270509831248424,2.85316954888546)-- (1.5,4.616525305762879);
\draw [line width=2.pt] (1.5,4.616525305762879)-- (-0.9270509831248419,2.853169548885461);
\draw [line width=2.pt] (-0.9270509831248419,2.853169548885461)-- (0.,0.);
\draw (1.313333333333333,5.194666666666663) node[anchor=north west] {$v_1$};
\draw (4.2253333333333325,3.104) node[anchor=north west] {$v_2$};
\draw (3.414666666666666,0.096) node[anchor=north west] {$v_3$};
\draw (-0.45733333333333326,-0.1706666666666659) node[anchor=north west] {$v_4$};
\draw (-1.396,3.232) node[anchor=north west] {$v_5$};
\draw (1.068,3.9253333333333305) node[anchor=north west] {$v_6$};
\draw (2.657333333333333,3.0613333333333315) node[anchor=north west] {$v_7$};
\draw (2.4333333333333327,1.2693333333333328) node[anchor=north west] {$v_8$};
\draw (0.18266666666666664,1.237333333333333) node[anchor=north west] {$v_9$};
\draw (-0.09466666666666665,3.104) node[anchor=north west] {$v_{10}$};
\draw [line width=2.pt] (1.5,4.616525305762879)-- (1.4946666666666664,3.466666666666664);
\draw [line width=2.pt] (-0.9270509831248419,2.853169548885461)-- (0.012,2.517333333333332);
\draw [line width=2.pt] (3.9270509831248424,2.85316954888546)-- (2.988,2.496);
\draw [line width=2.pt] (2.3906666666666663,0.7573333333333333)-- (3.,0.);
\draw [line width=2.pt] (0.5773333333333333,0.736)-- (0.,0.);
\draw [line width=2.pt] (0.012,2.517333333333332)-- (2.988,2.496);
\draw [line width=2.pt] (0.012,2.517333333333332)-- (2.3906666666666663,0.7573333333333333);
\draw [line width=2.pt] (1.4946666666666664,3.466666666666664)-- (0.5773333333333333,0.736);
\draw [line width=2.pt] (1.4946666666666664,3.466666666666664)-- (2.3906666666666663,0.7573333333333333);
\draw [line width=2.pt] (0.5773333333333333,0.736)-- (2.988,2.496);
\draw [line width=2.8pt,,color=qqqqff] (1.5,4.616525305762879)-- (3.9270509831248424,2.85316954888546);
\draw [line width=2.8pt,,color=qqqqff] (3.9270509831248424,2.85316954888546)-- (3.,0.);
\draw [line width=2.8pt,,color=qqqqff] (3.,0.)-- (0.,0.);
\draw [line width=2.8pt,,color=qqqqff] (0.,0.)-- (-0.9270509831248419,2.853169548885461);
\draw [line width=2.8pt,,color=qqqqff] (-0.9270509831248419,2.853169548885461)-- (0.012,2.517333333333332);
\draw [line width=2.8pt,,color=qqqqff] (0.012,2.517333333333332)-- (2.988,2.496);
\draw [line width=2.8pt,,color=qqqqff] (2.988,2.496)-- (0.5773333333333333,0.736);
\draw [line width=2.8pt,,color=qqqqff] (0.5773333333333333,0.736)-- (1.4946666666666664,3.466666666666664);
\draw [line width=2.8pt,,color=qqqqff] (1.4946666666666664,3.466666666666664)-- (2.3906666666666663,0.7573333333333333);
\begin{scriptsize}
\draw [fill=black] (0.,0.) circle (2.0pt);
\draw [fill=black] (3.,0.) circle (2.5pt);

\draw [fill=black] (3.9270509831248424,2.85316954888546) circle (2.0pt);
\draw [fill=black] (1.5,4.616525305762879) circle (2.0pt);
\draw [fill=black] (-0.9270509831248419,2.853169548885461) circle (2.0pt);
\draw [fill=black] (1.4946666666666664,3.466666666666664) circle (2.5pt);
\draw [fill=black] (0.012,2.517333333333332) circle (2.5pt);
\draw [fill=black] (0.5773333333333333,0.736) circle (2.5pt);
\draw [fill=black] (2.3906666666666663,0.7573333333333333) circle (2.5pt);
\draw [fill=black] (2.988,2.496) circle (2.5pt);
\end{scriptsize}
\end{tikzpicture}
\end{solucion}
\newpage

\begin{ejercicio}{1.10}
Obtén un árbol spanning por el algoritmo BFS del grafo de Grotzsch de la figura adjunta.
\end{ejercicio}
\begin{solucion}
\[
[v_1], [v_1, v_2], [v_1, v_2, v_5], [v_1, v_2, v_5, v_7], [v_1, v_2, v_5, v_7, v_{10}], [v_2, v_5, v_7, v_{10}],  [v_2, v_5, v_7, v_{10}, v_3], 
\]
\[
[v_2, v_5, v_7, v_{10}, v_3, v_6],[v_2, v_5, v_7, v_{10}, v_3, v_6, v_8],  [v_5, v_7, v_{10}, v_3, v_6, v_8],  [v_5, v_7, v_{10}, v_3, v_6, v_8, v_4],
\]
\[
[v_5, v_7, v_{10}, v_3, v_6, v_8, v_4, v_{9}], [v_7, v_{10}, v_3, v_6, v_8, v_4, v_{9}], [v_7, v_{10}, v_3, v_6, v_8, v_4, v_{9}, v_{11}]
\]
Como ya han aparecido todos los vértices paramos aquí.


\definecolor{qqqqff}{rgb}{0.,0.,1.}
\begin{tikzpicture}[line cap=round,line join=round,>=triangle 45,x=1.0cm,y=1.0cm]
\clip(-3.742666666666666,-0.7253333333333322) rectangle (8.470666666666665,5.28);
\draw[line width=2.pt] (0.,0.) -- (3.,0.) -- (3.9270509831248424,2.85316954888546) -- (1.5,4.616525305762879) -- (-0.9270509831248419,2.853169548885461) -- cycle;
\draw [line width=2.pt] (0.,0.)-- (3.,0.);
\draw [line width=2.pt] (3.,0.)-- (3.9270509831248424,2.85316954888546);
\draw [line width=2.pt] (3.9270509831248424,2.85316954888546)-- (1.5,4.616525305762879);
\draw [line width=2.pt] (1.5,4.616525305762879)-- (-0.9270509831248419,2.853169548885461);
\draw [line width=2.pt] (-0.9270509831248419,2.853169548885461)-- (0.,0.);
\draw [line width=2.pt] (-0.9270509831248419,2.853169548885461)-- (3.,0.);
\draw [line width=2.pt] (-0.9270509831248419,2.853169548885461)-- (3.9270509831248424,2.85316954888546);
\draw [line width=2.pt] (1.5,4.616525305762879)-- (0.,0.);
\draw [line width=2.pt] (1.5,4.616525305762879)-- (3.,0.);
\draw [line width=2.pt] (0.,0.)-- (3.9270509831248424,2.85316954888546);
\draw [line width=2.pt] (1.5,2.8531695488854605)-- (1.5,2.04);
\draw [line width=2.pt] (1.5,2.04)-- (0.7498492564294226,2.3077987118759378);
\draw [line width=2.pt] (1.0129721070609903,1.443660268553074)-- (1.5,2.04);
\draw [line width=2.pt] (1.5,2.04)-- (2.0449101167030137,1.4857141657331447);
\draw [line width=2.pt] (1.5,2.04)-- (2.2515228718441938,2.303575735277337);
\draw (1.313333333333333,5.194666666666663) node[anchor=north west] {$v_1$};
\draw (4.2253333333333325,3.104) node[anchor=north west] {$v_2$};
\draw (3.414666666666666,0.096) node[anchor=north west] {$v_3$};
\draw (-0.45733333333333326,-0.1706666666666659) node[anchor=north west] {$v_4$};
\draw (-1.396,3.232) node[anchor=north west] {$v_5$};
\draw (1.505333333333333,2.976) node[anchor=north west] {$v_6$};
\draw (2.252,2.432) node[anchor=north west] {$v_7$};
\draw (2.049333333333333,1.6106666666666658) node[anchor=north west] {$v_8$};
\draw (1.0146666666666664,1.568) node[anchor=north west] {$v_9$};
\draw (0.18266666666666664,2.6773333333333316) node[anchor=north west] {$v_{10}$};
\draw (1.505333333333333,2.165333333333332) node[anchor=north west] {$v_{11}$};
\draw [line width=2.8pt,color=qqqqff] (1.5,4.616525305762879)-- (3.9270509831248424,2.85316954888546);
\draw [line width=2.8pt,color=qqqqff] (3.9270509831248424,2.85316954888546)-- (3.,0.);
\draw [line width=2.8pt,color=qqqqff] (1.5,4.616525305762879)-- (2.2515228718441938,2.303575735277337);
\draw [line width=2.8pt,color=qqqqff] (1.5,4.616525305762879)-- (0.7498492564294226,2.3077987118759378);
\draw [line width=2.8pt,color=qqqqff] (1.5,4.616525305762879)-- (-0.9270509831248419,2.853169548885461);
\draw [line width=2.8pt,color=qqqqff] (3.9270509831248424,2.85316954888546)-- (1.5,2.8531695488854605);
\draw [line width=2.8pt,color=qqqqff] (3.9270509831248424,2.85316954888546)-- (2.0449101167030137,1.4857141657331447);
\draw [line width=2.8pt,color=qqqqff] (-0.9270509831248419,2.853169548885461)-- (0.,0.);
\draw [line width=2.8pt,color=qqqqff] (-0.9270509831248419,2.853169548885461)-- (1.0129721070609903,1.443660268553074);
\draw [line width=2.8pt,color=qqqqff] (2.2515228718441938,2.303575735277337)-- (1.5,2.04);
\begin{scriptsize}
\draw [fill=black] (0.,0.) circle (2.0pt);
\draw [fill=black] (3.,0.) circle (2.5pt);
\draw[color=black] (-0.596,1.4666666666666661) node {$j$};
\draw [fill=black] (3.9270509831248424,2.85316954888546) circle (2.0pt);
\draw [fill=black] (1.5,4.616525305762879) circle (2.0pt);
\draw [fill=black] (-0.9270509831248419,2.853169548885461) circle (2.0pt);
\draw [fill=black] (0.7498492564294226,2.3077987118759378) circle (2.5pt);
\draw [fill=black] (1.5,2.8531695488854605) circle (2.5pt);
\draw [fill=black] (2.2515228718441938,2.303575735277337) circle (2.5pt);
\draw [fill=black] (2.0449101167030137,1.4857141657331447) circle (2.5pt);
\draw [fill=black] (1.0129721070609903,1.443660268553074) circle (2.5pt);
\draw [fill=black] (1.5,2.04) circle (2.5pt);

\end{scriptsize}
\end{tikzpicture}
\end{solucion}

\newpage

\begin{ejercicio}{1.11}
Obtén un árbol spanning por el algoritmo BFS del grafo de Petersen de la figura adjunta.
\end{ejercicio}
\begin{solucion}\

\definecolor{qqqqff}{rgb}{0.,0.,1.}
\begin{tikzpicture}[line cap=round,line join=round,>=triangle 45,x=1.0cm,y=1.0cm]
\clip(-3.796,-0.928) rectangle (8.417333333333332,5.077333333333329);
\draw[line width=2.pt] (0.,0.) -- (3.,0.) -- (3.9270509831248424,2.85316954888546) -- (1.5,4.616525305762879) -- (-0.9270509831248419,2.853169548885461) -- cycle;
\draw [line width=2.pt] (0.,0.)-- (3.,0.);
\draw [line width=2.pt] (3.,0.)-- (3.9270509831248424,2.85316954888546);
\draw [line width=2.pt] (3.9270509831248424,2.85316954888546)-- (1.5,4.616525305762879);
\draw [line width=2.pt] (1.5,4.616525305762879)-- (-0.9270509831248419,2.853169548885461);
\draw [line width=2.pt] (-0.9270509831248419,2.853169548885461)-- (0.,0.);
\draw (1.313333333333333,5.194666666666663) node[anchor=north west] {$v_1$};
\draw (4.2253333333333325,3.104) node[anchor=north west] {$v_2$};
\draw (3.414666666666666,0.096) node[anchor=north west] {$v_3$};
\draw (-0.45733333333333326,-0.1706666666666659) node[anchor=north west] {$v_4$};
\draw (-1.396,3.232) node[anchor=north west] {$v_5$};
\draw (1.068,3.9253333333333305) node[anchor=north west] {$v_6$};
\draw (2.657333333333333,3.0613333333333315) node[anchor=north west] {$v_7$};
\draw (2.4333333333333327,1.2693333333333328) node[anchor=north west] {$v_8$};
\draw (0.18266666666666664,1.237333333333333) node[anchor=north west] {$v_9$};
\draw (-0.09466666666666665,3.104) node[anchor=north west] {$v_{10}$};
\draw [line width=2.pt] (1.5,4.616525305762879)-- (1.4946666666666664,3.466666666666664);
\draw [line width=2.pt] (-0.9270509831248419,2.853169548885461)-- (0.012,2.517333333333332);
\draw [line width=2.pt] (3.9270509831248424,2.85316954888546)-- (2.988,2.496);
\draw [line width=2.pt] (2.3906666666666663,0.7573333333333333)-- (3.,0.);
\draw [line width=2.pt] (0.5773333333333333,0.736)-- (0.,0.);
\draw [line width=2.pt] (0.012,2.517333333333332)-- (2.988,2.496);
\draw [line width=2.pt] (0.012,2.517333333333332)-- (2.3906666666666663,0.7573333333333333);
\draw [line width=2.pt] (1.4946666666666664,3.466666666666664)-- (0.5773333333333333,0.736);
\draw [line width=2.pt] (1.4946666666666664,3.466666666666664)-- (2.3906666666666663,0.7573333333333333);
\draw [line width=2.pt] (0.5773333333333333,0.736)-- (2.988,2.496);
\draw [line width=2.8pt,,color=qqqqff] (1.5,4.616525305762879)-- (3.9270509831248424,2.85316954888546);
\draw [line width=2.8pt,,color=qqqqff] (1.5,4.616525305762879)-- (-0.9270509831248419,2.853169548885461);
\draw [line width=2.8pt,,color=qqqqff] (1.5,4.616525305762879)-- (1.4946666666666664,3.466666666666664);
\draw [line width=2.8pt,,color=qqqqff] (3.9270509831248424,2.85316954888546)-- (3.,0.);
\draw [line width=2.8pt,,color=qqqqff] (3.9270509831248424,2.85316954888546)-- (2.988,2.496);
\draw [line width=2.8pt,,color=qqqqff] (-0.9270509831248419,2.853169548885461)-- (0.,0.);
\draw [line width=2.8pt,,color=qqqqff] (-0.9270509831248419,2.853169548885461)-- (0.012,2.517333333333332);
\draw [line width=2.8pt,,color=qqqqff] (1.4946666666666664,3.466666666666664)-- (2.3906666666666663,0.7573333333333333);
\draw [line width=2.8pt,,color=qqqqff] (1.4946666666666664,3.466666666666664)-- (0.5773333333333333,0.736);
\begin{scriptsize}
\draw [fill=black] (0.,0.) circle (2.0pt);
\draw [fill=black] (3.,0.) circle (2.5pt);

\draw [fill=black] (3.9270509831248424,2.85316954888546) circle (2.0pt);
\draw [fill=black] (1.5,4.616525305762879) circle (2.0pt);
\draw [fill=black] (-0.9270509831248419,2.853169548885461) circle (2.0pt);
\draw [fill=black] (1.4946666666666664,3.466666666666664) circle (2.5pt);
\draw [fill=black] (0.012,2.517333333333332) circle (2.5pt);
\draw [fill=black] (0.5773333333333333,0.736) circle (2.5pt);
\draw [fill=black] (2.3906666666666663,0.7573333333333333) circle (2.5pt);
\draw [fill=black] (2.988,2.496) circle (2.5pt);
\end{scriptsize}
\end{tikzpicture}


\end{solucion}

\newpage

\begin{figure}[h!]
\includegraphics[scale=0.8]{Rel1}
\end{figure}
%
%\newpage
%
%\begin{ejercicio}{1.12}
%
%\end{ejercicio}
%\begin{solucion}
%
%\end{solucion}

\end{document}
