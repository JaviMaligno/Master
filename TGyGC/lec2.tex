\documentclass[TGyGC.tex]{subfiles}

\begin{document}


%\hyphenation{equi-va-len-cia}\hyphenation{pro-pie-dad}\hyphenation{res-pec-ti-va-men-te}\hyphenation{sub-es-pa-cio}
\chapter{Coloración de grafos}
\section{Polinomio cromático}
En la diapo 76 sale eso porque el primer caso es ponerle a los dos esquinas el mismo o ponerles uno distinto. En la 84 selecciona una arista $e$ concreta, y recursivamente se van eligiendo las otras. $G/e$ denota la contracción de la arista. Para ver el caso 1 pensar en $C_4$. 
\end{document}
