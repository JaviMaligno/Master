	\documentclass[twoside]{article}
\usepackage{../estilo-ejercicios}
%\renewcommand{\baselinestretch}{1,3}
%--------------------------------------------------------
\begin{document}

\title{Ejercicios de Teoría de Representaciones}
\author{Javier Aguilar Martín y Rafael González López}
\maketitle

\section{Ejercicio}
\begin{ejercicio}{1}
Consideremos las subálgebras de $\mathfrak{gl}(n)$ definidas según
\begin{gather*}
t(n) = \{ M \in \mathfrak{gl}(n) \mid M\text{ es triangular superior}\}
\\
\nu(n) = \{ M \in \mathfrak{gl}(n) \mid M\text{ es estrctamente triangular superior}\}\\
\delta(n) = \{ M \in \mathfrak{gl}(n) \mid M\text{ es diagonal}\}
\end{gather*}
Probar los siguientes enunciados.
\begin{enumerate}
\item Probar que las álgebras $\mathfrak{sl}(n)$, $\mathfrak{sp}(2l+1)$ y $\mathfrak{o}(n)$ coinciden con sus álgebra derivadas.
\item Probar que dichas álgebras son autonormalizadas.
\item Probar que $t(n)$ y $\delta(n)$ son autonormalizada.
\item Probar que $\mathcal{N}(\nu(n)) = t(n)$
\end{enumerate}
\end{ejercicio}
\begin{solucion}
\begin{enumerate}
\item[]
\item Sabemos que siempre se tiene una contención. Basta ver que todo elemento de una base de las álgebras, como espacios vectoriales, está en su derivada. 

\item 

\item Al estar trabajado con álgebras, tenemos que estas están contenidas en su normalizador. Tenemos que ver que el recíproco es cierto. 
\newpage

Para el caso diagonal, hagamos la siguiente consideración. Sea $D$ una matriz diagonal y $M$ una matriz cualquiera. Entonces $DM$ corresponde a multiplicar la fila $i$ de $M$ por el elemento $ii$ de $D$. Queremos ver que no puede ser que $[D,M] \in \mathbb{N}(t(n))$, para todo $D\in t(n)$, sin ser $M$ matriz diagonal. Tomando $D=E_{ii}$ es claro que si imponemos que $[D,M]$ sea una matriz diagonal, necesariamente los elementos $M_{ij}$ y $M_{ji}$ para todo $i,j$ con $i\neq j$ tienen que ser cero. Por inducción sobre $i$, tenemos que $M$ es una matriz diagonal.
\item Se deduce de las siguientes consideraciones. Si $T$ y $H$ son matrices triangulares superiores, entonces su productos y sus diferencias también, luego $[T,H]$ también lo será. Además, como $TH$ y $HT$ tienen la misma traza, $[T,H]$ es estrictamente triangular superior aunque $T$ y $H$ sean solo triangulares superiores. Por tanto, $t(n) \subset \mathcal{N}(\nu(n))$, pero $t(n)$ es autonormalizado, luego es el mayor ideal que se contiene a sí mismo, por lo que debe darse la igualdad. 
\end{enumerate}
\end{solucion}
\end{document}
