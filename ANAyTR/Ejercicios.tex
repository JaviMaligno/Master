	\documentclass[twoside]{article}
\usepackage{../estilo-ejercicios}
%\renewcommand{\baselinestretch}{1,3}
%--------------------------------------------------------
\begin{document}

\title{Ejercicios de Álgebras de Lie}
\author{Javier Aguilar Martín y Rafael González López}
\maketitle

\begin{ejercicio}{1}
	Probar que $\mathcal{N}(W)$ es la mayor subálgebra que contiene a $W$ como ideal.
\end{ejercicio}
\begin{solucion}
		Ya hemos probado en clase que $\mathcal{N}(W)$ es una subálgebra que contiene a $W$ como ideal, así que tenemos que probar que es la mayor cumpliendo estas propiedades. Sea $V$ una subálgebra que contiene a $\mathcal{N}(W)$ y que contiene a $W$ como ideal. Entonces, para cualquier $w\in W$ y cualquier $v\in V$ se tiene que $[w,v]\in V$ por ser $W$ un ideal de $V$, pero esta es justamente la condición que debe cumplir $v$ para estar en el normalizador de $W$, luego el resultado se tiene trivialmente. 
	\end{solucion}
\newpage

\begin{ejercicio}{2}
Consideremos las subálgebras de $\mathfrak{gl}(n)$ definidas según
\begin{gather*}
t(n) = \{ M \in \mathfrak{gl}(n) \mid M\text{ es triangular superior}\}
\\
\eta(n) = \{ M \in \mathfrak{gl}(n) \mid M\text{ es estrctamente triangular superior}\}\\
\delta(n) = \{ M \in \mathfrak{gl}(n) \mid M\text{ es diagonal}\}
\end{gather*}
Probar los siguientes enunciados.
\begin{enumerate}
\item Probar que las álgebras $\mathfrak{sl}(n)$, $\mathfrak{sp}(2l)$ y $\mathfrak{o}(n)$ coinciden con sus álgebra derivadas.
\item Probar que dichas álgebras son autonormalizadas.
\item Probar que $t(n)$ y $\delta(n)$ son autonormalizadas.
\item Probar que $\mathcal{N}(\eta(n)) = t(n)$
\end{enumerate}
\end{ejercicio}
\begin{solucion}
\begin{enumerate}
\item[]
\item Sabemos que siempre se tiene una contención. Basta ver que todo elemento de una base de las álgebras, como espacios vectoriales, está en su derivada. Utilizaremos repetidamente la fórmula $[E_{ij},E_{kl}]=\delta_{jk}E_{il}-\delta_{li}E_{kj}$ para agilizar los cálculos.

En $\mathfrak{sl}(n)$ tenemos la base $\{E_{ij}, h_i=E_{i,i}-E_{i+1,i+1}\}_{1\leq i\leq n-1, i\neq j}$. Es fácil comprobar que para $k\neq i\neq j\neq k$, $E_{ij}=[E_{ik},E_{kj}]$ y que $h_i=[E_{i,i+1},E_{i+1,i}]$, por lo que tenemos el resultado. El primer resultado solo valdría para $n\geq 3$, pero para $n=1,2$ el resultado se verifica con un sencillo cálculo.

En el caso de $\mathfrak{sp}(2l)$ tenemos la base dada por la unión de los conjuntos
\begin{gather*}
\{E_{ii}-E_{l+i,l+i}, E_{i,l+i}\}_{1\leq i \leq l} \\
\{E_{ij}-E_{l+j,l+i}\}_{1\leq i\neq j\leq r} \qquad \{E_{i,l+j}+E_{j,l+i}\}_{i\leq i<j\leq l}
\end{gather*}
Veamos que todos los elementos de la base están también en el álgebra derivada. Utilizaremos que $E_{l+i,i} \in \mathfrak{sp}(2l)$ $1\leq i \leq l$, $E_{l+i,j}+E_{l+j,i} \in \mathfrak{sp}(2l)$ $i \leq i\neq j \leq l$.
\begin{gather*}
E_{ii}-E_{l+i,l+i}=[E_{i,l+i},E_{l+i,i}] \quad E_{i,l+i} = -\frac{1}{2}[E_{i,l+i},E_{ii}-E_{l+i,l+i}] \quad 1\leq i \leq l\\
E_{ij}-E_{l+j,l+i}=[E_{i,l+i},E_{l+i,j}+E_{l+j,i}] \quad i \leq i \neq j \leq r\\
E_{i,l+j}+E_{j,l+i}=[E_{ii}-E_{l+i,l+i},E_{i,l+j}+E_{j,l+i}]\quad i \leq i < j \leq l
\end{gather*}

Para $\mathfrak{o}(n)$ vamos a utilizar otra estrategia. Sabemos que si $\mathfrak{g}$ es un subálgebra simple, entonces $[\mathfrak{g},\mathfrak{g}] = g$, pues se da la contención como ideal y el $[\mathfrak{g},\mathfrak{g}]$ es no vacío, luego la simplicidad implica la igualdad.  Además, si un álgebra $\mathfrak{g}$ es semisimple, entonces es suma directa de ideales $\mathfrak{g}_1,\dotsc,\mathfrak{g}_m$ simples. Tenemos, por tanto,
$$
[\mathfrak{g},\mathfrak{g}] = \left[\bigoplus_{i=1}^m \mathfrak{g}_i,\bigoplus_{i=1}^m \mathfrak{g}_i\right] = \bigoplus_{i=1}^m [\mathfrak{g}_i,\mathfrak{g}_i] =\bigoplus_{i=1}^m \mathfrak{g}_i = \mathfrak{g}
$$
Es decir, el álgebra coincide con su derivada. Basta utilizar que para todo $n$ $\mathfrak{o}(n)$ es semisimple.

\item Al estar trabajando con álgebras, tenemos que estas están contenidas en su normalizador. Tenemos que ver que el recíproco es cierto. En el caso de $\mathfrak{sl}(n)$ no es cierto porque dada cualesquiera matrices $A,B\in\mathfrak{gl}(n)$, tenemos que $tr([A,B])=0$, luego en particular, para cualquier $W\in \mathfrak{sl}(n)$, tenemos que $tr([W,A])=0$, es decir, $[W,A]\in\mathfrak{sl}(n)$ y $\mathcal{N}(\mathfrak{sl}(n))=\mathfrak{gl}(n)$. %Para el caso de $\mathfrak{sl}(n)$, sea $H \in \mathcal{N}(\mathfrak{sl}(n))$ tal que $tr(H)\neq 0$. En particular, $\exists i$ tal que $H_{ii}\neq 0$.



\item 
Para el caso triangular superior, hagamos la siguiente consideración. Sea $D$ una matriz diagonal y $M$ una matriz cualquiera. Queremos ver que no puede ser que $[D,M] \in \mathcal{N}(t(n))$, para todo $D\in t(n)$, sin ser $M$ matriz diagonal. Tomando $D=E_{ii}$ es claro que $[D,M]$ es la matriz que tiene la fila $i$ de $M$, la columna $i$ de $M$ con signo opuesto, $0$ en la diagonal y el resto de elementos. Imponiendo que la zona triangular inferior sea nula, por inducción en $i$, llegamos  necesariamente los correspondientes elementos de $M$ son también nulos. Por tanto $M$ es triangular superior. Además, el mismo argumento imponiendo además que sea diagonal vale para $\delta(n)$.

\item Se deduce de las siguientes consideraciones. Si $T$ y $H$ son matrices triangulares superiores, entonces su productos y sus diferencias también, luego $[T,H]$ también lo será. Además, como $TH$ y $HT$ tienen la misma traza, $[T,H]$ es estrictamente triangular superior aunque $T$ y $H$ sean solo triangulares superiores. Por tanto, $t(n) \subset \mathcal{N}(\eta(n))$, pero $t(n)$ es autonormalizado, luego es el mayor subálgebra que se contiene a sí mismo como ideal, por lo que debe darse la igualdad. 
\end{enumerate}
\end{solucion}
\end{document}
