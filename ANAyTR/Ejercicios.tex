	\documentclass[twoside]{article}
\usepackage{../estilo-ejercicios}
%\renewcommand{\baselinestretch}{1,3}
%--------------------------------------------------------
\begin{document}

\title{Ejercicios de Álgebras de Lie}
\author{Javier Aguilar Martín y Rafael González López}
\maketitle

\begin{ejercicio}{1}
	Probar que $\mathcal{N}(W)$ es la mayor subálgebra que contiene a $W$ como ideal.
\end{ejercicio}
\begin{solucion}
		Ya hemos probado en clase que $\mathcal{N}(W)$ es una subálgebra que contiene a $W$ como ideal, así que tenemos que probar que es la mayor cumpliendo estas propiedades. Sea $V$ una subálgebra que contiene a $\mathcal{N}(W)$ y que contiene a $W$ como ideal. Entonces, para cualquier $w\in W$ y cualquier $v\in V$ se tiene que $[w,v]\in V$ por ser $W$ un ideal de $V$, pero esta es justamente la condición que debe cumplir $v$ para estar en el normalizador de $W$, luego el resultado se tiene trivialmente. 
	\end{solucion}



\newpage

\begin{ejercicio}{2}
Consideremos las subálgebras de $\mathfrak{gl}(n)$ definidas según
\begin{gather*}
t(n) = \{ M \in \mathfrak{gl}(n) \mid M\text{ es triangular superior}\}
\\
\eta(n) = \{ M \in \mathfrak{gl}(n) \mid M\text{ es estrctamente triangular superior}\}\\
\delta(n) = \{ M \in \mathfrak{gl}(n) \mid M\text{ es diagonal}\}
\end{gather*}
Probar los siguientes enunciados.
\begin{enumerate}
\item Probar que las álgebras $\mathfrak{sl}(n)$, $\mathfrak{sp}(2l)$ y $\mathfrak{o}(n)$ coinciden con sus álgebra derivadas.
\item Probar que dichas álgebras son autonormalizadas.
\item Probar que $t(n)$ y $\delta(n)$ son autonormalizadas.
\item Probar que $\mathcal{N}(\eta(n)) = t(n)$
\end{enumerate}
\end{ejercicio}
\begin{solucion}
\begin{enumerate}
\item[]
\item Sabemos que siempre se tiene una contención. Basta ver que todo elemento de una base de las álgebras, como espacios vectoriales, está en su derivada. Utilizaremos repetidamente la fórmula $[e_{ij},e_{kl}]=\delta_{jk}e_{il}-\delta_{li}e_{kj}$ para agilizar los cálculos.

En $\mathfrak{sl}(n)$ tenemos la base $\{e_{ij}, h_i=e_{i,i}-e_{i+1,i+1}\}_{1\leq i\leq n-1, i\neq j}$. Es fácil comprobar que para $i\neq j$, $e_{ij}=[e_{ij},e_{jj}]$ y que $h_i=[e_{i,i+1},e_{i+1,i}]$, por lo que tenemos el resultado.
%$[h_i,h_j]=0$ para todo $i,j$. 
%$$[h_i,h_j]=[e_{i,i}-e_{i+1,i+1},e_{j,j}-e_{j+1,j+1}]=[e_{i,i}-e_{i+1,i+1},e_{j,j}]-[e_{i,i}-e_{i+1,i+1},e_{j+1,j+1}]=$$
%$$ [e_{i,i},e_{j,j}]-[e_{i+1,i+1},e_{j,j}]-[e_{i,i},e_{j+1,j+1}]+[e_{i+1,i+1},e_{j+1,j+1}]=$$
%$$ \delta_{ij}e_{ij}-\delta_{ji}e_{ji}-(\delta_{i+1,j}e_{i+1,j}-\delta_{j,i+1}e_{j,i+1})-(\delta_{i,j+1}e_{i,j+1}-\delta_{j+1,i}e_{j+1,i})+\delta_{i+1,j+1}e_{i+1,j+1}-\delta_{j+1,i+1}e_{j+1,i+1}=$$
%Distingamos varios casos. Si $i=j$, entonces trivialmente $[h_i,h_j]=0$. Supongamos que $i=j+1$. Entonces $[h_i,h_j]=e_{j+1,j+1}-e_{j+1,j+1}=0$ y análogamente en el caso $j=i+1$. Si $|j-1|>1$ es claro que $[h_i,h_j]=0$.
%
%Calculemos $[h_i,e_{jk}]$ para las distintas posibles combinaciones de $i,j,k$. Supongamos $i=j$. %$$[h_i,e_{ik}]=e_{ii}e_{ik}-e_{i+1,i+1}e_{ik}-e_{ik}e_{ii}+e_{i+1,i+1}e_{ik}=e_{ik}$$
%$$[h_i,e_{ik}]=[e_{i,i}-e_{i+1,i+1},e_{ik}]=[e_{i,i},e_{ik}]-[e_{i+1,i+1},e_{ik}]=$$
%$$\delta_{ii}e_{ik}-\delta_{ki}e_{ii}-(\delta_{i+1,i}e_{i+1,k}-\delta_{k,i+1}e_{i,i+1})$$
%Si $i=k$ entonces $[h_i,e_{jk}]=0$, y también se anula si $k=i+1$. En otro caso, $[h_i,e_{jk}]=e_{ik}$. 
%
%Supongamos $i=k$ ($j\neq k$ pues el caso $i=j=k$ ya lo hemos tratado). Tenemos ahora análogamente 
%$$[h_i,e_{ji}]=[e_{i,i}-e_{i+1,i+1},e_{ji}]=[e_{i,i},e_{ji}]-[e_{i+1,i+1},e_{ji}]=$$
%$$\delta_{ij}e_{ii}-\delta_{ii}e_{ji}-(\delta_{i+1,j}e_{i+1,i}-\delta_{i,i+1}e_{j,i+1})$$
%
%Si $j=i+1$ obtenemos $[h_i,e_{ji}]=-2e_{i+1,i}$.
% Por último, si $i\neq j,k$, entonces (escribo ahora $e_{kl}$ para no liarme en la fórmula del corchete) 
% $$[h_i,e_{kl}]=[e_{i,i}-e_{i+1,i+1},e_{kl}]=[e_{i,i},e_{kl}]-[e_{i+1,i+1},e_{kl}]=$$
% $$\delta_{ik}e_{il}-\delta_{li}e_{ki}-(\delta_{i+1,k}e_{i+1,l}-\delta_{l,i+1}e_{k,i+1})$$
% 
%Si $k=l=i+1$, entonces $[h_i,e_{kl}]=0$, pero si solo uno de los dos es igual a $i+1$, entonces obtenemos en ambos casos $e_{i+1,i+1}$. En otro caso, $[h_i,e_{kl}]=0$.
%
%
%En definitiva, NO ME SALE NADA%todas las matrices de la base de $\mathfrak{sl}(n)$ están en el derivado.



%Una forma más conceptual de hacerlo es probar que si $x,y$ son matrices tales que $sx-x's=0$ y $sy-y's=0$ para $s=\begin{pmatrix}
%0 & I\\
%-I & 0
%\end{pmatrix}$, entonces $s[x,y]-[x,y]'s=0$. Es claro que la propiedad $sx-x's=0$ es cerrada para la suma, así que basta ver que lo es también para el producto. Para ello, multiplicamos la ecuación de $x$ por $y'$ a la izquierda y la de $y$ por $x$ a la derecha. Obtenemos entonces
%
%\begin{gather*}
%y'sx-y'x's=0\\
%syx-y'sx=0
%\end{gather*}
%
%Sumando ambas ecuaciones obtenemos $syx-y'x's=0$, que era lo que queríamos probar. 

En el caso de $\mathfrak{sp}(2r)$ tenemos la base $$\{e_{ii}-e_{i+r,i+r}, e_{ij}-e_{i+r,j+r},e_{i,i+r}, e_{i,j+r}+e_{j,i+r}, e_{i+r,i}, e_{i+r,j}+e_{j+r,i}\}_{1\leq i\neq j\leq r}.$$ 

Tenemos $e_{ii}-e_{i+r,i+r}=[e_{i,i+r},e_{i+r,i}]$, $e_{ij}-e_{j+r,i+r}=[e_{i,i+r},e_{i+r,j}+e_{j+r,i}]$, $e_{i,i+r}=-\frac{1}{2}[e_{i,i+r},e_{ii}-e_{i+r,i+r}]$, $e_{i,j+r}+e_{j,i+r}=[e_{ii}-e_{i+r,i+r},e_{i,j+r}+e_{j,i+r}]$, $e_{i+r,i}=\frac{1}{2}[e_{i+r,i},e_{ii}-e_{i+r,i+r}]$, $e_{i+r,j}+e_{j+r,i}=-[e_{ii}-e_{i+r,i+r},e_{i+r,j}+e_{j+r,i}]$. 

Así pues, tenemos para toda la base expresiones en función de corchetes de la base. 




Para $\mathfrak{o}(n)$ tenemos que considerar dos casos en función de la paridad de $n$. Comenzamos con $n=2l+1$. En este caso una base está dada por 
\begin{gather*}
\{e_{ii}-e_{i+l,i+l}\}_{2\leq i\leq l+1}\cup \{e_{1,i+l+1}-e_{i+1,1}, e_{1,i+1}-e_{i+l+1,1}\}_{1\leq i\leq l}\cup\{e_{i+1,j+l}-e_{j+l+1,i+l+1}\}_{1\leq i\neq j\leq l}\cup\\
\{e_{i+1,j+l+1}-e_{j+1,i+l+1}\}_{1\leq i<j\leq l}\cup \{e_{i+l+1,j+1}-e_{j+l+1,i+1}\}_{1\leq j<i\leq l}
\end{gather*}

Tenemos $e_{i+1,j+l}-e_{j+l+1,i+l+1}=[e_{ii}-e_{i+l,i+l},e_{i+1,j+l}-e_{j+l+1,i+l+1}]$, $e_{i+1,j+l+1}-e_{j+1,i+l+1}=[e_{ii}-e_{i+l,i+l},e_{i+1,j+l+1}-e_{j+1,i+l+1}]$, $e_{i+l+1,j+1}-e_{j+l+1,i+1}=[e_{ii}-e_{i+l,i+l},e_{i+l+1,j+1}-e_{j+l+1,i+1}]$, $e_{1,j+1}-e_{j+l+1}=[e_{1,i+l+1}-e_{i+1,1},e_{i+l+1,j+1}-e_{j+l+1,i+1}]$, $e_{1,j+1}-e_{j+l+1,1}=[e_{1,i+1}-e_{i+l+1,1},e_{i+1,j+l}-e_{j+l+1,i+l+1}]$ ME FALTA EL PRIMERO, EN EL SIGUIENTE TAMBIÉN

Tratamos ahora $n=2l$. Ahora tenemos la base
\[
\{e_{i+l,j}-e_{j+l,i}\}_{1\leq j<i\leq l}\cup\{e_{ii}-e_{i+l,i+l}\}_{1\leq i\leq l}\cup\{e_{ij}-e_{i+l,j+l}\}_{1\leq i\neq j\leq l}\cup\{e_{i,j+l}-e_{j,i+l}\}_{1\leq i<j\leq l}
\]

	Tenemos $e_{i+l,j}-e_{j+l,i}=[e_{i+l,j}-e_{j+l,i},e_{ii}-e_{i+l,i+l}]$, $e_{ij}-e_{i+l,j+l}=[e_{ii}-e_{i+l,i+l},e_{ij}-e_{i+l,j+l}]$ $e_{i,j+l}-e_{j,i+l}=[e_{ii}-e_{i+l,i+l},e_{i,j+l}-e_{j,i+l}]$, $e_{ii}-e_{i+l,i+l}=[]$. NO SACO EL QUE FALTA

\item Al estar trabajando con álgebras, tenemos que estas están contenidas en su normalizador. Tenemos que ver que el recíproco es cierto. En el caso de $\mathfrak{sl}(n)$ no es cierto porque dada cualesquiera matrices $A,B\in\mathfrak{gl}(n)$, tenemos que $tr([A,B])=0$, luego en particular, para cualquier $W\in \mathfrak{sl}(n)$, tenemos que $tr([W,A])=0$, es decir, $[W,A]\in\mathfrak{sl}(n)$ y $\mathcal{N}(\mathfrak{sl}(n))=\mathfrak{gl}(n)$. %Para el caso de $\mathfrak{sl}(n)$, sea $H \in \mathcal{N}(\mathfrak{sl}(n))$ tal que $tr(H)\neq 0$. En particular, $\exists i$ tal que $H_{ii}\neq 0$.

\item 
Para el caso triangular superior, hagamos la siguiente consideración. Sea $D$ una matriz diagonal y $M$ una matriz cualquiera. Queremos ver que no puede ser que $[D,M] \in \mathcal{N}(t(n))$, para todo $D\in t(n)$, sin ser $M$ matriz diagonal. Tomando $D=E_{ii}$ es claro que $[D,M]$ es la matriz que tiene la fila $i$ de $M$, la columna $i$ de $M$ con signo opuesto, $0$ en la diagonal y el resto de elementos. Imponiendo que la zona triangular inferior sea nula, por inducción en $i$, llegamos  necesariamente los correspondientes elementos de $M$ son también nulos. Por tanto $M$ es triangular superior. Además, el mismo argumento imponiendo además que sea diagonal vale para $\delta(n)$.

\item Se deduce de las siguientes consideraciones. Si $T$ y $H$ son matrices triangulares superiores, entonces su productos y sus diferencias también, luego $[T,H]$ también lo será. Además, como $TH$ y $HT$ tienen la misma traza, $[T,H]$ es estrictamente triangular superior aunque $T$ y $H$ sean solo triangulares superiores. Por tanto, $t(n) \subset \mathcal{N}(\eta(n))$, pero $t(n)$ es autonormalizado, luego es el mayor ideal que se contiene a sí mismo, por lo que debe darse la igualdad. 
\end{enumerate}
\end{solucion}
\end{document}
