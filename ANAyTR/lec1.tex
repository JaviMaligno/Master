\documentclass[ANAyTR.tex]{subfiles}

\begin{document}


%\hyphenation{equi-va-len-cia}\hyphenation{pro-pie-dad}\hyphenation{res-pec-ti-va-men-te}\hyphenation{sub-es-pa-cio}

\chapter{Teoría de Representaciones}
\section{Álgebras (asociativaas con unidad sobre un cuerpo $k$)}
Algunas álgebras habituales que usaremos seran las álgebras de polinomios, el álgebra de endomorfismos de un espacio vectorial, el álgebra de matrices cuadradas de orden $n$ (esta es la misma que la anterior para dimensión finita) y álgebras asociativas libres (polinomios no conmutativos). 

\begin{defi}
Una \emph{representación} de un álgebra $A$ es un $A$-módulo a la izquierda. También se puede interpretar como un espacio vectorial $V$ sobre $k$ (el mismo cuerpo que el álgebra) con una operación extra $A\times V\to V$ que extiende a la operación $k\times V\to V$ ``razonablemente''. 
\end{defi}

\begin{ej}[Ejemplo fundamental]
Sea $V$ un espacio vectorial de dimensión finita sobre $\C$ ($V=\C^n$). Sea $f\in\mathrm{End}(V)$ un endomorfismo con matriz asociada $A$ respecto a la base canónica. Si tenemos la expresión $(f^2-f-id)(v)$, podemos representarla por $(A^2-A-I)X$, donde $X$ es el vector de coordenadas de $v$. Vemos que ambas expresiones son el mismo polinomio evaluado en dos tipos de objetos distintos. Así que si consideramos un polinomio $P$, tenemos un producto $P\cdot v=P(f)(v)$. Esto refleja la idea de extender ``razonablemente'' el producto por escalar. Decimos entonces que $f$ tiene asociada una \emph{representación} del álgebra de polinomios $\C[x]$ (en general $k[x]$), que es una acción del álgebra de polinomios sobre el espacio vectorial.
\end{ej}


En vistas del ejemplo anterior, podemos precisar que la aplicación $\varphi:A\times V\to V$ debe cumplir las siguientes propiedades:
\begin{enumerate}
\item $\varphi(a,v+w)=\varphi(a,v)+\varphi(a,w)$
\item $\varphi(a,\lambda v)=\lambda\varphi(a,v)$
\item $\varphi(a+b,v)=\varphi(a,v)+\varphi(b,v)$
\item $\varphi(\lambda a,v)=\lambda\varphi(a,v)$
\item $\varphi(1,v)=v$
\item $\varphi(ab,v)=\varphi(a,\varphi(b,v))$. 
\end{enumerate}

Definimos $\rho:A\to \mathrm{End}(V)$, $\rho(a): v\mapsto\varphi(a,v)$. Las reglas anteriores nos dicen que $\rho$ es lineal, es decir, $\rho(a)(v+w)=\rho(a)(v)+\rho(a)(w)$, $\rho(a)(\lambda v)=\lambda\rho(a)(v)$. Además $\rho(a+b)=\rho(a)+\rho(b)$, $\rho(\lambda a)=\lambda\rho(a)$, $\rho(1)=id_V$ y $\rho(ab)=\rho(a)\circ\rho(b)$. Estas últimas propiedades se resumen en que $\rho$ es un morfismo de álgebras. Esto nos da la siguiente definición más concisa:

\begin{defi}
Una representación de un álgebra $A$ sobre $k$ es un morfismo de álgebras $A\to\mathrm{End}(V)$ donde $V$ es un espacio vectorial sobre $k$. Decimos también que $V$ es una representación de $A$.
\end{defi}

En el caso del ejemplo anterior, cada $f\in\End(V)$ da a $V$ una estructura de representación de $k[x]$ dada por $k[x]\to\End(V)$ donde $P\mapsto P(f)$. 

\begin{defi}
Una \emph{subrepresentación} de una representación $V$ de $A$ es un subespacio vectorial $L$ de $V$ estable bajo multiplicación por los elementos de $A$, esto es $\forall a\in A$, $\forall v\in L$, $a\cdot v\in L$ (identificamos la representación con $\rho:A\to\End(V)$ y $\rho(a)(v)$ con $a\cdot v$, además sería necesario definir $\rho'(a)=\rho(a)|_L$ como subrepresentación). 
\end{defi}

\begin{defi}
Si $V_1$ y $V_2$ son dos representaciones de $A$, un morfismo de representaciones de $V_1$ en $V_2$ es una alicación lineal $f:V_1\to V_2$ tal que $\forall a\in A$, $\forall v\in V_1$, $f(av)=af(v)$ (interpretar como $f\circ \rho_1(a)(v)=\rho_2(a)\circ f(v)$, donde $\rho_i$ son las representaciones correspondientes). Más guay si lo pensamos com ouna transformación natural: para todo $a\in A$ este diagrama conmuta
\[
\begin{tikzcd}
V_1\arrow[r, "f"]\arrow[d, "\rho_1(a)"] & V_2\arrow[d, "\rho_2(a)"]\\
V_1 \arrow[r, "f"] & V_2
\end{tikzcd}
\]
También se les llama a estos morfismos \emph{intertwining operators}. Un \emph{isomorfismo} de representaciones es un morfismo con two-sided inverse o equivalentemente, un morfismo biyectivo. 
\end{defi}

\begin{ej}
Sea $V$ un espacio vectorial de dimensión 3 sobre $\C$ y sea $f$ un endomorfismo de $V$ que admite, con respecto a cierta base $\{u_1,u_2,u_3\}$, la matriz
\[
\begin{pmatrix}
2 & 1 & 0\\
0 & 2 & 0\\
0 & 0 & 3 
\end{pmatrix}
\]
A $f$ se le asocia una representación de $\C[x]$, $\C[x]\to\End(V)$ mediante $P\mapsto P(f)$. Tenemos que $f(u_1)=2u_1$. Así que por definición $P\cdot u_1=P(f)(u_1)$. En particilar, $x\cdot u_1=f(u_1)=2u_1$. Con $x^2$ sería $f^2(u_1)=f(f(u_1))=2^2u_1$. En general, es fácil observar que $P\cdot u_1=P(2)\cdot u_1$. Tenemos que la recta $\C u_1$ es una subrepresentación de la representación considerada. Lo mismo se puede hacer con $u_3$, por lo que $\C u_3$ es otra subrepresentación. De la matriz obtenemos también que $\C u_1\oplus \C u_2$ es una subrepresentación. Se deja como ejercicio comprobar que cualquier otra recta no es estable bajo el producto, por lo que hemos obtenido todas las ubrepresentaciones. 

Si tuviéramos una matriz diagonal, las subrepresentaciones serían las de dimensión 1 y las generadas por sumas directas de ellas. Descompondríamos así la representación en subrepresentaciones \emph{irreducibles}, que las definimos a continuación.
\end{ej}

\begin{defi}
Una representación es \emph{irreducible} cuando no es $\{0\}$ y no contiene ninguan subrepresentación propia. Una representación es \emph{indecomponible} cuando no se puede obtener como suma directa de subrepresentaciones no triviales. 
\end{defi}

El ejemplo anterior muestra que las dos nociones definidas no son equivalentes. Se deja como ejercicio comprobar que para dimensión finita, se puede proceder a descomponer la representación en suma directa de subrepresentaciones y reiterar hasta que tenemos la representación descompuesta como suma directa de indecomponibles. 

Para cualquier endomorismo $f$ de un espacio vectorial $V$ de dimensión finita sobre $\C$ ha una base con respecto a la cual la matriz de $f$ tiene la forma normal de Jordan (diagonal por bloques, donde los bloques tienen $\lambda$ en la diagonal y 1 en la superdiagonal). Esto se interpreta en términos de representaciones como que cualquier representación de $\C[x]$ indecomponible de dimensión finita es del tipo $P\mapsto P(A)$, donde $A$ es un bloque de Jordan. Estas representaciones están indexadas por pares $(\lambda,r)$, donde $\lambda$ es el coeficiente de la diagonal y $r$ es el tamaño de la matriz. 


\begin{defi}
Dado un grupo finito $G$, el \emph{álgebra} $k[G]$ del grupo es el espacio vectorial con base los elementos de $G$ con el producto bilineal que existiende la multiplicación de $G$. No hay problema en definirlo para un grupo finito, pero no vamos a ver representaciones de grupos infinitos. 
\end{defi}

\begin{ej}
Sea $S_2=\{Id,\tau\}$ el grupo de dos elementos. Entonces $k[S_2]$ son las expresiones de la forma $(aId+b\tau)(cId+d\tau)$. 
\end{ej}

\begin{lemma}[de Schur]
Sea $A$ un álgebra, $V_1,V_2$ dos representaciones de $A$ y $f:V_1\to V_2$ un morfismo de representaciones. Entonces:
\begin{enumerate}
\item Si $V_1$ es irreducible, entonces $f$ es 0 o inyectiva.
\item Si $V_2$ es irreducible, entonces $f$ es 0 o sobreyectiva.
\item Si $V_1$ y $V_2$ son irreducibles, entonces $f$ es 0 o isomorfismo. 
\end{enumerate}
\end{lemma}
\begin{proof}\
\begin{enumerate}
\item $\ker f$ es una subrepresentación de $V_1$, es decir, $\forall a\in A$ y $\forall v\in\ker f$ se tiene que $av\in\ker f$. Esto es fácil de ver, si $v\in\ker f$, $f(v)=0$, pero entonces $af(v)=f(av)=0$, luego $av\in\ker f$. Como $V_1$ es irreducible, entonces solo hay dos posibilidades, $\ker f=0$ ($f$ inyectiva) o $\ker f=V_1$ ($f=0$). 
\item Para la imagen igual porque también es una subrepresentación de $V_2$.
\item Unir resultados anteriores. 
\end{enumerate}
\end{proof}

\begin{coro}[lema de Schur para $k$ algebraicamente cerrado]
Sea $A$ un álgeba y $V$ una representacion irreducible de $A$ de dimensión finita. Sea $f:V\to V$ un morfismo de representaciones. Como $k$ es algebraicamente cerrado, $f$ tiene un autovalor $\lambda\in k$. Se tiene que $f-\lambda Id$ es un morfismo de representaciones. Este morfismo no es inyectivo (si $v$ es un autovector, ($f-\lambda Id)(v)=0$), por lo que no puede ser un isomorfismo, así que es 0, es decir, $f=\lambda Id$. 
\end{coro}
\begin{proof}
$a(f-\lambda Id)(v)=af(v)-a\lambda v=f(av)-\lambda av=(f-\lambda Id)(av)$.
\end{proof}


\section{Ideales}

Definición

Weyl algebra (1.7)

\begin{prop}
Los monomios $\{x^iy^j\}$ son una base para el álgebra de Weyl $W$.
\end{prop}
\begin{dem}[Independencia lienal]
Consideramos la representación $W\to\End(\R[t])$ expresada como la derivada. Obsérvese que la derivada de $t\cdot f$ es $f+tf'$. En términos del operador $\partial=\frac{d}{dt}$ y la función $x: f\mapsto tf$ tenemos que $\partial x(f)=f+x\partial (f)$. Esta identidad es equvialente a $\partial x=1+x\partial$ (de aquí la relación del cociente). Esta representación está bien definida porque es la aplicación inducida en el cociente de $k\gene{x,y}\to\End(\R[t])$ dada por $x\mapsto t$ (multiplicación por $t$), $y\mapsto\partial$, ya que $yx-xy-1\mapsto \partial x-x\partial -1=0$. 

Sea $L=\sum a_{ij}x^iy^j\in W$ y suponemos que $L=0$. Debemos demostrar que todos los $a_{ij}=0$. $L$ se puede escribir también como $\sum P_j(x)y^j$ con $P_j(x)=\sum a_{ij}x^j$. Como $L=0$, también su imagen por la representación lo es, es decir, $\sum P_j(x)\partial^j=0$ como elemento de $\End(\R[t])$ (entiendiendo $x$ como multiplicación por $t$). Consideramos $(\sum P_j(x)\partial^j)1=P_0(x)1=P_0(t)$ (todas las derivaciones se anulan en la constante). Consideramos también 
\[
(\sum P_j(x)\partial^j)t=(P_0(x)+P_1(x)\partial)(t)=tP_0(t)+P_1(t)
\]
Podemos continuar aumentando el grado, y por hipótesis todos los polinomios se anulan. Por inducción se prueba que $P_i(t)=0$ para todo $i$. Esta prueba da problemas para característica positiva. En el libro se hace una modificación para salvar este problema.

\end{dem}


\section{Quivers}

Puñalada a quien le llame carcaj

Lo resumo en que un quiver es una categoría pequeña en la que no dibujamos las identidades ni las composiciones, y una representación de un Quiver es un functor entre este categoría y la categoría de espacios vectoriales sobre un cierto cuerpo. Por tanto, un morfismo de representaciones no es más que una transformación natural. 

Debo pensar cómo categorizar la definición de subrepresentación en términos de los morfismos de inclusión. 

\section{Álgebras de Lie}
La identidad de Jacobi sale de que el corchete es una derivaciónnnnnn

Recordemos que la antisimetría salía de que $[a,a]=0$.

\begin{teorema}
Todo álgebra de Lie de dimensión fintia es isomorfo a un subálgebra de un álgebra de Lie de matrices $M_n(k)$ con el corchete $[A,B]=AB-BA$. 
\end{teorema}
\end{document}

