	\documentclass[twoside]{article}
\usepackage{../../estilo-ejercicios}
%\renewcommand{\baselinestretch}{1,3}
%--------------------------------------------------------
\begin{document}

\title{Ejercicios de Teoría de Representaciones}
\author{Javier Aguilar Martín y Rafel González López}
\maketitle

\section{Ejercicio del profesor}


\begin{ejercicio}{1}
Demostrar que si $A$ es conmutativa y $k$ es algebraicamente cerrado, todas las representaciones irreducibles de dimensión finita de $A$ son de dimensión 1. 
\end{ejercicio}
\begin{solucion}
Sea $V$ una representación cualquiera de $A$. Para $a,b\in A$ cualquiesquiera, tenemos $ab=ba$, luego para todo $v\in V$ tenemos $ab(v)=ba(v)$
\[
\begin{tikzcd}
V\arrow[r, "b"] \arrow[d, "a"] & V\arrow[d, "a"]\\
V\arrow[r, "b"] & V
\end{tikzcd}
\]
De aquí deducimos que la multiplicación por $b$ es un morfismo de representaciones. Por tanto, $b=\lambda_b Id$ en virtud del lema de Schur, es decir, la multiplicación por $b$ es la multiplicación por un número $\lambda_b\in k$. Supongamos que $V$ tuviera dimensión al menos 2. Entonces podemos escoger dos vectores linealmente independientes $u,v$. Entonces cada recta $ku$ y $kv$ son subrepresentaciones de $V$, pues $bu=\lambda_bu$ y $bv=\lambda_bv$, de modo que las rectas son cerradas bajo el producto. Esto es una contradicción con el hecho de que $V$ sea irreducible. 

Forma alternativa de decirlo: todo subespacio de dimensión 1 de $V$ es subrepresentación y como $V$ es irreducible, $\dim(V)=1$. 

\end{solucion}

\newpage

\begin{ejercicio}{2}
Hallar una representación irreducible de dimensión 2 de $\R[x]$. 
\end{ejercicio}
\begin{solucion}
Tenemos que definir $\rho:\R[x]\to \End(\C)$ viendo $\C$ como $\R$-espacio vectorial. Definimos el homomorfismo de álgebras inducido por extensión lineal de $x\mapsto i$ (multiplicación por $i$). Como no hay ningún subespacio propio no trivial invariante por esta multiplicación, la representación es irreducible.  
\end{solucion}

\newpage

\begin{ejercicio}{3}
Si en $\End_A(V)$ hay solamente la identidad y sus múltiplos, entonces $V$ es indecomponible.
\end{ejercicio}
\begin{solucion}
Supongamos que existen una subrepresentaciones $W_1,W_2\subset V$ tales que $W_1\oplus W_2= V$. Dado $v\in V$, podemos descomponerlo de forma única como $v=w_1+w_2$, donde $w_1\in W_1$ y $w_2\in W_2$. Es fácil comprobar que las proyecciones $p_i:V\to W_i$ son morfismos de representaciones, pero $p_i$ es un múltiplo de la identidad si y solo si $W_i=V$ o bien $W_i=0$. 
\end{solucion}

\newpage

\section{Ejercicios del libro}

\begin{ejercicio}{1.20}
Sea $V$ una representación vectorial no trivial de dimensión finita de un álgebra $A$. Probar que tiene una subrepresentación irreducible. Probar entonces con un ejemplo que esto no siempre se cumple para representaciones de dimensión infinita.
\end{ejercicio}
\begin{solucion}
Si $V$ es irreducible, ya está. Si no, entonces tiene alguna subrepresentación propia $V_1$ no trivial de dimensión estrictamente menor. Repetimos recursivamente sobre este procedimiento con $V_1$ y como la dimensión es finita, el proceso termina, dando una subrepresentación irreducible.

Construimos ahora un contraejemplo para el caso de dimensión infinita. Sea $A=\C[x]$ y consideramos la representación regular (ejemplo 1.10) de $A$ actuando sobre sí mismo con la multiplicación del anillo. Entonces cualquier elemento no nulo de $a$ genera una subrepresentación isomorfa a $A$, por lo que no puede tener subrepresentaciones irreducibles. 

%Sea $A=\C[C_\infty]$ el álgebra del grupo cíclico infinito $C_\infty=\gene{c}$. Sea $V=\C[T,T^{-1}]$ el anillo de polinomios de Laurent, es decir, 
%\[
%V=\{a(T)=\sum_{i=m}^n a_iT^i\mid m,n\in\Z, m\leq n, a_i\in\C\}
%\]
%
%Podemos convertir $V$ en un $A$-módulo haciendo que el elemento $c^j\in C_\infty$ ($j\in\Z$) actúe con una traslación de $j$ posiciones, es decir,
%\[
%c^j\cdot a(T)=T^ja(T)
%\]
%Veamos que $V$ no contiene submódulos irreducibles. Supongamos por el contrario que existe un submódulo irreducible $W\neq\{0\}$. Si $a(T)=\sum_{i=m}^n a_iT^i\neq 0$ tiene $a_m\neq 0\neq a_n$, entonces llamamos a $|n-m|$ la \emph{anchura} de $a(T)$. Por tanto, todos los elementos no nulos de $V$ tienen anchura positiva. Esto implica que todo submódulo de $V$ contiene elementos de anchura mínima. Pero si $W$ es un submódulo irreducible, entonces el espacio $(T-1)W=\{(T-1)a(T)\mid a(T)\in W\}$ es un subespacio de $W$. 
%
%ME FALTA EL CONTRAEJEMPLO
%\url{https://math.stackexchange.com/questions/1041759/infinite-dimensional-representation-such-that-every-subrepresentation-is-reducib} (prefiero el segundo)
\end{solucion}

\newpage

\begin{ejercicio}{1.21}
Sea $A$ un álgebra sobre un cuerpo $k$ algebraicamente cerrado. El centro $Z(A)$ de $A$ es el conjunto de los elementos $z\in A$ que conmutan con todos los elementos de $A$. 
\begin{enumerate}[(a)]
\item  Probar que si $V$ es una representación irreducible de dimensión finita de $A$, entonces cualquier elemento $z\in Z(A)$ actúa en $V$ como multiplicación por algún escalar $\chi_V(z)$. Probar que $\chi_V:Z(A)\to k$ es un homomorfismo. Se llama \textbf{caracter central} de $V$.

\item Probar que si $V$ es una representación de dimensión finita indecomponible de $A$, entonces para cualquier $z\in Z(A)$, el operador $\rho(z)$ por el cual $z$ actúa en $V$ tiene solo un autovalor $\chi_V(z)$, igual al escalar por el cual $z$ actúa en alguna subrepresentación irreducible de $V$. Por tanto, $\chi_V:Z(A)\to k$ es un homomorfismo, que se llama de nuevo caracter central de $V$.

\item ¿Es $\rho(z)$ necesariamente un operador escalar?

\end{enumerate}
\end{ejercicio}
\begin{solucion}
\begin{enumerate}[(a)]
\item[]
\item Básciamente es seguir la demostración del ejercicio $1$. Si $z$ conmuta con todos los elementos de $a$, naturalmente para $v\in V$ y $a\in A$
$$
\rho(z)(\rho(a)(v)) = \rho(za)(v) = \rho(az)(v) = \rho(a)(\rho(z)(v))
$$ 
Por lo que $\rho(z)$ es un morfismo de representaciones. Por la hipótesis de que $V$ es irreducible y de dimensión finita, y al ser $k$ algebraicamente cerrado, deducimos que $\exists \X_V(z) \in k$ tal que $\rho(z)=\X_V(z)Id$.

Es claro que $Z(A)$ es una subálgebra, pues la unidad está y si $z,w$ conmutan con todo $A$, también lo hacen $z+w,zw,sz$ también lo hacen, con $s\in k$. Para ver que $\X_V(z)$ es un morfismo de álgebras (viendo $k$ como álgebra), sean $z,w\in Z(A)$, $s \in k$, entonces
\begin{align*}
\rho(z+w) &= \rho(z)+\rho(w) = \X_V(z)Id + \X_V(w)Id = (\X_V(z)+\X_V(w))Id \\
\rho(sz) &= s\rho(z) = s\X_V(z)Id = (s\X_V(z))Id\\
\rho(1_A)&=Id = (1_k)Id\\
\rho(zw) &= \rho(z)(\rho(w)) =\rho(z)(\X_V(w)Id) = \X_V(z)Id\X_V(w)Id = \X_V(z)\X_V(w)Id
\end{align*}
\item Tenemos que para todo $z\in Z(A)$ y todo $\lambda\in k$,  se verifica que, $\forall a \in A$ y $v \in V$
\begin{align*}
(\rho(z)-\lambda Id)(\rho(a)(v)) &= \rho(z)\circ \rho(a)(v) - \lambda Id (\rho(a)(v)) \\
&= \rho(za)(v) - \lambda \rho(a)(v)\\
&= \rho(az)(v) - \lambda \rho(a) \circ Id (v)\\
&= \rho(a) \circ \rho(z)(v) - \lambda \rho(a)\circ Id(v)\\
&= \rho(a)\circ ( \rho(z) - \lambda Id) (v)
\end{align*}
Por tanto, $\rho(z)-\lambda Id$ es un morfismo de representaciones. En particular, si $v$ es el autovector asociado a un autovalor $\lambda$, 

$$(\rho(z)-\lambda Id)(av)=a(\rho(z)-\lambda Id)(v)=0.$$

En particular, el espacio $V_1(\lambda)$ de autovectores asociados al autovalor $\lambda$ es invariante por esta representación, luego $V_1(\lambda)$ es una subrepresentación. Recordemos que autovectores asociados a distintos autovalores son linealmente independientes y suman, con suma directa, el espacio total por ser $k$ algebraicamente cerrado. Si hay $n$ autovalores distintos $\lambda_1,\dots, \lambda_n$ (debe haber una cantidad finita porque la representación es finita), entonces $V=\bigoplus_i V_{\lambda_i}$, donde $V_{\lambda_i}$ es el autoespacio asociado a $\lambda_i$. Como $V$ es indecomponible, necesariamente $n=1$.


\item No, para $A=k[x]$ ($Z(A)=A$) tras el ejemplo 1.19 se explica que $\rho(z)$ puede ser cualquier bloque de Jordan, por lo que en general no será necesariamente escalar. 

\end{enumerate}
\end{solucion}

\newpage

\begin{ejercicio}{1.22}
Sea $A$ un álgebra asociativa y $V$ una representación de $A$. Denotamos por $\End_A(V)$ al álgebra de morfismos de representaciones $V\to V$. Probar que $\End_A(A)=A^{op}$, el álgebra $A$ con la multiplicación opuesta.
\end{ejercicio}
\begin{solucion}
Definimos la aplicación $A^{op}\to \End_A(A)$ mediante $a\mapsto f_a \equiv (x\mapsto xa)$. Probamos que esta aplicacion es un morfismo de álgebras. Es claro que se trata de una aplicación lineal, por lo que comprobamos solamente que respeta el producto, que denotamos $a*b$, que por definición es $ba$. Tenemos por definición $a*b\mapsto f_{a*b}(x\mapsto x(a*b)=x(ba))$.

Veamos entonces que $f_a\circ f_b=f_{a*b}=f_{ba}$. 
\[
f_a(f_b(x))=f_a(xb)=xba=x(ba)
\]
por lo que la aplicación es un morfismo de álgebras. De hecho preserva la unidad puesto que trivialmente $1\mapsto Id_A$. Además tiene inverso dado por $f\mapsto f(1)$. Es fácil comprobar que efectivamente es el inverso, por lo que tenemos un isomorfismo. 
\end{solucion}

\newpage
\begin{ejercicio}{1.23}
Probar el siguiente ``lema infinito-dimensional de Schur'' (debido a Dixmier): sea $A$ un álgebra sobre $\C$ y $V$ una representación irreducible de $A$ con una base numerable. Entonces cualquier morfismo de representaciones $\phi:V\to V$ es un operador escalar.

Pista. Por el lemma de Schur usual, el álgebra $D=\End_A(V)$ es un álgebra con división. Probar que $D$ es como mucho de dimensión numerable. Suponer que $\phi$ no es escalar y considerar un subcuerpo $\C(\phi)\subset D$. Probar que $\C(\phi)$ es extensión trascendental de $\C$. Deducir de esto que $\C(\phi)$  tiene dimensión no numerable y obtener una contradicción.
\end{ejercicio}
\begin{solucion}
\end{solucion}

\newpage




\begin{ejercicio}{1.24}
Sea $A = k[x_1,\dotsc,x_n]$ y sea $I\neq A$ un ideal de $A$ que contiene todos los polinomios homogéneos de grado $\geq N$. Demostrar que $A/I$ es una representación indecomponible de $A$. 
\end{ejercicio}
\begin{solucion}
Consideremos la acción $\rho(a)(v) = \pi(a)\pi(v)$ donde $\pi$ es la aplicación de paso al cociente. Claramente $\pi$ es $k$-lineal y es compatible con el producto, además
$$
\rho(ab)(v) = \pi(ab)\pi(v) =\pi(a)\pi(bv) = \rho(a)\circ \rho(b)(v)
$$
Por tanto, $A/I$ es una representación de $A$. Para ver que es indecomponible, consideremos una base de $A/I$. Esta esta formada por las proyecciones de los monomios en $x_1,\dotsc,x_n$. Sea $W_1,\dotsc,W_k$ una desomposición propia en suma directa de $A/I$. $\exists! i$ tal que $\pi(1)\in W_i$. Tiene que existir un monomio $\bar{x}_1,\dotsc,\bar{x}_n$ que no pertenezca a $W_i$, pues en otro caso $W_i = A/I$. Supongamos sin pérdida de generalidad que es $\bar{x}_k$. En tal caso
$$
\rho(x_k)(1) = \pi(1)\pi(x_k) = \bar{x}_k \notin W_i
$$
Por tanto, $W_i$ no es subrepresentación. Como esto es independiente de la descomposición escogida, $A/I$ es una representación indecomponible.
\end{solucion}


\newpage

\begin{ejercicio}{1.25}
Sea $V\neq 0$ una representación de $A$. Decimos que un vector $v$ es cíclico si genera $V$, es decir, si $V= Av$. Una representación que admita un vector cíclico se dice cíclica. Demuestra que
\begin{enumerate}[(a)]
\item $V$ es irreducible si y solo si todos los vectores no nulos son cíclicos.
\item $V$ es cíclico si y solo si es isomorfo a $A/I$, donde $I$ es algún ideal de $A$. 
\item Da un ejemplo de una representación indecomponible que no sea cíclica.

\end{enumerate}
\end{ejercicio}
\begin{solucion}
\begin{enumerate}[(a)]
\item[]
\item Supongamos que $V$ es irreducible. Supongamos por reducción al absurdo que $\exists w$ no cíclico. En ese caso $Aw$ está estrictamente contenido en $V$ y, además, es un subespacio vectorial. Además, para todo $z \in Aw$ existe $b \in A$ tal que $\forall a \in A$ se cumple
$$
\rho(a)(z) = \rho(a)\circ \rho(b) (w) = \rho(ab) (w) \in Aw
$$
Por tanto, $Aw$ es una subrepresentación de $A$, lo cuál es una contradicción.


Recíprocamente, supongamos que todos los vectores no nulos son cíclicos. Si $V$ no es irreducible debe existir $W\subset V$ que sea subrepresentación, es decir, que $AW\subset W$. Sea $w\in W$ no nulos y sea $v \notin W$, debe existir un $a \in A$ tal que $aw = v$, lo cuál es una contradicción.
\item Supongamos que $V$ es cíclico. Entonces existe $v\in V$ tal que $Av = V$. Consideramos el morfismo de $A$-módulos $\phi=\rho(-)(v):A\to V$ definido como $a\mapsto \rho(a)(v)=av$. Como $Av=V$, este morfismo es sobreyectivo, y para $I=\ker\phi$ se tiene que $A/I\cong V$. Este isomorfismo es como $A$-módulos, pero $I$ no es solo un submódulo sino que es un ideal de $A$. 

El recíproco es trivial porque basta considerar la imagen de $\overline{1}\in A/I$ mediante el isomorfismo $A/I\cong V$. 

\item Sea $A = \C[x,y]/I_2$, visto como $\C$-ev (intuyo), donde $I_2$ es el ideal generado por los polinomios homogéneos de grado $\geq 2$ y $V = A^*$ donde $A^*$ es el espacio de los funcionales lineales de $A$ con la acción dada por $\rho(a)(f)(b) = f(ba)$. 

Veamos primeramente que $V$ no puede tener vectores cíclicos. Sea $f \in A^*$ no nulo, entonces 

Para ver que $V$ es indecomponible, supongamos 


Sea $C[x,y]$ e $I$ el ideal generado por los polinomios homogeneos de grado $2$. Entonces $C[x,y]/I$ es una representación indecomponible de $C[x,y]$ por el ejercicio anterior. 
\end{enumerate}
\end{solucion}


\end{document}
http://brianbi.ca/etingof/2.3