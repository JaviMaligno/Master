\documentclass[twoside]{article}
\usepackage{../../estilo-ejercicios}
\usepackage{manfnt}
%\renewcommand{\baselinestretch}{1,3}
%--------------------------------------------------------
\begin{document}

\title{Ejercicios Extras de Teoría de Representaciones}
\author{Javier Aguilar Martín, Rafael González López y Diego Pedraza López}
\maketitle

Estos son ejercicios no propuestos del curso de Teoría de Representaciones.

\begin{ejercicio}{1.49}
\mbox{}
\begin{enumerate}[(a)]
\item Let $U$ be any $k$-vector space.
Construct a natural bijection between bilinear maps $V \times W \to U$ and linear maps $V \otimes W \to U$.
\item Show that if $\{v_i\}$ is a basis of $V$ and $\{w_j\}$ is a basis of $W$ then $\{v_i \otimes w_j\}$ is a basis of $V \otimes W$.
\item Construct a natural isomorphism $V^* \otimes W \to \Hom(V, W)$ in the case when $V$ is finite dimensional (``natural'' means that the isomorphism is defined without choosing bases).

\item Let $V$ be a vector space over a field $k$.
Let $S^nV$ be the quotient of $V^{\otimes n}$ ($n$-fold tensor product of $V$) by the subspace spanned by the tensors $T - s(T)$ where $T \in V^{\otimes n}$, and $s$ is some transposition.
Also let $\wedge^n V$ be the quotient of $V^{⊗n}$ by the subspace spanned by the tensors $T$ such that $s(T) = T$ for some transposition $s$.
These spaces are called the $n$-th symmetric, respectively exterior, power of $V$.
If $\{v_i\}$ is a basis of V , can you construct a basis of $S^n V$, $\wedge^n V$?
If $\dim V = m$, what are their dimensions?
\item If $k$ has characteristic zero, find a natural identification of $S^n V$ with the space of $T ∈ V^{⊗n}$ such that $T = sT$ for all transpositions $s$, and of $\wedge^n V$ with the space of $T ∈ V^{⊗n}$ such that $T = -sT$ for all transpositions $s$.
\item Let $A : V \to W$ be a linear operator.
Then we have an operator $A^{⊗n} : V^{⊗n} \to W^{⊗n}$, and its symmetric and exterior powers $S^n A : S^n V \to S^n W$, $\wedge^n A : \wedge^n V → \wedge^n W$ which are defined in an obvious way.
Suppose $V = W$ and has dimension $N$, and assume that the eigenvalues of $A$ are $λ_1, \dots, λ_N$.
Find $Tr(S^n A)$, $Tr(\wedge^n A)$.
\item Show that $\wedge^N A = det(A)Id$, and use this equality to give a one-line proof of the fact that $\det(AB) = \det(A) \det(B)$.
\end{enumerate}
\end{ejercicio}

\newpage

\begin{ejercicio}{1.51} Tercer Problema de Hilbert.
It is known that if $A$ and $B$ are two polygons of the same area then $A$ can be cut by finitely many straight cuts into pieces from which one can make $B$.
David Hilbert asked in 1900 whether it is true for polyhedra in 3 dimensions.
In particular, is it true for a cube and a regular tetrahedron of the same volume?

The answer is ``no'', as was found by Dehn in 1901.
The proof is very beautiful.
Namely, to any polyhedron $A$ let us attach its ``Dehn invariant'' $D(A)$ in $V = \R ⊗ (\R/\Q)$ (the tensor product of $\Q$-vector spaces).
Namely,
\[ D(A) = \sum_a l(a) ⊗ \frac{\beta(a)}{\pi} \]
where $a$ runs over edges of $A$, and $l(a)$, $β(a)$ are the length of $a$ and the angle at $a$.
\begin{enumerate}[(a)]
\item Show that if you cut $A$ into $B$ and $C$ by a straight cut, then $D(A) = D(B) + D(C)$.
\item Show that $α = \arccos(1/3)/π$ is not a rational number.
Hint. Assume that $α = 2m/n$, for integers $m, n$.
Deduce that roots of the equation $x+x^{-1} = 2/3$ are roots of unity of degree $n$.
Conclude that $x^k + x^{-k}$ has denominator $3^k$ and get a contradiction.
\item Using (a) and (b), show that the answer to Hilbert's question is negative. (Compute the Dehn invariant of the regular tetrahedron and the cube).
\end{enumerate}
\end{ejercicio}
\begin{solucion}
\begin{enumerate}[(a)]
\item Una aclaración: el ángulo $\beta(a)$ se llama también ángulo diedro, se corresponde con el ángulo interior de las dos caras que se intersecan en $a$.

Nuestro objetivo es probar que $D(B)+D(C) = D(A)$.
Para una arista $a$, definimos $\tau_A(a) = l(a) ⊗ \frac{\beta(a)}{\pi}$, de manera que $D(A) = \sum_a \tau_A(a)$.
Nuestra estrategia será ver como afecta el corte a $\tau(a)$.
Lo primero que podemos decir es que si $a$ pertenece exclusivamente a $B$ o a $C$, entonces el corte no afecta al ángulo o a la longitud de $a$.
En otro caso, si $a$ --o un corte de $a$-- aparece como $a_1 \in B$ y $a_2 \in C$, hay que ver que $\tau_B(a_1) + \tau_C(a_2) = \tau_A(a)$.
Por último, cualquier arista $a$ que se añada debe tener valor nulo.
Veamos los cortes que puede hacer el plano sobre cada cara del polihedro.
\begin{enumerate}
	\item \textit{Corte a un plano:} De esa forma se crea una nueva arista $a$ que pertenece a $B$ y a $C$. Los ángulos diedros $\alpha$ y $\alpha'$ deben sumar $\pi$ grados.
	Entonces (recuérdese que en $\R/\Q$, todo racional es congruente al $0$):
	\[ l(a)\otimes\frac{\alpha}{\pi} + l(a)\otimes\frac{\alpha'}{\pi} = l(a)\otimes\frac{\alpha+\alpha'}{\pi} = l(a) \otimes 0_{\R/\Q} = 0_V \]
	\item \textit{Corte transversal a una arista:} De esa forma la arista $e$ se divide en dos aristas $e_1$ y $e_2$ tal que $l(e)=l(e_1)+l(e_2)$.
	El ángulo diedro es el mismo $\alpha$ en las dos aristas. Entonces:
	\[ l(a_1)\otimes\frac{\alpha}{\pi} + l(a_2)\otimes\frac{\alpha}{\pi} = l(a_1 + a_2)\otimes\frac{\alpha}{\pi} = l(a)\otimes\frac{\alpha}{\pi} \]
	Por lo tanto $\tau_B(a_1) + \tau_C(a_2) = \tau_A(a)$
	\item \textit{Corte a lo largo de una arista:} De esta forma la arista $a$ aparece en $B$ y $C$, pero el ángulo diedro se divide en dos ángulos $\alpha_1$ y $\alpha_2$ con $\alpha_1+\alpha_2=\alpha$ donde $\alpha$ era el ángulo diedro de $a$ en $A$.
	Siguiendo el mismo argumento que antes se llega a que $\tau_B(a) + \tau_C(a) = \tau_A(a)$.
\end{enumerate}
Por lo tanto, $D(B)+D(C) = D(A)$.

\item Supongamos que $α$ fuera racional.
Escribámoslo como $α = 2m/n$, donde $m, n \in \Z$ y $n > 0$.

\[ \pi\alpha = \arccos(\frac{1}{3}) \]
\[ \cos(\pi\alpha) = \frac{1}{3} \]
\[ \cos(\pi\alpha)+i\sin(\pi\alpha)+\cos(-\pi\alpha)+i\sin(-\pi\alpha) = \frac{2}{3} \]
\[ e^{i\pi\alpha}+e^{-i\pi\alpha} = \frac{2}{3} \]
Luego $e^{i\pi\alpha}$ es una raíz de $x+x^{-1}=2/3$.
Además $e^{i\pi\alpha}$ es una $n$-raíz de la unidad.

Por otro lado, para toda raíz de $x+x^{-1}=2/3$ podemos demostrar por inducción que divide a $x^k+x^{-k} = \frac{a_k}{3^kb_k}$, donde $a_k$ y $b_k$ son enteros no divisibles por $3$.
Haciendo el paso de inducción suponiendo cierto la hipótesis $k$ y observando que:
\begin{align*}
(x^k+x^{-k})(x+x^{-1}) & = x^{k+1}+x^{-k+1}+x^{k-1}+x^{-k-1}\\
 & = (x^{k+1}+x^{-(k+1)})+(x^{k-1}+x^{-(k-1)})
\end{align*}
Entonces:
\[ x^{k+1}+x^{-(k+1)} = \frac{a_k}{3^k b_k} \cdot \frac{2}{3}- \frac{a_{k-1}}{3^{k-1}b_{k-1}}
= \frac{2a_k b_{k-1} - a_{k-1}3^2 b_k}{3^{k+1} b_k b_{k+1}} \]
El numerador necesariamente no es divisible por $3$, por ser la diferencia de un múltiplo y un no múltiplo de $3$.

Con esto se deduce que $3^k$ divide a $x^k+x^{-k}$.
Sin embargo, tomando $k=n$ llegamos a que $x^n+x^{-n}=1+1=2$ es divisible por $3^n$, lo cual es absurdo.

\item La invariante de Dehn del cubo es sencilla, pues todos sus $12$ ángulos son $\pi/2$. Como $\beta(a)/\pi$ es racional para toda arista $a$, necesariamente
\[ D(\mbox{\mancube}) = 0 \]
Para calcular la invariante de Dehn del tetraedro daremos por sabido que todos los ángulos diedros del sólido son $\arccos(1/3)$ (úsese trigonometría y la fórmula generalizada de Pitágoras).
Recordando que $\arccos(1/3)/\pi$ es irracional por el apartado (b), la invariante de Dehn del tetraedro es:
\[ D(\triangle) = 6 l \otimes [\arccos(1/3)/\pi] \]
Donde $l$ es la longitud de las aristas y $[\arccos(1/3)/\pi]$ es una clase no nula en $\R/\Q$.

Como la invariante de Dehn es aditiva respecto al corte como hemos visto en (a), no podemos cortar el cubo en una cantidad finita de trozos que podamos recomponer en un tetraedro.
Hemos encontrado pues un contraejemplo a la hipótesis del tercer problema de Hilbert.
\end{enumerate}
\end{solucion}

\newpage

\begin{ejercicio}{1.54}
Let $V$, $W$, $U$ be finite dimensional representations of a Lie algebra $\mathfrak{g}$.
Show that the space $\Hom_\mathfrak{g} (V ⊗ W, U)$ is isomorphic to $\Hom_\mathfrak{g} (V, U ⊗ W^∗)$. (Here $\Hom_\mathfrak{g} := \Hom_{U (\mathfrak{g})}$).
\end{ejercicio}
\end{document}