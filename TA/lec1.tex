\documentclass[TA.tex]{subfiles}

\begin{document}


%\hyphenation{equi-va-len-cia}\hyphenation{pro-pie-dad}\hyphenation{res-pec-ti-va-men-te}\hyphenation{sub-es-pa-cio}

\chapter{Homología}

Para nociones de homología simplicial, consultar los apuntes de Homología Simplicial. Atención: en homología ordenada no existen los símplices en orden distinto al prefijado, no se toma cociente. 
\section{Homología singular}

Dado un espacio topológico $X$, definimos un $n$-\emph{símplice singular} como una aplicación continua $\sigma_n:\Delta^n\to X$, donde $\Delta^n$ es el $n$-símplice estándar. Denotamos $C_n(X)$ al grupo abeliano libre de todos los $n$-símplices singulares. A las combinaciones de la forma $\sum_{i=1}^n\lambda_i\sigma_n^i$ con $\lambda_i\in\Z$ se las llama $n$-\emph{cadenas} (estos son de hecho todos los elementos de $C_n(X)$. Tenemos definido también el operador borde $d_n:C_n(X)\to C_{n-1}(X)$, que es un homomorfismo de grupos abelianos definido como $d_n(\sigma_n)=\sum_{i=0}^n \sigma_n|_{[v_0,\dots, \hat{v_i},\dots, v_n]}$. Este operador verifica las mismas propiedades que el operador borde entre complejos simpliciales, con idéntica demostración. Así que hemos construido un complejo de cadenas singulares $\{C_n(X), d_n\}_{n\geq 1}=C_*(X)$. En caso de usar coeficientes en un grupo $G$ se indicará como $C_*(X,G)$. Análogamente a la homología simplicial, si $z\in C_n(X)$ y $d(z)=0$, decimos que $z$ es un $n$-ciclo. Si $z=d_{n+1}(z')$ para algún $z'\in C_{n+1}(X)$, decimos que $z$ es un $n$-borde. Como tenemos $B_n(X)=\Ima d_{n+1}\subseteq \ker d_n=Z_n(X)$, definimos el $n$-ésimo grupo de homología singular $H_n(X)=Z_n(X)/B_n(X)$. 


\begin{prop}
Si $X=\sqcup X_i$ (cantidad finita) es una descomposición de $X$ en sus componentes conexas por caminos, entonces $H_n(X)\cong \oplus_i H_n(X_i)$. 
\end{prop}
\begin{dem}
Como un símplice singular siempre tiene imagen conexa por caminos, $C_n(X)$ se escribe como suma directa de $C_n(X_i)$. Como $d_n$ respeta esta descomposición, $\ker d$ e $\Ima d$ se descomponen de la misma forma, por lo que $H_n(X)\cong \oplus_i H_n(X_i)$. \QED
\end{dem}

\begin{prop}\label{2.7}
Si $X\neq\emptyset$ es arco-conexo, entonces $H_0(X)\cong\Z$. 
\end{prop}
\begin{dem}
Por definición $H_0(X)=Z_0(X)/B_0(X)=C_0(X)/Ima d_1$. Definimos una aumentación $\varepsilon:C_0(X)\to\Z$ como $\varepsilon(\sigma_0)=1$ para todo 0-símplice singular. Como $X\neq\emptyset$, $\varepsilon$ es sobreyectivo, porque podemos un el 0-símplice. Vamos a probar que $\ker\varepsilon=\Ima d_1$, con lo que usando el primer teorema de isomorfía tendremos que $H_0(X)\cong \Ima\varepsilon=\Z$. 

Veamos primero $\Ima d_1\subseteq\ker\varepsilon$. Sea $\sigma_1:\Delta^1\to X$, entonces $\varepsilon(d_1(\sigma_1))=\varepsilon(\sigma_1|_{v_1})-\varepsilon(\sigma_1|_{v_0})=0$.

Ahora la inclusión contraria, $\ker\varepsilon\subseteq\Ima d_1$. Sea $z\in C_0(X)$ tl que $\varepsilon(z)=0$. Por definición $z=\sum_{i=1}^k \lambda_i\sigma^i$, donde $\sigma_i$ son aplicaciones constantes $x_i$. Entonces $\varepsilon(z)=\sum_{i=1}^k\lambda_i$. Consideramos los 1-símplices singulares $\tau_i$ que son un camino entre $x_0$ y $x_i$ y definimos $\tau=\sum_{i=1}^k\lambda_i\tau_i\in C_1(X)$. Entonces
\[
d_1(\tau)=\sum_{i=1}^k \lambda_id(\tau_i)=\sum_{i=1}^k \lambda_i(x_i-x_0)=\sum_{i=1}^k\lambda_i\sigma_i-(\sum_{i=1}^k\lambda_i)x_0
\]
Como $\sum_{i=1}^k\lambda_i=0$ por ser $\varepsilon(z)=0$,  $d_1(\tau)=z$. \QED
\end{dem}

\begin{prop}
Si $X$ es un punto, entonces $H_n(X)=0$ para $n>0$ y $H_0(X)=\Z$.
\end{prop}

\begin{dem}
$H_0(X)=\Z$ se tiene por ser $X$ arco-conexo. Para el resto de caso miramos directamente el complejo de cadenas. $C_n(X,\Z)$ tiene solamente un $n$-símplice singular $\sigma_n$, y el operador borde $d(\sigma_n)=\sum_i (-1)^i\sigma_{n-1}$, que es 0 para $n$ impar y $\sigma_{n-1}$ para $n$ par. Entonces tenemos el complejo de cadenas
\[
\cdots\Z\xrightarrow{\cong}\Z\xrightarrow{0}\Z\xrightarrow{\cong}\Z\xrightarrow{0}\Z\to 0
\]
cuya homología es trvial para $n>0$. 
\QED
\end{dem}

Una variación del complejo de cadenas singulares es el complejo de cadenas singulares aumentado, definiendo $\varepsilon:C_0(X)\to \Z$ como $\varepsilon(\sigma_0)=1$ para todo 0-símplice singular. Denotamos como $\widetilde{C}_n$ a los grupos de este complejo, que en realidad coinciden en todos los niveles salvo en -1, que es $\Z$. Hemos visto en la demostración de la proposición \ref{2.7} que $\varepsilon d_1=0$, así que podemos definir la homología reducida con valor
\[
\widetilde{H}_n(X)=\begin{cases}
H_n(X) & n>0\\
H_0(X)/\Z & n=0
\end{cases}
\]
El primer caso es trivial, vamos a probar el segundo, aunque usaremos algunas nociones de álgebra que veremos más adelante.  Tenemos el diagrama conmutativo
\[
\begin{tikzcd}
\ker\varepsilon\arrow[rd,twoheadrightarrow]\arrow[r,twoheadrightarrow] & \ker\overline{\varepsilon}\arrow[r, rightarrowtail]& H_0(X)\arrow[d, twoheadrightarrow, "\overline{\varepsilon}"]\\
C_1(X)\arrow[r, "d_1"]\arrow[d, twoheadrightarrow] & C_0(X)\arrow[r, twoheadrightarrow, "\varepsilon"]\arrow[ur,twoheadrightarrow]  & \Z\\
\Ima{d_1}\arrow[ur, rightarrowtail]
\end{tikzcd}
\]
donde $\overline{\varepsilon}$ es la aplicación inducida en $H_0(X)$ por $\varepsilon$ (es fácil comprobar que está bien definida). El hecho de que $\varepsilon\circ d_1=0$ nos proporciona además $\Ima d_1 \rightarrowtail\ker\varepsilon$. Esto nos da la sucesión exacta corta
\[
\Ima d_1 \rightarrowtail\ker\varepsilon\twoheadrightarrow\ker\overline{\varepsilon}
\]
El primer teorema de isomorfía nos da además $\ker\overline{\varepsilon}\cong \ker\varepsilon/\Ima d_1=\widetilde{H}_0(X)$. Así sustituyendo en el diagrama anterior obtenemos la sucesión exacta corta
\[
\widetilde{H}_0(X)\rightarrowtail H_0(X)\overset{\overline{\varepsilon}}{\twoheadrightarrow}\Z
\]
Como el grupo de la derecha es abeliano libre, la sucesión escinde, con lo que $H_0(X)=\Z\oplus \widetilde{H}_0(X)$.

%$\twoheadleftarrow \twoheadrightarrow \rightarrowtail \leftarrowtail$

\subsection{Invariancia homotópica}

%f_\sharp

Si tenemos una aplicación continua $X\to Y$, se induce una aplicación $f_*: C_n(X)\to C_n(Y)$ definido como $\sigma\mapsto f\circ\sigma$, que conmuta con el operador borde (es un morfismo de complejos de cadenas), como vamos a comprobar.  Si $\sigma\in C_n(X)$, 
\[
f_*(d_n(\sigma))=\sum_{i=0}^n (-1)^i (f\circ\sigma)|_{[v_0,\dots, \hat{v_i}, \dots, v_n]}=d_n(f\circ\sigma)
\]

Esto implica que si $z\in C_n(X)$ con $d(z)=0$, entonces $d_n(f_*(z))=f_*(d_n(z))=0$, por lo que $f_*$ envía ciclos en ciclos. De igual manera, envía bordes en bordes. De este modo, se induce un homomorfismo dentado igual, $f_*:H_n(X)\to H_n(Y)$. 

Además $(g\circ f)_*=g_*\circ f_*$ y $(Id_X)_*=Id_*$ tanto en complejos de cadenas como en homología.


\begin{teorema}
Si $f,g:X\to Y$ son continuas con $f\simeq g$, entonces $f_*=g_*:H_n(X)\to H_n(Y)$ para todo $n\geq 0$. 
\end{teorema}\
 \opencutright
\begin{dem}
%This is where the table goes with text wrapping around it. You may 
%embed tabular environment inside wraptable environment and customize as you like.

%------------------------------------------
%\begin{wrapfigure}{r}{4cm}
%\caption{A wrapped figure going nicely inside the text.}\label{wrap-fig:1}
%\includegraphics[width=4cm]{cilindrosimplice}
%\end{wrapfigure} 
%------------------------------------------
\def\windowpagestuff{\flushright 
   \includegraphics[width=2.8cm,height=4.5cm]{cilindrosimplice}}
   
  
   \begin{cutout}{1}{0.75\textwidth}{0pt}{6}
     \noindent
El primer paso será subdividir $\Delta^n\times I$ en simplices. Sea $\Delta^n\times\{0\}=[v_0,\dots, v_n]$ y $\Delta^n\times\{1\}=[w_0,\dots, w_n]$, donde $v_i$ tiene la misma imagen que $w_i$ mediante la proyección $\Delta^n\times I\to \Delta$. Para cada $0\leq i\leq n$ consideramos el $(n-1)$-símplice $[v_0,\dots, v_i, w_i,\dots, w_n]$. Este símplice se ha conseguido simplemente recorriendo $\Delta^n\times\{0\}$ hasta el vértice $v_i$ y después saltando a $w_i$ para recorrer el resto de $\Delta^n\times\{1\}$, para luego volver a $v_0$. La figura muestra los casos $n=1,2$.
\end{cutout}\
%INTENTAR CON CUTWIN \url{https://tex.stackexchange.com/questions/40806/how-to-wrap-text-around-a-figure-revised}
%\begin{figure}[h!]
%\includegraphics[scale=0.7]{cilindrosimplice}
%\end{figure}

Dada una homotopía $F:X\times I\to Y$ y un símplice singular $\sigma:\Delta^n\to X$, podemos formar la composición $F\circ(\sigma\times Id):\Delta^n\times I\to X\times I\to Y$. Usando esto podemos definir el \emph{operador prisa} $P:C_n(X)\to C_{n+1}(Y)$ como sigue
\[
P(\sigma)=\sum_i(-1)^iF\circ(\sigma\times I)|_{[v_0,\dots, v_i, w_i,\dots, w_n]}
\]
Vamos a demostrar que el operador prisma satisface la relación
\[
dP=g_*-f_*-Pd
\]
Para probarlo calculamos
\[
dP(\sigma)=\sum_{j≤i}
(−1)^i(−1)^jF\circ(σ \times Id)|_{ 
[v_0, \dots, \hat{v}_j , \dots ,v_i,w_i, \dots ,w_n]}
+\sum_{j≥i}
(−1)^i(−1)^{j+1}F\circ(σ\times Id)|_{
[v_0, \dots ,v_i,w_i, \dots ,\hat{w}_j , \dots ,w_n]}
\]
Los términos para $i=j$ se cancelan excepto para $F\circ(σ \times Id)|_{ 
[\hat{v}_0,w_0,\dots ,w_n]}$, que es $g\circ\sigma=g_*(\sigma)$; y $-F\circ(σ \times Id)|_{ 
[v_0,,\dots ,v_n,\hat{w}_n]}$, que es $-f\circ \sigma=-f_*(\sigma)$. Los términos para $i\neq j$ son exactamente $-Pd(\sigma)$. 

Así, si $\alpha\in Z_n(X)$, entonces $g_*(\alpha)-f_*(\alpha)=Pd(\alpha)+dP(\alpha)=dP(\alpha)$ por ser $d\alpha=0$. Entonces $g_*(\alpha)$ y $f_*(\alpha)$ determinan la misma clase de homología al ser su diferencia un borde. 
%Para el caso aumentado creo que P=0 (comprobar).
\end{dem}

En la prueba hemos usado la relación $dP-Pd=f_*-g_*$, que es la que define una \emph{homotopía de complejos de cadenas}. Hemos probado además que una homotopía de complejos de cadenas induce isomorfismo en la homología. Para el caso de homología reducida también es válido este resultado definiendo $f_*:\Z\to\Z$ como 0 y $P:\Z\to C_0(Y)$ como la aplicación nula.

\begin{coro}
Si $f:X\to Y$ es equivalencia homotópica, entonces $f_*$ es isomorfismo.
\end{coro}

\section{Álgebra Homológica}

Asumimos conocidas las definiciones y propiedades básicas de las sucesiones exactas. 

\begin{lemma}[Splitting lemma]
Dada una sucesión exacta de módulos $0\to A\xrightarrow{f}B\xrightarrow{g}C\to 0$ son equivalentes:
\begin{enumerate}[a)]
\item Hay un morfismo $p:B\to A$ tal que $pf=Id:A\to A$.
\item Hay un morfismo $s:C\to B$ tal que $gs=Id:C\to C$. 
\item Existe un isomorfismo $B\cong A\oplus C$ compatible con la sucesión exacta corta, es decir, que hace conmutar el siguiente diagrama
\[
\begin{tikzcd}
A\arrow[r, rightarrowtail, "f"]\arrow[d, equals] & B\arrow[r, twoheadrightarrow, "g"]\arrow[d, "u"] & C\arrow[d, equals] \\
A\arrow[r, "i"] & A\oplus C\arrow[r, "\pi"]& C
\end{tikzcd}
\]
\end{enumerate}
\end{lemma}
\begin{proof}
Que c) implica a) y b) es trivial.

Para ver que a) implica c), sea $p:B\to A$ un homomorfismo que verifique $pf=Id$. En este caso, afirmamos que $B=f(A)\oplus\ker p$. Si $b\in B$, entonces $b=fp(b)+(m-fp(b))$. Es claro que $fp(b)\in f(A)$ y además $p(b-fp(b)=p(b)-pfp(b)=p(b)-p(b)=0$, luego $b-fp(b)\in\ker p$. Se tiene también que si $b\in f(A)\cap\ker p$, entonces $b=f(a)$ y $0=p(b)=pf(a)=a$, por lo que $b=0$, con lo que la afirmación es cierta. Una vez que tenemos $b=f(a)+b'$, definimos el homomorfismo $u(b)=a+g(b')$. Se tiene de hecho que $u(b)=a+g(b)$, pues si $b-b'=f(a)$ y por exactitud $g(b-b')=0$. Esta definición no es ambigua por ser $f$ inyectiva. Por la definición que se ha hecho, el diagrama es conmutativo y el lema de los 5 implica que $u$ es isomorfismo.

Por último veamos que b) implica c). Sea $s:C\to B$ tal que $gs=Id$. Afirmamos que $B=\ker g\oplus s(C)$. Sea $b\in B$, lo escribimos como $b=(b-sg(b))+sg(b)$. Es claro que $sg(b)\in s(C)$ y además $g(b-sg(b))=g(b)-gsg(b)=g(b)-g(b)=0$, por lo que $b-sg(b)\in\ker g$. Si $b\in\ker g\cap s(C)$, entonces $b=s(c)$ y $0=g(b)=gs(c)=c$, por lo que $b=0$, lo que prueba la afirmación. Ahora, como $\ker g=\Ima f$, podemos expresar $b=f(a)+s(c)$. Definimos entonces $u(b)=a+c$, lo cual tiene sentido porque tanto $f$ como $s$ son inyectivas y además $\Ima s\cap\Ima f=\Ima s\cap\ker g=0$. La definición que hemos hecho hace conmutar el diagrama y esto hace que $u$ sea isomorfismo por el lema de los cinco. 
\end{proof}

Cuando $C$ es un objeto proyectivo se verifica la escisión. En particular, cuando sea un objeto libre. Se recuerda que una sucesión de complejos  de cadenas $0\to A^*\to B^*\to C^*\to 0$ se dice exacta si lo es en cada nivel. Se advierte que el hecho de que se tenga escisión en cada nivel no implica que $B^*\cong A^*\oplus C^*$ pues los isomorfismos de cada nivel no tienen por qué ser compatibles con el operador borde. 

\begin{teorema}
Toda sucesión exacta corta de complejos de cadenas $0\to A\xrightarrow{f} B\xrightarrow{g} C\to 0$ induce una sucesión exacta larga
\[
\cdots \to H_n(A)\xrightarrow{f_*}H_n(B)\xrightarrow{g_*}H_n(C)\xrightarrow{\partial}H_{n-1}(A)\to\cdots
\] 
%\xrightarrow[g] pone la g debajo
\end{teorema}
La prueba sigue la clásica estrategia de \emph{diagram chasing} y se puede encontrar a partir de la página 116 de \emph{Algebraic Topology} de Allen Hatcher. 
\section{Homología relativa}

Dado un subespacio $A\subseteq X$, vamos a construir el complejo de cadenas relativo $C_*(X,A)$ asociado al par $(X,A)$, lo que nos dará la homología relativa $H_*(X,A)$. Esto se consigue definiendo $C_n(X,A)=C_n(X)/C_n(A)$. Como $d$ lleva $C_n(A)$ en $C_{n-1}(A)$, podemos definir $d:C_n(X,A)\to C_{n-1}(X,A)$. 

A partir de las sucesión exacta corta evidente $0\to C_*(A)\to C_*(X)\to C_(X,A)\to 0$ surge la sucesión exacta larga de homología relativa
\[
\cdots \to H_n(A)\xrightarrow{f_*}H_n(X)\xrightarrow{g_*}H_n(X,A)\xrightarrow{\partial}H_{n-1}(A)\to\cdots
\]



\end{document}
