	\documentclass[twoside]{article}
\usepackage{../../estilo-ejercicios}
\renewcommand{\baselinestretch}{1,3}
%--------------------------------------------------------
\begin{document}

\title{Ejercicios de Topología Algebraica}
\author{Javier Aguilar Martín}
\maketitle

\begin{ejercicio}{2.1.16}\
\begin{enumerate}[(a)]
\item Show that $H_0(X,A) = 0$ iff $A$ meets each path-component of $X$.
\item Show that $H_1(X,A) = 0$ iff $H_1(A)→H_1(X)$ is surjective and each path-component
of $X$ contains at most one path-component of $A$.
\end{enumerate}
 

\end{ejercicio}
\begin{solucion}\
\begin{enumerate}[(a)]
\item En la sucesión exacta larga de homología relativa, que $H_0(X,A)=0$ nos da
\[
\cdots\to H_0(A)\to H_0(X)\to 0
\]
de donde deducimos que la aplicación es sobreyectiva. Es decir, para cada generador $[\alpha]\in H_0(X)$, que representa una componente conexa por caminos de $X$, existe un elemento $[\beta]\in H_0(A)$ (representando una componente conexa por caminos de $A$) con $i_*[\beta]=[i\circ \beta]=[\alpha]$, donde $i_*:H_0(A)\to H_0(X)$ es la aplicación inducida por la inclusión. Con lo cual la imagen de $\beta$ está contenida en la misma componente conexa por caminos que la de $\alpha$. Así que necesariamente hay al menos una componente conexa por caminos de $A$ contenida en la componente conexa por caminos que representa $[\alpha]$. Como esto se cumple para todos los generadores de $H_0(X)$ tenemos una de las implicaciones.

Recíprocamente, si $A$ interseca cada componente conexa por caminos de $X$, entonces $H_0(A)\to H_0(X)$ es sobreyectiva, pues un generador $[\alpha]\in H_0(A)$ de una componente conexa por caminos de $A$, lo será también de la componente conexa de $X$ en la que se encuentra, esto es, $i_*[\alpha]$ es generador de $H_0(X)$, y como esto ocurre en todas las componentes, tenemos que $i_*(H_0(A))=H_0(X)$. Así, la sucesión exacta larga de homología relativa nos da 
\[
\cdots\to H_0(A)\to H_0(X)\to H_0(X,A)\to 0
\]
Por el primer teorema de isomorfía, denotando $j_*:H_0(X)\to H_0(X,A)$,  $H_0(X)/\ker j_*\cong \Ima j_*$, que con los obtenido anteriormente y la exactitud resulta en $0=H_0(X)/H_0(X)\cong H_0(X,A)$, con lo que hemos probado el resultado. 



\item Usando la sucesión exacta larga de homología relativa, si $H_1(X,A)=0$ tenemos entonces
\[
\cdots\to H_1(A)\to H_1(X)\to 0
\]
que por exactitud nos da que la aplicación es sobreyectiva. Si nos fijamos en lo que viene después del 0 tenemos
\[
0\to H_0(A)\to H_0(X)\to\cdots
\]
de donde se tiene que $i_*:H_0(A)\to H_0(X)$ es inyectiva. Como los generadores de $H_0(A)$ representan las componentes conexas por caminos de $A$ y los generadores de $H_0(X)$ representan las componentes conexas por caminos, el hecho de que la aplicación inducida por la inclusión sea inyectiva implica que cada componente conexa por caminos de $X$ contiene como mucho una componente conexa por caminos de $A$, ya que si tenemos dos elementos distintos $[\alpha],[\beta]\in H_0(A)$, $i\circ\alpha$ e $i\circ\beta$ tienen imagen en la componente conexa de $X$ donde estuvieran las imágenes de $\alpha$ y $\beta$ respectivamente, que no puede ser la misma para ambas por hipótesis. 

Recíprocamente, si cada componente conexa por caminos de $X$ contiene a lo sumo una componente conexa por caminos de $A$, la aplicación $H_0(A)\to H_0(X)$ debe ser inyectiva, de lo contrario habría dos ciclos $\alpha,\beta\in Z_0(A)$ tales que $i_*([\alpha])=i_*([\beta])=[\gamma]$ para $\gamma\in Z_0(X)$. Pero $i_*([\alpha])=[i\circ\alpha]$ e $i_*([\beta])=[i\circ\beta]$. Entonces $i\circ\alpha$ e $i\circ\beta$ representarían ciclos homólogos, para lo cual su imagen tendría que estar en la misma componente conexa de $X$. 

Si denotamos $\partial:H_1(X,A)\to H_0(A)$, por exactitud junto a lo probado en el párrafo anterior tenemos que $\Ima\partial=0$. Por otro lado tenemos la hipótesis de que $i_*:H_1(A)\to H_1(X)$ es sobreyectiva, luego denotando $j_*:H_1(X)\to H_1(X,A)$ tenemos que $\ker j_*=\Ima i_*=H_1(X)$, lo cual implica que $j_*$ es la aplicación nula y por tanto $0=\Ima j_*=\ker\partial$. Ya no queda más que usar el primer teorema de isomorfía para obtener $$H_1(A,X)\cong H_1(A,X)/\ker\partial\cong \Ima\partial=0.$$
\end{enumerate}
\end{solucion}

\newpage

\begin{ejercicio}{2.1.18}
Probar que para el subespacio $\Q\subset\R$, el grupo de homología relativa $H_1(\R,\Q)$ es libre abeliano y encontrar una base.
\end{ejercicio}
\begin{solucion}
%\url{https://math.stackexchange.com/questions/2608865/exact-sequence-of-reduced-homology-groups}
%\url{https://pages.uoregon.edu/njp/635hw2solutions.pdf}
%\url{https://math.stackexchange.com/questions/2674821/homology-groups-in-a-totally-path-disconnected-space}

Sabemos que $\R$ es contráctil, por lo que $H_0(\R)\cong\Z$ y $H_n(\R)=0$ $\forall n\neq 0$. Por otro lado, $\Q$ es totalmente disconexo por caminos y numerable, así que $H_0(\Q)\cong\bigoplus_{i=0}^{\infty}\Z$. Vamos a probar que además $H_n(\Q)=0$ $\forall n\neq 0$ (esta prueba vale para cualquier espacio totalmente disconexo por caminos, con algún matiz en el caso de que no sea numerable). 

Como $\Delta^n$ es conexo, cualquier símplice singular $\sigma:\Delta^n\to\Q$ es una aplicación constante, así que $C_n(\Q)\cong\bigoplus_{i=0}^{\infty}\Z$, esto es, una copia de $\Z$ por cada aplicación constante, o lo que es lo mismo, por cada racional. Como $\sigma:\Delta^n\to\Q$ es constante, también lo es su restricción a cualquiera de las caras, luego $d(\sigma)=0$ si $n$ es par y $d(\sigma)=\sigma$ si $n>0$ es impar. Esto nos da por cálculo directo el resultado anunciado.


Vamos ahora a la sucesión exacta larga de homología reducida, donde sustituyendo los datos calculados obtenemos
\[
0\to H_1(\R,\Q)\xrightarrow{\partial} \bigoplus_{i=0}^{\infty}\Z\xrightarrow{i_*}\Z\to H_0(\R,\Q)\to 0.
\]
La exactitud junto con con el primer teorema de isomorfía nos da $H_1(\R,\Q)\cong\ker i_*$. La aplicación $i_*$ consiste en enviar cualquier generador $[\alpha_q]\in H_0(\Q)$ a $1\in\Z\cong H_0(\R)$, ya que los símplices singulares sobre $\Q$ son constantes, y estas aplicaciones en al pasarlas a $\R$ son homólogas a la misma constante, así que no representan el elemento neutro. Esto significa que $\ker i_*=\{[\alpha_q]-[\alpha_0]\mid 0\neq q\in\Q\}$. Esto prueba que $H_1(\R,\Q)$ es libre abeliano y que tiene base $\{\partial^{-1}([\alpha_q]-[\alpha_0])\mid 0\neq q\in\Q\}$ (la inversa está bien definida porque $\partial$ es inyectiva en este caso).
\end{solucion}
\end{document}
