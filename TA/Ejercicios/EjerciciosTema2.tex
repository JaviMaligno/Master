	\documentclass[twoside]{article}
\usepackage{../../estilo-ejercicios}
\renewcommand{\baselinestretch}{1,3}
%--------------------------------------------------------
\begin{document}

\title{Ejercicios de Topología Algebraica}
\author{Javier Aguilar Martín}
\maketitle

\begin{ejercicio}{2.1.9}\
Calcular los grupos de homología del $\Delta$-complejo $X$ obtenido de $\Delta^n$ identificando todas las caras de la misma dimensión. Esto es, $X$ tiene un solo $k$-símplice para cada $k\leq n$. 

 

\end{ejercicio}
\begin{solucion}\

\end{solucion}

\newpage

\begin{ejercicio}{2.1.14}
Determinar si existe una sucesión exacta corta $$0\to\Z_4\to \Z_8\oplus\Z_2\to \Z_4\to 0.$$ Más generalmente, determinar qué grupos abelianos $A$ forman una sucesión exacta corta $0\to\Z_{p^m}\to A\to\Z_{p^n}\to 0$ con $p$ primo. ¿Y en el caso de las sucesiones exactas cortas $0\to\Z\to A\to\Z_n\to 0$?
\end{ejercicio}
\begin{solucion}

\end{solucion}

\newpage

\begin{ejercicio}{2.2.3}
Sea $f:S^n\to S^n$ una aplicación de grado cero. Probar que existen puntos $x,y\in S^n$ con $f(x)=x$ y $f(y)=-y$. Usar esto para probar que si $F$ es un campo vectorial continuo en la bola unidad $D^n$ en $\R^n$ tal que $F(x)\neq 0$ para todo $x$, entonces existe un punto en $\partial D^n$ donde $F$ apunta radialmente hacia fuera y otro punto en $\partial D^n$ donde $F$ apunta radialmente hacia dentro.
\end{ejercicio}
\begin{solucion}
\end{solucion}

\newpage

\begin{ejercicio}{2.2.10}
Sea $X$ el espacio cociente de $S^2$ bajo las identificaciónes $x\sim-x$ para $x$ en el ecuador $S^1$. Calcular los grupos homología $H_i(X)$. Hacer lo mismo para $S^3$ con puntos antipodales del ecuador $S^2\subset S^3$ identificados. 
\end{ejercicio}
\begin{solucion}
\end{solucion}

\newpage


\begin{ejercicio}{Naranjas}
Para el espacio $X$ de la imagen (donde las esferas se consideran huecas), calcular la homología local de todos sus puntos, hallar la mayor cantidad posible de subespacios $A\subseteq X$ tales que cualquier homeomorfismo $f:A\to X$ satisfaga $f(A)\subseteq A$, y estudiar si es posible asegurar que alguna potencia $f^n$ tenga puntos fijos, y en su caso cuántos. 
\end{ejercicio}

\end{document}
