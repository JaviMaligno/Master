	\documentclass[twoside]{article}
\usepackage{../../estilo-ejercicios}

%--------------------------------------------------------
\begin{document}

\title{Ejercicios de Algebraic Topology, Capítulo 0}
\author{Javier Aguilar Martín}
\maketitle

\begin{lemma}[Ejercicio 0.4]
Un \textbf{retracto de deformación en el sentido débil} de un espacio $X$ sobre un subespacio $A$ es una homotopía $f_t:X\to X$ tal que $f_0=Id$, $f_1(X)\subseteq A$ y $f_t(A)\subseteq A$. Si $X$ retrae con deformación sobre $A$ en el sentido débil, entonces la inclusión $A\hookrightarrow X$ es una equivalencia de homotopía. 

\end{lemma}


%Pongo también el 4 porque hace falta para el 6

\begin{ejercicio}{0.5}
Probar que si un espacio $X$ retrae con deformación a un punto $x\in X$, entonces para cada entorno $U$ de $x$ existe un entorno $V\subseteq U$ tal que la inclusión $V\hookrightarrow U$ es nulhomotópica.
\end{ejercicio}

\begin{solucion}

\end{solucion}



\begin{ejercicio}{0.6}\
\begin{enumerate}

\item[(a)] Sea $X$ el subespacio de $\R^2$ consistente en el segmento horizontal $[0,1]\times\{0\}$ junto con todos los segmentos verticales $\{r\}\times[0,1-r]$ para $r$ racional en $[0,1]$. Probar que $X$ retrae con deformación a cualquier punto del segmento $[0,1]\times\{0\}$, pero no a ningún otro punto. [Ver el problema anterior.]

\begin{figure}[h!]
\centering
\includegraphics[scale=0.7]{peine}
\end{figure}

\item[(b)] Sea $Y$ el subespacio de $\R^2$ que es la unión de una cantidad infinita de copias de $X$ tal como aparece enla figura más abajo. Probar que $Y$ es contráctil, pero que no retrae con deformación sobre ningún punto.
\begin{figure}[h!]
\centering
\includegraphics[scale=0.7]{peines}
\end{figure}

\item[(c)] Sea $Z$ el subespacio zigzag de $Y$ homeomorfo a $\R$ indicado por la línea más gruesa. Probar que hay un retracto con deformación en el sentido débil de $Y$ sobre $Z$, pero no un verdadero retracto con deformación. 
\end{enumerate}
\end{ejercicio}
\begin{solucion}

\end{solucion}

\newpage





%\end{solucion}
%
%\newpage
%
%\begin{ejercicio}{1.6}
%
%\end{ejercicio}
%\begin{solucion}
%
%\end{solucion}
%
%\newpage
%
%\begin{ejercicio}{1.7}
%
%\end{ejercicio}
%\begin{solucion}
%
%\end{solucion}
%
%\newpage
%
%\begin{ejercicio}{1.9}
%
%\end{ejercicio}
%\begin{solucion}
%
%
%
%
%
%\end{solucion}
%
%\newpage
%
%\begin{ejercicio}{1.10}
%
%\end{ejercicio}
%\begin{solucion}
%
%\end{solucion}
%
%\newpage
%
%\begin{ejercicio}{1.11}
%
%\end{ejercicio}
%\begin{solucion}
%
%
%
%\end{solucion}
%
%\newpage
%
%\begin{ejercicio}{1.12}
%
%\end{ejercicio}
%\begin{solucion}
%
%\end{solucion}

\end{document}
