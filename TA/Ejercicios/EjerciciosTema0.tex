	\documentclass[twoside]{article}
\usepackage{../../estilo-ejercicios}

%--------------------------------------------------------
\begin{document}

\title{Ejercicios de Algebraic Topology, Capítulo 0}
\author{Javier Aguilar Martín}
\maketitle

\begin{lemma}[Ejercicio 0.4]
Un \textbf{retracto de deformación en el sentido débil} de un espacio $X$ sobre un subespacio $A$ es una homotopía $f_t:X\to X$ tal que $f_0=Id$, $f_1(X)\subseteq A$ y $f_t(A)\subseteq A$. Si $X$ retrae con deformación sobre $A$ en el sentido débil, entonces la inclusión $A\hookrightarrow X$ es una equivalencia de homotopía. 

\end{lemma}


%Pongo también el 4 porque hace falta para el 6

\begin{ejercicio}{0.5}
Probar que si un espacio $X$ retrae con deformación a un punto $x\in X$, entonces para cada entorno $U$ de $x$ existe un entorno $V\subseteq U$ tal que la inclusión $V\hookrightarrow U$ es nulhomotópica.
\end{ejercicio}

\begin{solucion}
El hecho de que $x$ sea retracto de deformación de $X$ nos da una homotopía $f_t:X\to X$ tal que $f_0=Id_X$, $f_1=x$ y $f_t(x)=x$ para todo $t\in[0,1]$. Si dado un entorno $V\subseteq U$ de $x$, restringimos $f_t$ a $V$ esto nos da una homotopía $f_t|_V: V\to X$. Por tanto, si encontramos un $V$ los bastante pequeño como para que $f_t(V)\subseteq U$ para todo $t\in [0,1]$ tendremos la homotopía deseada entre $V\hookrightarrow U$ y la constante $x$. . 

Consideremos el conjunto $f^{-1}(U)=\{(y,t)\in X\times I, f_s(y)\in U\}$, que es abierto de $X\times I$, que tiene la topología producto. Por tanto, para todo $t\in [0,1]$ existen abiertos $U_t\subseteq X$ conteniendo a $x$ y $G_t\subseteq I$ conteniendo a $t$ tales que $U_t\times G_t\subseteq f^{-1}(U)$. Es claro que $\{x\}\times I\subseteq f^{-1}(U)$ y que variando $t$ podemos recubrir todo $\{x\}\times I$ por conjuntos de la forma $U_t\times G_t$. Al ser $\{x_0\}\times I$, podemos conseguir un recubrimiento finito $\bigcup_k U_{t_k}\times G_{t_k}$ de $\{x_0\}\times I$. Definimos entonces $V=\bigcap_k U_{t_k}$ , que por definición es abierto y verifica $f_t(V)\subseteq U$ para todo $t\in [0,1]$. 

HACER DIBUJO
%https://www3.nd.edu/~lnicolae/ProblemsHatcher.pdf
 
\end{solucion}



\begin{ejercicio}{0.6}\
\begin{enumerate}

\item[(a)] Sea $X$ el subespacio de $\R^2$ consistente en el segmento horizontal $[0,1]\times\{0\}$ junto con todos los segmentos verticales $\{r\}\times[0,1-r]$ para $r$ racional en $[0,1]$. Probar que $X$ retrae con deformación a cualquier punto del segmento $[0,1]\times\{0\}$, pero no a ningún otro punto. [Ver el problema anterior.]

\begin{figure}[h!]
\centering
\includegraphics[scale=0.7]{peine}
\end{figure}

\item[(b)] Sea $Y$ el subespacio de $\R^2$ que es la unión de una cantidad infinita de copias de $X$ tal como aparece enla figura más abajo. Probar que $Y$ es contráctil, pero que no retrae con deformación sobre ningún punto.
\begin{figure}[h!]
\centering
\includegraphics[scale=0.7]{peines}
\end{figure}

\item[(c)] Sea $Z$ el subespacio zigzag de $Y$ homeomorfo a $\R$ indicado por la línea más gruesa. Probar que hay un retracto con deformación en el sentido débil de $Y$ sobre $Z$, pero no un verdadero retracto con deformación. 
\end{enumerate}
\end{ejercicio}
\begin{solucion}\
\begin{enumerate}
\item[(a)] Para probar que $X$ retrae sobre cualquier punto del segmento $[0,1]\times\{0\}$ basta tomar la siguiente homotopía $f_t:X\to X$, siendo $x_0\in [0,1]$ fijo:

\[
f_t((x,y))=\begin{cases}
(1-2t)(x,y)+2t(x,0) & 0\leq t\leq \frac{1}{2}\\
(2-2t)(x,0)+(2t-1)(x_0,0) & \frac{1}{2}\leq t\leq 1
\end{cases}
\]

Para ver que $X$ no se retrae con deformación a ningún otro punto usamos el ejercicio \ref{ejer:0.5}. Sea $x\in X\setminus([0,1)\times\{0\})$. Sea entonces un entorno $U$ de $x$ que no corte a $[0,1]\times\{0\}$. Cualquier entorno $V\subseteq U$ de $x$ tampoco corta a $[0,1]\times\{0\}$, y podemos garantizar que tanto $U$ como $V$ son disconexos por densidad de los racionales. Si $X$ retrae con deformación a $x$, entonces la inclusión $V\hookrightarrow U$ es nulhomotópica, es decir homotópicamente equivalente a una constante, lo cual implicaría que $V$ es contráctil en $U$, y esto no es posible porque $V$ es disconexo. 

\item[(b)]

\item[(c)]
\end{enumerate}

\end{solucion}

\newpage





%\end{solucion}
%
%\newpage
%
%\begin{ejercicio}{1.6}
%
%\end{ejercicio}
%\begin{solucion}
%
%\end{solucion}
%
%\newpage
%
%\begin{ejercicio}{1.7}
%
%\end{ejercicio}
%\begin{solucion}
%
%\end{solucion}
%
%\newpage
%
%\begin{ejercicio}{1.9}
%
%\end{ejercicio}
%\begin{solucion}
%
%
%
%
%
%\end{solucion}
%
%\newpage
%
%\begin{ejercicio}{1.10}
%
%\end{ejercicio}
%\begin{solucion}
%
%\end{solucion}
%
%\newpage
%
%\begin{ejercicio}{1.11}
%
%\end{ejercicio}
%\begin{solucion}
%
%
%
%\end{solucion}
%
%\newpage
%
%\begin{ejercicio}{1.12}
%
%\end{ejercicio}
%\begin{solucion}
%
%\end{solucion}

\end{document}
