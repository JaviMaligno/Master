\documentclass[twoside]{article}
\usepackage{../estilo-ejercicios}
\usepackage{tikz}
\usetikzlibrary{automata,positioning}
\newcommand{\x}{{\mathbf{x}}}
\newcommand{\y}{{\mathbf{y}}}
%--------------------------------------------------------
\begin{document}

\title{Fractales y Procesos Iterativos}
\author{Rafael González López}
\maketitle

\begin{ejercicio}{3}
Sea $\gamma:[0,1]\to \R^n$ una curva en $\R^n$. La curva $\gamma$ se dice rectificable si cumple
$$
L(\gamma) = \sup \sum_{j=1}^N d(\gamma(t_{j-1}),\gamma(t_j)) < +\infty
$$
donde el supremo se toma en todas las particiones $0 = t_0 < t_1< \cdots< t_N = 1$ de $[0,1]$. En el caso de que $\gamma$ sea rectificable se define la longitud de $\gamma$ como $L(\gamma)$. 

Probar que si $\gamma:[0,1]\to \R^n$ es continua, inyectiva y rectificable, entonces
$$
L(\gamma) = H^1(\gamma([0,1]))
$$
Deducir que la dimensión de la imagen de cualquier curva continua en $\R^n$, inyectiva salvo a lo más en un número finito de puntos y que sea rectificable es $1$.
\end{ejercicio}
\begin{solucion}
En primer lugar, recordemos la definición de $H^1_\varepsilon$.
$$
H^1_\varepsilon(E) = \inf\left\{\sum\limits_{i=1}^\infty\text{diam}(A_i)\,\bigg|\,\bigcup\limits_{i=1}^\infty A_i \supseteq E,\,\text{diam}(A_i)\leq \varepsilon\right\}.
$$
Probemos la igualdad. Si $L(\gamma)$ es el valor del supremo, consideremos $\varepsilon >0$ y una partición $t_0<\cdots < t_1$ que 
$$
\sum_{j=1}^n d(\gamma(t_{j-1}),\gamma(t_j)) > L -\varepsilon
$$
Podemos suponer que $d(\gamma(t_{j-1}),\gamma(t_j))<\varepsilon$. Tomamos $A_j = \gamma([t_{j-1},t_j])$.
\end{solucion}
\end{document}