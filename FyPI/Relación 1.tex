\documentclass[twoside]{article}
\usepackage{../estilo-ejercicios}
\usepackage{tikz}
\usetikzlibrary{automata,positioning}
\newcommand{\x}{{\mathbf{x}}}
\newcommand{\y}{{\mathbf{y}}}
%--------------------------------------------------------
\begin{document}

\title{Fractales y Procesos Iterativos}
\author{Rafael González López}
\maketitle

\begin{ejercicio}{3}
Sea $\gamma:[0,1]\to \R^n$ una curva en $\R^n$. La curva $\gamma$ se dice rectificable si cumple
$$
L(\gamma) = \sup \sum_{j=1}^N d(\gamma(t_{j-1}),\gamma(t_j)) < +\infty
$$
donde el supremo se toma en todas las particiones $0 = t_0 < t_1< \cdots< t_N = 1$ de $[0,1]$. En el caso de que $\gamma$ sea rectificable se define la longitud de $\gamma$ como $L(\gamma)$. 

Probar que si $\gamma:[0,1]\to \R^n$ es continua, inyectiva y rectificable, entonces
$$
L(\gamma) = H^1(\gamma([0,1]))
$$
Deducir que la dimensión de la imagen de cualquier curva continua en $\R^n$, inyectiva salvo a lo más en un número finito de puntos y que sea rectificable es $1$.
\end{ejercicio}
\begin{solucion}
En primer lugar, recordemos la definición de $H^1_\varepsilon$.
$$
H^1_\varepsilon(E) = \inf\left\{\sum\limits_{i=1}^\infty\text{diam}(A_i)\,\bigg|\,\bigcup\limits_{i=1}^\infty A_i \supseteq E,\,\text{diam}(A_i)\leq \varepsilon\right\}.
$$
Probemos la igualdad. Sea $E\subset \R^n$ conexo. En primer lugar, notemos que la función $f_a:x\to d(x,a)$, para $a\in E$ fijo, tiene como imagen de $E$ un intervalo. Tenemos que $H^1(E)\geq H_1(f_p(E))$. Pero la segunda medida, por ser $f_p(E)$ un intervalo de $\R$ es precisamente $\text{diam}(E)$. Aplicándolo a $E = \gamma([t_{j-1},t_j])$ tenemos que
$$
L(\gamma) = \sum_{j=1}^N d(\gamma(t_{j-1}),\gamma(t_j)) \leq \sum_{j=1}^N \text{diam}(\gamma([t_{j-1},t_j])) \leq \sum_{j=1}^N H^1(\gamma([t_{j-1},t_j])) = H^1(\gamma([0,1])
$$
En la última igualdad utilizamos que $H^1$ es aditiva sobre conjunto disjuntos y $\gamma$ es inyectiva.
\newpage
Pasemos a la desigualdad recíproca. Naturalmente, se verifica que $\text{diam}(\gamma([0,1])) \leq L(\gamma)$, pues si el diámetro se alcanza para $0\leq x < y\leq 1$, basta tomar la partición formada por $\{0,x,y,1\}$ (pudiendo ser algunos puntos iguales) y usar que la definición de $L(\gamma)$. Por la inyectividad de $\gamma$ y que claramente la longitud es un operado aditivo sobre curvas con imagen disjuntas (o que tienen un punto de enlace), si denotamos $\gamma_j$ a las curvas que se inducen al restringir $\gamma$ a cada intervalo de una partición,
$$
\sum_{j=1}^n \text{diam}(\gamma([t_{j-1},t_j])) \leq \sum_{j=1} L(\gamma_j) = L(\gamma)
$$
Naturalmente, tomando particiones suficientemente pequeñas, podemos suponer sin pérdida de generalidad que $\text{diam}(\gamma([t_{j-1},t_j]))<\varepsilon$ para $\varepsilon>0$ fijo. Por tanto, 
$$
H^1_\varepsilon(\gamma([0,1])) \leq L(\gamma)
$$
dado que $\varepsilon \to H^1_\varepsilon(S)$ es un aplicación uniformemente continua, tomando $\varepsilon\to 0$ tenemos la otra desigualdad.

La última parte del ejercicio es trivial, pues si $\gamma$ no es inyectiva, consideramos las $k$ aplicaciones inyectivas que se induce al restringirnos y aplicamos el resultado anterior a cada una de ellas.
\end{solucion}
\end{document}