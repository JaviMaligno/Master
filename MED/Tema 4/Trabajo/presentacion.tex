\documentclass{beamer}
\usepackage[utf8]{inputenc}
\usetheme{Berlin}
\usepackage[spanish]{babel}
\usepackage{multirow}
%\usepackage{estilo-apuntes}
\usepackage[]{graphicx}
\usepackage{svg}
\usepackage{multicol}
\theoremstyle{definition}

\newtheorem{teorema}{Teorema}
\newtheorem{defi}{Definición}
\newtheorem{prop}[teorema]{Proposición}
\newcommand{\Z}{\mathbb{Z}}
\newcommand{\C}{\mathbb{C}}
\newcommand{\D}{\mathbb{D}}
\newcommand{\E}{\mathbb{E}}
\newcommand{\R}{\mathbb{R}}

\providecommand{\conv}[1]{\overset{#1}{\longrightarrow}}
\providecommand{\convcs}{\xrightarrow{CS}}
\providecommand{\gene}[1]{\langle{#1}\rangle}
\providecommand{\posi}[1]{\left[#1\right]^+}
\addtobeamertemplate{navigation symbols}{}{%
    \usebeamerfont{footline}%
    \usebeamercolor[fg]{footline}%
    \hspace{1em}%
    \insertframenumber/\inserttotalframenumber
}
\setbeamercolor{footline}{fg=black}
\setbeamerfont{footline}{series=\bfseries}

%-----------------------------------------------------------

\title{Minería Estadística de Datos\\
Tema 4 - Redes Neuronales	}
\author{Rafael González López 
}

\institute{
Universidad de Sevilla}
\date{}
 
\begin{document}
\frame{\titlepage}

\begin{frame}
\frametitle{Tabla de contenidos}
\tableofcontents
\end{frame}

\section{Datos}

\begin{frame}
\frametitle{Introducción a los datos}
\begin{table}[ht]
\centering
\begin{tabular}{rrrrrrr}
  \hline
 & V780 & V781 & V782 & V783 & V784 & V785 \\ 
  \hline
1 &   0 &   0 &   0 &   0 &   0 &   2 \\ 
  2 &   0 &   0 &   0 &   0 &   0 &   3 \\ 
  3 &   0 &   0 &   0 &   0 &   0 &   0 \\ 
  4 &   0 &   0 &   0 &   0 &   0 &   0 \\ 
  5 &   0 &   0 &   0 &   0 &   0 &   2 \\ 
  6 &   0 &   0 &   0 &   0 &   0 &   7 \\ 
  7 &   0 &   0 &   0 &   0 &   0 &   5 \\ 
  8 &   0 &   0 &   0 &   0 &   0 &   2 \\ 
  9 &   0 &   0 &   0 &   0 &   0 &   6 \\ 
  10 &   0 &   0 &   0 &   0 &   0 &   8 \\ 
   \hline
\end{tabular}
\end{table}

\end{frame}




\begin{frame}
\frametitle{Conjunto Entrenamiento}
\begin{figure}[h!]
\includegraphics[scale=0.4]{train.png}
\end{figure}
\end{frame}



\begin{frame}
\frametitle{Conjunto Test}
\begin{figure}[h!]
\includegraphics[scale=0.4]{test.png}
\end{figure}
\end{frame}

\section{Modelo nnet usando ACP}


\begin{frame}
\frametitle{Acierto en el conjunto de Entrenamiento}
\begin{table}[ht]
\centering
\begin{tabular}{rrrrrrr}
  \hline
 & 0 & 1 & 2 & 3 & 9 & Acierto Ent \\ 
  \hline
0 & 603 & 0 & 1 & 2 & 2 & 99.20 \\ 
  1 & 0 & 638 & 3 & 2 & 2 & 98.90 \\ 
  2 & 3 & 0 & 577 & 17 & 3 & 96.20 \\ 
  3 & 0 & 0 & 13 & 613 & 1 & 97.80 \\ 
  9 & 1 & 0 & 2 & 3 & 514 & 98.80 \\ 
   \hline
\end{tabular}
\end{table}
\end{frame}



\begin{frame}
\frametitle{Acierto en el conjunto Test}
\begin{table}[ht]
\centering
\begin{tabular}{rrrrrrr}
  \hline
 & 0 & 1 & 2 & 3 & 9 & Acierto Test \\ 
  \hline
0 & 191 & 0 & 0 & 0 & 0 & 100.00 \\ 
  1 & 0 & 184 & 1 & 2 & 1 & 97.90 \\ 
  2 & 5 & 1 & 203 & 9 & 2 & 92.30 \\ 
  3 & 1 & 0 & 4 & 199 & 4 & 95.70 \\ 
  9 & 1 & 1 & 5 & 3 & 183 & 94.80 \\ 
   \hline
\end{tabular}
\end{table}
\end{frame}


\section{Modelo Deep Learning con H20}

\begin{frame}
\frametitle{Acierto en el conjunto Test}
\begin{table}[ht]
\centering
\begin{tabular}{rrrrrrr}
  \hline
 & 0 & 1 & 2 & 3 & 9 & Acierto Test \\ 
  \hline
0 & 190.00 & 0.00 & 0.00 & 0.00 & 1.00 & 99.50 \\ 
  1 & 0.00 & 186.00 & 0.00 & 2.00 & 0.00 & 98.90 \\ 
  2 & 3.00 & 0.00 & 212.00 & 5.00 & 0.00 & 96.40 \\ 
  3 & 0.00 & 0.00 & 2.00 & 204.00 & 2.00 & 98.10 \\ 
  9 & 1.00 & 1.00 & 2.00 & 2.00 & 187.00 & 96.90 \\ 
   \hline
\end{tabular}
\end{table}
\end{frame}

\begin{frame}
\begin{center}
\huge{¡Gracias por su atención!}
\end{center} 
\end{frame}
\end{document}
