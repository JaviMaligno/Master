\documentclass[TFM.tex]{subfiles}

\begin{document}


\chapter{Proof of Deligne Conjecture}

AQUÍ LO QUE PUEDA SACAR DE HINICH

\url{https://en.wikipedia.org/wiki/Bialgebra}

\url{https://en.wikipedia.org/wiki/Coalgebra}

SECCIÓN 5.5 (CREO QUE PUEDO ESQUIVAR LO DE DG BIALGEBRA DANDO DIRECTAMENTE LA ESTRUCTURA EN EL GRADED VECTOR SPACE)

CREO QUE DE 6 ME LO PUEDO SALTAR, PORQUE AUNQUE CONSTRUYE LA APLICACIÓN, REQUIERE LAS CONSTRUCCIONES RARAS DEL PRINCIPIO (SALVO QUE EN ALGÚN MOMENTO LAS ENTIENDA, TIENEN UNA DESCRIPCIÓN EXPLÍCITA EN LAS PÁGINAS 5-6 PERO NO SÉ QUÉ REPRESENTA $S_n$ EN CADA CASO) ASÍ QUE SOLO COMENTAR QUE HACE FALTA LO DE ETINGOF-KAZHDAN


EN ALGUNO DE LOS PAPERS PONE QUE KOSZUL ES CUANDO ES CUADRÁTICA (GENERADA POR LA OPERACIÓN DE GRADO 2) Y LAS RELACIONES ESTÁN EN $V\oplus (V\otimes V)$ ASÍ QUE ENTIENDO QUE VALENCIA 0 ES EN EL CUERPO Y POR ESO VALENCIA 3 ES AHÍ

QUIÉN COÑO ES $G_\infty$? ¿ES SIMPLEMENTE PONERLE EL INFINITO COMO EN 3.1.4 (PARA LO CUAL TENGO QUE ENTENDER MEJOR LAS CONSTRUCCIONES ESAS DE MIERDA)? (Y DE ALGÚN MODO SALE EL OPERAD, QUIZÁ POR LO QUE SE CUENTA ANTES COMO LAS CONMUTATIVAS Y LAS ASOCIATIVAS Y LAS LIE) EN ESE CASO EL EJEMPLO 3.1.7 DARÍA YA UNA APLICACIÓN $G_\infty\to G$ PERO SUPONGO QUE HAY QUE DE TODOS MODOS HAY QUE PROBAR QUE SE TIENE UNA APLICACIÓN QUE ES EQUIVALENCIA

\end{document}
