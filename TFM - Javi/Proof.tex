\documentclass[TFM.tex]{subfiles}

\begin{document}


\chapter{Proof of Deligne Conjecture}

Now that we know the quasi-isomorphism $C_*(E_2)\simeq H_*(E_2)$, we can prove the Deligne conjecture if we find an action of $H_*(E_2)$ on the Hochschild complex $C^*(A;A)$ inducing the Gerstenhaber algebra structure on $H^*(A;A)$. In this section, we will denote $G=H_*(E_2)$. For an operad $\OO$ of a certain kind that we will call \emph{quadratic}, there is an operad $\OO_\infty$ and a map $\OO_\infty\to\OO$, which is a quasi-isomorphism under certain conditions that $G$ satisfies, so we have a quasi-isomorphism $G_\infty\to G$. From this point, the idea is to factor this map through
\[
G_\infty\to \BB,\BB\to G
\]
where $\BB$ is an operad which acts on $C^*(A;A)$ inducing the Gerstenhaber algebra structure on $H^*(A;A)$.  This map won't be constructed explicitly since it relies on an isomoprhic operad $\widetilde{\BB}\cong\BB$, which is obtained using tools of Etingof-Kazhdan quantization theory \cite{EK} (see also Section 7 of \cite{Hinich}). Finally, since $\BB\simeq G$, we obtain the wanted action of $G$ on $C^*(A;A)$. 


\section{Quadratic operads and operad of homotopy Gerstenhaber algebras}

\begin{defi}
An operad $\OO$ of graded vector spaces is called \emph{quadratic} if it is generated as an operad by $O(2)$ and relations spanned by composition of two operations. 
\end{defi}

In chapter 1 we explained that $G=H_*(E_2)$ was generated by $G(2)=H_*(E_2(2))$ and their relations involved compositions of $\mu$ and $l$ as we saw in the previous chapter. %en cada composición intervienen 2, no hay una tirple composición

AQUÍ LO QUE PUEDA SACAR DE HINICH

\url{https://en.wikipedia.org/wiki/Bialgebra}

\url{https://en.wikipedia.org/wiki/Coalgebra}

SECCIÓN 5.5 (CREO QUE PUEDO ESQUIVAR LO DE DG BIALGEBRA DANDO DIRECTAMENTE LA ESTRUCTURA EN EL GRADED VECTOR SPACE)

CREO QUE DE 6 ME LO PUEDO SALTAR, PORQUE AUNQUE CONSTRUYE LA APLICACIÓN, REQUIERE LAS CONSTRUCCIONES RARAS DEL PRINCIPIO (SALVO QUE EN ALGÚN MOMENTO LAS ENTIENDA, TIENEN UNA DESCRIPCIÓN EXPLÍCITA EN LAS PÁGINAS 5-6 PERO NO SÉ QUÉ REPRESENTA $S_n$ EN CADA CASO) ASÍ QUE SOLO COMENTAR QUE HACE FALTA LO DE ETINGOF-KAZHDAN


EN ALGUNO DE LOS PAPERS PONE QUE KOSZUL ES CUANDO ES CUADRÁTICA (GENERADA POR LA OPERACIÓN DE GRADO 2) Y LAS RELACIONES ESTÁN EN $V\oplus (V\otimes V)$ ASÍ QUE ENTIENDO QUE VALENCIA 0 ES EN EL CUERPO Y POR ESO VALENCIA 3 ES AHÍ

%QUIÉN COÑO ES $G_\infty$? ¿ES SIMPLEMENTE PONERLE EL INFINITO COMO EN 3.1.4 (PARA LO CUAL TENGO QUE ENTENDER MEJOR LAS CONSTRUCCIONES ESAS DE MIERDA)? (Y DE ALGÚN MODO SALE EL OPERAD, QUIZÁ POR LO QUE SE CUENTA ANTES COMO LAS CONMUTATIVAS Y LAS ASOCIATIVAS Y LAS LIE) EN ESE CASO EL EJEMPLO 3.1.7 DARÍA YA UNA APLICACIÓN $G_\infty\to G$ PERO SUPONGO QUE HAY QUE DE TODOS MODOS HAY QUE PROBAR QUE SE TIENE UNA APLICACIÓN QUE ES EQUIVALENCIA

---------------------------------------------------------------------------------------------------------------------

PARA DESCRIBIR $G_\infty$ (AUNQUE COMENTARÉ QUE SE PUEDE DESCRIBIR COMO LA RESOLUCIÓN DE KOSZUL DE $G$): GALVEZ-CARRILLO PROPOSITION 16 PAGINA 559 (21 DEL PDF). PRIMERO NECESITO LA DEFINICIÓN DE S DE LA CONVENCIÓN 0.1 EN LA PÁGINA 542 (4 DEL PDF), PARA LA BARRA PÁGINA 554 (16 DEL PDF). AUNQUE TENIENDO EN CUENTA QUE NO ME DA LA APLICACIÓN A $G$, CREO QUE ME TRAE MÁS CUENTA TRATAR DE HACERLO COMO HINICH, QUE NO ES TAN COMPLICADO DE DEFINIR, SOLO DE ENTENDER, PERO CREO QUE PUEDO (Y COMENTAR QUE EN GALVEZ-CARRILLO ESTÁ MÁS EXPLÍCITO)

LA PROPIEDAD DE QUE SEA UN QUASISOMORFISMO ES PARA QUADRATIC OPERAD (ALGEBRAIC OPERADS 7.1.1)

\section{Operad $\BB$ and its action on the Hochschild complex}

\end{document}
