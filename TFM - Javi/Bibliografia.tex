\documentclass[TFM.tex]{subfiles}
\cleardoublepage %para la versión impresa esto dependiendo del número de páginas puede ser mejor quitarlo pero para el pdf es mejor dejarlo
\phantomsection
\addcontentsline{toc}{chapter}{References}

\begin{document}
%\renewcommand\chaptername{\Huge Tema}
%
%\titleformat{\chapter}[display]
%    {\normalfont\huge\bfseries}{\chaptertitlename\ \thechapter}{10pt}{\Huge}
%\titlespacing*{\chapter}{0pt}{-1cm}{10pt}


\begin{thebibliography}{}



\bibitem[CD]{CW} Giusti, Chad; Sinha, Dev.
\emph{Fox-Neuwirth cell structures and the cohomology of symmetric groups}. (English summary) Configuration spaces, 273–298, 
CRM Series, 14, Ed. Norm., Pisa, 2012.


\bibitem[CW]{tesis} Christensen, Steffen; Wahl, Nathalie. \emph{Equivalenace of The Little Disk and
Cacti operads}. Thesis for the Master degree in Mathematics. Department of Mathematical Sciences, University of Copenhagen. %POSIBLEMENTE CAMBIAR POR \url{http://math.uchicago.edu/~may/PAPERS/handout.pdf}

 \url{https://www.math.ku.dk/english/research/tfa/top/paststudents/ms-theses/thesis_steffen_christensen.pdf}


\bibitem[DBN]{1deTamarkin} Bar-Natan, D. \emph{On associators and Grothendieck–Teichmuller group I}, q-alg/9606025


\bibitem[EK]{EK}  Etingof, P., Kazhdan, D. \emph{Quantization of Lie bialgebras}, 2. Selecta Mathematica, New
Series, 4 (1998), 213–231


\bibitem[EMC]{EZ} Samuel Eilenberg, Saunders MacLane, \emph{On the groups $H(Π,n)$}, I, Ann. of Math. (2) 58, (1953), 55–106. (jstor) %1953


\bibitem[GJ]{GJHinich} Getzler, E., Jones, J. \emph{Operads, homotopy algebra and iterated integrals for double loop
spaces}; preprint hep-th/9403055


\bibitem[GM]{Gerstenhaber} Gerstenhaber, Murray. \emph{The cohomology structure of an associative ring}. 
Ann. of Math. (2) 78 1963 267–288. 
16.90 (18.00)  %1963


\bibitem[GTV]{Galvez}  Gálvez-Carrillo, Imma; Tonks, Andrew; Vallette, Bruno. \emph{Homotopy Batalin-Vilkovisky algebras}. J. Noncommut. Geom. 6 (2012), no. 3, 539–602.


\bibitem[HA]{Hatcher} Hatcher, A. (2002). \emph{Algebraic topology}. Cambridge: Cambridge University Press. ISBN: 0-521-79160-X; 0-521-79540-0 


\bibitem[HV]{Hinich} Hinich, Vladimir. \emph{Tamarkin's proof of Kontsevich formality theorem}. (English summary) 
Forum Math. 15 (2003), no. 4, 591–614. 


\bibitem[JGM]{Meneses} J. González-Meneses. \emph{Basic results on braid groups}. Annales mathématiques, Blaise Pascal Working version – October 5, 2010. \url{https://arxiv.org/abs/1010.0321v1}


\bibitem[KM]{Kontsevich} Kontsevich, M. \emph{Operads and motives in deformation quantization}, Lett. Math. Phys.
48(1) (1999), 35–72.

\bibitem[LP]{strong} van der Laan, Pepijn. \emph{Coloured Koszul duality and strongly homotopy operads}. (2004) Available at \url{https://arxiv.org/pdf/math/0312147.pdf}

\bibitem[LV]{AlgebraicOperads} Loday, Jean-Louis; Vallette, Bruno. \emph{Algebraic operads}. Grundlehren der Mathematischen Wissenschaften [Fundamental Principles of Mathematical Sciences], 346. Springer, Heidelberg, 2012. xxiv+634 pp. ISBN: 978-3-642-30361-6

\bibitem[MDJ]{Madsen} Madsen, I., \& Tornehave, J. \emph{From calculus to cohomology : de Rham cohomology and characteristic classes}. Cambridge: Cambridge University Press (1999).

\bibitem[MJ]{Milnor} J. Milnor. \emph{Construction of universal bundles I}. Annals of Mathematics, 63 (1956), 272-284. %1956


\bibitem[MLC]{cuentas} Cohen, Frederick R.; Lada, Thomas J.; May, J. Peter. \emph{The homology of iterated loop spaces}. Lecture Notes in Mathematics, Vol. 533. Springer-Verlag, Berlin-New York, 1976. vii+490 pp. %1976


\bibitem[MP1]{simplicial}  May, J. Peter. \emph{Simplicial objects in algebraic topology}. Reprint of the 1967 original. Chicago Lectures in Mathematics. University of Chicago Press, Chicago, IL, 1992. viii+161 pp. ISBN: 0-226-51181-2


\bibitem[MP2]{May} May, J. P.
\emph{The geometry of iterated loop spaces}. 
Lectures Notes in Mathematics, Vol. 271. Springer-Verlag, Berlin-New York, 1972. viii+175 pp. 


\bibitem[MP3]{tensor-hom} May, J.P.; Sigurdsson, J. (2006). \emph{Parametrized Homotopy Theory}. A.M.S. p. 253. ISBN 0-8218-3922-5. \url{http://www.math.uchicago.edu/~may/EXTHEORY/MaySig.pdf}


\bibitem[OH]{bundle} Osborn, Howard. \emph{Vector Bundles. Volume 1: Foundations and Stiefel-Whitney Classes}. Pure and Applied Mathematics (Academic Pr), January 11, 1983, ISBN-10: 0125293011


\bibitem[RC]{gerstenhaberalgebra} Roger, Claude. \emph{Gerstenhaber and Batalin-Vilkovisky algebras; algebraic, geometric and physical aspects}. Archivum Mathematicum (BRNO), Tomus 45 (2009), 301–324. 

\url{https://www.emis.de/journals/AM/09-4/roger.pdf}


\bibitem[SG]{casi} Segal, Graeme. \emph{Classifying spaces and spectral sequences}. Publications Mathématiques de l'IHÉS, Tome 34 (1968) pp. 105-112.

 \url{http://www.numdam.org/item/PMIHES_1968__34__105_0/} %1968


\bibitem[TD]{Tamarkin} Tamarkin, Dmitry E. \emph{Formality of chain operad of little discs}. Lett. Math. Phys. 66 (2003), no. 1-2, 65–72. 


\bibitem[TFG]{tfg} Aguilar Martín, Javier; González-Meneses López, Juan; Flores Díaz, Ramón Jesús. \emph{El problema de la palabra en los grupos de trenzas}.

 \url{https://idus.us.es/xmlui/handle/11441/77489}
 
 

\bibitem[VG1]{higher} Voronov, A. A.; Gerstenhaber, M.
\emph{Higher-order operations on the Hochschild complex}. (Russian. Russian summary) 
Funktsional. Anal. i Prilozhen. 29 (1995), no. 1, 1--6, 96; translation in 
Funct. Anal. Appl. 29 (1995), no. 1, 1–5 
 
\bibitem[VG2]{VGH}  Gerstenhaber, Murray; Voronov, Alexander A. \emph{Homotopy $G$-algebras and moduli space operad}. Internat. Math. Res. Notices 1995, no. 3, 141–153.
 
\bibitem[VGD]{Drinfeld} Drinfeld, V. G. \emph{Quasi-Hopf algebras}, Leningrad Math. J. (1990), 1419–1457.



\bibitem[VO]{VO} Voronov, Alexander A. \emph{Homotopy Gerstenhaber algebras}. Conférence Moshé Flato 1999, Vol. II (Dijon), 307–331, Math. Phys. Stud., 22, Kluwer Acad. Publ., Dordrecht, 2000.


\bibitem[YD]{Yau} Yau, Donald. \emph{Colored operads}. Graduate Studies in Mathematics, 170. American Mathematical Society, Providence, RI, 2016. xxviii+428 pp. ISBN: 978-1-4704-2723-8


\bibitem[ZF]{bar} Fiedorowicz, Z. \emph{The symmetric bar construction}, preprint, available at 

\url{https://people.math.osu.edu/fiedorowicz.1/symbar.ps.gz}




%La referencia para lo de fibrado sobre contráctil es trivial \url{https://www.amazon.com/Vector-Bundles-Foundations-Stiefel-Whitney-Mathematics/dp/0125293011} es la proposición 3.5 (comentado en el tex el google books para verla) %\url{https://books.google.es/books?id=9A3bhgoTdicC&pg=PA73&lpg=PA73&dq=fiber+bundle+over+contractible+space&source=bl&ots=UJVyzZUF8_&sig=ACfU3U1MYacjylxaQbgpLujR2XTrngpBsw&hl=es&sa=X&ved=2ahUKEwivj8_txvjgAhXtx4UKHb1AAH4Q6AEwCXoECAIQAQ#v=onepage&q=fiber%20bundle%20over%20contractible%20space&f=false}





\end{thebibliography}

\end{document}