\documentclass[TFM.tex]{subfiles}
\cleardoublepage %para la versión impresa esto dependiendo del número de páginas puede ser mejor quitarlo pero para el pdf es mejor dejarlo
\phantomsection
\addcontentsline{toc}{chapter}{References}

\begin{document}
%\renewcommand\chaptername{\Huge Tema}
%
%\titleformat{\chapter}[display]
%    {\normalfont\huge\bfseries}{\chaptertitlename\ \thechapter}{10pt}{\Huge}
%\titlespacing*{\chapter}{0pt}{-1cm}{10pt}


\begin{thebibliography}{}






\bibitem[GH1963]{Gerstenhaber} hacer citaciones con lo que pone en MathSciNet

\bibitem{Tamarkin} TAMARKIN

\bibitem{Hinich} HINICH

\bibitem{gerstenhaberalgebra} \url{https://www.emis.de/journals/AM/09-4/roger.pdf}

\bibitem{simplicial}




\bibitem{Yau} YAU
\bibitem{May} THE GEOMETRY OF ITERATED LOOP SPACES
\bibitem[HA]{Hatcher} HACTHER

\bibitem[CLM1976]{cuentas} COHEN
\bibitem{bar} ESTE ES FIEDEROWICZ

\bibitem[EMC1953]{EZ} EILENBERG-ZILBER MAP Samuel Eilenberg, Saunders MacLane, On the groups H(Π,n), I, Ann. of Math. (2) 58, (1953), 55–106. (jstor)


\bibitem[ML1956]{Milnor} J. Milnor. \emph{Construction of universal bundles I}. Annals of Mathematics, 63 (1956), 272-284. 

\bibitem{bundle} La referencia para lo de fibrado sobre contráctil es trivial \url{https://www.amazon.com/Vector-Bundles-Foundations-Stiefel-Whitney-Mathematics/dp/0125293011} es la proposición 3.5 (comentado en el tex el google books para verla) %\url{https://books.google.es/books?id=9A3bhgoTdicC&pg=PA73&lpg=PA73&dq=fiber+bundle+over+contractible+space&source=bl&ots=UJVyzZUF8_&sig=ACfU3U1MYacjylxaQbgpLujR2XTrngpBsw&hl=es&sa=X&ved=2ahUKEwivj8_txvjgAhXtx4UKHb1AAH4Q6AEwCXoECAIQAQ#v=onepage&q=fiber%20bundle%20over%20contractible%20space&f=false}

\bibitem{tensor-hom} May, J.P.; Sigurdsson, J. (2006). Parametrized Homotopy Theory. A.M.S. p. 253. ISBN 0-8218-3922-5. \url{http://www.math.uchicago.edu/~may/EXTHEORY/MaySig.pdf}

\bibitem[GS1968]{casi} \url{http://www.numdam.org/item/?id=PMIHES_1968__34__105_0}

\bibitem{1deTamarkin}

\bibitem{GJHinich}

\bibitem{AlgebraicOperads}

\bibitem{EK}  Etingof-Kazhdan (EK en Hinich)

\bibitem{tesis}
\end{thebibliography}

\end{document}