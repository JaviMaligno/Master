\documentclass[TFM.tex]{subfiles}

\begin{document}


\mychapter{0}{Resumen}

La cohomología de Hochschild de un álgebra asociativa sobre un anillo tiene estructura de álgebra de Gerstenhaber, que es equivalente a una acción de la homología del operad de discos pequeños. La conjetura de Deligne pregunta si esta acción proviene de una acción del complejo de cadenas singulares del operad de discos pequeños en el complejo de Hochschild del álgebra asociativa. En este trabajo examinamos la primera prueba histórica de esta conjetura, debida a Tamarkin in Kontsevich, que es válida cuando el anillo es un cuerpo de característica cero.
\end{document}