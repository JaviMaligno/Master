\documentclass[TFM.tex]{subfiles}

\begin{document}



\mychapter{0}{Introduction}

PREGUNTAR A FERNANDO  POR SI LO DEJO COMO ABSTRACT LARGO

HABLAR MUY SUCINTAMENTE DE LA COHOMOLOGÍA DE HOCHSCHILD Y DE QUE TIENE ESTRUCTURA DE ÁLGEBRA DE GERSTENHABER, QUE ESTAS ÁLGEBRAS SON ÁLGEBRAS SOBRE OPERADS DE DISCOS PEQUEÑOS DICIENDO INTUITIVAMENTE LO QUE SON LAS OPERADS Y DE QUÉ VA LO DE LOS DISCOS PEQUEÑOS, DECIR ENTONCES LO DE LA CONJETURA CITANDO DIVERSAS PRUEBAS. DESPUÉS ESTRUCTURA DEL TRABAJO POR CAPÍTULO Y POR ÚLTIMO COMENTAR LO DE LOS CONOCIMIENTOS PREVIOS DE CATEGORÍAS Y TRENZAS

\url{https://ncatlab.org/nlab/show/Deligne+conjecture}
%\url{https://en.wikipedia.org/wiki/Tensor_algebra}
%\url{https://ncatlab.org/nlab/show/Eilenberg-Zilber+map} HACE FALTA LA EXPRESIÓN PARA PROBAR LA ASOCIATIVIDAD DE LA OPERACIÓN INDUCIDA POR EL OPERAD QUE HAGA QUE EL COMPLEJO SIGA SIENDO OPERAD

\url{https://en.wikipedia.org/wiki/S-object} los S-objetos de Hinich

CITAR MI TFG PARA TRENZAS Y MACLANE PARA CATEGORÍAS
\end{document}