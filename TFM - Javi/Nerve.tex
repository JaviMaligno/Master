\documentclass[TFM.tex]{subfiles}

\begin{document}

%\hyphenation{equi-va-len-cia}\hyphenation{pro-pie-dad}\hyphenation{res-pec-ti-va-men-te}\hyphenation{sub-es-pa-cio}
\chapter{Simplicial Sets, Geometric Realization and Classifying Space}

\url{https://en.wikipedia.org/wiki/Simplicial_set} FORMAL DEFINITION Y LAS APLICACIONES FACE Y DEGENERANCY (REPETIR EL VÉRTICE SE TRADUCE EN COORDENADAS COMO PONERLE UN 0 AL SIGUIENTE)

TAMBIÉN ME GUSTA AQUÍ \url{https://ncatlab.org/nlab/show/simplicial+set} VIENE EL NERVE TAMBIÉN EN LOS DOS, AUNQUE QUIZÁ MEJOR COGERLO DE HATCHER


DEBERÍA COMPROBAR QUE LAS APLICACIONES DE LAS QUE SON IMAGENES LAS CARAS Y DEGENERACIÓN CUMPLEN LAS RELACIONES DUALES PARA QUE SEA VERDAD QUE ESTAS CUMPLEN LAS IDENTIDADES SIMPLICIALES


COMENTAR SIMPLICIAL OBJECT EN GENERAL, QUEDA MUY BIEN EL CASO DE AB DE MAY


\section{Simplicial Sets}
\begin{ej}[Nerve of a Category]
\end{ej}
\subsection{Simplicial Objects}

\section{Geometric Realization}

\section{Classifying Spaces}
EN VERDAD ESTO NO SÉ SI DA PARA UNA SECCIÓN PROPIA O VA CON EL NERVIO 
\end{document}

%isotopía \url{https://link.springer.com/chapter/10.1007%2F978-94-015-9319-9_6}
%Orientation preserving (que las cartas conservan la orientación) https://math.stackexchange.com/questions/1319234/meaning-of-the-expression-orientation-preserving-homeomorphism 
%O sea, conserva la orientación como superficie