\documentclass[TFM.tex]{subfiles}

\begin{document}

%\hyphenation{equi-va-len-cia}\hyphenation{pro-pie-dad}\hyphenation{res-pec-ti-va-men-te}\hyphenation{sub-es-pa-cio}
\chapter{Simplicial Sets and Geometric Realization}\label{ch2}




In this chapter we introduce an ubiquitous subject in topology which are simplicial sets (and the generalized notion of simplicial object) and their geometric realization. For example, simplicial objects occur in May's work on recognition principles for iterated loop spaces \cite{simplicial}, which is closely related to our study.

%TAMBIÉN ME GUSTA AQUÍ \url{https://ncatlab.org/nlab/show/simplicial+set} VIENE EL NERVE TAMBIÉN EN LOS DOS, AUNQUE QUIZÁ MEJOR COGERLO DE HATCHER
%
%
%\url{https://ncatlab.org/nlab/show/nerve+and+realization}




\section{Simplicial Sets}

Let $\mathbf{\Delta}$ denote the \emph{simplex category}, whose objects are ordered sets of the form $[n]=\{0<1<\dots< n\}$ and whose morphisms are non-decreasing functions. 

\begin{defi}
A \emph{simplicial set} $X$ is a functor $X:\mathbf{\Delta}^{op}\to\Set$ from the opposite category of $\mathbf{\Delta}$ to the category of sets. The category of such functors is usually denoted $\SSet$. 
\end{defi}

We can make this definition more explicit by looking at the image of the functor $X$. First, consider the unique injection $\delta_i:[n-1]\to[n]$ such that $i$ is not in its image and the unique surjection $\sigma_i:[n+1]\to [n]$ that hits $i$ twice, for $0\leq i\leq n$. It is easily seen that every other morphism of $\mathbf{\Delta}$ can be expressed in terms of such maps. Note that the following relations between $\delta_i$ and $\sigma_i$ hold:
\begin{itemize}
\item $\delta_j\circ\delta_i=\delta_i\circ \delta_{j-1}$ if $i<j$,
\item $\sigma_j\circ \sigma_i=\sigma_{i-1}\circ \sigma_j$ if $i>j$,
\item $\sigma_j\circ\delta_i=\begin{cases}
\delta_i\circ \sigma_{j-1} &\text{if } i<j\\
Id & \text{if } i=j\text{ or }i=j+1\\
\delta_{i-1}\circ \sigma_j & \text{if }i>j+1
\end{cases}$
\end{itemize}
Here, by abuse of notation, we're identifying $\delta_i$ with any map $[n]\to[m]$ that is equal to $\delta_i$ on the subset $[n-1]\subseteq [m]$ and the identity on the rest, and similarly with $\sigma_i$. Hence, we can define a simplicial set $X$ by giving the following data:
\begin{enumerate}
\item for each $n\in\N$, a set $X_n$ called the set of \emph{$n$-simplices};
\item for each $\delta_i:[n-1]\to[n]$ ($0\leq i\leq n$) a map $d_i:X_n\to X_{n-1}$ called the \emph{$i$-th face map};
\item for each $\sigma_i:[n]\to[n+1]$ ($0\leq i\leq n$) a map $s_i:X_n\to X_{n+1}$ called the \emph{$i$-th degeneracy map}.
\end{enumerate} 
These maps are required to satisfy the following equations:
\begin{itemize}
\item $d_i \circ d_j  = d_{j-1} \circ d_i$ if $i<j$,
\item $s_i \circ s_j  = s_j \circ s_{i-1}$ if $i>j$,
\item $d_i \circ s_j =  \begin{cases} s_{j-1} \circ d_i &  \text{if }  i < j \\ Id & \text{if }  i = j \text{ or } i = j+1 \\ s_j \circ d_{i-1} &  \text{if }i > j+1  \end{cases}$
\end{itemize}


\begin{defi}
A \emph{simplicial map} is a morphism between simplicial sets $X$ and $Y$, i.e., a natural transformation $f: X\to Y$.
\end{defi}

Again, we can express this concept in terms of the families $X_n$ and $Y_n$. A simiplicial map $f:X\to Y$ consists of a family $f_n:X_n\to Y_n$ that commutes with the face and degeneracy maps:
\[f_n\circ d_i=d_i\circ f_{n+1}\]
\[f_n\circ s_i=s_i\circ f_{n-1}\]

More generally, one can also define a \emph{simplicial object} in a category $\CC$ as a functor $X:\mathbf{\Delta}^{op}\to\CC$. The category of simplicial objects in $\CC$ is denoted by $\SCC$. 
%QUEDA MUY BIEN EL CASO DE AB DE MAY. AÑADIR MAY A LA BIBLIOGRAFÍA
\begin{ex}[Nerve of a Category]
Let $\CC$ be a small category. We are going to build a simplicial set from the category of morphisms in $C$, called the \emph{nerve} of $\CC$ and written $N(\CC)$. Its $n$-simplices for $n>0$ are the strings $X_0→X_1→\cdots →X_n$ of morphisms in $\CC$. For $n=0$ there is a $0$-simplex for each object of $\CC$. The faces $d_i:N(\CC)_n\to N(\CC)_{n-1}$ of a $n$-simplex are obtained by deleting $X_i$, and then composing the two adjacent morphisms if $i ≠ 0,n$. Thus when $n = 2$ the three faces of
$X_0→X_1→X_2$ are $X_0→X_1$, $X_1→X_2$, and the composed morphism $X_0→X_2$. When $i\in \{0,n\}$, the face maps consist of removing $X_0$ or $X_n$.
The degenerancy maps $s_i:N(\CC)_{n-1}\to N(\CC)_n$ are given by inserting an identity morphism at the object $X_i$. 

\end{ex}




\section{Geometric Realization}

As in \cite[Chapter III]{simplicial}, denote by $\Delta_n$ the topological $n$-simplex, $\Delta_n=\{(t_0,\dots, t_n)\mid 0\leq t_i\leq 1, \sum_i t_i=1\}\subseteq\R^{n+1}$. We define maps $\delta_i:\Delta_{n-1}\to\Delta_n$ and $\sigma_i:\Delta_{n+1}\to\Delta_n$ by
\[
\delta_i(t_0,\dots, t_{n-1})=(t_0,\dots, t_{i-1},0,t_i,\dots, t_{n-1}),
\]
\[\sigma_i(t_0,\dots, t_{n+1})=(t_0,\dots, t_{i-1},t_i+t_{i+1},t_{i+2},\dots, t_{n+1}).\]

Let $X$ be a simplicial set. Give each $X_n$ the discrete topology and form the disjoint union $\overline{X}=\coprod_{n\geq 0}(X_n\times\Delta_n)$. Define an equivalence relation $\sim$ in $\overline{X}$ by
\[(d_ix_n,u_{n-1})\sim(x_n,\delta_i u_{n-1}),\ x_n\in X_n, u_{n-1}\in\Delta_{n-1},\]
\[(s_ix_n,u_{n+1})\sim(x_n,\sigma_iu_{n+1}),\ x_n\in X_n, u_{n+1}\in\Delta_{n+1}.\]
The quotient space $|X|=\overline{X}/\sim$ is called the \emph{geometric realization} of $X$. %Geometrically, there is one $n$-simplex for each $X_n$ and we identify the face and degeneracy maps of the simplicial set with the usual geometric face and degeneracy maps of simplicial complexes. 



\begin{ex}[Clasifying space]\label{exnerve}
To each small category $\CC$ there is associated a simplicial topological space $B\CC$ called the \emph{classifying
space} of $\CC$, which is the geometric realization of its nerve. In
case $\CC$ has a single object and the morphisms of $\CC$ form a group $G$, then $B\CC$ is a $K(G, 1)$ space \cite[Section \textsection 2.3]{Hatcher}.

If we start with a
simplicial complex $X$ and regard its set of simplices as a partially ordered set $C(X)$ under
the relation of inclusion of faces (which, as every poset, becomes a category whose objects are the elements of the set and whose arrows are ``less than or equal'' relations), then $BC(X)$ is the barycentric subdivision of $X$.

A functor $F :\CC→\DD$ induces a map $B\CC→B\DD$. This is the simplicial map that sends an
$n$-simplex $X_0→X_1→\cdots →X_n$ to the $n$-simplex $F(X_0)→F(X_1)→\cdots →F(X_n)$.
A natural transformation from a functor $F$ to a functor $G$ induces a homotopy
between the induced maps of classifying spaces in the following way. Let $H:F\to G$ a natural transformation. $H$ can be regarded as a functor $\CC\times \{0<1\}\to \DD$ such that the restrictions to $\CC \times \{0\}$ and $\mathcal{C} \times \{1\}$ are $F$ and $G$, respectively. Here $\{0<1\}$ is just $\{0,1\}$ regarded as an ordered set and therefore a category. So $H$ induces a map $B(\CC\times\{0,1\})\to B\DD$. But $B(\CC\times\{0,1\})=B\CC\times B\{0,1\}$ by the theorem below. Since $B\{0,1\}\cong [0,1]$, the induced map is a homotopy. %aparece en Hatcher propuesto como ejercicio pero no de los listados como ejercicio
\end{ex}

\begin{thm}\cite[Theorem 14.3]{simplicial}
If $K$ and $L$ are simplicial sets, then there exists a bijection $\eta:|K\times L|\to|K|\times|L|$. Furthermore, if $|K|\times|L|$ is a CW-complex, $\eta$ is a homeomorphism.
\end{thm}

\begin{remark}\cite[Theorem A.6]{Hatcher}\label{countable}
The hypothesis that $|K|\times|L|$ is a CW-complex holds if $K$ and $L$ are both countable. 
\end{remark}

\end{document}

%isotopía \url{https://link.springer.com/chapter/10.1007%2F978-94-015-9319-9_6}
%Orientation preserving (que las cartas conservan la orientación) https://math.stackexchange.com/questions/1319234/meaning-of-the-expression-orientation-preserving-homeomorphism 
%O sea, conserva la orientación como superficie