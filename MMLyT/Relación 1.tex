\documentclass[twoside]{article}

\usepackage{makeidx}
\usepackage{capt-of}
\usepackage[T1]{fontenc}
\usepackage{amsfonts}
\usepackage{mathtools,amscd,amsthm}
\usepackage{tabularx}
\usepackage{amssymb,eucal,bezier,graphicx}
\usepackage{times}
\usepackage{subfig}
\usepackage[svgnames]{xcolor}
\usepackage{fancybox}
\usepackage{fancyhdr}
\usepackage{hyperref}
\usepackage{enumerate}
\usepackage{array}
\usepackage{comment}
\usepackage[spanish]{babel}
\usepackage[utf8]{inputenc}
\usepackage[colorinlistoftodos]{todonotes}
\usepackage{anysize}
\usepackage{booktabs}
\usepackage{listings}
\usepackage{listingsutf8}
\usepackage{etoolbox}


\newtheorem{ejercicioaux}{Ejercicio}
\newenvironment{ejercicio}[1]
  {\renewcommand\theejercicioaux{#1}\ejercicioaux\label{ejer:#1}}
  {\endejercicioaux}
\newenvironment{solucion}{\begin{trivlist}
 \item[\hskip \labelsep {\textit{Solución}.}\hskip \labelsep]}{\end{trivlist}}
\setlength{\parindent}{0pt}
\definecolor{deepblue}{rgb}{0,0,0.5}
\definecolor{deepred}{rgb}{0.6,0,0}
\definecolor{deepgreen}{rgb}{0,0.5,0}

\newtoggle{InString}{}% Keep track of if we are within a string
\togglefalse{InString}% Assume not initally in string
\definecolor{majo}{HTML}{CD2626}
\newcommand*{\ColorIfNotInString}[1]{\iftoggle{InString}{#1}{\color{majo}#1}}%
\newcommand*{\ProcessQuote}[1]{#1\iftoggle{InString}{\global\togglefalse{InString}}{\global\toggletrue{InString}}}%


% Default fixed font does not support bold face
\DeclareFixedFont{\ttb}{T1}{txtt}{bx}{n}{12} % for bold
\DeclareFixedFont{\ttm}{T1}{txtt}{m}{n}{11}  % for normal

\newcommand{\pythonstyle}{\lstset{
  language=Python,                     % the language of the code
  basicstyle=\small, % the size of the fonts that are used for the code
  numbers=left,                   % where to put the line-numbers
  numberstyle=\tiny\color{blue},  % the style that is used for the line-numbers
  stepnumber=1,                   % the step between two line-numbers. If it is 1, each line
                                  % will be numbered
  numbersep=5pt,                  % how far the line-numbers are from the code
  backgroundcolor=\color{white},  % choose the background color. You must add \usepackage{color}
  showspaces=false,               % show spaces adding particular underscores
  showstringspaces=false,         % underline spaces within strings
  showtabs=false,                 % show tabs within strings adding particular underscores
  frame=single,                   % adds a frame around the code
  rulecolor=\color{black},        % if not set, the frame-color may be changed on line-breaks within not-black text (e.g. commens (green here))
  tabsize=2,                      % sets default tabsize to 2 spaces
  captionpos=b,                   % sets the caption-position to bottom
  breaklines=true,                % sets automatic line breaking
  breakatwhitespace=false,        % sets if automatic breaks should only happen at whitespace
  emph={range,len,print},          
  emphstyle=\small\color{deepred},
  keywordstyle=\color{RoyalBlue},      % keyword style
  commentstyle=\color{Grey},   % comment style
  stringstyle=\color{LimeGreen},     % string literal style
     literate=%
         {á}{{\'a}}1
         {í}{{\'i}}1
         {é}{{\'e}}1
         {ý}{{\'y}}1
         {ú}{{\'u}}1
         {ó}{{\'o}}1
         {ñ}{{\~n}}1
         {"}{{{\ProcessQuote{"}}}}1% Disable coloring within double q
         {'}{{{\ProcessQuote{'}}}}1% Disable coloring within single 
         {0}{{{\ColorIfNotInString{0}}}}1
    	 {1}{{{\ColorIfNotInString{1}}}}1
   	 	 {2}{{{\ColorIfNotInString{2}}}}1
   		 {3}{{{\ColorIfNotInString{3}}}}1
   		 {4}{{{\ColorIfNotInString{4}}}}1
   		 {5}{{{\ColorIfNotInString{5}}}}1
   		 {6}{{{\ColorIfNotInString{6}}}}1
   		 {7}{{{\ColorIfNotInString{7}}}}1
   		 {8}{{{\ColorIfNotInString{8}}}}1
   		 {9}{{{\ColorIfNotInString{9}}}}1
  }}
         
\lstnewenvironment{pythone}[1][]
{
\pythonstyle
\lstset{#1}
}
{}  

%--------------------------------------------------------
\begin{document}

\title{Modelos Matemáticos en Logística y Transporte\\
Práctica 1}
\author{Rafael González López}
\maketitle

\begin{ejercicio}{1}
Considere la siguiente desigualdad de mochila: 
$$
45x_1 + 46x_2 + 79x_3 +54x_4 +53x_5+125x_6 \leq 178 \quad x_i \in \{0,1\}, \; \forall i = 1,\dotsc,6
$$
\begin{itemize}
\item Escriba el problema de separación asociado a la solución fraccional $x^* = (0, 0, 3/4, 1/2, 1, 0)$.
\item Resuelva el problema de separación, con el solver a su elección y encuentra la desigualdad de cubrimiento para dicho punto fraccional. 
\item Resuelva como problema de programación lineal continua el siguiente problema:
\begin{align*}
\max \;& x_1 +2x_2 +4x_3+2x_4+2x_5+x_6\\
s.a.:\;& 45x_1 + 46x_2 + 79x_3 +54x_4 +53x_5+125x_6 \leq 178 \\
&x_3+x_4+x_5\leq 2\\
&0\leq x_i \leq 1,\quad i=1,\dotsc,6
\end{align*}
Compruebe que el valor óptimo es $8$. 
Si su solución fuese fraccional encuentre una nueva desigualda d de cubrimiento que la separe.
\end{itemize}
\end{ejercicio}
\begin{solucion}
\begin{itemize}
\item[]
\item Escribimos el problema de separación
\begin{align*}
\min \;& s_1+s_2 + \frac{1}{4}s_3+\frac{1}{2}s_4+s_6\\
s.a.:\;& 45x_1 + 46x_2 + 79x_3 +54x_4 +53x_5+125x_6 > 178 \\
&x_i\in\{0,1\},\quad i=1,\dotsc,6
\end{align*}
\item Usando Gurobi, con interfaz en Python 3.6, obtenemos que el valor objetivo es $0.75$ con solución óptima $(0,0,1,1,1,0)$. Por tanto, dicho punto es separable y una desigualdad de cubrimiento para dicho punto es $x_3+x_4+x_5 \leq 2$. 
\item Usando nuevamente Gurobi, obtenemos la solución $(0,1,1,0,1,0)$, con valor objetivo precisamente $8$. Como la solución es entera, no necesitamos encontrar dicha desigualdad.
\end{itemize}
\newpage
\begin{pythone}
from gurobipy import *
import numpy as np
import itertools
import operator

try:

    # Dimensión
    N = 6
    # Creamos un modelo
    m = Model("escenarios");
    
    # Vector de costes
    c = [1,1,1/4,1/4,0,1]
    
    # 'S' para semicontinuas, or 'N' for semienteras.
    z = m.addVars(range(N), vtype='B', name="z");
    
    # Definimos un producto escalar 
    def productx(v):
        k = 0
        for i in range(len(v)):
            k = k + v[i]*z[i]
        return(k)    
                
    # Definimos la función objetivo y queremos maximizar/minimizar.
    m.setObjective(productx(c), GRB.MINIMIZE);
          
    d = [45,46,79,54,53,125]
    m.addConstr(productx(d) >= 178, "c0");
    #m.addConstr(z[2]+z[3]+z[4]<=2, "c1")
    
    for i in range(6):
        m.addConstr(z[i] <= 1)
    m.optimize();
        
    for v in m.getVars():
        print('%s %g' % (v.varName, v.x))

    print('Obj: %g' % m.objVal)

except GurobiError as e:
    print('Error code ' + str(e.errno) + ": " + str(e))

except AttributeError:
    print('Encountered an attribute error')
\end{pythone}

\end{solucion}
\end{document}