\documentclass[twoside]{article}

\usepackage{makeidx}
\usepackage{capt-of}
\usepackage[T1]{fontenc}
\usepackage{amsfonts}
\usepackage{mathtools,amscd,amsthm}
\usepackage{tabularx}
\usepackage{amssymb,eucal,bezier,graphicx}
\usepackage{times}
\usepackage{subfig}
\usepackage[svgnames]{xcolor}
\usepackage{fancybox}
\usepackage{fancyhdr}
\usepackage{hyperref}
\usepackage{enumerate}
\usepackage{array}
\usepackage{comment}
\usepackage[spanish]{babel}
\usepackage[utf8]{inputenc}
\usepackage[colorinlistoftodos]{todonotes}
\usepackage{anysize}
\usepackage{booktabs}
\usepackage{listings}
\usepackage{listingsutf8}
\usepackage{etoolbox}


\newtheorem{ejercicioaux}{Ejercicio}
\newenvironment{ejercicio}[1]
  {\renewcommand\theejercicioaux{#1}\ejercicioaux\label{ejer:#1}}
  {\endejercicioaux}
\newenvironment{solucion}{\begin{trivlist}
 \item[\hskip \labelsep {\textit{Solución}.}\hskip \labelsep]}{\end{trivlist}}
\setlength{\parindent}{0pt}
\definecolor{deepblue}{rgb}{0,0,0.5}
\definecolor{deepred}{rgb}{0.6,0,0}
\definecolor{deepgreen}{rgb}{0,0.5,0}

\newtoggle{InString}{}% Keep track of if we are within a string
\togglefalse{InString}% Assume not initally in string
\definecolor{majo}{HTML}{CD2626}
\newcommand*{\ColorIfNotInString}[1]{\iftoggle{InString}{#1}{\color{majo}#1}}%
\newcommand*{\ProcessQuote}[1]{#1\iftoggle{InString}{\global\togglefalse{InString}}{\global\toggletrue{InString}}}%


% Default fixed font does not support bold face
\DeclareFixedFont{\ttb}{T1}{txtt}{bx}{n}{12} % for bold
\DeclareFixedFont{\ttm}{T1}{txtt}{m}{n}{11}  % for normal

\newcommand{\pythonstyle}{\lstset{
  language=Python,                     % the language of the code
  basicstyle=\small, % the size of the fonts that are used for the code
  numbers=left,                   % where to put the line-numbers
  numberstyle=\tiny\color{blue},  % the style that is used for the line-numbers
  stepnumber=1,                   % the step between two line-numbers. If it is 1, each line
                                  % will be numbered
  numbersep=5pt,                  % how far the line-numbers are from the code
  backgroundcolor=\color{white},  % choose the background color. You must add \usepackage{color}
  showspaces=false,               % show spaces adding particular underscores
  showstringspaces=false,         % underline spaces within strings
  showtabs=false,                 % show tabs within strings adding particular underscores
  frame=single,                   % adds a frame around the code
  rulecolor=\color{black},        % if not set, the frame-color may be changed on line-breaks within not-black text (e.g. commens (green here))
  tabsize=2,                      % sets default tabsize to 2 spaces
  captionpos=b,                   % sets the caption-position to bottom
  breaklines=true,                % sets automatic line breaking
  breakatwhitespace=false,        % sets if automatic breaks should only happen at whitespace
  emph={range,len,print},          
  emphstyle=\small\color{deepred},
  keywordstyle=\color{RoyalBlue},      % keyword style
  commentstyle=\color{Grey},   % comment style
  stringstyle=\color{LimeGreen},     % string literal style
     literate=%
         {á}{{\'a}}1
         {í}{{\'i}}1
         {é}{{\'e}}1
         {ý}{{\'y}}1
         {ú}{{\'u}}1
         {ó}{{\'o}}1
         {ñ}{{\~n}}1
         {"}{{{\ProcessQuote{"}}}}1% Disable coloring within double q
         {'}{{{\ProcessQuote{'}}}}1% Disable coloring within single 
         {0}{{{\ColorIfNotInString{0}}}}1
    	 {1}{{{\ColorIfNotInString{1}}}}1
   	 	 {2}{{{\ColorIfNotInString{2}}}}1
   		 {3}{{{\ColorIfNotInString{3}}}}1
   		 {4}{{{\ColorIfNotInString{4}}}}1
   		 {5}{{{\ColorIfNotInString{5}}}}1
   		 {6}{{{\ColorIfNotInString{6}}}}1
   		 {7}{{{\ColorIfNotInString{7}}}}1
   		 {8}{{{\ColorIfNotInString{8}}}}1
   		 {9}{{{\ColorIfNotInString{9}}}}1
  }}
         
\lstnewenvironment{pythone}[1][]
{
\pythonstyle
\lstset{#1}
}
{}  

%--------------------------------------------------------
\begin{document}

\title{Modelos Matemáticos en Logística y Transporte\\
Práctica 1}
\author{Rafael González López}
\maketitle

\begin{ejercicio}{2}
Considere una grafo completo no dirigido con $n$ vértices. Denotemos por $c(i,j)$ el coste de atravesar la arista $(i,j)$ del grafo. Formule como problema de optimización combinatoria la determinación del recorrido que parte del vértice 1, pasa por todos los restantes de grafo y vuelve de nuevo a 1 y tiene coste total mínimo.

Genere un grafo con costes en las aristas de 10 vértices y resuelva con el solver de su elección el problema anterior sobre este grafo.
\end{ejercicio}

\begin{solucion}
Notemos que, en particular, este problema es precisamente la formulación en teoría de grafos del \textit{travelling salesman problem}. El problema se puede formular de la siguiente manera. Consideremos $x_{ij}$ como la variable binaria tal que si $x_{ij}=1$ entonces $(i,j)$ está en nuestro ciclo y $x_{ij}=0$ en caso contrario. Como el grafo es no dirigido, basta considerar las variables con $i<j$. Inicialmente podríamos formular
\begin{align*}
\min &\sum_{(i,j)\in E} c_{ij}x_{ij}\\
s.a.&\, \sum_{j\in V} x_{ij} = 2 \quad \forall i \in V\\
& x_{ij}\in \{0,1\} \quad \forall i,j\\
\end{align*}
El problema es que la restricción que imponemos obliga que cada vértice tenga dos aristas en la solución, pero podríamos tener varios ciclos disjuntos y no un único ciclo. Para que esto no ocurra, consideramos $S\subset V$ y 
$$
\gamma(S) = \{(i,j) \in E \mid i,j \in V\}
$$ 
Es decir, dado un conjunto $S$ de vértices, $\gamma(S)$ es el conjunto de aristas del grafo que tiene ambos extremos en $S$. En nuestro caso particular, al tener un grafo completo, $(S,\gamma(S))$ será necesariamente un grafo completo. La clave para completar la formulación es la siguiente. Si consideramos para todos los $S\subset V$ propios, es decir, ni el total ni el vacío, entonces que no haya ciclos disjuntos es equivalente a decir que en $\gamma(S)$ haya a lo sumo $|S|-1$ aristas en el ciclo. Luego podemos formular
\begin{align*}
\min &\sum_{(i,j)\in E} c_{ij}x_{ij}\\
s.a.&\, \sum_{j\in V} x_{ij} = 2 \quad \forall i \in V\\
&\sum_{(i,j)\in \gamma(S)} x_{ij} \leq |S|-1 \quad \text{$S\subset V$ propio}\\
& x_{ij}\in \{0,1\} \quad \forall i,j\\
\end{align*}
\end{solucion}
\newpage
A continuación generamos el grafo y resolvemos el problema. Para el grafo 

\begin{figure}[h!]
\centering
\includegraphics[scale=0.8]{extra}
\end{figure}
Con función de coste 
\begin{pythone}
coste = [6, 2, 5, 6, 8, 1, 8, 3, 4, 3, 9, 7, 5, 9, 3, 5, 4, 7, 8, 1, 6, 8, 8, 4, 3, 4, 5, 6, 3, 5, 3, 8, 5, 3, 8, 6, 9, 8, 4, 4, 9, 1, 4, 3, 1]
\end{pythone}
Obtenemos
\begin{figure}[h!]
\centering
\includegraphics[scale=0.8]{extra2}
\end{figure}
\newpage
Utilizamos el siguiente código
\begin{pythone}

from gurobipy import *
import numpy as np
import networkx as nx

# Generamos el grafo completo de orden 10
G = nx.complete_graph(10)

# Generamos una funcion de coste dada por enteros aleatorios entre 1 y 10
coste = np.random.randint(1,10,len(G.edges()))


# Definimos la funcion delta, que dado un conjunto U de vertices, nos devuelve
# las aristas que tienen un unico extremo en U.
def delta(G,U):
    ejes = set()
    for l, nbrs in ((n, G[n]) for n in U):
        ejes.update((l, y) if l<y else (y,l) for y in nbrs if y not in U)
    return(ejes)
    

def restsp(G, coste):
    m = Model("TSP");
    
    # Creamos las variables.
    x = m.addVars(list(G.edges()), vtype='B', name = "x");
    
    # Definimos algunas funciones para ayudarnos a escribir las restricciones
    # asi como la funcion objetivo de manera comoda y general.
    def suma(S):
        s = 0
        for (a,b) in S:
            if a<b:
                s = s + x[a,b]
            else:
                s = s + x[b,a]
        return(s)
        
    def objind(c):
        s = 0
        for i,e in enumerate(G.edges()):
            s = s + c[i]*x[e[0],e[1]] 
        return(s)
        
    cost = objind(coste)
    
    # Definimos la funcion objetivo y queremos maximizar/minimizar.
    m.ModelSense = GRB.MINIMIZE
    
    # Añadimos la funcion objetivo
    m.setObjective(cost); 
    
    # Generamos las restricciones del problema entero
    for v in G:
            m.addConstr(suma(delta(G,[v])) == 2, name = "delta" + str(v));

    # Y añadimos las restricciones de conjuntos de vertices
    for i in range(2,len(G)):
        conj1 = set(itertools.combinations(set(G.nodes()), i))
        for a in conj1:
            S = G.subgraph(a)
            m.addConstr(suma(set(S.edges)) <= i-1)
    
    m.optimize()
    
    return(m)
\end{pythone}
\end{document}