	\documentclass[twoside]{article}
\usepackage{../estilo-ejercicios}
%\renewcommand{\baselinestretch}{1,3}
%--------------------------------------------------------
\begin{document}

\title{Ejercicios de Modelado y Simulación Topológica\\ Teoría de Morse Discreta}
\author{Javier Aguilar Martín}
\maketitle

\begin{ejercicio}{1}
Considerar $K$ una triangulación de la banda de Möbius en la que dos puntos distintos del borde se han conectado mediante una arista. Se pide:
\begin{enumerate}[(a)]
\item Definir sobre $K$ una función de Morse discreta óptima $f$.
\item Construir el campo gradiente inducido por $f$. 
\item Determinar en ese caso el CW-complejo con el mismo tipo de homotopía que $K$ dado por el teorema fundamental de la TMD. 
\item Repetir todo el ejercicio, tomando como $K$ una triangulación de la 2-esfera con una membrana ecuatorial. 
\end{enumerate}
\end{ejercicio}
\begin{solucion}

\end{solucion}
\newpage

\begin{ejercicio}{2}
Estudiar si es posible (y si lo es, hazlo) definir en una triangulación de la 2-esfera una función de Morse discreta $f$ con $m_0(f)=2$, $m_1(f)=0$ y $m_2(f)=1$. Repetir el mismo estudio con $m_0(f)=2$, $m_1(f)=1$ y $m_2(f)=1$. Determinar, en caso de existir, la optimalidad de tales funciones. Si alguna no lo es, transformarla en una óptima mediante cancelación. 
\end{ejercicio}
\begin{solucion}
La primera no es posible porque $\chi(S^2)=2$, mientras que $m_0(f)-m_1(f)+m_2(f)=3$. SEGUIR
\end{solucion}

\newpage

\begin{ejercicio}{3}
Probar que el mínimo global de una función de Morse discreta corresponde siempre a un vértice crítico. Estudiar qué ocurre en el caso del máximo global.
\end{ejercicio}
\begin{solucion}
Si $\sigma$ es un mínimo global de $f$, entonces todas sus caras tienen al menos el mismo valor de $f$ que $\sigma$. Cualquier $n$-símplice con $n>0$ tiene al menos 2 caras, si $\sigma$ no es un vértice, entonces $f$ al ser de Morse (como mucho puede haber una cara con valor mayor) debería tomar un valor estrictamente menor en alguna de ellas, lo cual es una contradicción. Por tanto, necesariamente $\sigma$ es un vértice. 
\end{solucion}

\newpage

\begin{ejercicio}{4}
Qué propiedad topológicamente relevante tiene un complejo simplicial sobre el que puede definirse una función de Morse discreta con un único símplice crítico.
\end{ejercicio}
\begin{solucion}
Se puede tomar el campo gradiente asociado a $f$, que converge al símplice crítico. Este campo nos da un colapso hacia ese símplice, luego el complejo es del mismo tipo de homotopía simple que el símplice crítico ESTO HAY QUE PROBARLO MÁS RIGUROSAMENTE.
\end{solucion}

\newpage

\begin{ejercicio}{5}
Considerar el siguiente complejo simplicial $K$ con la función de Morse discreta $f$ indicada:

\begin{tikzpicture}[line cap=round,line join=round,>=triangle 45,x=1.0cm,y=1.0cm]
\clip(-3.04,0.53333333333333) rectangle (12.293333333333337,5.973333333333328);
\fill[line width=2.pt,fill=black,fill opacity=0.10000000149011612] (4.026666666666667,1.773333333333333) -- (2.,2.7466666666666653) -- (3.,5.) -- cycle;
\fill[line width=2.pt,fill=black,fill opacity=0.10000000149011612] (4.026666666666667,1.773333333333333) -- (5.653333333333334,4.56) -- (3.,5.) -- cycle;
\draw [line width=2.pt] (1.,1.)-- (2.,2.7466666666666653);
\draw [line width=2.pt] (1.,1.)-- (4.026666666666667,1.773333333333333);
\draw [line width=2.pt] (4.026666666666667,1.773333333333333)-- (2.,2.7466666666666653);
\draw [line width=2.pt] (2.,2.7466666666666653)-- (3.,5.);
\draw [line width=2.pt] (3.,5.)-- (4.026666666666667,1.773333333333333);
\draw [line width=2.pt] (4.026666666666667,1.773333333333333)-- (5.653333333333334,4.56);
\draw [line width=2.pt] (5.653333333333334,4.56)-- (3.,5.);
\draw [line width=2.pt] (3.,5.)-- (4.026666666666667,1.773333333333333);
\draw [line width=2.pt] (5.653333333333334,4.56)-- (8.,4.);
\draw (0.7333333333333338,1.12) node[anchor=north west] {$7$};
\draw (2.4266666666666676,1.4266666666666665) node[anchor=north west] {$8$};
\draw (8.04,4.506666666666663) node[anchor=north west] {$0$};
\draw (1.12,2.4) node[anchor=north west] {$6$};
\draw (1.653333333333334,3.2666666666666644) node[anchor=north west] {$5$};
\draw (2.1066666666666674,4.4) node[anchor=north west] {$4$};
\draw (2.3466666666666676,2.52) node[anchor=north west] {$11$};
\draw (2.653333333333334,3.546666666666664) node[anchor=north west] {$12$};
\draw (3.5066666666666673,3.76) node[anchor=north west] {$13$};
\draw (4.28,4.24) node[anchor=north west] {$14$};
\draw (2.7733333333333343,5.653333333333329) node[anchor=north west] {$3$};
\draw (4.08,5.44) node[anchor=north west] {$2$};
\draw (5.586666666666668,5.266666666666662) node[anchor=north west] {$1$};
\draw (6.773333333333334,4.933333333333329) node[anchor=north west] {$1$};
\draw (4.84,3.1866666666666648) node[anchor=north west] {$10$};
\draw (4,1.866666666666648) node[anchor=north west] {$9$};
\end{tikzpicture}
Se pide:
\begin{enumerate}[(a)]
\item Determinar los símplices críticos de $f$, así como su campo gradiente inducido.
\item Estudiar si $f$ es óptima. En caso negativo, transformarla en una función de Morse discreta óptima $g$ definida sobre $K$.
\item Aplicando el teorema fundamental de la teoría de Morse discreta a $g$, construir un CW-complejo con el mismo tipo de homotopía que $K$.
\item Construir el diagrama de Hasse de $K$.
\item Construir el emparejamiento de Morse inducido por $f$, comprobando que no es máximo.
\item Determinar mediante transferencia un emparejamiento de Morse máximo. 
\end{enumerate}
\end{ejercicio}
\begin{solucion}\
\begin{enumerate}[(a)]
\item Los símplices críticos son: el vértice con 0, la arista con 10 y la cara con 14. El campo gradiente lo dibujo aquí debajo

\begin{tikzpicture}[line cap=round,line join=round,>=triangle 45,x=1.0cm,y=1.0cm]
\clip(-3.04,0.53333333333333) rectangle (12.293333333333337,5.973333333333328);

\fill[line width=2.pt,fill=black,fill opacity=0.10000000149011612] (4.026666666666667,1.773333333333333) -- (2.,2.7466666666666653) -- (3.,5.) -- cycle;
\fill[line width=2.pt,fill=black,fill opacity=0.10000000149011612] (4.026666666666667,1.773333333333333) -- (5.653333333333334,4.56) -- (3.,5.) -- cycle;
\draw [line width=2.pt] (1.,1.)-- (2.,2.7466666666666653);
\draw [line width=2.pt] (1.,1.)-- (4.026666666666667,1.773333333333333);
\draw [line width=2.pt] (4.026666666666667,1.773333333333333)-- (2.,2.7466666666666653);
\draw [line width=2.pt] (2.,2.7466666666666653)-- (3.,5.);
\draw [line width=2.pt] (3.,5.)-- (4.026666666666667,1.773333333333333);

\draw [line width=2.pt] (4.026666666666667,1.773333333333333)-- (5.653333333333334,4.56);
\draw [line width=2.pt] (5.653333333333334,4.56)-- (3.,5.);
\draw [line width=2.pt] (3.,5.)-- (4.026666666666667,1.773333333333333);
\draw [line width=2.pt] (5.653333333333334,4.56)-- (8.,4.);
\draw (0.7333333333333338,1.12) node[anchor=north west] {$7$};
\draw (2.4266666666666676,1.4266666666666665) node[anchor=north west] {$8$};
\draw (8.04,4.506666666666663) node[anchor=north west] {$0$};
\draw (1.12,2.4) node[anchor=north west] {$6$};
\draw (1.653333333333334,3.2666666666666644) node[anchor=north west] {$5$};
\draw (2.1066666666666674,4.4) node[anchor=north west] {$4$};
\draw (2.3466666666666676,2.52) node[anchor=north west] {$11$};
\draw (2.653333333333334,3.546666666666664) node[anchor=north west] {$12$};
\draw (3.5066666666666673,3.76) node[anchor=north west] {$13$};
\draw (4.28,4.24) node[anchor=north west] {$14$};
\draw (2.7733333333333343,5.653333333333329) node[anchor=north west] {$3$};
\draw (4.08,5.44) node[anchor=north west] {$2$};
\draw (5.586666666666668,5.266666666666662) node[anchor=north west] {$1$};
\draw (6.773333333333334,4.933333333333329) node[anchor=north west] {$1$};
\draw (4.84,3.1866666666666648) node[anchor=north west] {$10$};
\draw (4,1.866666666666648) node[anchor=north west] {$9$};


\draw [line width=2.pt,->, color=blue] (1.,1.)-- (2.,2.7466666666666653);
\draw [line width=2.pt,->,color=blue] (4.026666666666667,1.773333333333333)--(1.,1.);
\draw [line width=2.pt,->,color=blue] (2.,2.7466666666666653)-- (3.,5.);
\draw [line width=2.pt,->,color=blue] (3.,5.)-- (5.6666666666667,4.55);
\draw [line width=2.pt,->,color=blue] (5.6666666666667,4.55)--(8.1,3.95);
\draw [line width=2.pt,->,color=blue] (3.5,3.5)--(2.5,3.);
\end{tikzpicture}
\item No es óptima porque no es $\R$-perfecta, ya que tiene un 2-símplice crítico y $b_2=0$. TRANSFORMARLA

\item 

\item ponte debajo dibujo cabrón

\begin{tikzpicture}[line cap=round,line join=round,>=triangle 45,x=1.0cm,y=1.0cm]
\clip(-0.5,-1.38) rectangle (11.,4.5);
\draw (1.,0.12) node[anchor=north west] {$0$};
\draw (2.,0.12) node[anchor=north west] {$1$};
\draw (3.,0.12) node[anchor=north west] {$3$};
\draw (4.,0.12) node[anchor=north west] {$5$};
\draw (5.,0.12) node[anchor=north west] {$7$};
\draw (6.,0.12) node[anchor=north west] {$9$};
\draw (-0.05,2.38) node[anchor=north west] {$1$};
\draw (0.94,2.39) node[anchor=north west] {$2$};
\draw (1.83,2.33) node[anchor=north west] {$4$};
\draw (2.93,2.38) node[anchor=north west] {$6$};
\draw (3.93,2.37) node[anchor=north west] {$8$};
\draw (5.01,2.37) node[anchor=north west] {$10$};
\draw (5.92,2.38) node[anchor=north west] {$11$};
\draw (6.94,2.35) node[anchor=north west] {$13$};
\draw (2.89,4.4) node[anchor=north west] {$12$};
\draw (3.88,4.38) node[anchor=north west] {$14$};
\draw [line width=2.pt] (1.,0.)-- (0.,2.);
\draw [line width=2.pt] (2.,0.)-- (0.,2.);
\draw [line width=2.pt] (2.,0.)-- (1.,2.);
\draw [line width=2.pt] (2.,0.)-- (5.,2.);
\draw [line width=2.pt] (3.,0.)-- (1.,2.);
\draw [line width=2.pt] (3.,0.)-- (2.,2.);
\draw [line width=2.pt] (3.,0.)-- (7.,2.);
\draw [line width=2.pt] (4.,0.)-- (2.,2.);
\draw [line width=2.pt] (4.,0.)-- (3.,2.);
\draw [line width=2.pt] (4.,0.)-- (6.,2.);
\draw [line width=2.pt] (5.,0.)-- (3.,2.);
\draw [line width=2.pt] (5.,0.)-- (4.,2.);
\draw [line width=2.pt] (6.,0.)-- (6.,2.);
\draw [line width=2.pt] (6.,0.)-- (5.,2.);
\draw [line width=2.pt] (6.,0.)-- (7.,2.);
\draw [line width=2.pt] (1.,2.)-- (4.,4.);
\draw [line width=2.pt] (2.,2.)-- (3.,4.);
\draw [line width=2.pt] (6.,2.)-- (3.,4.);
\draw [line width=2.pt] (5.,2.)-- (4.,4.);
\draw [line width=2.pt] (7.,2.)-- (3.,4.);
\draw [line width=2.pt] (7.,2.)-- (4.,4.);
\draw [line width=2.pt] (6,0.)-- (4.,2.);
\begin{scriptsize}
\draw [fill=black] (1.,0.) circle (2.5pt);
\draw [fill=black] (2.,0.) circle (2.5pt);
\draw [fill=black] (3.,0.) circle (2.5pt);
\draw [fill=black] (4.,0.) circle (2.5pt);
\draw [fill=black] (5.,0.) circle (2.5pt);
\draw [fill=black] (6.,0.) circle (2.5pt);
\draw [fill=black] (0.,2.) circle (2.5pt);
\draw [fill=black] (7.,2.) circle (2.5pt);
\draw [fill=black] (1.,2.) circle (2.5pt);
\draw [fill=black] (2.,2.) circle (2.5pt);
\draw [fill=black] (3.,2.) circle (2.5pt);
\draw [fill=black] (4.,2.) circle (2.5pt);
\draw [fill=black] (5.,2.) circle (2.5pt);
\draw [fill=black] (6.,2.) circle (2.5pt);
\draw [fill=black] (3.,4.) circle (2.5pt);
\draw [fill=black] (4.,4.) circle (2.5pt);
\end{scriptsize}
\end{tikzpicture}

\item TENGO EL DIBUJO, PERO NO SE PUEDE HACER UN EMPAREJAMIENTO QUE DEJE FUERA SOLO LOS CRÍTICOS (CREO QUE ESO ES LO QUE HACE QUE NO SEA MÁXIMO, PERO NO SÉ CUÁL DE LAS POSIBILIDADES ES EL INDUCIDO POR F)

\item
\end{enumerate}
\end{solucion}

\end{document}
