	\documentclass[twoside]{article}
\usepackage{../estilo-ejercicios}
%\renewcommand{\baselinestretch}{1,3}
%--------------------------------------------------------
\begin{document}

\title{Examen 1 de Modelado y Simulación Topológica}
\author{Javier Aguilar Martín}
\maketitle

\begin{ejercicio}{1}
Se considera el siguiente complejo simplicial $K$, sobre el que está definida la función de Morse discreta $f$ indicada:
\begin{tikzpicture}[line cap=round,line join=round,>=triangle 45,x=1.0cm,y=1.0cm]
\clip(-2.3933333333333335,-2.72) rectangle (12.94,4.786666666666666);
\fill[line width=2.pt,fill=black,fill opacity=0.10000000149011612] (1.513333333333334,2.013333333333333) -- (2.473333333333334,0.) -- (3.46,2.) -- cycle;
\fill[line width=2.pt,fill=black,fill opacity=0.10000000149011612] (2.473333333333334,0.) -- (5.,0.) -- (3.42,-2.0133333333333305) -- cycle;
\draw [line width=2.pt] (0.,0.)-- (1.513333333333334,2.013333333333333);
\draw [line width=2.pt] (1.513333333333334,2.013333333333333)-- (3.46,2.);
\draw [line width=2.pt] (3.46,2.)-- (5.,0.);
\draw [line width=2.pt] (1.513333333333334,2.013333333333333)-- (2.473333333333334,0.);
\draw [line width=2.pt] (2.473333333333334,0.)-- (3.46,2.);
\draw [line width=2.pt] (0.,0.)-- (5.,0.);
\draw [line width=2.pt] (2.473333333333334,0.)-- (3.42,-2.0133333333333305);
\draw [line width=2.pt] (3.42,-2.0133333333333305)-- (5.,0.);
\draw [line width=2.pt] (2.473333333333334,0.)-- (1.313333333333334,-2.0266666666666637);
\draw [line width=2.pt] (1.513333333333334,2.013333333333333)-- (2.473333333333334,0.);
\draw [line width=2.pt] (2.473333333333334,0.)-- (3.46,2.);
\draw [line width=2.pt] (3.46,2.)-- (1.513333333333334,2.013333333333333);
\draw [line width=2.pt] (2.473333333333334,0.)-- (5.,0.);
\draw [line width=2.pt] (5.,0.)-- (3.42,-2.0133333333333305);
\draw [line width=2.pt] (3.42,-2.0133333333333305)-- (2.473333333333334,0.);
\draw (-0.38,0.2666666666666695) node[anchor=north west] {$1$};
\draw (1.193333333333334,-0.08) node[anchor=north west] {$1$};
\draw (0.58,1.28) node[anchor=north west] {$5$};
\draw (1.8066666666666673,0.96) node[anchor=north west] {$4$};
\draw (2.5,1.56) node[anchor=north west] {$6$};
\draw (2.38,2.426666666666668) node[anchor=north west] {$3$};
\draw (1.4866666666666672,2.373333333333335) node[anchor=north west] {$3$};
\draw (3.74,2.266666666666668) node[anchor=north west] {$2$};
\draw (3.073333333333334,1.026666666666669) node[anchor=north west] {$2$};
\draw (4.593333333333335,1.1866666666666688) node[anchor=north west] {$3$};
\draw (3.78,0.32) node[anchor=north west] {$1$};
\draw (5.366666666666668,0.2) node[anchor=north west] {$1$};
\draw (3.8466666666666676,-0.36) node[anchor=north west] {$7$};
\draw (4.5,-0.946666666666663) node[anchor=north west] {$2$};
\draw (2.8733333333333344,-1.0266666666666628) node[anchor=north west] {$7$};
\draw (3.3533333333333344,-2.133333333333329) node[anchor=north west] {$2$};
\draw (1.6066666666666674,-0.826666666666663) node[anchor=north west] {$1$};
\draw (1.313333333333334,-1.8666666666666623) node[anchor=north west] {$1$};
\begin{scriptsize}
\draw [fill=black] (0.,0.) circle (2.0pt);
\draw [fill=black] (1.513333333333334,2.013333333333333) circle (2.0pt);
\draw [fill=black] (3.46,2.) circle (2.0pt);
\draw [fill=black] (5.,0.) circle (2.0pt);
\draw [fill=black] (2.473333333333334,0.) circle (2.0pt);
\draw [fill=black] (3.42,-2.0133333333333305) circle (2.0pt);
\draw [fill=black] (1.313333333333334,-2.0266666666666637) circle (2.0pt);
\end{scriptsize}
\end{tikzpicture}
Se pide:
\begin{enumerate}[(a)]
\item Determina los puntos críticos de $f$, así como el campo gradiente inducido.
\item Estudia si $f$ es óptima. En caso negativo, transfórmala en una función de Morse discreta óptima $g$ definida sobre $K$. 
\item Aplicando el teorema fundamental de la teoría de Morse discreta a $g$, construye un $CW$-complejo con el mismo tipo de homotopía que $K$.
\item Si se considera la filtración dada por los subcomplejos de nivel asociados a los valores críticos de $f$, obtén los diagramas de persistencia y los códigos de barras correspondientes para $0\leq p\leq \dim(K)$.
\end{enumerate}
\end{ejercicio}
\begin{solucion}\
\begin{enumerate}
\item
\end{enumerate}



\end{solucion}


\end{document}
