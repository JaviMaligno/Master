	\documentclass[twoside]{article}
\usepackage{../../estilo-ejercicios}
%\renewcommand{\baselinestretch}{1,3}
%--------------------------------------------------------
\begin{document}

\title{Ejercicios sobre Bitcoin}
\author{Javier Aguilar Martín y Jesús López}
\maketitle


\begin{ejercicio}{1}
Comprueba que la verificación de la firma digital sobre curvas elípticas es correcta. 

%\begin{nota}
%Se recuerda que el algoritmo de firma digital consistía en lo siguiente:
%\begin{enumerate}
%	\item Calcular $e=h(m)$.
%	\item Sea $z$ el conjunto de $l$ dígitos más a la izquierda de $e$, donde $l$ es la longitud en bits de $N$.
%	\item Elegir un número entero aleatorio $k\in\{1,\dots,N-1\}$.
%	\item Calcular el punto $(a,b)=kP$ usando la multiplicación por escalar de la curva elíptica. 
%	\item Encontrar $r\equiv a\mod N$. Si $r=0$, vuelta al paso 3.
%	\item Encontrar $s\equiv (z+rx)k^{-1}\mod N$. Si $s=0$, vuelta al paso 1. 
%	\item La firma es el par $(r,s)$. 
%\end{enumerate}
%Y la verificación consistía en:
%\begin{enumerate}
%	\item Verificar que $1\leq r,s\leq N-1$. En caso contrario la firma no es válida.
%	\item Calcular $e=h(m)$ con la misma función hash de antes.
%	\item Tomar $z$ como los $l$ bits más a la izquierda de $e$ del mismo modo que antes.
%	\item Calcular $w\equiv s^{-1}\mod N$.
%	\item Calcular $u\equiv zw\mod N$ y $v\equiv rw\mod N$. 
%	\item Calcular el punto $(a,b)=uP+vY$.
%	\item Verificar que $r\equiv a\mod N$. La firma es inválida en caso contrario.
%\end{enumerate}
%\end{nota}
\end{ejercicio}


\begin{ejercicio}{2}
	Contesta a las siguientes preguntas.
	\begin{enumerate}
		\item ¿En qué consiste una prueba de trabajo?
	\item  ¿Cómo evita Bitcoin el doble gasto?
\end{enumerate}
\end{ejercicio}
\end{document}