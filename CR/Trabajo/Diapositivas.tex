\documentclass{beamer}
\usepackage[utf8]{inputenc}
\usetheme{Copenhagen}
\usepackage[spanish]{babel}
\usepackage{multirow}
%\usepackage{estilo-apuntes}
\usepackage{braids}
\usepackage[]{graphicx}
\usepackage{rotating}
\usepackage{pgf,tikz}
\usepackage{pgfplots}
\usepackage{tikz-cd}
\usetikzlibrary{arrows}
\usetikzlibrary{cd}
\usetikzlibrary{babel}
\pgfplotsset{compat=1.13}
\usetikzlibrary{decorations.shapes}
\pgfkeyssetvalue{/tikz/braid height}{1cm} %no parece hacer nada
\pgfkeyssetvalue{/tikz/braid width}{1cm}
\pgfkeyssetvalue{/tikz/braid start}{(0,0)}
\pgfkeyssetvalue{/tikz/braid colour}{black}

\theoremstyle{definition}

\newtheorem{teorema}{Teorema}
\newtheorem{defi}{Definición}
\newtheorem{prop}[teorema]{Proposición}

\newcommand{\Z}{\mathbb{Z}}
\newcommand{\C}{\mathbb{C}}
\newcommand{\D}{\mathbb{D}}
\providecommand{\gene}[1]{\langle{#1}\rangle}


\addtobeamertemplate{navigation symbols}{}{%
    \usebeamerfont{footline}%
    \usebeamercolor[fg]{footline}%
    \hspace{1em}%
    \insertframenumber/\inserttotalframenumber
}
\setbeamercolor{footline}{fg=black}
\setbeamerfont{footline}{series=\bfseries}

%-----------------------------------------------------------

\title{Funcionamiento de Bitcoin}
\author{Javier Aguilar Martín y Jesús López}
\institute{Universidad de Sevilla}
\date{}
 
\begin{document}
\frame{\titlepage}
%\begin{frame}
%
%
%\title[About Beamer] %optional
%{About the Beamer class in presentation making}
% 
%\subtitle{A short story}
% 
%\author[Arthur, Doe] % (optional, for multiple authors)
%{A.~B.~Arthur\inst{1} \and J.~Doe\inst{2}}
% 
%\institute[VFU] % (optional)
%{
%  \inst{1}%
%  Faculty of Physics\\
%  Very Famous University
%  \and
%  \inst{2}%
%  Faculty of Chemistry\\
%  Very Famous University
%}

% 
%\date[VLC 2013] % (optional)
%{Very Large Conference, April 2013}


%\end{frame}
\setbeamercovered{highly dynamic}

\newcounter{saveenumi}
\newcommand{\seti}{\setcounter{saveenumi}{\value{enumi}}}
\newcommand{\conti}{\setcounter{enumi}{\value{saveenumi}}}

\resetcounteronoverlays{saveenumi}
%\AtBeginSection[]{
\begin{frame}
\frametitle{Tabla de contenidos}
\tableofcontents
\end{frame}
%}


\section{Overview}
\begin{frame}
	\frametitle{Overview}
	\begin{itemize}
		\item<1-> Bitcoin es una criptomoneda sin banco central creada en 2008 por el seudónimo Satoshi Nakamoto.
		\item<2-> Cada usuario de Bitcoin tiene una dirección (clave pública) con la que recibe Bitcoins y una clave privada con la que los gasta.
		\item<3-> Cada intercambio de Bitcoins se llama transacción.
		\item<4-> Las transacciones se almacenan en una cadena de bloques (blockchain) que sirve como libro de cuentas.
		\item<5-> Bitcoin usa el algoritmo de firma digital sobre la curva elíptica ecp256k1.
	\end{itemize}
	
	
	
	
	
	
	
\end{frame}



\section{Preliminares}
\subsection{Árboles de Merkle}

\begin{frame}
	\frametitle{Árboles de Merkle}
\end{frame}

\subsection{Prueba de Trabajo}
\begin{frame}
\frametitle{Prueba de trabajo}
\begin{defi}
	Una \textbf{prueba de trabajo} es un reto criptográfico para asegurar que una parte ha realizado una determinada cantidad de trabajo computacional. La verificación de la prueba debe poder realizarse de forma eficiente.
\end{defi}\pause

Supongamos que Alice quiere pedirle a Bob que realice un trabajo computacional en cada mensaje que le envíe. Alice puede pedirle a Bob una cadena de texto cuyo hash verifique una cierta estructura. Encontrar esa cadena tiene una probabilidad de acierto que determinará la cantidad media de trabajo realizada por Bob.
\end{frame}

\begin{frame}
	\frametitle{Prueba de trabajo de Bitcoin}
	\begin{itemize}
	\item<1-> Bitcoin utiliza como algoritmo de hash el doble $SHA256$ ($SHA256^2$).
	\item<2-> La estructura predefinida es que el hash (en hexadecimal) sea menor o igual que un valor $T$.
	\item<3-> Para un mensaje $m$, la probabilidad de encontrar un \textbf{nonce} $n$ que verifique $H=SHA256^2(m||n)\leq T$ es
	$$P(H\leq T)=\frac{T}{2^{256}}$$
	\item<4-> La verificación de la prueba consiste simplemente en comprobar $SHA256^2(m||n)\leq T$.
	\end{itemize}
\end{frame}


\section{Transacciones}
\subsection{Transacciones regulares}
\begin{frame}
\frametitle{Transacciones regulares}
\begin{defi}
	Una \textbf{transacción regular} es una transferencia de Bitcoins entre usuarios distintos. Coniste en un archivo que contiene los siguientes datos: un número de versión ($nVersion$), un vector de inputs ($vin$), un vector de outputs ($vout$) y una fecha ($nLockTime$). Incluye además $\# vin$ y $\# vout$ (el número de elementos de $vin$ y $vout$, respectivamente). 
\end{defi}

\end{frame}

\begin{frame}
	TABLA (NO SÉ SI COPIARLA O HACERLA PARA TRADUCIRLA)
\end{frame}


\begin{frame}
	\begin{block}{nVersion}
		Almacena el número de versión del formato de transacción. Actualmente es 1.
	\end{block}\pause

\begin{block}{vin}
	Almacena un vector con una o más transacciones. Cada input está compuesto por una referencia a un output previo ($hash,n$), el campo de firma ($scriptSig$), la longitud del campo de firma en bytes ($scriptSigLen$) y un número de secuencia de transacción ($nSequence$).
\end{block}
\end{frame}

\begin{frame}
\begin{itemize}
	\item<1-> $(hash,n)$: un output previo se identifica con la tupla $(hash,n)$. El campo $hash$, o \textbf{ID de transacción} ($TxID$) se calcula como el doble $SHA256$ de la transacción anterior: $$TxID=SHA256^2(Transaction)$$
	El valor $n$ almacenado en el $vout$ de la transacción anterior es la cantidad de Bitcoin transferida.
\end{itemize}
\end{frame}

\begin{frame}
	FIGURA 3.2
\end{frame}

\begin{frame}
	\begin{itemize}
		\item<1-> $scriptSig$: es un espacio para introducir una firma digital que se calcula a partir de la clave pública de la anterior transacción. Contiene además la prueba de conocimiento para verificar la firma. %es el mensaje que se firma
		\item<2-> $nSquence$. DETALLES TÉCNICOS QUE NO CREO QUE IMPORTEN
		\end{itemize}
\end{frame}

\begin{frame}
	\begin{block}{vout}
		Almacena un vector con una o más transacciones output. Cada transacción output está compuesta por una cantidad de BTC para ser gastada ($nValue$), la clave pública ($scriptPubkey$) y la longitud de la clave pública ($scriptPubkeyLen$).
	\end{block}
\begin{itemize}
	\item<2-> $nValue$: almacena la cantidad de BTC transferida. Esta cantidad está representada en \textbf{Satoshis}, esto es $10^{-18}$ BTC.
	\item<3->$scriptPubkey$: es un espacio en el que se introduce la clave pública del destinatario y contiene la prueba de conocimiento para que solo el destinatario pueda demostrar que su clave privada se corresponde a esa clave pública.
\end{itemize}
	
\end{frame}

\begin{frame}

	\begin{block}{nLockTime}
		Almacena un tiempo de bloqueo, es decir, el momento (en formato UNIX) a partir del cual la transacción debería ser incluida en un bloque.
	\end{block}
\end{frame}

\subsection{Transacciones Coinbase}
\begin{frame}
	\frametitle{Transacciones Coinbase}
\end{frame}
\section{Blockchain}
\subsection{Bloques}
\begin{frame}
	\frametitle{Bloques}
\end{frame}
\subsection{Formación de la cadena}
\begin{frame}
	\frametitle{Formación de la cadena}
\end{frame}


\section{¿En qué consiste tener Bitcoin?}
\begin{frame}
	\frametitle{¿En qué consiste tener Bitcoin?}
\end{frame}



\end{document}
