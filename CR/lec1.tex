\documentclass[CR.tex]{subfiles}

\begin{document}


%\hyphenation{equi-va-len-cia}\hyphenation{pro-pie-dad}\hyphenation{res-pec-ti-va-men-te}\hyphenation{sub-es-pa-cio}

\chapter{Algoritmos para problemas criptográficos en teoría de grupos}
\section{Preliminares}
Usamos la versión del teorema chino del resto que dice que si $a,b\in\Z_{>0}$ son tales que $\gcd(a,b)=1$ entonces $\Z_{ab}\cong \Z_a\times\Z_b$ mediante el isomorfismo $x\mapsto (x\mod a,x\mod b)$. 

Recordemos que un elemento de $a\in \Z_N$ es unidad si y solo si $\gcd(a,N)=1$: si $ab=1\mod N$, entonces $N|ab-1$, y de la identidad de Bézout se deduce el resultado. Recíprocamente, si $\gcd(a,b)=1$, el inverso es el $\alpha$ de la identidad de Bézout. Por ello, $\varphi(N)$ se puede definir como los primos con $N$ o como el cardinal de las unidades.

\begin{lemma}
$\varphi$ es multiplicativo si $\gcd(a,b)=1$. 
\end{lemma}
\begin{proof}
Por el teorema chino del resto, $\Z_{ab}\cong\Z_a\times\Z_b$, y como el isomorfismo respeta las unidades, se tiene también $(\Z_{ab})^*\cong(\Z_a)^*\times(\Z_b)^*$. Tomando cardinal se obtiene $\varphi(ab)=\varphi(a)\varphi(b)$. 
\end{proof}
\begin{lemma}
$\varphi(p^e)=p^{e-1}(p-1)$ para $p$ primo y $e\in\Z_{>0}$.
\end{lemma}
\begin{proof}
$\varphi(p^e)=\sharp\{a=1,\dots, p^e-1\mid p\not| a\}=p^e-\sharp\{a=1,\dots, p^e-1\mid p|a\}=p^e-\sharp\{p,2p,\dots\}=p^e-p^{e-1}$.
\end{proof}


\begin{lemma}
Para todo $p$ primo, el grupo $(\Z_p)^*$ es cíclico 
\end{lemma}
\begin{proof}
Queremos probar que existe $a\in\Z_p$ tal que $ord(a)=p-1$. Sea $mcm\{ord(b)\mid b\in\Z_p\}=m$. Si $p^e|m$ y $p^{e+1}\not| m$; entonces existe $b$ tal que $p^e| ord(b)$, luego $ord(b)=rp^e$, así que $b^{r}$ tiene orden $p^e$. Probamos que $ord(\prod_{p|m}b_p^{r_p})=m$. Si $m=p-1$, ya está. Si no, consideramos $f(x)=x^m-1\in\Z_p(x)$. Para todo $b\in\Z_p$, $f(b)=b^m-1=0$. Esto significa que $f$ tiene $p-1$ raíces distintas, por lo que $m\geq p-1$. Por otra parte, $m\leq p-1$ porque $ord(b)|p-1$ por el teorema de Lagrange, y por ello también lo divide el mínimo común múltiplo de los órdenes, $m$. 
\end{proof}

\section{Test de primalidad}
Nos basamos en el pequeño teorema de Fermat, esto es, si $p$ es primo y $a\geq 1$, $a\not| p$, entonces $a^{p-1}\equiv 1\mod p$. Esto es consecuencia del teorema de Lagrange, pues al ser $a$ unidad por no dividir a $p$, su orden debe dividir a $p-1$. 

Sea $N\geq 2$. Sea $a\geq 1$ tal que $\gcd(a,N)=1$ (si no fuera 1, tendríamos divisores de $N$, y por tanto $N$ no es primo). Si $a^{N-1}\not\equiv 1\mod N$, entonces $N$ no es primo.

\begin{ej}
$N=91$, $a=3$. Haciendo reducciones llegamos a que $3^{90}\equiv 1\mod 91$. Esto no es concluyente. De hecho $91=7\cdot 13$. 

Para $a=2$, el test nos da $2^{90}\equiv 64\mod 91$, lo que sí concluye que no es primo, aunque no conocemos sus factores. 
\end{ej}


\begin{defi}
Diremos que $N\geq 2$ no primo es un número de Carmichael si para tod $a$ con $\gcd(a,N)=1$, $a^{N-1}\equiv 1\mod N$.
\end{defi}

Por ejemplo, 561 o 1105. 

\begin{prop}
Si $N>2$ no es de Carmichael y es compuesto, entonces 
\[
\#\{a\mid 1\leq a\leq N-1, gcd(a,N)=1, a^{N-1}\equiv 1\mod N\}\leq \varphi(N)/2 
\]
\end{prop}
\begin{dem}
Sea $H=\{a\in\Z_N\mid$ mismas condiciones que en el enunciado$\}\subseteq (\Z_N)^*$. Es fácil ver que $H$ es un subgrupo. Si $N$ no es de Carmichael y es compuesto, $H\neq (\Z_N)^*$ porque hay algún $a$ que no verifica la ecuación. Entonces $\#(H)c=\varphi(N)$ con $c\geq 2$, y ya está. 
\end{dem}

\begin{prop}
Para $N\geq 2$ son equivalentes:
\begin{enumerate}
\item $N$ es de Carmichael
\item Para todo $p|N$: $p$ es impar, $p^2\not| N$ y $(p-1)\mid (N-1)$. 
\end{enumerate}
\end{prop}
\begin{dem}
$1\Rightarrow 2$. 
\begin{enumerate}
\item Por 2, $2^2\not| N$, luego existe $p>2$ primo dividiendo a $N$.  Por 3, $p-1|N-1$, pero $p-1$ es par y $N-1$ es impar. 
\item $a^{N-1}\equiv 1\mod N\Rightarrow a^N\equiv a\mod N$. Para $a=p$, $N| p^N-p$, así que $p=p^N-cN$ para algún $c$.
\item Sea $a$ generador $(\Z_p)^*$, de modo que $ord(a)=p-1$ y $a^{N-1}\equiv 1\mod p$. 
\end{enumerate}

$2\Rightarrow 1$. Por el teorema chino del resto usando la descomposición $N=p_1\cdots p_r$ ($N$ es libre de cuadrados), tenemos que $a^{N-1}\mod N\mapsto (a^{N-1}\mod p_1,\dots, a^{N-1}\mod p_r)$, y el $1$ va al $(1,\dots, 1)$, de mod oque $a^{N-1}=1\mod p_i$ para todo $i$, de donde se deduce que $a^{N-1}=1\mod p_1\cdots p_r$. 
\end{dem}

\begin{ej}
$N=561$ es de Carmichael. $3-1,11-1$ y $17-1$ dividen a $561-1$. 
\end{ej}

\begin{prop}
$\#\{x\leq N, x$ es de Carmichael$\}\geq O(N^{2/7})$.
\end{prop}

Vamos a cambiar el test por $a^{\frac{N-1}{2}}\equiv 2\mod N$ (para $N$ impar). 

\begin{defi}
Diremos que $a\geq 0$ es un cuadrado $\mod p$ o un residuo cuadrático si $\exists b\geq 0$ tal que $b^2=a\mod p$.
\end{defi}

\begin{ej}
$p=3$, podemos hacer todas las cuentas y $2$ no es un cuadrado
\end{ej}

\begin{lemma}
Sea $p>2$ primo. Hay $\frac{p-1}{2}$ residuos no nulos  y la misma cantidad de no residuos. 
\end{lemma}
\begin{proof}
Sea $\varphi:(\Z_p)^*\to(\Z_p)^*$, $x\mapsto x^2$. $\ker\varphi\supseteq \{1,-1\}$. El polinomio $x^2-1\in\Z_p[x]$ tiene a lo más 2 raíces en $\Z_p$, por lo que la inclusión anterior es una igualdad. Primer teorema de isomorfía, $\Ima\varphi\cong (\Z_p)^*/\ker\varphi$. Como la imagen tiene $\frac{p-1}{2}$ elementos, y estos son los residuos cuadráticos, se deduce el lema. 
\end{proof}


\begin{prop}
Sea $N=p_1^{e_1}\cdots p_r^{e_r}$. Sea $a\geq 1$, entonces $a$ es residuo módulo $N$ si y solo si $a$ es residuo módulo $p_i^{e_i}$. 
\end{prop}
\begin{dem}
Hacia la derecha es trivial. Para el otro lado, teorema chino del resto.
\end{dem}

\begin{lemma}
Sea $p$ un primo impar y $a\in\Z$ que no divide a $p$. Son equivalentes:
\begin{enumerate}
\item $a$ es residuo módulo $p$.
\item $a$ es residuo módulo $p^n$ para todo $p^n$. 
\end{enumerate}
\end{lemma}
\begin{proof}

\end{proof}

Con estos resultados tenemos que $a$ es cuadrado $\mod N$ si y solo si para todo $i$ $a$ es un cuadrado $\mod p_i^{e_i}$ si y solo si $a$ es cuadrado $\mod  p_i$ ($p_i>2$). El caso $p_i=2$ lo cubre el siguiente lema.

\begin{lemma}
Sea $a\in\Z$ impar. Son equivalentes:
\begin{enumerate}
\item $a$ es residuo cuadrático $\mod 2^n$
\item Se verifica alguno de los siguientes: $n=1$, $n=2$ y $a\equiv 1\mod 4$, $n\geq 3$ y $a\equiv mod 8$.
\end{enumerate}
\end{lemma}
La prueba es por inducción, probando a mano los casos $n=1$, $n=2$. 

\begin{defi}
Se define el símbolo de Legendre como sigue. Para todo primo $p$ y para todo $a\in\Z$, 
$$\left(\frac{a}{p}\right)=\begin{cases}
0 & p|a\\
1 & a \text{residuo no nulo }\mod p\\
-1 & c.c.
\end{cases}$$
\end{defi}

\begin{prop}[Euler]
Sea $p>2$ primo y $a\in\Z$. Entonces $a^{\frac{p-1}{2}}\equiv (\frac{a}{p}) \mod p$.
\end{prop}
\begin{dem}
Si $p|a$, entonces el resultado es trivial porque $a=0\mod p$. En otro caso, definimos $\phi:(\Z_p)^*\to (\Z_p)^*$ definido como $\phi(x)=x^{\frac{p-1}{2}}$. Se tiene que $\Ima\phi\subseteq \{\pm 1\}$ porque $a^{p-1}\equiv 1\mod p$ y $b=a^{\frac{p-1}{2}}$ es tal que $b^2\equiv 1\mod p$, que como $\Z_p$ es un cuerpo no puede tener más de dos soluciones. Por otro lado, $\ker\phi=\{a, a^{(p-1)/2}=1\}$. 

Tenemos que como $(\Z_p)^*$ es cíclico existe $c\not| p$ tal que $\{1,c,\dots, c^{(p-1)/2},\dot, c^{p-2}\}$ son todos distintos, por lo que $\phi(c)\neq 1$, es decir, $\phi(c)=-1$. Por ello $\Im\phi=\{\pm 1\}$ (tiene cardinal 2). Por el primer teorema de isomorfía, $\ker\phi$ tiene cardinal $\frac{p-1}{2}$.  Queremos probar $\ker\phi=\{b^2+\gene{p}\mid b=1,\dots, p-1\}$. Como tienen el mismo cardinal basta ver una inclusión. La inclusión fácil es hacia la izquierda, si tomamos un cuadrado $b^2+\gene{p}$, entonces $\phi(b^2+\gene{p})=b^{p-1}+\gene{p}=1+\gene{p}$ por el pequeño teorema de Fermat, y por tanto el cuadrado está en el núcleo. 

Tenemos entonces que $(\frac{a}{p})$ y $a^{(p-1)/2}$ toman el valor 1 en los mismos casos, por lo que valen lo mismo $\mod p$. 
\end{dem}

Queremos ver que la congruencia anterior se da también módulo $N$ cuando $N$ es compuesto definiéndola adecuadamente. 

\begin{defi}
Sea $N=p_1^{e_1}\cdots p_r^{e_r}$ y $a\in\Z$. Se define el símbolo de Jacobi como 
\[
\left(\frac{a}{N}\right)=\prod_{i=1}^r \left(\frac{a}{p_i}\right)^{e_i}
\]
\end{defi}

\begin{lemma}\
\begin{enumerate}
\item $(\frac{a}{N})$ solo depende de la clase de $a\mod N$.
\item $(\frac{ab}{N})=(\frac{a}{N})(\frac{b}{N})$
\end{enumerate}
\end{lemma}
\begin{proof}
Se deduce que de la multiplicatividad se deduce de la del símbolo de Legendre. Si $p=2$ es trivial. Si $p|a$ o $p|b$ entonces es 0 en ambos lados. En otro caso, se usa la proposición anterior.
\end{proof}

\begin{teorema}[Ley de reciprocidad cuadrática de Gauss]
Sea $p$ un primo impar. \url{https://en.wikipedia.org/wiki/Quadratic_reciprocity}. Además un par de casos concretos

\begin{enumerate}
\item $(\frac{-1}{p})=(-1)^{(p-1)/2}=\begin{cases}
1 & p\equiv 1\mod 4\\
-1 & p\equiv 3\mod 4
\end{cases}$

\item 
\item $(\frac{2}{p})=(-1)^{(p^2-1)/8}=\begin{cases}
1 & p\equiv 1,7\mod 8\\
-1 & p\equiv 3,5\mod 8
\end{cases}$
\end{enumerate}
\end{teorema}
\begin{dem}

\begin{lemma}
Sea $p$ un primo impar, $n\geq 1$ entero no divisible por $p$. Entonces existe un cuerpo $\F_p(n)\supseteq \F_p$ tal que $\F_p(n)^*$ contiene un subgrupo cíclico de orden $n$. 
\end{lemma}
\begin{proof}[Creo que esto es la prueba del lema]
Consideramos una clasura algebraica $\overline{\F}_p$ de $\F_p$. Sea $f(c)=x^n-1\in\F_p[x]$. No tiene raíces dobles porque $f'(x)=nx^{n-1}\neq 0$ porque $p$ no divide a $n$ y además $\gcd(f(x),f'(x))=1$ ($f(x)$ es separable), esto implica que $f(x)$ no tiene raíces múltiples. Sean $\{\alpha_1,\dots,\alpha_n\}\subseteq\overline{F}_p$ las $n$ raíces de $f(x)$. Sea $\F_p(n)=\F_p(\alpha_1,\dots, \alpha_n)$. Es claro que $H=\{\alpha_1,\dots,\alpha_n\}$ es un subgrupo cíclico de $\F_p(n)^*$. $\F_p(n)$ es un cuerpo finito porque se obtiene mediante $n$ extensiones de grado finito de un cuerpo finito, luego $\F_p(n)^*$ es cíclico.  Como $H$ es subgrupo de un cíclico, entonces es cíclico.
\end{proof}

Demostramos la parte 3. Sea $\F_p(8)$ el cuerpo del lema anterior. Vamos a dar un elemento $\gamma\in\F_p(8)$ tal que $(\frac{2}{p})=1\Leftrightarrow \gamma\in\F_p$. Definimos $\gamma=\alpha+\alpha^{-1}$. Calculamos $\gamma^2=\alpha^2+\alpha^{-2}+2$. Como $\alpha^8=1$ en $\F_p(8)$, $\alpha^4=\pm 1$, y como $\alpha$ tiene orden 8, el signo es negativo. Entonces $(\alpha^2+\alpha^{-2})^2=0$, por lo que $\alpha^2+\alpha^{-2}=0$ también puedes hacer $\alpha^{-2}=-\alpha^4\alpha^{-2}=-\alpha^2$). Deducimos que $\gamma^2=2$. Si $\gamma\in\F_p$, $\gamma$ es un cuadrado, luego $(\frac{2}{p})=1$. 

Recíprocamente, si $(\frac{2}{p})=1$, existe $a\in\F_p$ con $a^2=2$. Tenemos que $a^2=2=\gamma^2$, luego $a^2-\gamma^2=0\Rightarrow (a+\gamma)(a-\gamma)=0\Rightarrow \gamma=\pm a\in\F_p$. 

Para ver ahora la igualdad del teorema, basta ver que si $p\equiv 1,7\mod 8$ entonces $\gamma\in\F_p$ ($\Leftrightarrow\gamma^p=\gamma$) y que si $p\equiv 3,5\mod 8$ entonces $\gamma\notin\F_p$ ($\Leftrightarrow \gamma^p\neq \gamma$).

\begin{itemize}
\item $p=1+8m$. $\gamma^p=(\alpha+\alpha^{-1})^p=\alpha^p+\alpha^{-p}$ como $\alpha^8=1$, esto es $\gamma$. 
\item $p=3+8m$. $\gamma^p=\alpha^{3+8m}+\alpha^{-3-8m}$. Usando $\alpha^4=-1$ tenemos que $\alpha^3+\alpha^{-3}=-\alpha-\alpha^{-1}=-\gamma\neq\gamma$.
\end{itemize}
Y to lo demás igual.


Ahora probamos la parte 2, que es el caso general. La idea es buscar $\beta\in\F_p(q)$ tal que ``algo'' sea cuadrado $\mod p$ si y solo si $\beta\in\F_p$. Ese ``algo'' serán las sumas de Gauss. Fijamos $\alpha\in\F_p(q)$ de orden $q$. Definimos
\[
S_p(q)=\sum_{k=1}^{q-1}\left(\frac{k}{q}\right)\alpha^k
\]
$S_p(q)$ depende de la elección de $\alpha$. El índice se puede cambiar por $k\in\F_q^*$.
\begin{lemma}
Si $\beta^q=1$, $\beta\neq 1$, entonces $\sum_{j=0}^{q-1}\beta^j=0$. 
\end{lemma} 
\begin{proof}
$\sum_{j=0}^{q-1}\beta^j=0\Leftrightarrow (1-\beta)\sum_{j=0}^{q-1}\beta^j=0\Leftrightarrow 1-\beta^q=0$, que se tiene por hipótesis. 
\end{proof}

\begin{lemma}
$S_p(q)=(\frac{-1}{q}q\in\F_p(q)$
\end{lemma}
\begin{proof}
Por la multiplicatividad de Legendre, escribimos el cuadrado como producto por sí mismo y lo que sale es
\[
\sum_{k,m\in\F_p^*}(\frac{km}{q})\alpha^{k+m}
\]
Cambiamos de variable, $m=kr$
\[
\sum_{k}\sum_r(\frac{k^2r}{q})\alpha^{k(1+r)}
\]
Tenemos que que $(\frac{k^2r}{q})=(\frac{k}{q})^2(\frac{r}{q})=(\frac{r}{q})$. Así que la suma se reduce a
\[
\sum_r(\frac{r}{q})\sum_k(\alpha^{1+r})^k
\]
Como $\alpha_{k=0}^{q-1}(\alpha^{1+r})^k=0$, la suma desde 1 puede valer $q-1$ si $\alpha^{1+r}=1$ ($1+r=q$) o $-1$ en caso contrario. 
\[
\sum_{r=1}^{q-2}(\frac{r}{q})(1)+(\frac{-1}{q})(q-1)=\sum_{r=1}^{q-1}(\frac{r}{q})(-1)+q(\frac{-1}{q})
\]
Basta ver que $\sum_{r=1}^{q-1}(\frac{r}{q})=0$. Como hay el mismo número de no residuos que de residuos no nulos $\mod  p$, la suma se cancela y da 0. Hemos probado que $S_p(q)^2=q(\frac{-1}{q})\in\F_p(q)$ (en particular no depende de $\alpha$). Además, $S_p(q)\neq 0$. 

\begin{lemma}[Fórmula cerrada para el símbolo de Legendre]
$\left(\frac{p}{q}\right)=S_p(q)^{p-1}$
\end{lemma}
\begin{proof}
$S_p(q)^p=(\sum_{n\in\F_q^*}(\frac{n}{q})\alpha^n)^p=\sum_{n\in\F^*_q}(\frac{n}{q})\alpha^{np}$
Como $p$ es impar y el símbolo de Legendre es $\pm 1$, la potencia no afecta. 
Tenemos que por algún motivo (seguramente multiplicatividad) la suma de antes es igual a 
\[
\sum(\frac{np^2}{q}\alpha^{np}=\sum_{r=np\in \F_q^*}(\frac{rp}{q})\alpha^r=(\frac{p}{q})\sum_r(\frac{r}{q})\alpha^r=(\frac{p}{q})S_p(q)
\]
Como $S_p(q)\neq 0$, entonces $S_p(q)^{p-1})=(\frac{p}{q})$.
\end{proof}
Tenemos entonces que $(\frac{p}{q})=1\Leftrightarrow S_p(q)^{p-1}=1\Leftrightarrow S_p(q)\in\F_p\Leftrightarrow q(-1)^{(q-1)/2}$ es un cuadrado en $\F_p$. La última equivalencia es porque $S_p(q)^2=q(\frac{-1}{q})=q(-1)^{(q-1)/2}$. Continuamos la cadena de equivalencias con $(\frac{q(-1)^{(q-1)/2}}{q})=1$, que es equivalente usando la multiplicatividad a que $(\frac{p}{q})(-1)^{(p-1)/2 (q-1)/2}=1$. 

\end{proof}
\end{dem}

\subsection{Algoritmo de factorización}
Dado $N$, encontrar $d|N$. 

\textbf{Algoritmo $p-1$ de Pollard}.

Dados $N,R\geq 1$, es fácil calcular $d=\gcd(N,R)$. Si $d\neq 1,N$, $d$ es un divisor propio de $N$. Idea: si $a$ es tal que $\gcd(a,N)=1$ verifica $a^{p-1}\equiv 1\mod p$ para todo $p|N$. Si $p|N$, $\gcd(a^{p-1}-1,N)$ no es divisible por $p$ (en particular no es 1). La idea del algoritmo es tomar $M$ divisible por potencias de primos pequeños y calcular $\gcd(a^M-1,N)$. Si $p-1|M$, entonces $\gcd(a^M-1,N)\neq 1$. 

El algoritmo consiste en lo siguiente. Dado $N$ y $B>0$ (una cota), escogemos $1\leq a\leq N-1$. Si $\gcd(a,N)\neq 1$, todo ok. Sea $M=mcm\{b:b<B\}$, calculamos $d=\gcd(a^M-1,N)$. Si $d=1$, aumentamos $B$, si $1<d<N$ todo ok. Si $d=N$, cambiamos $a$ y repetimos. 

\begin{prop}
Sea $N=pq$ con $p,q$ primos distintos. Supongamos que existe $M$ con $p-1|M$ y $q-1\not| M$. Entonces
\[
\#\{1\leq a\leq N\mid\gcd(a,N)=1, \gcd(a^M-1,N)=p\}\geq (p-1)\varphi(q-1)
\]
\end{prop}
\begin{dem}
Por el teorema chino del resto, $(\Z_N)^*\cong(\Z_p)^*\times(\Z_q)^*$. Que $\gcd(a^M-1,N)=p$ es equivalente a $q\not| a^M-1$. Si $a$ es un generador de $\F_q^*$, entonces $ord(a)=q-1$ y $q\not| a^M-1\Leftrightarrow (q-1)\not| M$. Hay $\varphi(q-1)$ elementos de orden $q-1$ en $\Z_q^*$ y tenemos los $p-1$ elementos de $\Z_p^*$ de donde sale la cota.
\end{dem}



\begin{ej}
$(\frac{127}{31}(=(\frac{3}{31})$ porque $127=4\cdot 31+3$, y por tanto es igual a $(-1)^{(3-1)/2 (31-1)/2}(\frac{31}{3})=-(\frac{1}{3})=-1$, por ser $31=3\cdot 10+1$.
\end{ej}

\begin{teorema}[Ley de reciprocidad de Jacobi]
Sean $m,k\geq 3$ enteros impares. Entonces se cumplen los dos primeros apartados del teorema anterior para $p=m$ y adema´s
\[
(\frac{m}{k})=(-1)^{(k-1)/2 (m-1)/2}(\frac{k}{m})
\]
\end{teorema}

\begin{lemma}
\begin{enumerate}
\item $\varepsilon:(\Z_4)^*\to \{\pm 1\}$, $m\mapsto (-1)^{(m-1)/2}$ es morfismo de grupos
\item $\omega:(\Z_8)\to\{\pm 1\}$, $m\mapsto (-)^{(m^2-1)/8}$ también.
\end{enumerate}
\end{lemma}
La prueba es sencilla. 

\begin{dem}[sketch del teorema]
Escribimos $m=\prod_i p_i^{e_i}$. Para las dos priemras congruencias basta utilizar $\omega$ y $\tau$ del lema anterior sobre $m$. Para la parte interesante, 
\[
(\frac{m}{k})(\frac{k}{m})^{-1}=(\frac{m}{k})(\frac{k}{m})
\]
Aplicar la definición y la multiplicatividad, luego reciprocidad cuadrática. Después usar $\varphi$ y listo. 
\end{dem}

\begin{prop}
Sea $N>2$ impar, entonces $N$ es primo si y solo si 
\[
\left(\frac{a}{N}\right)\equiv a^{\frac{N-1}{2}}\mod N
\]
para todo $a$ con $\gcd(a,N)=1$.
\end{prop}
\begin{dem}
Para todo $a$ primo con $N$, tenemos que $a^{N-1}\equiv (\frac{a}{N})^2\equiv 1\mod N$. Entonces $N$ es primo de Carmichael. Supngamos que $N$ es de Carmichael con $N=p_1\dots p_3$ donde los $p_i$ son distintos e impares. Sea $(\Z_{p_i})^*$ generado por $a+\gene{p_i}$. Tenemos que $a^{(p-1)/2}\equiv -1\mod p_i$. Sea $b\in\Z$ tal que $b\equiv a\mod p_i$ y es congruente con 1 módulo cualquier otro de los primos, que existe por el teorema chino del resto. Se tiene de estas ecuaciones que $\gcd(b,N)=1$. Además $b^{(N-1)/2}=(\frac{b}{N})\mod N$. Como $(\frac{b}{p_i})=(\frac{a}{p_i})=-1$ y 1 para cualquier otro de los primos. Tenemos entonces  $b^{N-1)/2}=-1\mod N$ y $1=1^{N-1)/2}=-1\mod p_2$, lo cual es una contradicción.
\end{dem}

\subsection{Test de primalidad de Solovay-Strassen}
Sea $N>2$ impar. Escojo $a=1,\dots, N-1$, $d=\gcd(a,N)$. Si $d\neq 1$, entonces $N$ no es primo. Si $d=1$, calculamos $a^{(N-1)/2}$ y $(\frac{a}{N})$. Si no coinciden, $N$ no es primo. Si coinciden, repetimos.

El cardinal de los número que no se descartan para un $a$ concreto está acotado por $\varphi(N)/2$. El razonamiento es similar a uno hecho anteriormente para un cardinal definiendo $\theta:(\Z_N)^*\to\{\pm 1\}$ como $a\mapsto a^{(N-1)/2}(\frac{a}{N})$. 

\subsection{Método $\rho$ de Pollard}
Está inspirado en la paradoja del cumpleaños. El cálculo de probabilidad en esta paradoja nos da un producto $\prod_{i=1}^{k-1}(1-\frac{i}{n})$. Como $e^x=\lim (1+\frac{x}{n})^{1/n}$, podemos aproximar el producto anterior por $\prod_{i=1}^{k-1}e^{-i/n}=e^{-k(k-1)/2n}\approx e^{-k^2/2n}$. Al hacer la probabilidad del complementario tendríamos entonces $1-e^{-k^2/2n}$. Para que la probabilidad sea al menos de $1/2$, tomando logaritmo obtenemos $\log 2=k^2/2n$, por lo que $k\sim O(\sqrt{n})$. 

Basado en esto, queremos un factor de $N$ en el conunto $S=\{1,\dots, N\}$. Tomamos $f(x)=x^2+1\mod N$ y un punto de partida $s\in\{1,\dots, N\}$. Se construye la sucesión seudoaleatoria $s_0=s$, $s_i=f(s_{i-1})$. Podemos esperar por la paradoja del cumpleaños que $s_i=s_j$ para $i\sim\sqrt{N}$. Sea ahora $p|N$ y consideremos la sucesión análoga a la anterior generada por $g(x)=x^2+1\mod p$. Aquí la repetición se esperará en $O(\sqrt{p})$ pasos, que es menor que $O(\sqrt{p})$. De esta forma, $s_i-s_j$ sería divisible por $p$ pero no por $N$. Por tanto, $p|\gcd(s_i-s_j,N)$ y es distinto de $N$, luego será un divisor propio de $N$. El algoritmo sigue estos pasos:

\begin{enumerate}
\item Calcular $s_0,s_1,\cdots$.
\item Para cada $i$, calcular $\gcd(s_i-s_j,N)$ para todo $j<i$. 
\end{enumerate}
El motivo del nombre se puede ver en \url{https://en.wikipedia.org/wiki/Pollard\%27s_rho_algorithm}. Para encontrar el ciclo donde aparecen las repeticiones se sigue el método de Floyd de la libere y la tortuga (que también aparece en el enlace). La liebre va al doble de velocidad que la tortuga, ¿cuándo pilla la liebre a la tortuga? Es decir, queremos encontrar una repecición $s_i=s_{2i}$. Si $s_i=s_j$, entonces $f(s_i)=f(s_j)$, es decir, $s_{i
i}=s_{j+1}$. Si llamamos $l=i-j$ ($i=j+l$), para todo $m\geq j$ tendremos $f^{m+l})(s_0)=f^m(s_0)$, e iterando $f^{m+kl}(s_0)=f^m(s_0)$.  Dividiendo $j$ entre $l$, $j=ql+r$. Para $m=(q+1)l\geq j$, $f^m(s_0)=f^{m+kl}(s_0)$ para todo $k$, en particular para $k=q+1$, o sea, $f^m(s_0)=f^{2m}(s_0)$, esto es $s_m=s_{2m}$. Tenemos ademas que como $m=(q+1)l=(j-r)+l$, $i-r\sim \sqrt{N}$. Así que podemos refinar el algoritmo
\begin{enumerate}
\item Escogemos $s\in\{1,\dots, N\}$, $s_0=s$. 
\item Para $i\geq 1$ calculamos $s_i=f(s_{i-1})$, $s_{2i}=f(s_{2i-1})$, $\gcd(s_{2i}-s_i,N)$.
\end{enumerate}

Para $N=2057$, $s_0=N\equiv 0\mod N$ y $f(x)+x^2+1$ se obtiene en el sexto paso $s_{2i}-s_i=966$, que verifica $\gcd(966,N)=23\neq 1$, luego tenemos un divisor. Además, $\sqrt{23}\sim 4.7$ que no es lejano a 6 (aunque con un ejemplo tan pequeño es normal que se distancie un poco). 

\section{Problema del logaritmo dicreto}
Dado un grupo $G=\gene{g}$ cíclico y $h\in G$, $n\in\N$ tal que $h=g^n$. Veremos primero un algoritmo general para resolverlo (todos los generales son exponenciales) y después veremos algunos algoritmos más eficientes para grupos concretos, donde los criptosistemas basados en estos grupos no serán tan seguros.

\subsection{Método $\rho$ de Pollard para el logaritmo discreto}
Sea $N=|G|$. Como $g^N=1$, busamos $n\in\{1,\dots, N\}$. Vamos a construir una sucesión $x_i=h^{a_i}g^{b_i}$ con $(a_i,b_i)\in\{1,\dots, N\}^2$ aleatorios. Si $x_i=x_j$, $h^{a_i}g^{b_i}=h^{a_j}g^{b_j}\Rightarrow g^{n(a_i-a_j)}=h^{a_i-a_j}=g^{b_j-b_i}\Rightarrow N|n(a_i-a_j)-(b_j-b_i)$. Si $\gcd(N,a_i-a_j)=1$, entonces $n\equiv (b_j-b_i)(a_i-a_j)^{-1}\mod N$. Si no, sea $d=\gcd(N,a_i-a_j)$. Tenemos $n(a_i-a_j)\equiv b_j-b_i\mod N$, luego $n(a_i-a_j)/d\equiv (b_j-b_i)/d\mod N/d$. Podemos calcular $n\mod N/d$, con lo que hay $d$ posibilidades $\mod N$, $n_1,\dots, n_d$. Se pueden calcular las respectivas potencias de $g$ y ver cuál es $h$. 

¿Cómo calculamos $(a_i,b_i)$? Se divide $G=G_1\cup G_2\cup G_3$ en tres partes aproximadamente iguales. Partimos de $a_0=b_0=0$. Entonces 
\[
(a_{i+1},b_{i+1})=\begin{cases}
(a_i+1,b_i)\ (\Rightarrow x_{i+1}=hx_i) & x_i\in G_1\\
(2a_i,2b_i)\ (\Rightarrow x_{i+1}=x_i^2) & x_i\in G_2\\
(a_i,b_i+1\ (\Rightarrow x_{i+1}=gx_i) & x_i\in G_3
\end{cases}
\]
Esto funciona porque funciona.

\subsection{Método de cáculo del índice (index calculus)}
Sea $p$ primo, $G=\Z_p^*$, $g$ un generador y $h\in G$. Queremos calcular $n$ tal que $g^n=h$. 

\begin{enumerate}
\item[Paso 1:] pre-calcular el logaritmo discreto para primos pequeños. Fijamos $B\in\N$ y definimos $S=\{p\leq B, p\text{ primo }\}=\{p_1,\dots, p_k\}$. Vamos a calcular $e_i\in N$ tales que $g^{e_i}=p_i\mod p\in\Z_p^*$. Buscamos exponentes $s$ tales que $g^s=p_1^{r_1}\cdots p_k^{r_k}$ en $\Z$ (estamos eligiendo un representante de $g$, que podemos ir cambiándolo). Entonces $s=e_1r_1+\cdots+e_tr_t\mod p-1$. Esto nos da una ecuación para $e_1,\dots e_k$. Vamos buscando exponentes $s_1,\dots, s_k$  hasta que el sistema de ecuaciones esté formado por $k$ ecuaciones y esté determinado.
\item[Paso 2:] Buscamos un exponente $R$ tal que $hg^R=p_1^{t_i}\dots, p_k^{t_k}$ (representantes). De esto, $n+R=\sum e_it_i\Rightarrow n=\sum e_it_i-R$. 
\end{enumerate}

La complejidad de este algoritmo es subesponencial, concretamente $e^{c\sqrt{\log N\log\log N}}$, con $c$ constante. Tomando $G=\F_{p^r}^*$ también podemos aplicar el agoritmo. Lo que hemos usando de $\Z_p^*$ es que tenemos una sobreyección $Z\to \Z_p$ y en $\Z$ tenemos factorización en primos. Podemos usar la sobreyección $\F_p[x]\to \F_{p^r}$ dada por $\F_{p^r}\cong \F_p[x]/\gene{f(x)}$ con $f(x)$ un polinomio irreducible de grado $r$. Como $\F_p[x]$ es un dominio euclídeo tenemos factorización en elementos irreducibles y podemos hacer un algoritmo análogo.

\begin{enumerate}
\item[Paso 1:] pre-calcular el logaritmo discreto de unos ciertos polinomios irreducibles $f_1,\dots, f_k$ de grado pequeño ($\leq B$). Fijado $g\in F_{p^r}^*$ generador, sea $pC(x)\in\F_p[x]$ representante de $g$. Calcular $p^s(x)\mod f(x)$ de forma que $p^s(x)=f_1^{e_1}\cdots f_k^{e_k}$. Necesitamos buscar ecuaciones independientes igual que en el caso anterior.
\item[Paso 2:] Se $h\in\F_{p^r}^*$ y $q(x)$ un representante. Buscamos $R$ tal que  $q(x)p(x)^R=f_1^{t_1}\cdots f_k^{t_k}\mod f(x)\Rightarrow n=t_1\log f_1+\cdots t_k\log f_k-R\mod p^r-1$. 
\end{enumerate} 

\begin{defi}
Sea $k$ un cuerpo y $f\in k[x]$. Se define el orden de $f(x)=\sum_{i=0}^na_ix^i$ en $x=0$ como $ord(f)=\min\{i:a_i\neq 0\}$. 
\end{defi}

\begin{lemma}
$f,g\in k[x]$
\begin{enumerate}
\item $ord(f\cdot g)=ord(f)+ord(g)$.
\item $ord(f+g)\geq\min\{ord(f),ord(g)\}$, dándose la igualdad si $ord(f)\neq ord(g)$.
\item $ord(f\circ g)=ord(f)ord(g)$. 
\end{enumerate}
\end{lemma}
La prueba es muy sencilla.

\begin{nota}
$ord$ es una valoración discreta sobre $k[x]$. 
\end{nota}

\begin{defi}
Si $\psi(x)=k(x)$ con $\psi(x)=\frac{f(x)}{g(x)}$ ($f,g\in k[x], g\neq 0$). Se define $ord\phi=ord f-ord g$.
\end{defi}
Se comprueba fácilmente que el orden de una función racional no depende de la expresión como fracción. Además se verifican las propiedades para polinomios.

\begin{defi}[Multiplicidad de intersección entre una curva plana sobre una recta en un punto]
Sea $C/k=\{f(x,y,z)=0\}$ una curva, $L=L(\alpha,\beta,\gamma)/k$ una recta y $P=[a:b:c]$. Si $P\notin L$, se define $m(C,L,P)=0$. Si $P\in L$, escogemos $P'=[a':b':c']\in L$ distinto de $P$. Consideramos $\psi(t)=f(a+ta',b+tb',c+tc')$. Se define entonces $m(C,L,P)=ord\psi(t)$ (con respecto a la variable $t$, o sea en $t=0$ siguiendo la definición de orden). 
\end{defi}

Se comprueba que esta definición no depende de ninguna de las elecciones.

\begin{lemma}
Sea $C_f$ una curva plana, $L$ una recta y $P\in L$. Entonces 
\begin{enumerate}
\item $m(C_f,L,P)\geq 1\Leftrightarrow P\in C_f$. 
\item $P\in C_f$ y $L$ es la tangente a $C_f$ en $P$ implican que $m(C_f,L,P)\geq 2$.
\end{enumerate}
\end{lemma}
\begin{proof}
Mirar a partir de la definition 2.3.7 de Elliptische Kurven a ver si está por ahí
\end{proof}

\begin{lemma}
Sea $A\in GL_3(k)$ y $C_f$ una curva. $P\in C_{f_A}$. $P$ es singular en $C_{f_A}$ si y solo si $\varphi_A(P)$ es singular en $C_f$.  
\end{lemma}
EL PHI ESE NO SÉ DE DÓNDE SALE, MIRAR EL LIBRO

\section{Curvas elípticas}
Muchas veces para definirlas se usa el género\footnote{\url{https://www.encyclopediaofmath.org/index.php/Genus_of_a_curve}} de una curva. 

Ecuación de Weierstrass (tras definición 2.3.1) y definición 2.3.1 (curva elíptica = curva lisa/no singular dada por una ecuación de Weierstrass)

\begin{lemma}
Si $f(x,y,z)=0$ es una ecuación de Weierstrass, entonces si
\[
A=\begin{pmatrix}
u^2 & 0 & r\\
u^2s & u^3 & t\\
0 & 0 & 1
\end{pmatrix}\in GL_3(k)
\]
con $u,s,t\in k$, se tiene que $f_A(x,y,z)=0$ también es de Weierstrass.
\end{lemma}
La prueba es directa. CREO QUE EL SUBÍNDICE A ES MULTIPLICAR LA MATRIZ POR EL POLINOMIO VISTO COMO VECTOR

Descripción del punto $O$ y no singularidad. Además la tangente en $O$ a $C_f$ es $L(0,0,1)$. $m(C_f,L,O)=ord\psi(t)$ con $\psi(5)=f(0+t,1+0t,0+0t)=f(t,1,0)$. Sea $P'=[1:0:0]\in L(0,0,1)$. Entonces sale que $m(C_f,L,O)=3$, por lo que tiene multiplicidad de tangente. 

Números $b_i$, $\Delta$ y $j$ de las páginas 27 y 28. 


Proposition 2.3.2, aunque creo que la hemso puesto algo distinta:

\begin{lemma}
Si la característica de $k$ no es 2, sea $g(x,y,z)=y^2z-x^3-\frac{1}{4}b_2x^2z-\frac{1}{2}b_4xz^2-\frac{1}{4}b_6z^3$. Entonces $\varphi:\PP^2\to \PP^2$ dada por $[a:b:c]\mapsto [a:b+\frac{a_1}{2}+\frac{a_3}{2}c:c]$ verifica que $C_f=\varphi(C_f)$.
\end{lemma}

\begin{lemma}[en la página 26]
Si la característica de $k$ no es 2 ni 3, y $g(x,y,z)=y^2z-x^3+27c_4xz^2+54c_6z^3$. Entonces el cambio de variables $\varphi:[a:b:c]\mapsto [36a+2b_2c:216b:c]$ lleva $f$ en $g$. 
\end{lemma}

\begin{prop}[2.3.3]
Sea $f(x,y,z)=0$ ecuación de Weierstrass. Entonces $C_f$ es lisa si y solo si $\Delta=0$.
\end{prop}

Esto nos da como conclusión que $C_f$ es elíptica si y solo si $\Delta\neq 0$.
\end{document}

\begin{prop}[Satz 2.3.13 und 2.3.14]
\end{prop}

Subgrupo de torsión (CREO página 61 en adelante) \url{https://en.wikipedia.org/wiki/Tate_module}

DIFFIE-HELMMAN EN CURVAS ELÍPTICAS

