\documentclass[CR.tex]{subfiles}

\begin{document}


%\hyphenation{equi-va-len-cia}\hyphenation{pro-pie-dad}\hyphenation{res-pec-ti-va-men-te}\hyphenation{sub-es-pa-cio}

\chapter{Tests de primalidad}
\section{Preliminares}
Usamos la versión del teorema chino del resto que dice que si $a,b\in\Z_{>0}$ son tales que $\gcd(a,b)=1$ entonces $\Z_{ab}\cong \Z_a\times\Z_b$ mediante el isomorfismo $x\mapsto (x\mod a,x\mod b)$. 

Recordemos que un elemento de $a\in \Z_N$ es unidad si y solo si $\gcd(a,N)=1$: si $ab=1\mod N$, entonces $N|ab-1$, y de la identidad de Bézout se deduce el resultado. Recíprocamente, si $\gcd(a,b)=1$, el inverso es el $\alpha$ de la identidad de Bézout. Por ello, $\varphi(N)$ se puede definir como los primos con $N$ o como el cardinal de las unidades.

\begin{lemma}
$\varphi$ es multiplicativo si $\gcd(a,b)=1$. 
\end{lemma}
\begin{proof}
Por el teorema chino del resto, $\Z_{ab}\cong\Z_a\times\Z_b$, y como el isomorfismo respeta las unidades, se tiene también $(\Z_{ab})^*\cong(\Z_a)^*\times(\Z_b)^*$. Tomando cardinal se obtiene $\varphi(ab)=\varphi(a)\varphi(b)$. 
\end{proof}
\begin{lemma}
$\varphi(p^e)=p^{e-1}(p-1)$ para $p$ primo y $e\in\Z_{>0}$.
\end{lemma}
\begin{proof}
$\varphi(p^e)=\sharp\{a=1,\dots, p^e-1\mid p\not| a\}=p^e-\sharp\{a=1,\dots, p^e-1\mid p|a\}=p^e-\sharp\{p,2p,\dots\}=p^e-p^{e-1}$.
\end{proof}


\begin{lemma}
Para todo $p$ primo, el grupo $(\Z_p)^*$ es cíclico 
\end{lemma}
\begin{proof}
Queremos probar que existe $a\in\Z_p$ tal que $ord(a)=p-1$. Sea $mcm\{ord(b)\mid b\in\Z_p\}=m$. Si $p^e|m$ y $p^{e+1}\not| m$; entonces existe $b$ tal que $p^e| ord(b)$, luego $ord(b)=rp^e$, así que $b^{r}$ tiene orden $p^e$. Probamos que $ord(\prod_{p|m}b_p^{r_p})=m$. Si $m=p-1$, ya está. Si no, consideramos $f(x)=x^m-1\in\Z_p(x)$. Para todo $b\in\Z_p$, $f(b)=b^m-1=0$. Esto significa que $f$ tiene $p-1$ raíces distintas, por lo que $m\geq p-1$. Por otra parte, $m\leq p-1$ porque $ord(b)|p-1$ por el teorema de Lagrange, y por ello también lo divide el mínimo común múltiplo de los órdenes, $m$. 
\end{proof}

\section{Test de primalidad}
Nos basamos en el pequeño teorema de Fermat, esto es, si $p$ es primo y $a\geq 1$, $a\not| p$, entonces $a^{p-1}\equiv 1\mod p$. Esto es consecuencia del teorema de Lagrange, pues al ser $a$ unidad por no dividir a $p$, su orden debe dividir a $p-1$. 

Sea $N\geq 2$. Sea $a\geq 1$ tal que $\gcd(a,N)=1$ (si no fuera 1, tendríamos divisores de $N$, y por tanto $N$ no es primo). Si $a^{N-1}\not\equiv 1\mod N$, entonces $N$ no es primo.

\begin{ej}
$N=91$, $a=3$. Haciendo reducciones llegamos a que $3^{90}\equiv 1\mod 91$. Esto no es concluyente. De hecho $91=7\cdot 13$. 

Para $a=2$, el test nos da $2^{90}\equiv 64\mod 91$, lo que sí concluye que no es primo, aunque no conocemos sus factores. 
\end{ej}


\begin{defi}
Diremos que $N\geq 2$ no primo es un número de Carmichael si para tod $a$ con $\gcd(a,N)=1$, $a^{N-1}\equiv 1\mod N$.
\end{defi}

Por ejemplo, 561 o 1105. 

\begin{prop}
Si $N>2$ no es de Carmichael y es compuesto, entonces 
\[
\#\{a\mid 1\leq a\leq N-1, gcd(a,N)=1, a^{N-1}\equiv 1\mod N\}\leq \varphi(N)/2 
\]
\end{prop}
\begin{dem}
Sea $H=\{a\in\Z_N\mid$ mismas condiciones que en el enunciado$\}\subseteq (\Z_N)^*$. Es fácil ver que $H$ es un subgrupo. Si $N$ no es de Carmichael y es compuesto, $H\neq (\Z_N)^*$ porque hay algún $a$ que no verifica la ecuación. Entonces $\#(H)c=\varphi(N)$ con $c\geq 2$, y ya está. 
\end{dem}

\begin{prop}
Para $N\geq 2$ son equivalentes:
\begin{enumerate}
\item $N$ es de Carmichael
\item Para todo $p|N$: $p$ es impar, $p^2\not| N$ y $(p-1)\mid (N-1)$. 
\end{enumerate}
\end{prop}
\begin{dem}
$1\Rightarrow 2$. 
\begin{enumerate}
\item Por 2, $2^2\not| N$, luego existe $p>2$ primo dividiendo a $N$.  Por 3, $p-1|N-1$, pero $p-1$ es par y $N-1$ es impar. 
\item $a^{N-1}\equiv 1\mod N\Rightarrow a^N\equiv a\mod N$. Para $a=p$, $N| p^N-p$, así que $p=p^N-cN$ para algún $c$.
\item Sea $a$ generador $(\Z_p)^*$, de modo que $ord(a)=p-1$ y $a^{N-1}\equiv 1\mod p$. 
\end{enumerate}

$2\Rightarrow 1$. Por el teorema chino del resto usando la descomposición $N=p_1\cdots p_r$ ($N$ es libre de cuadrados), tenemos que $a^{N-1}\mod N\mapsto (a^{N-1}\mod p_1,\dots, a^{N-1}\mod p_r)$, y el $1$ va al $(1,\dots, 1)$, de mod oque $a^{N-1}=1\mod p_i$ para todo $i$, de donde se deduce que $a^{N-1}=1\mod p_1\cdots p_r$. 
\end{dem}

\begin{ej}
$N=561$ es de Carmichael. $3-1,11-1$ y $17-1$ dividen a $561-1$. 
\end{ej}

\begin{prop}
$\#\{x\leq N, x$ es de Carmichael$\}\geq O(N^{2/7})$.
\end{prop}

Vamos a cambiar el test por $a^{\frac{N-1}{2}}\equiv 2\mod N$ (para $N$ impar). 

\begin{defi}
Diremos que $a\geq 0$ es un cuadrado $\mod p$ o un residuo cuadrático si $\exists b\geq 0$ tal que $b^2=a\mod p$.
\end{defi}

\begin{ej}
$p=3$, podemos hacer todas las cuentas y $2$ no es un cuadrado
\end{ej}

\begin{lemma}
Sea $p>2$ primo. Hay $\frac{p-1}{2}$ residuos no nulos  y la misma cantidad de no residuos. 
\end{lemma}
\begin{proof}
Sea $\varphi:(\Z_p)^*\to(\Z_p)^*$, $x\mapsto x^2$. $\ker\varphi\supseteq \{1,-1\}$. El polinomio $x^2-1\in\Z_p[x]$ tiene a lo más 2 raíces en $\Z_p$, por lo que la inclusión anterior es una igualdad. Primer teorema de isomorfía, $\Ima\varphi\cong (\Z_p)^*/\ker\varphi$. Como la imagen tiene $\frac{p-1}{2}$ elementos, y estos son los residuos cuadráticos, se deduce el lema. 
\end{proof}


\begin{prop}
Sea $N=p_1^{e_1}\cdots p_r^{e_r}$. Sea $a\geq 1$, entonces $a$ es residuo módulo $N$ si y solo si $a$ es residuo módulo $p_i^{e_i}$. 
\end{prop}
\begin{dem}
Hacia la derecha es trivial. Para el otro lado, teorema chino del resto.
\end{dem}

\begin{lemma}
Sea $p$ un primo impar y $a\in\Z$ que no divide a $p$. Son equivalentes:
\begin{enumerate}
\item $a$ es residuo módulo $p$.
\item $a$ es residuo módulo $p^n$ para todo $p^n$. 
\end{enumerate}
\end{lemma}
\begin{proof}

\end{proof}

Con estos resultados tenemos que $a$ es cuadrado $\mod N$ si y solo si para todo $i$ $a$ es un cuadrado $\mod p_i^{e_i}$ si y solo si $a$ es cuadrado $\mod  p_i$ ($p_i>2$). El caso $p_i=2$ lo cubre el siguiente lema.

\begin{lemma}
Sea $a\in\Z$ impar. Son equivalentes:
\begin{enumerate}
\item $a$ es residuo cuadrático $\mod 2^n$
\item Se verifica alguno de los siguientes: $n=1$, $n=2$ y $a\equiv 1\mod 4$, $n\geq 3$ y $a\equiv mod 8$.
\end{enumerate}
\end{lemma}
La prueba es por inducción, probando a mano los casos $n=1$, $n=2$. 

\begin{defi}
Se define el símbolo de Legendre como sigue. Para todo primo $p$ y para todo $a\in\Z$, 
$$\left(\frac{a}{p}\right)=\begin{cases}
0 & p|a\\
1 & a \text{residuo no nulo }\mod p\\
-1 & c.c.
\end{cases}$$
\end{defi}

\begin{prop}[Euler]
Sea $p>2$ primo y $a\in\Z$. Entonces $a^{\frac{p-1}{2}}\equiv (\frac{a}{p}) \mod p$.
\end{prop}
\begin{dem}
Si $p|a$, entonces el resultado es trivial porque $a=0\mod p$. En otro caso, definimos $\phi:(\Z_p)^*\to (\Z_p)^*$ definido como $\phi(x)=x^{\frac{p-1}{2}}$. Se tiene que $\Ima\phi\subseteq \{\pm 1\}$ porque $a^{p-1}\equiv 1\mod p$ y $b=a^{\frac{p-1}{2}}$ es tal que $b^2\equiv 1\mod p$, que como $\Z_p$ es un cuerpo no puede tener más de dos soluciones. Por otro lado, $\ker\phi=\{a, a^{(p-1)/2}=1\}$. 

Tenemos que como $(\Z_p)^*$ es cíclico existe $c\not| p$ tal que $\{1,c,\dots, c^{(p-1)/2},\dot, c^{p-2}\}$ son todos distintos, por lo que $\phi(c)\neq 1$, es decir, $\phi(c)=-1$. Por ello $\Im\phi=\{\pm 1\}$ (tiene cardinal 2). Por el primer teorema de isomorfía, $\ker\phi$ tiene cardinal $\frac{p-1}{2}$.  Queremos probar $\ker\phi=\{b^2+\gene{p}\mid b=1,\dots, p-1\}$. Como tienen el mismo cardinal basta ver una inclusión. La inclusión fácil es hacia la izquierda, si tomamos un cuadrado $b^2+\gene{p}$, entonces $\phi(b^2+\gene{p})=b^{p-1}+\gene{p}=1+\gene{p}$ por el pequeño teorema de Fermat, y por tanto el cuadrado está en el núcleo. 

Tenemos entonces que $(\frac{a}{p})$ y $a^{(p-1)/2}$ toman el valor 1 en los mismos casos, por lo que valen lo mismo $\mod p$. 
\end{dem}

Queremos ver que la congruencia anterior se da también módulo $N$ cuando $N$ es compuesto definiéndola adecuadamente. 

\begin{defi}
Sea $N=p_1^{e_1}\cdots p_r^{e_r}$ y $a\in\Z$. Se define el símbolo de Jacobi como 
\[
\left(\frac{a}{N}\right)=\prod_{i=1}^r \left(\frac{a}{p_i}\right)^{e_i}
\]
\end{defi}

\begin{lemma}\
\begin{enumerate}
\item $(\frac{a}{N})$ solo depende de la clase de $a\mod N$.
\item $(\frac{ab}{N})=(\frac{a}{N})(\frac{b}{N})$
\end{enumerate}
\end{lemma}
\begin{proof}
Se deduce que de la multiplicatividad se deduce de la del símbolo de Legendre. Si $p=2$ es trivial. Si $p|a$ o $p|b$ entonces es 0 en ambos lados. En otro caso, se usa la proposición anterior.
\end{proof}

\begin{teorema}[Ley de reciprocidad cuadrática de Gauss]
Sea $p$ un primo impar. \url{https://en.wikipedia.org/wiki/Quadratic_reciprocity}. Además un par de casos concretos

\begin{enumerate}
\item $(\frac{-1}{p})=(-1)^{(p-1)/2}=\begin{cases}
1 & p\equiv 1\mod 4\\
-1 & p\equiv 3\mod 4
\end{cases}$

\item {enumerate}
\item $(\frac{2}{p})=(-1)^{(p^2-1)/8}=\begin{cases}
1 & p\equiv 1,7\mod 8\\
-1 & p\equiv 3,5\mod 8
\end{cases}$
\end{enumerate}
\end{teorema}
\begin{dem}
\end{dem}

\begin{ej}
$(\frac{127}{31}(=(\frac{3}{31})$ porque $127=4\cdot 31+3$, y por tanto es igual a $(-1)^{(3-1)/2 (31-1)/2}(\frac{31}{3})=-(\frac{1}{3})=-1$, por ser $31=3\cdot 10+1$.
\end{ej}

\begin{teorema}[Ley de reciprocidad de Jacobi]
Sean $m,k\geq 3$ enteros impares. Entonces se cumplen los dos primeros apartados del teorema anterior para $p=m$ y adema´s
\[
(\frac{m}{k})=(-1)^{(k-1)/2 (m-1)/2}(\frac{k}{m})
\]
\end{teorema}

\begin{lemma}
\begin{enumerate}
\item $\varepsilon:(\Z_4)^*\to \{\pm 1\}$, $m\mapsto (-1)^{(m-1)/2}$ es morfismo de grupos
\item $\omega:(\Z_8)\to\{\pm 1\}$, $m\mapsto (-)^{(m^2-1)/8}$ también.
\end{enumerate}
\end{lemma}
La prueba es sencilla. 

\begin{dem}[sketch del teorema]
Escribimos $m=\prod_i p_i^{e_i}$. Para las dos priemras congruencias basta utilizar $\omega$ y $\tau$ del lema anterior sobre $m$. Para la parte interesante, 
\[
(\frac{m}{k})(\frac{k}{m})^{-1}=(\frac{m}{k})(\frac{k}{m})
\]
Aplicar la definición y la multiplicatividad, luego reciprocidad cuadrática. Después usar $\varphi$ y listo. 
\end{dem}

\begin{prop}
Sea $N>2$ impar, entonces $N$ es primo si y solo si 
\[
\left(\frac{a}{N}\right)\equiv a^{\frac{N-1}{2}}\mod N
\]
para todo $a$ con $\gcd(a,N)=1$.
\end{prop}
\begin{dem}
Para todo $a$ primo con $N$, tenemos que $a^{N-1}\equiv (\frac{a}{N})^2\equiv 1\mod N$. Entonces $N$ es primo de Carmichael. Supngamos que $N$ es de Carmichael con $N=p_1\dots p_3$ donde los $p_i$ son distintos e impares. Sea $(\Z_{p_i})^*$ generado por $a+\gene{p_i}$. Tenemos que $a^{(p-1)/2}\equiv -1\mod p_i$. Sea $b\in\Z$ tal que $b\equiv a\mod p_i$ y es congruente con 1 módulo cualquier otro de los primos, que existe por el teorema chino del resto. Se tiene de estas ecuaciones que $\gcd(b,N)=1$. Además $b^{(N-1)/2}=(\frac{b}{N})\mod N$. Como $(\frac{b}{p_i})=(\frac{a}{p_i})=-1$ y 1 para cualquier otro de los primos. Tenemos entonces  $b^{N-1)/2}=-1\mod N$ y $1=1^{N-1)/2}=-1\mod p_2$, lo cual es una contradicción.
\end{dem}

\subsection{Test de primalidad de Solovay-Strassen}
Sea $N>2$ impar. Escojo $a=1,\dots, N-1$, $d=\gcd(a,N)$. Si $d\neq 1$, entonces $N$ no es primo. Si $d=1$, calculamos $a^{(N-1)/2}$ y $(\frac{a}{N})$. Si no coinciden, $N$ no es primo. Si coinciden, repetimos.

El cardinal de los número que no se descartan para un $a$ concreto está acotado por $\varphi(N)/2$. El razonamiento es similar a uno hecho anteriormente para un cardinal definiendo $\theta:(\Z_N)^*\to\{\pm 1\}$ como $a\mapsto a^{(N-1)/2}(\frac{a}{N})$. 

\end{document}

