\documentclass[CR.tex]{subfiles}

\begin{document}


%\hyphenation{equi-va-len-cia}\hyphenation{pro-pie-dad}\hyphenation{res-pec-ti-va-men-te}\hyphenation{sub-es-pa-cio}

\chapter{Elliptische Kurven in der Kryptographie}
\section{Ebene Kurven (Affine und Projektive)}
Sea $k$ un cuerpo. 
\begin{defi}
El plano afín $\A^2(k)=\{(a,b)\mid a,b\in k\}$. Los puntos afines son elementos de $\A^2(k)$. Para un polinomio $f$ con coeficientes en $k$ se define $C_f(k)=\{(a,b)\in\A^2(k)\mid f(a,b)=0\}$. 
\end{defi}

Consideremos $\{(C,f)\mid C\subseteq\A^2(k), f\in k[x,y], C_f(k)=C\}$. Consideramos una relación de equivalencia: $(C_1,f_1)\sim (C_2,f_2)\Leftrightarrow C_1=C_2$ y $f_1=\lambda f_2$ para algún $\lambda\in k^*$. 

\begin{nota}
Sea $f\neq 0$ en $k[x,y]$. Entonces $(C_f,f)\sim (C_{\lambda f},\lambda f)$ para cualquier $\lambda\neq 0$ de $k$. Esto tiene que ver con que $\gene{f}=\gene{\lambda f}$.
\end{nota}

El plano proyectivo ya sabemos todos lo que es. ConsIderamos solo $\PP^2(k)=k^3\setminus\{0\}/\sim$. Hay 3 maneras naturales de meter $\A^2(k)$ en $\PP^2(k)$ ($i_1$, $i_2$, $i_3$), que es poniendo un 1 en alguna coordenada, y de hecho $\PP^2(k)$ es unión de estras 3 inclusiones. Normalmente usaremos $i=i_3$. Denotamos $H=i_1(\{(a,0):a\in k\})$ y $\theta=[0:1:0]$. Tenemos la unión disjunta $\PP^2(k)=\A^2(k)\cup H\cup \{\theta\}$. 

\begin{defi}
Sea $d\geq 1$ y $f\in k[x,y,z]$ no nulo. Diremos que $f$ es homogéneo de grado $d$ si todos sus monomios tienen grado $d$. 
\end{defi}

Es claro que si $f$ es homogéneo, entonces $f(a,b,c)=0\Leftrightarrow f(\lambda a,\lambda b,\lambda c)=0$ para todo $\lambda\neq 0$. Podemos definir entonces $C_f(k)$ en $\PP^2(k)$ y la misma relación de antes. 

Si $f\in k[x,y]$, homogeneizar es añadir $z$ en cada monomio tantas veces como haga falta para que salga un polinomio homogéneo en $f^*\in k[x,y,z]$, concretamente, si $d=\max\{n_1+n_2, a_{n_1 n_2}\neq 0\}$, siendo $n_1+n_2$ los grados de los monomios en $x,y$, entonces añadimos $z^{d-n_1-n_2}$ en cada caso. A $C_{f^*}$ se le llama proyectivización de $C_f$. 

\begin{lemma}
Sea $f\in k[x,y]$ no constante. Entonces $i(C_f(k))=C_{f^*}(k)\cap i(\A^2(k))$. 
\end{lemma}

\begin{lemma}\
\begin{enumerate}[(a)]
\item Sean $P_1,P_2\in\PP^2(k)$ puntos distintos. Existe una única recta proyectiava $L$ que pasa por $P_1$ y $P_2$. 
\item Sean $L_1$ y $L_2$ dos rectas proyectivas distintas, entonces se cortan en un único punto. 
\end{enumerate}
\end{lemma}
\begin{proof}\
\begin{enumerate}[(a)]
\item $P_i=[a_i:b_i:c_i]$. Busco $L=L(\alpha,\beta,\gamma)$ tal que 
\[
\begin{pmatrix}
a_1 & b_1 & c_1\\
a_2 & b_2 & c_2
\end{pmatrix}\begin{pmatrix}
\alpha\\
\beta\\
\gamma
\end{pmatrix}=0
\]
El sistema tiene rango 2 porque $P_1\neq P_2$. Supongamos sin pérdida de generalidad que es el el primer cuadrado el que tiene determinante no nulo. Buscamos $\alpha,\beta$ tales que
\begin{equation}\label{syst}
\begin{pmatrix}
a_1 & b_1\\
a_2 & b_2
\end{pmatrix}\begin{pmatrix}
\alpha\\
\beta
\end{pmatrix}=\begin{pmatrix}
-c_1\\
-c_2
\end{pmatrix}
\end{equation}
Este sistema tiene solución única $L(\alpha,\beta,1)$, que es una recta que pasa por $P_1$ y $P_2$. Si $L(\alpha',\beta',\alpha')$ también pasa por $P_1$ y $P_2$, entonces
\[
\begin{cases}
\alpha'a_1+\beta'b_1=-\gamma'c_1\\
\alpha'a_1+\beta'b_2=-\gamma'c_2
\end{cases}
\]
Si $\gamma'\neq 0$, $(\frac{\alpha'}{\gamma'},\frac{\beta'}{\gamma'})$ verifica el sistema \ref{syst}, luego
\[
\alpha=\alpha'/\gamma'; \beta=\beta'/\gamma'; 1=\gamma'/\gamma'
\]
luego $L(\alpha,\beta,1)=L(\alpha',\beta',\gamma')$.

\item Análogo.
\end{enumerate}
\end{proof}

\subsection{Cambio de base}
Sea $K\subseteq E$ una extensión de cuerpos y $f\in K[x,y,z]\subseteq E[x,y]$. Tenemos $C_f/K=(C_f(K),f)/\sim$ y $C_f/E=(C_f(E),f)/\sim$. En general, la inclusión $C_f(K)\subseteq C_f(E)$ conmuta con la inclusión $\A^2(K)\subseteq \A^2(E)$ y $C_f(K)=C_f(E)\cap \A^2(K)$. 

En el plano proyectivo tenemos también estas inclusiones e igualdades de forma natural. 

\begin{defi}[Recta tangente a una curva en un punto]
La recta tangente de $f$ es la componente homogénea de grado 1, $f_1$, cuando no es nula. Si es nula, pues no está difinida, qué le vamos a hacer.
\end{defi}


\begin{defi}[2.2.7]
Sea $K$ cuyo $f(x,y)\in K[x,y]$ no nulo y $P=(a,b)\in \A^2(K)$ con $f(a,b)=0$. Decimos que $P$ es singular si las derivadas parciales en $P$ se anulan simultaniamente. Si $P$ es no singular, se dice regular. Decimos que $C$ es lisa si para alguna clausura algebraica $\overline{K}$ de $K$, todo punto $P\in C_f(K)$ es regular. 
\end{defi}

Las derivadas paraciales de un polinomio homogéneo son polinomios homogéneos, luego podemos hacer las mismas definiciones para el caso proyectivo.

\begin{nota}
$P=(a,b)\in C_f/K$ es singular si y solo si $[a:b:1]$ es singular en $C_{f^*}/K$. Una curva $C_f$ puede ser lisa pero no serlo $C_{f^*}$ (pueden pasar cosas en el infinito). Por ejemplo $f(x,y)=x^3+yx^2-y$.
\end{nota}
\begin{defi}
Sea $C_f/K$ una curva proyectiva y $P=[a:b:c]\in C_f(K)$ no singular. Se define la recta tangente a $C_f$ en $P$ como $L(\parcial{f}{x}(a,b,c),\parcial{f}{y}(a,b,c),\parcial{f}{z}(a,b,c))$. 
\end{defi}
\begin{defi}
Sea $k$ un cuerpo y $f\in k[x]$. Se define el orden de $f(x)=\sum_{i=0}^na_ix^i$ en $x=0$ como $ord(f)=\min\{i:a_i\neq 0\}$. 
\end{defi}

\begin{lemma}
$f,g\in k[x]$
\begin{enumerate}
\item $ord(f\cdot g)=ord(f)+ord(g)$.
\item $ord(f+g)\geq\min\{ord(f),ord(g)\}$, dándose la igualdad si $ord(f)\neq ord(g)$.
\item $ord(f\circ g)=ord(f)ord(g)$. 
\end{enumerate}
\end{lemma}
La prueba es muy sencilla.

\begin{nota}
$ord$ es una valoración discreta sobre $k[x]$. 
\end{nota}

\begin{defi}
Si $\psi(x)=k(x)$ con $\psi(x)=\frac{f(x)}{g(x)}$ ($f,g\in k[x], g\neq 0$). Se define $ord\phi=ord f-ord g$.
\end{defi}
Se comprueba fácilmente que el orden de una función racional no depende de la expresión como fracción. Además se verifican las propiedades para polinomios.

\begin{defi}[Multiplicidad de intersección entre una curva plana sobre una recta en un punto]
Sea $C/k=\{f(x,y,z)=0\}$ una curva, $L=L(\alpha,\beta,\gamma)/k$ una recta y $P=[a:b:c]$. Si $P\notin L$, se define $m(C,L,P)=0$. Si $P\in L$, escogemos $P'=[a':b':c']\in L$ distinto de $P$. Consideramos $\psi(t)=f(a+ta',b+tb',c+tc')$. Se define entonces $m(C,L,P)=ord\psi(t)$ (con respecto a la variable $t$, o sea en $t=0$ siguiendo la definición de orden). 
\end{defi}

Se comprueba que esta definición no depende de ninguna de las elecciones.

\begin{lemma}
Sea $C_f$ una curva plana, $L$ una recta y $P\in L$. Entonces 
\begin{enumerate}
\item $m(C_f,L,P)\geq 1\Leftrightarrow P\in C_f$. 
\item $P\in C_f$ y $L$ es la tangente a $C_f$ en $P$ implican que $m(C_f,L,P)\geq 2$.
\end{enumerate}
\end{lemma}

\begin{lemma}
Sea $A\in GL_3(k)$ y $C_f$ una curva. $P\in C_{f_A}$. $P$ es singular en $C_{f_A}$ si y solo si $\varphi_A(P)$ es singular en $C_f$, donde el subíndice $A$ representa la multiplicación de $A$ por los puntos correspondientes.   
\end{lemma}



\section{Curvas elípticas}

Muchas veces para definirlas se usa el género\footnote{\url{https://www.encyclopediaofmath.org/index.php/Genus_of_a_curve}} de una curva. 

\begin{defi}
Una curva elíptica es una curva projectiva no singular (lisa) $C_g(F)$ donde $g$ es de la forma
\[
g(X,Y,Z)=Y^2Z-a_1XYZ+a_3YZ^2-X^3-a_2X^2Z-a_4XZ^2-a_6Z^3.
\]
La ecuación que surge de igualar a 0 este polinomio se llama ecuación de Weierstrass. 
\end{defi}


\begin{lemma}
Si $f(x,y,z)=0$ es una ecuación de Weierstrass, entonces si
\[
A=\begin{pmatrix}
u^2 & 0 & r\\
u^2s & u^3 & t\\
0 & 0 & 1
\end{pmatrix}\in GL_3(k)
\]
con $u,s,t\in k$, se tiene que $f_A(x,y,z)=0$ también es de Weierstrass.
\end{lemma}
La prueba es directa. 

El único punto no afín que hay en una curva elíptica es $O=[0:1:0]$. Este punto es no singular porque $\parcial{g}{Z}(0,1,0)=1$. Además la tangente en $O$ a $C_f$ es $L(0,0,1)$. Se tiene $m(C_f,L,O)=ord\psi(t)$ con $\psi(t)=f(0+t,1+0t,0+0t)=f(t,1,0)$. Sea $P'=[1:0:0]\in L(0,0,1)$. Entonces sale que $m(C_f,L,O)=3$, por lo que tiene multiplicidad de tangente. 

\begin{defi}
Se define el discriminante de una curva elíptica como 
\[
\Delta=-b_2^2b_8-8b_4^3-27b_6^2+9b_2b_4b_6
\]
donde
\begin{align*}
&b_2=a_1^2+4a_2\\
&b_4=2a_4+a_1a_3\\
&b_6=a_3^2+4a_6\\
&b_8=a_1^2a_6+4a_2a_6-a_1a_3a_4+a_2a_3^2-a_4^2
\end{align*}

Se define además el $j$-invariante como 
\[
j=\frac{(b_2^2-24b_4)^3}{\Delta}=\frac{c_4^3}{\Delta}.
\]
\end{defi}



Proposition 2.3.2, aunque creo que la hemso puesto algo distinta:

\begin{lemma}
Si la característica de $k$ no es 2, sea $g(x,y,z)=y^2z-x^3-\frac{1}{4}b_2x^2z-\frac{1}{2}b_4xz^2-\frac{1}{4}b_6z^3$. Entonces $\varphi:\PP^2\to \PP^2$ dada por $[a:b:c]\mapsto [a:b+\frac{a_1}{2}+\frac{a_3}{2}c:c]$ verifica que $C_f=\varphi(C_f)$.
\end{lemma}

\begin{lemma}[en la página 26]
Si la característica de $k$ no es 2 ni 3, y $g(x,y,z)=y^2z-x^3+27c_4xz^2+54c_6z^3$. Entonces el cambio de variables $\varphi:[a:b:c]\mapsto [36a+2b_2c:216b:c]$ lleva $f$ en $g$. 
\end{lemma}

\begin{prop}[2.3.3]
Sea $f(x,y,z)=0$ ecuación de Weierstrass. Entonces $C_f$ es lisa si y solo si $\Delta=0$.
\end{prop}

Esto nos da como conclusión que $C_f$ es elíptica si y solo si $\Delta\neq 0$.

\begin{defi}[Definition 2.3.10 usw]
Existe un asuma en el conjunto de puntos de una curva elíptica. Esta suma forma un grupo abeliano con elemento neutro el punto $O$. La conmutatividad es evidente porque la recta que uno P con Q es la misma que une $Q$ con $P$. El inverso de $P$ es el punto intermedio en el proceso de suma $P*O$. 
\end{defi}

Para dibujar la recta que corta al infinito se hace una paralela al eje $Y$.



\begin{prop}[Satz 2.3.13 und 2.3.14]
Se puede dar una expresión explícita de la suma en términos de las coordenadas afines de los puntos. 
\end{prop}

%Subgrupo de torsión (CREO página 61 en adelante) \url{https://en.wikipedia.org/wiki/Tate_module}


\begin{lemma}[Satz 2.3.8]
$L=L(\alpha,\beta,\gamma)$, $E$ una curva elíptica. Entonces $\sum_{P\in\PP^2(k)}m(E,L,P)\in\{0,1,3\}$.
\end{lemma}
\begin{proof}[Aclaraciones de la prueba]
Si la raíz tiene multiplicidad 3, solo hay un punto en la intersección, ya que es la única raíz del polinomio y luego el punto de la recta queda determinado por los coeficientes. 


Si 2 raíces están en el cuerpo, la otra también porque puedes hacer la división. Pero puede pasar que tenga solo una raíz en el cuerpo y las otras dos en la clausura algebraica. Entonces la suma da 1. Si no hay ninguna raíz en el cuerpo, la suma sale 0. 
\end{proof}

\begin{coro}[Korollar 2.3.9]
\begin{enumerate}
\item como tanto $P_1$ como $P_2$ están en la intersección de $L$ con $E$, al menos $\sum_{P\in\PP^2}m(E,L,P)\geq 2$, por lo que es necesariamente 3. Así que hay algún punto más (posiblemente uno de ellos que tenga multiplicidad 2). 
\item Esto sigue siendo si $P_1=P_2$ con multiplicidad al menos 2, es decir, considerando la recta tangente. 
\end{enumerate}
\end{coro}


\section{Puntos de torsión}

%Creo que tiene que ver con lo que viene a partir de la página 61.

Sea $E/k$ una curva elíptica y $n\in\Z_{\geq 1}$. Definimos la aplicación
\[
n:E\to E
\]
\[P\mapsto nP=P+\cdots+P\]
Sea $\overline{k}$ una clausura algebraica de $k$. Consideramos $E[n]=\{P\in E(\overline{k})\mid nP=0\}$, al que llamamos conjunto de puntos de $n$-torsión. Este conjunto es un subgrupo de $E$. Se puede probar por inducción, ya que por definición $1(P+Q)=P+Q=1P+1Q$ y $(n+1)(P+Q)=n(P+Q)+1(P+Q)$. Esto tiene como consecuencia que $E[n]=\ker n$. 

\begin{prop}
Sea $m\in\Z_{\geq 1}$. \begin{enumerate}
\item Si $char(k)\not| m$, entonces $E[m]\cong \Z/n\Z\times \Z/n\Z$. 
\item Si $p=char(k)$
\[
E[p^e]=\begin{cases}
\{0\} & o\\
\Z/p^e\Z & 
\end{cases}
\]
\end{enumerate} 
\end{prop}


\begin{prop}
Si $k$ es un cuerpo finito, entonces o bien $E(k)\cong \Z/n\Z$ o bien $E(k)\cong\Z/ n_1\Z\times \Z/n_2\Z$.
\end{prop}
\begin{dem}
Tenemos que $E(k)$ es finito por estar contenido en $\PP^2(k)$, así que por el teorema de clasificación de grupos abelianos $E(k)\cong\Z/n_1\Z\times\cdots\times\Z/n_r\Z$ con $n_i|n_{i+1}$. Tenemos que probar que $r\leq 2$. Consideremos $A=\{P\in E(k)\mid n_1P=0\}\subseteq E[n_1]$. Se tiene que $\# A\leq \# E[n_1]\leq n_1^2$. Por otro lado, se tiene por magia infusa que $A\cong\Z/n_1/Z\times\cdots\Z/n_1\Z$ ($r$ veces), luego $\# A=n_1^r$, de modo que $r\leq 2$. 
\end{dem}

\section{Diffie-Hellman en curvas elípticas}
$A$ quiere ponerse de acuerdo con $B$ en una clave secreta. Entonces:
\begin{enumerate}
\item $A$ y $B$ fijan una curva elíptica $E/k$ sobre un cuerpo finito $k$. 
\item Sea $P\in E(k)$ fijado con orden grande. $A$ escoge en secreto $n_A$ y calcula $n_AP=O_A$, que se hace público. Analógamente, $B$ calcula $n_BP=O_B$. 
\item $A$ calcula $n_AO_B=n_An_BP$ y $B$ calcula $n_BO_A=n_Bn_AP$. El secreto común es $n_An_BP$. 
\end{enumerate}


\end{document}
