	\documentclass[twoside]{article}
\usepackage{../../estilo-ejercicios}
%\renewcommand{\baselinestretch}{1,3}
%--------------------------------------------------------
\begin{document}

\title{Ejercicios de Criptografía}
\author{Javier Aguilar Martín}
\maketitle


\begin{ejercicio}{1}
\begin{enumerate}
\item Sea $k ∈ \Z_{≥0}$, y supongamos que $2^k + 1$ es un número primo. Demuestra que
existe un número natural $l$ tal que $k = 2^l$

(Ayuda: Para todo par de números naturales
$m$, $n$ con $m$ impar, se cumple la siguiente fórmula: $a^{mn}+1 = (a^n+1)(a^{n(m−1)}−a^{n(m−2)}+\cdots + (−1)^{i+1}a^{n(m − i)} +\cdots + 1)$).

Para cada número natural $n$, diremos que $F_n := 2^{2^n}+ 1$ es un primo de Fermat si $F_n$ es
primo.

\item Sea $n ∈ \Z_{≥0}$ tal que $F_n$ es un primo de Fermat. Demuestra que si $a ∈ \Z$ verifica que
$F_n \not| a$ y que $a$ no es un residuo cuadrático módulo $F_n$, entonces el elemento $a + (Fn)$
genera el grupo multiplicativo $(\Z/F_n\Z)^×$.

\item Sea $p ≥ 5$ un número primo . Demuestra que se verifica la siguiente fórmula:
\[
\left(\frac{3}{p}\right)
=\begin{cases}
+1 & \text{if }p ≡ ±1 (\mod 12)\\
−1 & \text{if }p ≡ ±5 (\mod 12).
\end{cases}
\]
Ayuda: Usa la ley de reciprocidad cuadrática de Gauss.

\item Sea $n ∈ \Z_{≥1}$ y consideremos el número $F_n = 2^{2^n}+1$. Prueba que las siguientes condiciones
son equivalentes:
\begin{enumerate}[(i)]
\item $F_n$ es primo.
\item $3^{(F_n−1)/2} ≡ −1 (\mod F_n)$.
\end{enumerate}
Ayuda para la implicación $(ii) → (i)$: Usando el Teorema Chino del Resto, demuestra
que $F_n$ es primo si y sólo si el cardinal del grupo multiplicativo $(\Z/F_n\Z)^×$ es $F_n − 1$.
Calcula el orden del grupo generado por $3 + (F_n)$ en el grupo $(\Z/F_n\Z)^×$
\end{enumerate}
\end{ejercicio}
\begin{solucion}
\begin{enumerate}
\item Si $k$ no es de la forma $2^l$ para ningún $l$, entonces $k=rs$ con $r>2$ impar y $s\in\N$. Por la ayuda tenemos $2^s+1$ divide a $2^k+1$. Como $1<2^s+1<2^k+1$, se tiene que $2^k+1$ no es primo y se tiene el resultado por contrarrecíproco. 




%\url{https://en.wikipedia.org/wiki/Fermat_number#Other_theorems_about_Fermat_numbers}
\item Por las hipótesis sobre $a$ tenemos que $\left(\frac{a}{F_n}\right)=-1$, luego $a^{2^{2^n-1}}=a^{\frac{F_n-1}{2}}\equiv -1\mod F_n$. Sabemos que $(\Z/F_n\Z)^\times$ tiene orden $2^{2^n}$ y $2^{2^n-1}$ generadores, que son los que no son cuadrados, ya que un elemento de un grupo cíclico de orden $r$ es generador si y solo si no es una potencia $q$-ésima para ningún factor primo $q$ de $r$, luego los generadores son justamente los no cuadrados. 

\item Usaré la propiedad siguiente que parece que todo el mundo sabe menos yo: si $p\equiv q\equiv 3\mod 4$, entonces $(p/q)=-(q/p)$ y se da la igualdad sin signo en otro caso. También que si $a\equiv b\mod p$, entonces $(a/p)=(b/p)$. 

Obsérvese que como $p\geq 5$ (es decir, $p\neq 2,3$), tenemos que $p\equiv \pm 1\mod 12$ o $p\equiv \pm 5\mod 12$. Además el símbolo de Legendre en este caso no puede ser 0. 
\begin{itemize}
\item Si $p=1+12k$, entonces $p\equiv 1\mod 3,\mod 4$, luego $(3/p)=(p/3)=(1/3)=1$.
\item Si $p=-1+12k$, entonces $p\equiv -1\mod 3,\mod 3$ ($p\equiv 2\mod 3$ y $p\equiv 3\mod 4$), luego $(3/p)=-(p/3)=-(2/3)=1$.
\item Si $p=5+12k$, entonces $p\equiv 2\mod 3$ y $p\equiv 1\mod 4$, luego $(3/p)=(p/3)=(2/3)=-1$.
\item Si $p=5+12k$, entonces $p\equiv 1\mod 3$ y $p\equiv 3\mod 4$, luego $(3/p)=-(p/3)=-(1/3)=-1$. 
\end{itemize}
%\url{http://mathonline.wikidot.com/legendre-symbol-rules-for-3-p-and-6-p}, \url{}

\item $(i)\Rightarrow (ii)$ es consecuencia de la proposición de Euler. Efectivamente, tenemos que $3^{(F_n−1)/2} ≡ \left(\frac{3}{F_n}\right) (\mod F_n)$. Es fácil ver que $2^{2^n}\equiv 1\mod 3$, con lo que $F_n\equiv 2\mod 3$, así que $(F_n/3)=-1$. Como $F_n\equiv 1\mod 4$, concluimos que $(3/F_n)=-1$ y aplicando la proposición se deduce el resultado. 


Para $(ii)\Rightarrow (i)$, vemos primero que $p$ es primo si y solo si $\#(\Z/p\Z)^\times=p-1$. La implicación directa es consecuencia del teorema de Lagrange, para la inversa, por el teorema Chino del resto tenemos que si $p=ab$ con $ab$ primos, entonces $\#(\Z/p\Z)^\times=(a-1)(b-1)\neq p-1$ (el caso general de una descomposición en primos es similar). 

Ahora, supongamos que $3^{(F_n−1)/2} ≡ −1 (\mod F_n)$, lo cual implica que $3^{F_n−1} ≡ 1 (\mod F_n)$. Esto significa que el orde de 3 divide a $F_n-1$, que es una potencia de 2, pero no divide a $\frac{F_n-1}{2}$, luego es exactamente $F_n-1$. En particular, hay al menos $F_n-1$ coprimos con $F_n$, lo cual solo puede pasar si $F_n$ es primo por lo probado al principio del apartado. 

%\url{https://en.wikipedia.org/wiki/P%C3%A9pin%27s_test}



 
\end{enumerate}
\end{solucion}

\newpage

\begin{ejercicio}{2}
Para cada uno de los valores de N que se indican, calcula la factorización de $N$
utilizando el método $p − 1$ de Pollard. 
\begin{enumerate}[1)]
\item $N = 63373$;
\item $N = 75203$;
\item $N = 75359$;
\item $N = 98537$.
\end{enumerate}
\end{ejercicio}
\begin{solucion}
\begin{enumerate}
\item  $N = 63373$, $B=7$, $M=420$, $a=2$. $\gcd(2^{420}-1, N)=127$ (primo), $N/127=499$ (primo). 
\item  $N = 75203$, $B=13$, $M=60060$, $a=2$. $\gcd(2^M-1,N)=157$ (primo), $N/157=479$ (primo). 
\item  $N = 75359$, $B=7$, $M=420$, $a=2$. $\gcd(2^M-1,N)=421$ (primo), $N/421=179$ (primo).
\item  $N = 98537$, $B=7$, $M=420$, $a=2$. $\gcd(2^M-1,N)=211$ (primo), $N/211=467$ (primo). 
\end{enumerate}
\end{solucion}

\newpage

\begin{ejercicio}{3}
Para cada uno de los valores de $N$ que se indican, calcula la factorización de $N$
utilizando el método $ρ$ de Pollard.
\begin{enumerate}[1)]
\item $N = 7081$;
\item $N = 9047$.
\end{enumerate}
\end{ejercicio}
\begin{solucion}
He traducido a Sage el algoritmo con el siguiente código

\begin{verbatim}
def rho(N,s):
    i=0
    y=s
    d=1
    ZN.<x>=PolynomialRing(Integers(N))
    g=x^2+1
    while d==1:
        i=i+1
        s=g(s)
        y=g(g(y))
        d=gcd(s-y, N)
    if d==N:
        return 'Failure'
    else:
        dd=Integer(d)
        return dd, i, N/dd, (N//dd).is_prime()
\end{verbatim}
Como en este ejercicio ambos casos da que los números son producto de solo dos primos es suficiente con el que acabo de dar, pero para un caso algo más general se puede usar el que añado al final. En ambos casos he usado la semilla $s=2$. 
\begin{enumerate}
\item $7081=97\cdot 73$.
\item $9047=83\cdot 109$.
\end{enumerate}
\begin{verbatim}
def rho2(N,s):
    n=N
    r=rho(N,s)[0]
    L=[r]
    while L[-1]=='Failure':
        l=L[-1]
        n=n/dd
        t=rho(n,s)[0]
        L=L.append(t)
    return L
\end{verbatim}
\end{solucion}
\newpage
\begin{ejercicio}{4}
Para cada uno de los valores de $p$ y $g$ que se indican, consideramos el grupo
multiplicativo $\F^×_p$. Calcula, utilizando el método $ρ$ de Pollard, el logaritmo discreto de $h =131 ∈ \F^×_p$.
\begin{enumerate}[1)]
\item $p = 491$, $g = 2$;
\item $p = 499$, $g = 7$.
\end{enumerate}
\end{ejercicio}
\begin{solucion}
\begin{enumerate}
\item
\item
\end{enumerate}
\end{solucion}
\newpage
\begin{ejercicio}{5}
Para cada uno de los valores de $p$, $g$, $B$ que se indican, consideramos el grupo
multiplicativo $\F^×_p$. Calcula: 
(1) los logaritmos discretos de los primos menores o iguales que
$B$ en base $g$ (basta utilizar los valores de $g^i$
, para $i ≤ 50$, en todos los casos); (2) El logaritmo
discreto del elemento $h = 131$ ∈ $\F^×_p$, utilizando el método de ``index-calculus''.
\begin{enumerate}[1)] 
\item $p = 223$, $g = 3$, $B = 19$;
\item $p = 227$, $g = 2$, $B = 11$.
\end{enumerate}


\end{ejercicio}
\begin{solucion}
 
\end{solucion}
\newpage 

Consideremos las siguientes curvas proyectivas planas, definidas sobre el cuerpo $\F_3$ de
tres elementos:
\begin{enumerate}[1)]
\item $E_1 :$ $Y^2Z = X^3 + 2X^2Z + XZ^2 + Z^3$;
\item $E_2 :$ $Y^2Z + XYZ = X^3 + 2X^2Z + XZ^2 + Z^3$;
\item $E_3 :$ $Y^2Z + XYZ = X^3 + 2X^2Z + XZ^2$;
\item $E_4 :$ $Y^2Z + YZ^2 = X^3 + 2X^2Z + XZ^2$.
\end{enumerate}
Para cada uno de los casos $i = 1, \dots , 4$, se proponen los siguientes ejercicios:


\begin{ejercicio}{6}
Demuestra que la curva $E_i$ es lisa.
\end{ejercicio}
\begin{solucion}

\end{solucion}

\newpage

\begin{ejercicio}{7}
Sean $L_1 = L(0, 0, 1)$ la recta proyectiva definida por la ecuación $Z = 0$ y
$L_2 = L(1, 0, 1)$ la recta proyectiva definida por la ecuación $X + Z = 0$. Para cada punto
proyectivo $P ∈ \PP^2(\F_3)$ y cada índice $j = 1, 2$, denotamos por $m(E_i, L_j , P)$ a la multiplicidad
de intersección de $E_i$ y $L_j$ en $P$. Calcula la suma
\[
\sum_{P ∈\PP^2(\F_3)}m(E_i, L_j , P)
\]
para $j = 1, 2$.
\end{ejercicio}
\begin{solucion}

\end{solucion}

\newpage

\begin{ejercicio}{8}
 Calcula el conjunto $E_i(\F_3)$ de puntos de la curva elíptica con coeficientes en $\F_3$.
Sabemos que este conjunto es un grupo abeliano. Calcula la tabla de multiplicar de este grupo
\end{ejercicio}
\begin{solucion}

\end{solucion}



\end{document}
