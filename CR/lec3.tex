\documentclass[CR.tex]{subfiles}

\begin{document}


%\hyphenation{equi-va-len-cia}\hyphenation{pro-pie-dad}\hyphenation{res-pec-ti-va-men-te}\hyphenation{sub-es-pa-cio}

\chapter{Algoritmos en curvas elípticas}
\section{Diffie-Hellman en curvas elípticas}
$A$ quiere ponerse de acuerdo con $B$ en una clave secreta. Entonces:
\begin{enumerate}
\item $A$ y $B$ fijan una curva elíptica $E/k$ sobre un cuerpo finito $k$. 
\item Sea $P\in E(k)$ fijado con orden grande. $A$ escoge en secreto $n_A$ y calcula $n_AP=O_A$, que se hace público. Analógamente, $B$ calcula $n_BP=O_B$. 
\item $A$ calcula $n_AO_B=n_An_BP$ y $B$ calcula $n_BO_A=n_Bn_AP$. El secreto común es $n_An_BP$. 
\end{enumerate}

\section{El problema del logaritmo discreto en curvas elítpticas (sección 4, página 75)}

Dado $P\in E(k)$ y dado $Q\in\gene{P}$, encontrar $n$ con $nP=Q$. 

\subsection{Problemas}
\begin{itemize}
\item Construir curvas elípticas con cardinal controlado.
\item Calcular el orden de $P$ (que coincide con el cardinal del subgrupo que genera).
\item Cómo evitar ataques al problema del logaritmo discreto. 
\end{itemize}

\section{Curvas super singulares (3.4)}
\begin{defi}[3.4.1]
$E(F)$ es supersingular si $p=char(F)$ divide a $q+1-\#E(F)$ (No confundir con singular, porque las curvas elípticas no pueden ser singulares). 
\end{defi}

\begin{ej}
La curva sobre $\F_2$ dada en afín por $y^2+y=x^3+x+1$ tiene solo $O$, con lo que el número sale 2, que es divisible por 2. 
\end{ej}

\begin{prop}[3.4.3]
Criterio para determinar si una curva es supersingular cuando $p\geq 3$. 
\end{prop}
\begin{prop}[3.4.4]
 cuando $p=2$. 
\end{prop}

\begin{defi}
Sea $n\in\Z$ y $q=p^r$. El grado de inmersión de $n$ en $\F_q$ es el mínimo $l$ tal que $\mu_n(\overline{\F}_q)\subseteq \F_{q^l}$. 
\end{defi}

\section{Algoritmo MOV (4.2)}

\begin{teorema}[Formula de Weil, 3.2.1]
$\#E(\F_q)-q-1\leq q$ y $|\#E(\F_q)-q-1|\leq 2\sqrt{q}$
\end{teorema}

\begin{prop}[4.2.2]

\end{prop}

WEIL PAIRING \url{https://en.wikipedia.org/wiki/Weil_pairing}(pag 83)

Dados $E/\F_q$, $P$ de orden $n$ y $Q$.
\begin{enumerate}
\item Calcular $\#(E)$, $l$, $d$ (en las fotos $l\rightarrow d$, $d\rightarrow f$).
\item Elgir $R'\in E$ cualquiera y $R=\frac{d}{n}R'$.
\item Calcular $a=e_n(P,R)$ y $b=e_n(Q,R)$
\item Resolver el logaritmo discreto $b=a^k$ en $\F_{q^l}^*$
\end{enumerate}
\begin{nota}
Para que sea seguro, hay que comprobar que para todo $l$ con PLD soluble se tiene que $n$ no divide a $q^l-1$.
\end{nota}

\begin{lemma}
Son equivalentes:
\begin{enumerate}
\item $l$ es el grado de inmersión de $n$ en $\F_q$. 
\item $l$ es el mínimo exponente con $n|q^l-1$. 
\end{enumerate}
\end{lemma}
\begin{proof}
$1\Rightarrow 2$. Si $\mu_n(\F_q)\subseteq \F_{q^l}^*$, por Lagrange $n|q^l-1$ y si para algún $l'<l$ se verifica esta divisibilidad, como $\F_{q^l}$ es cíclico, dado $g$ un elemento primitivo, tendríamos que $o(g^{\frac{q^{l'}-1}{n}})$, con lo que el grupo que genera sería justamente $\mu_n(\overline{\F}_q)$, lo cual es una contradicción. 

$2\Rightarrow 1$. Sea $\gene{g}=\F_{q^l}$, de modo que $\mu_n=\gene{g^{\frac{q^{l'}-1}{n}}}$, luego $n|q^l-1$ y $l$ es mínimo porque esta divisibilidad para $l'$ implica que $\mu_n\subseteq \F_{q^{l'}}$
\end{proof}


Así que en el algoritmo tenemos otro modo de calcular el grado de inmersión y elegir $R$ de modo que $a$ sea una raíz primitiva. 
\section{Algoritmo de factorización de Lenstra}
\url{https://en.wikipedia.org/wiki/Lenstra_elliptic-curve_factorization} 

Para factorizar $n\in\Z$ consideraremos la curva elíptica $E=E(\Z/n\Z)=\{[x:y:z]\in\PP^2(\Z/n\Z)\mid y^2z=x^3+axz^2+bz^3$. El algoritmo usará que cuando $\Z/n\Z$ la adición no da una operación de grupo.
\begin{enumerate}
\item Escoger $x,y,a\in\Z/n\Z$ con $a\neq 0$. 
\item Calcular $b=y^2-x^3-ax$. La curva elíptica $E$ está dada entonces por la ecuación de Weierstrass $Y^2=X^3+aX+b$, que en coordenadas proyectivas se traduce en $ZY^2=X^3+aZ^2X+bZ^3$, la cual contiene el punto $[x:y:1]$. 
\item Elegir una cota superior $B\in\Z$. Se encontrarán factores primos $p$ si $|E(\Z/p\Z)|$ tiene solamente factores menores que $B$.
\item Calcular $k=mcm(1,\dots, B)$. Calcular $kP$. Si $n$ es primo, entonces el resultado es $[0:1:0]$, pero si no es primo se puede encontrar un factor primo tal como sale en el enlace.
\item Si no se encuentra, volver a 2. 

\end{enumerate}



\end{document}

