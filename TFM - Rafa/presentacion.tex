\documentclass{beamer}

\definecolor{kugreen}{RGB}{46,83,84}
\definecolor{kugreenlys}{RGB}{86,157,160}
\definecolor{kugreenlyslys}{RGB}{165,165,165}
\definecolor{kugreenlyslyslys}{RGB}{242,242,242}
\definecolor{kugreenclaro}{RGB}{192, 213, 184}
\definecolor{ocre}{RGB}{244, 236, 187}
\definecolor{ocreclaro}{RGB}{251, 249, 217}
\mode<presentation>
{
  \usetheme{Berlin}      % or try Darmstadt, Madrid, Warsaw, ...
  \usecolortheme[named=kugreen]{structure} % or try albatross, beaver, crane, ...
  \usefonttheme{structureitalicserif}  % or try serif, structurebold, ...
\setbeamertemplate{navigation symbols}{}
  \setbeamertemplate{caption}[numbered]
} 
\addtobeamertemplate{navigation symbols}{}{%
    \usebeamerfont{footline}%
    \usebeamercolor[fg]{footline}%
    \hspace{1em}%
    \insertframenumber/\inserttotalframenumber
}
\setbeamercolor{footline}{fg=black}
\setbeamerfont{footline}{series=\bfseries}

\usepackage[english]{babel}
\usepackage[utf8x]{inputenc}

\usepackage[]{graphicx}
\usepackage[]{color}
\usepackage{geometry}
\usepackage{multimedia}




\newtheorem{defi}{Definici\'on}[section]
\newtheorem{ej}{Ejemplo}[section]
\newtheorem{ejs}{Ejemplos}[section]
\newtheorem{prop}{Proposici\'on}[section]
\newtheorem{nota}{Nota}[section]
\newtheorem{notac}{Notación}[section]
\newtheorem{rem}{Observaci\'on}[section]
\newtheorem{thm}{Teorema}[section]
\newtheorem{cor}{Corolario}[section]
\newtheorem{lem}{Lema}[section]
\newtheorem*{dem}{Demostración}

\title[El problema paramétrico del emparejamiento en grafos y problema de emparejamiento con dos objetivos]{\textbf{\textup{El problema paramétrico del emparejamiento en grafos y problema de emparejamiento con dos objetivos}}}
\author{Rafael González López\\ Supervisado por: Justo Puerto Albondoz\\}
\institute{Universidad de Sevilla}
\date{\textit{Departamento de Estadística e investigación Operativa}}

\begin{document}

%%%%%%%%%%%%%%%%%%%%%%%%%%%%%%%%%%%%%%%%%%%%%%%%%%%%%%%%%%%%%%%%%%%%%

\begin{frame}
  \titlepage
\end{frame}

%%%%%%%%%%%%%%%%%%%%%%%%%%%%%%%%%%%%%%%%%%%%%%%%%%%%%%%%%%%%%%%%%%%%%%

\begin{frame}{Índice}
 \tableofcontents
\end{frame}

%%%%%%%%%%%%%%%%%%%%%%%%%%%%%%%%%%%%%%%%%%%%%%%%%%%%%%%%%%%%%%%%%%%%%%


\section{Introducción}

\begin{frame}{¿Qué es un emparejamiento en un grafo?}
\begin{figure}[h!]
\includegraphics[scale=0.45]{opt}
\end{figure}
\end{frame}

%%%%%%%%%%%%%%%%%%%%%%%%%%%%%%%%%%%%%%%%%%%%%%%%%%%%%%%%%%%%%%%%%%%%%%
\subsection{Formulación del problema}
\begin{frame}{Formulación del problema}
\begin{figure}[h!]
\includegraphics[scale=0.45]{opt2}
\end{figure}
\end{frame}

\begin{frame}{Añadiendo costes}
\begin{itemize}
\item Dada una arista $(u,v)$ podemos considerar un coste $c_{uv}$ asociado. 
\item Estos costes pueden ser números enteros, reales, reales positivos, etc.
\item Sea $M$ un conjunto de aristas, podemos definir
\end{itemize}
$$
c(M) = \sum_{(u,v)\in M} c_{uv}
$$

\end{frame}
\begin{frame}{Problema de optimización}
Sea $G=(V,E)$ un grafo. Entonces podemos considerar:
\begin{align*}
\max_{x} &\; \sum_{(u,v)\in E} x_{uv}c_{uv}  \nonumber\\ 
s.a.\;  &  Ax\leq 1_n \\
& x\in\{0,1\}^n\nonumber
\end{align*}
donde $c$ es el vector de pesos, $A$ es la matriz de incidencia del grafo, $|E|=n$ y la variable $x_{uv}=1$ si la arista $(u,v)$ está en el emparejamiento y $0$ en caso contrario.
\end{frame}
\begin{frame}{Definiciones previas}
\begin{defi}
Sea $G=(V,E)$ y $W\subset V$. Entonces 
\begin{gather*}
\delta(W) = \{(u,v)\in E \mid \text{Exactamente $u \in W$ o $w \in W$} \}\\
\gamma(W) = \{(u,v)\in E \mid u,v \in W\}
\end{gather*}
\end{defi}
\end{frame}
\begin{frame}
\begin{thm}
Sea $G=(V,E)$ un grafo. Sea $B$ el conjunto
$$
B = \{S\subset V \mid |S| \text{ es impar},\;|S|\geq 3\}
$$
Entonces, la envolvente convexa del politopo del matching viene dada por
\begin{align*}
\sum_{(u,v)\in\delta(u)} x_{uv} &\leq 1, \quad \forall u\in V\\
x_{uv} &\geq 0\\
\sum_{(u,v)\in \gamma(S)} x_{uv}& \leq \frac{1}{2}(|S|-1)\quad \forall S \in B	
\end{align*}
\end{thm}
\end{frame}
\begin{frame}{Emparejamiento perfecto de mínimo peso}
\begin{align*}
\min_x & \sum_{(u,v) \in E}x_{uv}c_{uv}\\
s.a.&\;\sum_{(u,v)\in\delta(u)} x_{uv} = 1, \quad \forall u \in V\\
&\sum_{(u,v)\in \gamma(S)} x_{uv} \leq \frac{1}{2}(|S|-1)\quad \forall S \in B	\\
&x_{uv} \geq 0 \qquad \forall(u,v)\in E
\end{align*}
\end{frame}
\begin{frame}{Problema paramétrico}\begin{align*}
\min_x & \sum_{(u,v) \in E}x_{uv} (c^1_{uv} + \lambda c^2_{uv})\\
s.a.&\;\sum_{(u,v)\in\delta(u)} x_{uv} \leq 1, \quad \forall u \in V\\
&\sum_{(u,v)\in \gamma(S)} x_{uv} \leq \frac{1}{2}(|S|-1)\quad \forall S \in B	\\
&x_{uv} \geq 0 \qquad \forall(u,v)\in E
\end{align*}
\end{frame}
\begin{frame}{Problema biobjetivo}
\begin{align*}
\min_x & \left(\sum_{(u,v) \in E}x_{uv}c^1_{uv},\sum_{(u,v) \in E}x_{uv} c^2_{uv}\right)\\
s.a.&\;\sum_{(u,v)\in\delta(u)} x_{uv} \leq 1, \quad \forall u \in V\\
&\sum_{(u,v)\in \gamma(S)} x_{uv} \leq \frac{1}{2}(|S|-1)\quad \forall S \in B	\\
&x_{uv} \geq 0 \qquad \forall(u,v)\in E
\end{align*}
\end{frame}
\subsection{Grötschel-Holland}
\begin{frame}{Problemas de la formulación de Edmonds}
\begin{itemize}
\item La formulación de Edmonds tiene un número exponencial de restricciones en el número de vértices.
\item Para grafos incluso medianos resulta infactible computacionalmente.
\item Grötschel y Holland proponen un algoritmo para lidiar con este problema y utilizar Programación Lineal.
\end{itemize}
\end{frame}

\begin{frame}{Algoritmo de Grötschel-Holland}
\begin{enumerate}
\item[1] Determinamos $E'\subset E$ de ejes canditados. Para cada nodo escogemos $NN$ ejes de menor coste. 
\item[2] Utilizamos un algoritmo voraz para encontrar un matching perfecto inicial. En caso de no haberlo, añadimos aristas para garantizar que exista.
\item[3] Consideramos el problema
\begin{align*}
\min_x & \sum_{(u,v) \in E'}x_{uv}c_{uv}\\
s.a.&\;\sum_{(u,v)\in\delta(u)} x_{uv} = 1, \quad \forall u \in V\\
&x_{uv} \geq 0 \qquad \forall(u,v)\in E'
\end{align*}
\end{enumerate}
\end{frame}

\begin{frame}{Algoritmo de Grötschel-Holland}
\begin{enumerate}
\item[4] Resolvemos el problema considerado. 
\item [5] Sea $x^*$ la solución obtenida.
\item[6] Si $x^*$ es entero, induce un matching. En este caso comprobamos la optimalidad. En otro caso, añadimos aristas.
\item[7] Si $x^*$ no es entero, planteamos dos heurísticas para encontrar planos de corte. En caso de no conseguirlos, utilizamos el procedimiento de Padberg-Rao.
\item[8] Añadimos los planos de corte encontrados al problema y volvemos la Paso 4.
\end{enumerate}
\end{frame}
\subsection{Aplicaciones}
\begin{frame}{Problema de asignación}
\begin{itemize}
\item Supongamos que tenemos un conjunto de recursos y un conjunto de tareas.
\item Cada recurso realiza una tarea y cada tarea solo puede ser realizada por un recurso.
\item Ejemplos típicos: Empleados y puestos de trabajo, compañeros de habitación, etc.
\end{itemize}
\end{frame}

\begin{frame}{Problema del orden de inventario}
\begin{itemize}
\item Supongamos que tenemos una pila de $p$ objetos que ganan o pierden valor en el tiempo. Por simplificar, todos del mismo tipo.  
\item Sea $v(t)$ la función que nos da la utilidad esperada de un objeto en el tiempo $t$  y sea $a_i$ la edad del objeto $i$.
\item Supongamos que sabemos los tiempos $t_i$ en los que necesitaremos sacar un objeto de la pila. 
\end{itemize}
$$
c_{ij} = v(a_i+t_j)
$$
\end{frame}
\section{Algoritmo del SAP}
\subsection{Fundamentos}
\begin{frame}

\end{frame}

\section{Análisis computacional}
\subsection{Comparando métodos}
\begin{frame}{Tiempo medio de ejecución}
\begin{figure}[h!]
\centering
\includegraphics[scale=0.45]{bebesota}
\end{figure}
\end{frame}

\begin{frame}{Tiempo medio de ejecución}
\begin{figure}[h!]
\centering
\includegraphics[scale=0.45]{plot3}
\end{figure}
\end{frame}

\begin{frame}{Media de llamadas al simplex}
\begin{figure}[h!]
\centering
\includegraphics[scale=0.45]{plot2}
\end{figure}
\end{frame}

\subsection{Reoptimización}

\begin{frame}{Frontera de Pareto}
\begin{figure}[h!]
\centering
\includegraphics[scale=0.33]{Front_pareto}
\end{figure}
\end{frame}

\begin{frame}{Escalarización lineal}
\begin{align*}
\min_x & \sum_{(u,v) \in E}x_{uv}((1-\lambda)c^1_{uv}+\lambda c^2_{uv})\\
s.a.&\;\sum_{(u,v)\in\delta(u)} x_{uv} \leq 1, \quad \forall u \in V\\
&\sum_{(u,v)\in \gamma(S)} x_{uv} \leq \frac{1}{2}(|S|-1)\quad \forall S \in B	\\
&x_{uv} \geq 0 \qquad \forall(u,v)\in E
\end{align*}
\end{frame}

%\begin{frame}{Reoptimización en Gurobi}
%\begin{align*}
%\max&\, \lambda\\
%s.a.&\,  l \leq (1-\lambda) c^1 + \lambda c^2\\
%& u \geq (1-\lambda) c^1 + \lambda c^2 \\
%& \lambda\geq \lambda^* 
%\end{align*}
%\end{frame}

\begin{frame}{Ejemplo}
\begin{figure}[h!]
\centering
\includegraphics[scale=0.5]{opt}
\end{figure}
\end{frame}

\begin{frame}
\begin{figure}[h!]
\centering
\includegraphics[scale=0.5]{opt2}
\end{figure}
\end{frame}

\begin{frame}
\begin{figure}[h!]
\centering
\includegraphics[scale=0.5]{opt3}
\end{figure}
\end{frame}

\begin{frame}
\begin{figure}[h!]
\centering
\includegraphics[scale=0.5]{opt4}
\end{figure}
\end{frame}


\begin{frame}{Media de soluciones encontradas}
\begin{table}[ht]
\centering
\begin{tabular}{rr}
  \hline
  Nodos & Sols \\ 
  \hline
20 & 10.00 \\ 
40 & 19.80 \\ 
60 & 31.00 \\ 
80 & 45.60 \\ 
100 & 51.60 \\ 
120 & 63.80 \\ 
140 & 68.00 \\ 
   \hline
\end{tabular}
\end{table}
\end{frame}

\begin{frame}{Número total de soluciones}
\begin{figure}[h!]
\centering
\includegraphics[scale=0.47]{plot10}
\end{figure}
\end{frame}
\begin{frame}{Porcentaje de soluciones no soportadas}
\begin{figure}[h!]
\centering
\includegraphics[scale=0.47]{plot11}
\end{figure}
\end{frame}

\section{Conclusiones}
\begin{frame}
\begin{center}
\huge{¡Gracias por su atención!}
\end{center} 
\end{frame}
\end{document}