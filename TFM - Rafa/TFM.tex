\documentclass[twoside,a4paper,openright,12pt,tikz]{book}
\usepackage[T1]{fontenc}
\usepackage{amsfonts}
\usepackage{mathtools,amscd,amsthm}
\usepackage{tabularx}
\usepackage{amssymb,eucal,bezier,graphicx}
\usepackage{times}
\usepackage{subfig}
\usepackage[svgnames]{xcolor}
\usepackage{fancybox}
\usepackage{fancyhdr}
\usepackage{hyperref}
\usepackage{enumerate}
\usepackage{comment}
\usepackage[spanish]{babel}
\usepackage[utf8]{inputenc}
\usepackage{anysize}
\usepackage{listings}
\usepackage{etoolbox}
\usepackage{tikz}


% Plantillas de código
\newcommand{\rstyle}{\lstset{ 
  language=R,                     % the language of the code
  basicstyle=\small\ttfamily, % the size of the fonts that are used for the code
  numbers=left,                   % where to put the line-numbers
  numberstyle=\tiny\color{Blue},  % the style that is used for the line-numbers
  stepnumber=1,                   % the step between two line-numbers. If it is 1, each line
                                  % will be numbered
  numbersep=5pt,                  % how far the line-numbers are from the code
  backgroundcolor=\color{white},  % choose the background color. You must add \usepackage{color}
  showspaces=false,               % show spaces adding particular underscores
  showstringspaces=false,         % underline spaces within strings
  showtabs=false,                 % show tabs within strings adding particular underscores
  frame=single,                   % adds a frame around the code
  rulecolor=\color{black},        % if not set, the frame-color may be changed on line-breaks within not-black text (e.g. commens (green here))
  tabsize=2,                      % sets default tabsize to 2 spaces
  captionpos=b,                   % sets the caption-position to bottom
  breaklines=true,                % sets automatic line breaking
  breakatwhitespace=false,        % sets if automatic breaks should only happen at whitespace
  keywordstyle=\color{RoyalBlue},      % keyword style
  commentstyle=\color{YellowGreen},   % comment style
  stringstyle=\color{ForestGreen},     % string literal style
     literate=%
         {á}{{\'a}}1
         {í}{{\'i}}1
         {é}{{\'e}}1
         {ý}{{\'y}}1
         {ú}{{\'u}}1
         {ó}{{\'o}}1
         {ñ}{{\~n}}1}}
         
\lstnewenvironment{erre}[1][]
{
\rstyle
\lstset{#1}
}
{}         

\definecolor{deepblue}{rgb}{0,0,0.5}
\definecolor{deepred}{rgb}{0.6,0,0}
\definecolor{deepgreen}{rgb}{0,0.5,0}

\newtoggle{InString}{}% Keep track of if we are within a string
\togglefalse{InString}% Assume not initally in string
\definecolor{majo}{HTML}{CD2626}
\newcommand*{\ColorIfNotInString}[1]{\iftoggle{InString}{#1}{\color{majo}#1}}%
\newcommand*{\ProcessQuote}[1]{#1\iftoggle{InString}{\global\togglefalse{InString}}{\global\toggletrue{InString}}}%


\DeclareFixedFont{\ttb}{T1}{txtt}{bx}{n}{12} % for bold
\DeclareFixedFont{\ttm}{T1}{txtt}{m}{n}{11}  % for normal

\newcommand{\pythonstyle}{\lstset{
  language=Python,                     % the language of the code
  basicstyle=\small\ttm, % the size of the fonts that are used for the code
  numbers=left,                   % where to put the line-numbers
  numberstyle=\tiny\color{DarkBlue},  % the style that is used for the line-numbers
  stepnumber=1,                   % the step between two line-numbers. If it is 1, each line
                                  % will be numbered
  numbersep=5pt,                  % how far the line-numbers are from the code
  backgroundcolor=\color{white},  % choose the background color. You must add \usepackage{color}
  showspaces=false,               % show spaces adding particular underscores
  showstringspaces=false,         % underline spaces within strings
  showtabs=false,                 % show tabs within strings adding particular underscores
  frame=single,                   % adds a frame around the code
  rulecolor=\color{black},        % if not set, the frame-color may be changed on line-breaks within not-black text (e.g. commens (green here))
  tabsize=2,                      % sets default tabsize to 2 spaces
  captionpos=b,                   % sets the caption-position to bottom
  breaklines=true,                % sets automatic line breaking
  breakatwhitespace=false,        % sets if automatic breaks should only happen at whitespace
  emph={range,len,print},          
  emphstyle=\ttb\color{deepred},
  keywordstyle=\color{RoyalBlue},      % keyword style
  commentstyle=\color{Grey},   % comment style
  stringstyle=\color{LimeGreen},     % string literal style
     literate=%
         {á}{{\'a}}1
         {í}{{\'i}}1
         {é}{{\'e}}1
         {ý}{{\'y}}1
         {ú}{{\'u}}1
         {ó}{{\'o}}1
         {ñ}{{\~n}}1
         {"}{{{\ProcessQuote{"}}}}1% Disable coloring within double q
         {'}{{{\ProcessQuote{'}}}}1% Disable coloring within single 
         {0}{{{\ColorIfNotInString{0}}}}1
    	 {1}{{{\ColorIfNotInString{1}}}}1
   	 	 {2}{{{\ColorIfNotInString{2}}}}1
   		 {3}{{{\ColorIfNotInString{3}}}}1
   		 {4}{{{\ColorIfNotInString{4}}}}1
   		 {5}{{{\ColorIfNotInString{5}}}}1
   		 {6}{{{\ColorIfNotInString{6}}}}1
   		 {7}{{{\ColorIfNotInString{7}}}}1
   		 {8}{{{\ColorIfNotInString{8}}}}1
   		 {9}{{{\ColorIfNotInString{9}}}}1
  }}
         
\lstnewenvironment{pythone}[1][]
{
\pythonstyle
\lstset{#1}
}
{}


\newcommand{\graphfromadj}[3][arc/.try]{
    \foreach [count=\r] \row in {#3}{
        \foreach [count=\c] \cell in \row{
            \ifnum\cell=1%
                \draw[arc/.try=\cell, #1] (#2\r) edge (#2\c);
            \fi
        }
    }
}

\newcommand{\weigthgraphfromadj}[3][draw,->]{
    \foreach [count=\r] \row in {#3}{
        \foreach [count=\c] \cell in \row{
            \if0\cell%
            \else
                \draw[arc/.try=\cell, #1] (#2\r) edge node[arc label/.try=\cell]{\cell} (#2\c);
            \fi
        }
    }
}


\marginsize{3.25cm}{3.25cm}{3cm}{3cm}

\newtheorem{defi}{Definici\'on}[section]
\newtheorem{ej}{Ejemplo}[section]
\newtheorem{ejs}{Ejemplos}[section]
\newtheorem{prop}{Proposici\'on}[section]
\newtheorem{nota}{Nota}[section]
\newtheorem{notac}{Notación}[section]
\newtheorem{rem}{Observaci\'on}[section]
\newtheorem{thm}{Teorema}[section]
\newtheorem{cor}{Corolario}[section]
\newtheorem{lem}{Lema}[section]
\newtheorem*{dem}{Demostración}

\providecommand{\abs}[1]{\left|{#1}\right|}
\providecommand{\conv}[1]{\overset{#1}{\longrightarrow}}
\providecommand{\convcs}{\xrightarrow{CS}}
\providecommand{\conve}{\xrightarrow{e}}
\providecommand{\func}[2]{\colon{#1}\longrightarrow{#2}}
\newcommand{\efe}{\hat{f}}
\newcommand{\R}{\mathbb{R}}
\newcommand{\D}{\mathbb{D}}
\newcommand{\Z}{\mathbb{Z}}
\newcommand{\N}{\mathbb{N}}
\newcommand{\E}{\mathbb{E}}
\newcommand{\fn}{\hat{f}_{0N}}
\newcommand{\X}{\overline{X}}
\newcommand{\dis}{\displaystyle}
\providecommand{\norm}[1]{\left\lVert#1\right\rVert}
\providecommand{\posi}[1]{\left[#1\right]^+}

% IRENITA--------------------------------
\renewcommand{\headrulewidth}{0.4pt} 
\fancyhead[RO,LE]{\thepage} 
\fancyhead[LO]{\nouppercase{\leftmark}}
\fancyhead[RE]{\nouppercase{\rightmark}}
\fancyfoot{}
\newcommand{\va}{\hat{\vartheta}_N}
\pagestyle{fancy}
% --------------------------------------

\setcounter{secnumdepth}{3}
\setcounter{tocdepth}{3}


\begin{document}
% ---------------------PORTADA
\begin{titlepage}

\vspace*{1in}
\begin{center}
\vspace*{-1in}
\begin{figure}[htb]
\begin{center}
\begin{large}
TRABAJO FIN DE MÁSTER\\
\end{large}
\rule{80mm}{0.1mm}\\
\vspace*{0.1in}
\end{center}
\end{figure}
\begin{large}
\end{large}

\vspace*{0.2in}
\begin{Large}
{\huge \bfseries El problema paramétrico del emparejamiento en grafos y
problema de emparejamiento con dos objetivos}\\[2cm]
\end{Large}

\begin{center} \Large
\emph{Presentado por:}\\
\textsc{ \bf{Rafael González López}}
\end{center}

\vspace*{0.2in}
\begin{center} \large
\emph{Supervisado por:} \\
\textsc{Dr.~Justo Puerto Albondoz}\\
\end{center}
\vspace*{0.2in}

\centering
\includegraphics[width =7cm]{logo}



\begin{large}
\centering
FACULTAD DE MATEMÁTICAS \\
\end{large} 

\begin{large}
Departamento de Estadística e Investigación Operativa\\
\end{large}


\begin{large}
\centering
Sevilla, Junio 2019\\
 \end{large}
\end{center}


\end{titlepage}

\newpage
\thispagestyle{empty}
%------------------------------------------------------------------------

\tableofcontents
\newpage
\thispagestyle{empty}

\chapter*{Abstract}
\addcontentsline{toc}{chapter}{Abstract}
PENDIENTE
%The sample average approximation (SAA) method is an approach for solving stochastic optimization problems by using Monte Carlo simulation. The basic idea of such method is that we can approximate the expected objetive function by the corresponding sample average function using a random sample. We solve the obtained sample average approximating problem by deterministic optimization techniques, and the process is repeated several times with different samples to obtain candidate solutions along with statistical estimates of their optimality gaps until a stopping criterion is satisfied.


%In section 1 we describe the expected value and sample average approximation problems and give a few examples of real cases in which it can be useful. In section 2 we show many results related to convergence of estimators (objective value, optimal solution, etc) under certain assumptions. In section 3  we discuss convergence rates of objetive values. In section 4 we implement the method to study two problems (that involve different random variables) to illustrate the power of the method.
\newpage
\thispagestyle{empty}

\chapter*{Introducci\'on}\label{cap.introduccion}
\addcontentsline{toc}{chapter}{Introducción}
%En ocasiones, cuando tratamos de resolver problemas de programación estocástica que involucran valores esperados, tenemos que lidiar un tamaño del espacio muestral excesivamente grande o con cálculos de esperanzas que pueden ser tremendamente costosos desde un punto de vista computacional. En este trabajo desarrollamos una técnica que nos permite en una gran variedad de casos aproximar estos problemas mediante mediante el método de Monte Carlo. Esta técnica se conoce como Método de aproximación por media muestral.

%Tras una presentación inicial del método, proseguiremos dando resultados relacionados con la convergencia, bajo hipótesis adecuadas, de las soluciones, valor objetivo y otros estimadores de los problemas muestrales a los correspondientes del problema real, así como una primera aproximación al orden de convergencia del método. 

%Finalmente, ejemplificaremos la eficacia del método con experimentos reales. Expondremos dos casos donde utilizaremos una normal multivariante y una mixtura de normales para tener una primera impresión sobre la capacidad de aproximación del método y la utilidad para casos donde, amén de heurística, resulta infactible resolver los problemas por fuerza bruta.


\newpage
\thispagestyle{empty}
%\pagenumbering{arabic} % para empezar la numeración con números
\chapter{El problema del emparejamiento}
\section{El problema del $b$-emparejamiento}
Dado que nuestro objetivo es el estudio paramétrico del problema del emparejamiento y la versión multiobjetivo del mismo, comenzamos este trabajo exponiendo en qué consiste el problema del ejemparejamiento o problema del \textit{matching}. Para ello, comenzamos definiendo algunos conceptos básicos en el marco de la teoría de grafos.
\begin{defi}
Sea $G=(V,E)$ un grafo y sea $S \subset V$, definimos $\delta(S)$ como el conjunto de aristas con un único extremo en $S$. En el caso de un conjunto unitario $\{i\}$, denotamos $\delta(i):=\delta(\{i\})$. Usualmente se denomina \textbf{grado del vértice $i$} al cardinal de $\delta(i)$. Definimos además $\gamma(S)$ como el conjunto de aristas que tienen ambos extremos en $S$.
\end{defi}

En adelante consideramos siempre $G=(V,E)$ un grafo no dirigido. El problema del \textit{matching} consiste en encontrar un subconjunto $M\subset E$ con la propiedad de que en el subgrafo inducido $G(M)=(V,M)$ ningún vértice tenga grado mayor que $1$, es decir, que ninguna arista tenga vértices en común. Naturalmente, este problema es fácilmente generalizable al problema del $b$-emparejamiento o $b$-\textit{matching}, en el cual cada vértice $v$ debe tener un grado no mayor que $b_v$, donde $b_v$ es un entero positivo. El problema original pasaría a ser el caso particular en el que $b_v = 1$ $\forall v \in V$.

\begin{defi}
Sea $G=(V,E)$ un grafo y sea $M\subset E$ un $b$-emparejamiento. Diremos que $M$ es un \textbf{emparejamiento perfecto} si $|\delta(v)|=b_v$ $\forall v \in V$, es decir, si las restricciones se verifican con igualdad.
\end{defi}
Para cada $(u,v)\in E$ podemos considerar el peso o coste $c_{uv}$ asociado. Dependiendo del contexto en el que estemos trabajando estos pesos pueden ser números reales, reales positivos, enteros no negativos, etc. En este trabajo consideraremos que los costes son reales no negativos. Dado un conjunto de aristas $E'\subset E$, tiene sentido considerar
$$
c(E')=\sum_{(u,v)\in E'} c_{uv}
$$
El problema del $b$-emparejamiento de coste máximo o \textit{weighted $b$-matching problem} consiste en encontrar el $b$-emparejamiento que maximiza la función $c(\cdot)$. Si nos ceñimos únicamente a los emparejamientos perfectos, también tiene sentido considerar el problema de encontrar el que tiene peso mínimo. En general, cuando $c_{uv}=1$ $\forall (u,v)\in E$, el problema se denomina de cardinalidad o \textit{cardinality problem}.

El problema del matching puede ser formulado como un problema de programación entera
\begin{align*}
\max_{x} &\; \sum_{(u,v)\in E} x_{uv}c_{uv}  \nonumber\\ 
s.a.\;  &  Ax\leq b \\
& x\in\{0,1\}^n\nonumber
\end{align*}
donde $c$ es el vector de pesos, $A$ es la matriz de incidencia del grafo, $|E|=n$ y la variable $x_{uv}=1$ si la arista $(u,v)$ está en el emparejamiento y $0$ en caso contrario. Nótese que si $G$ es un grafo bipartito entonces $A$ es una matriz totalmente unimodular y, en ese caso, los puntos extremos del poliedro $\{x \in \R^n_+\mid Ax\leq b\}$ son precisamente los $b$-emparejamientos.

Es claro que la formulación como problema de programación entera no suele resultar la más conveniente. La mayoría de técnicas y algoritmos para resolver este problema de manera eficiente utilizan instrumentos basados en la dualidad de la Programación Lineal. Esto es posible gracias a un importante resultado que probó Edmonds en \cite{edmond}, que pasamos a enunciar.
\begin{thm}
Sea $G=(V,E)$ un grafo. Sea $B$ el conjunto
$$
B = \{S\subset V \mid |S| \text{ es impar},\;|S|\geq 3\}
$$
Entonces, la envolvente convexa del politopo del matching viene dada por
\begin{align*}
\sum_{(u,v)\in\delta(u)} x_{uv} &\leq 1, \quad \forall u\in V\\
x_{uv} &\geq 0\\
\sum_{(u,v)\in \gamma(S)} x_{uv}& \leq \frac{1}{2}(|S|-1)\quad \forall S \in B	
\end{align*}
Además, si nos restringimos a los emparejamientos perfectos, entonces basta considerar únicamente la igualdad en el primer bloque de restricciones.
\end{thm}

Resumiendo nuestras hipótesis, consideraremos que $c_{uv}\geq 0$ $\forall (u,v)\in E$, es decir, los pesos son no negativos. Además, imponemos que los grafos estudiados admiten un matching perfecto. Principalmente, vamos a estudiar técnicas para resolver y reoptimizar el problema del emparejamiento perfecto de coste mínimo o \textit{minimum-cost perfect matching problem} (MCPM). Usando el teorema anterior, este puede ser formulado como
\begin{align*}
\min_x & \sum_{(u,v) \in E}x_{uv}c_{uv}\\
s.a.&\;\sum_{(u,v)\in\delta(u)} x_{uv} \leq 1, \quad \forall u \in V\\
&\sum_{(u,v)\in \gamma(S)} x_{uv} \leq \frac{1}{2}(|S|-1)\quad \forall S \in B	\\
&x_{uv} \geq 0 \qquad \forall(u,v)\in E
\end{align*}
Esta formulación es ciertamente útil, pues tenemos una primera aproximación al problema del matching con una formulación propia de la Programación Lineal. De hecho, podemos dar la formulación del problema dual
\begin{align*}
\max_y & \sum_{u\in V}y_u + \sum_{S\in B}\frac{1}{2}(|S|-1)y_S\\
s.a.&\;y_u+y_v - \sum_{S\in B,\,(u,v)\in \gamma(S)} y_S \leq c_{uv} \quad \forall (u,v)\in E\\
&y_u \geq 0 \qquad \forall u\in V\\
&y_S \geq 0 \qquad \forall S\in B
\end{align*}
De manera que podemos hacer la siguiente definición.
\begin{defi}
Definimos el coste reducido de una variable $x_{uv}$ asociada a una arista $(u,v)$ con respecto a una solución dual $y$ como
$$
c_{uv}' = c_{uv} -y_u - y_v + \sum_{S\in B,\,(u,v)\in \gamma(S)}y_S 
$$
También notaremos el vector de costes reducidos como $\overline{c}$.
\end{defi}

A pesar de la utilidad del teorema de Edmonds para el desarrollo de numerosos algoritmos, más adelante veremos algunas modificaciones del de dicho teorema que nos darán algunas formulaciones equivalentes pero más interesantes de cara al paradigma de la resolución computacional. Como introducción a esta idea, tenemos la equivalencia
$$
\sum_{(u,v)\in \gamma(S)} x_{uv} \leq \frac{1}{2}(|S|-1) \Leftrightarrow \sum_{(u,v)\in \delta(S)} x_{uv} \geq 1
$$
\newpage
\subsection{El problema del emparejamiento biobjetivo}
Una vez que hemos presentado los conceptos básicos para entender el MCPM, podemos definir una clase más general de problemas a los cuáles dedicaremos especial atención durante los procesos de reoptimización. Definimos el \textit{problema del emparejamiento paramétrico respecto de $R\subset E$} con parámetro $\lambda$ como
\begin{align*}
\min_x & \sum_{(u,v) \in E}x_{uv} (c_{uv} + \lambda d_{uv})\\
s.a.&\;\sum_{(u,v)\in\delta(u)} x_{uv} \leq 1, \quad \forall u \in V\\
&\sum_{(u,v)\in \gamma(S)} x_{uv} \leq \frac{1}{2}(|S|-1)\quad \forall S \in B	\\
&x_{uv} \geq 0 \qquad \forall(u,v)\in E
\end{align*}
donde 
$$
d_{uv} = \begin{cases}
1 & (u,v)\in R\\
0 & (u,v)\notin R
\end{cases}
$$
Este problema puede verse como la relajación lagrangiana del problema del matching con una cota superior y es ampliamente estudiado por Ball y Taverna en \cite{balltab}. 
Más generalmente, si consideramos dos funciones de coste $c^1$ y $c^2$ cualesquiera y un parámetro $\lambda$ escalar, podemos formular el \textit{problema del emparejamiento biobjetivo}
\begin{align*}
\min_x & \sum_{(u,v) \in E}x_{uv}(c^1_{uv}+\lambda c^2_{uv})\\
s.a.&\;\sum_{(u,v)\in\delta(u)} x_{uv} \leq 1, \quad \forall u \in V\\
&\sum_{(u,v)\in \gamma(S)} x_{uv} \leq \frac{1}{2}(|S|-1)\quad \forall S \in B	\\
&x_{uv} \geq 0 \qquad \forall(u,v)\in E
\end{align*}
\section{El algoritmo de Grötschel-Holland}
A partir del Teorema 1.1.1, Edmonds desarrolló basándose en ideas de la programación lineal un método combinatorio \cite{edmond} para obtener una solución del problema de emparejamiento en tiempo polinómico. Una de las características más importantes del problema de Programación Lineal asociado es que tiene un número de restricciones exponencial en el número de ejes. A pesar de que, en un principio, encontraríamos esta condición prohibitiva a la hora de resolver el problema, los resultados de Edmonds muestran que la estructura de las restricciones son más importantes que su número. 

A pesar de que es el algoritmo que finalmente utilizaremos para el proceso de repotimización, Grötschel y Holland presentan en \cite{holland} un método para resolver el problema del emparejamiento basado en el método del simplex. Este algoritmo utiliza varias heurísticas que según los autores mejoran considerablemente su tiempo de ejecución. Teóricamente, el algoritmo que pasamos a presentar no es polinomial, no obstante en la práctica es competitivo con los algoritmos combinatorios conocidos para el problema del emparejamiento.

Para evitar redundar en desarrollo teórico de un algoritmo que no utilizaremos en secciones posteriores pero con el fin de ilustrarlo de manera suficientemente técnica, presentamos seguidamente el algoritmo y explicamos las ideas claves que lo componen. Notemos que originalmente el algoritmo fue detallado por sus autores para un grafo completo $K_n$, lo cuál será respetado en el siguiente análisis.
\begin{align*}
\text{Paso 1: }&\text{Determinar un conjunto $E'\subset E$ de ejes candidatos.}\\
&\text{Vamos al Paso 2.}\\
\text{Paso 2: }&\text{Utilizar una heurísitca para encontrar un matching $M$.}\\
&\text{Establecemos $E':=E'\cup M$.}\\
&\text{Vamos al Paso 3.}\\
\text{Paso 3: }& \text{Establecemos el problema de Programación Lineal}\\
& \min_x  \sum_{(u,v) \in E}x_{uv}c_{uv}\\
&s.a.\;\sum_{(u,v)\in\delta(u)} x_{uv} = 1, \quad \forall u \in V\\
&\qquad x_{uv} \geq 0 \qquad \forall(u,v)\in E'\\
&\text{Vamos al Paso 4.}\\
\text{Paso 4: }&\text{Obtener una solución $x^*$ del problema anterior.}\\
&\text{Vamos al Paso 5.}\\
\text{Paso 5: }&\text{Construimos el grafo solución $G_{x^*}$.}\\
&\text{Comprobamos si $x^*$ es entero.}\\
&\text{Vamos al Paso 6.}\\
\text{Paso 6: }& \text{En caso afirmativo, vamos al Paso 9.}\\
&\text{En otro caso, vamos al Paso 7.}\\
\text{Paso 7: }&\text{Encontramos un hiperplano de corte para $x^*$. Para ello realizamos:}\\
&\text{Utilizamos la Heurística 1 con $G_{x^*}$. Si encontramos un hiperplano}\\
&\text{vamos al Paso 8. En otro caso utilizamos la Heurística 2 en $G^2_{x^*}$.}\\
\end{align*}
\begin{align*}
&\text{Si lo encontramos, vamos al Paso 8. En otro caso, utilizamos el}\\
&\text{procedimiento Padberg-Rao en $G'_{x^*}$. Obtenemos un hiperplano y}\\
&\text{vamos al Paso 8.}\\
\text{Paso 8: }&\text{Añadimos el hiperplano a nuestro problema de Programación Lineal.}\\
&\text{Vamos al Paso 4.}\\
\text{Paso 9: }&\text{Determinar las varaibles $VAR$.}\\
&\text{Si $VAR=\emptyset$, la solución es óptima. }\\
&\text{Si $VAR\neq\emptyset$, establecemos $E'=E\cup VAR$.}\\
&\text{Vamos al Paso 4.}
\end{align*}
A continuación describimos brevemente cada uno de los pasos del algoritmo.
\begin{itemize}
\item[Paso 1.] Determinamos un conjunto $E'\subset E$ que denominamos ejes candidatos, que utilizaremos como variables para el problema de Programación Lineal inicial. Para cada nodo determinamos los $NN$ ($1\leq NN \leq |V|-1$) ejes de menor coste incidentes en él. $E'$ será la unión de estos ejes. Los autores del algoritmo han probado computacionalmente distintos valores para $NN$ y encuentran que $5\leq NN \leq 10$ es una buena elección para grafos de tamaño mediano, es decir, de tamaño entre 500 y 1000. Esto significa que el número de varaibles que utilizamos, en principio, supone menos del 1\% del número total de variables. 
\item[Paso 2.] Los ejes de $E'$ son ordenados (por ejemplo, a través del algoritmo Quickshort) de manera no decreciente con respecto a sus pesos. Utilizamos un algoritmo voraz para encontrar un matching $M$ inicial. Si $M$ no fuese perfecto, tomamos pares arbitrariamente pares de nodos no conectados en $M$ y añadiendo aristas de $E\setminus E'$ a $M$ y, para garantizar la factibilidad del problema, al propio conjunto $E'$. 
\item[Paso 3.] El primer problema de Programación Lineal que resolveremos es la relajación trivial 
\begin{align*}
\min_x & \sum_{(u,v) \in E}x_{uv}c_{uv}\\
&s.a.\;\sum_{(u,v)\in\delta(u)} x_{uv} = 1, \quad \forall u \in V\\
&x_{uv} \geq 0 \qquad \forall(u,v)\in E'
\end{align*}
del problema inducido por $E'$. Notemos que el grado $\delta(v)$ debe tomarse en el subgrafo $(V,E')$. Como solución inicial de este problema utilizamos el matching $M$ calculado anteriormente.
\newpage
\item[Paso 4.] Resolvemos el problema de Programación Lineal considerado. Los autores del algoritmo recomiendan usar IBM-MPSX/370 utilizando. De manera general, cada vez que vayamos a este paso utilizamos la solución óptima obtenida en la llamada anterior como base inicial. En el caso de que hayamos añadido nuevas restricciones, esta base será infactible para el problema primal, pero sí será factible para el dual, que podemos utilizar para encontrar una solución del primal. Sea $x^*$ la solución obtenida.
\item[Paso 5.] Consideramos el grafo $G_{x^*}=(V,E_{x^*})$, donde $E_{x^*} =\{(u,v)\in E'\mid x^*_{uv}>0\}$. Este grafo servirá como input para la fase de detección de planos de corte. Definimos los costes $k_{uv}:=x^*_{uv}$. Para no escanear dos veces $x^*$, comprobamos si $x^*$ es un vector de coordenadas enteras.
\item[Paso 6.] Si $x^*$ es entero, entonces induce un matching perfecto en el subproblema considerado. En este caso, comprobamos la optimalidad global (Paso 9). En otro caso, necesitamos determinar planos de corte para obtener una nueva solución $x^*$, es decir, vamos al Paso 7.
\item[Paso 7.] Si la solución $x^*$ no es entera, entonces no puede inducir un matching en nuestro grafo. Necesitamos encontrar planos de corte y, naturalmente, deseamos encontrarlos en el menor tiempo posible.

Comenzamos utilizando búsqueda en profundidad, comprobamos si $G_{x^*}$ tiene componentes conexas con un número impar de nodos. Si lo hay, sean $(V_i,E_i)$ estas componentes de $G_{x^*}$, estas dan lugar a unas desigualdades
$$
\sum_{(u,v)\in E_i} x_{uv} \leq \frac{1}{2}(|V_i|-1)
$$ 
que es violada por $x^*$. En este caso no necesitamos buscar más planos de corte, sino directamente utilizamos estos y continuamos al Paso 8.

Si nuestra solución no viola las desigualdades anteriores, consideramos el grafo $G^2_{x^*}$ obtenido al eliminar las aristas de $E_{x^*}$ cuyos costes -recordemos, los $k_{uv}$ definidos anteriormente- sean menores que un cierto $\varepsilon$. Aplicamos ahora el procedimiento inicial del Paso 7 al grafo $G^2_{x^*}$. Incluso si encontramos componentes conexas de cardinalidad impar, estas no tendrían por qué inducir desigualdades que se violasen necesariamente, por lo que tenemos que comprobarlo. Si es así, hemos encontrado planos de corte y pasamos al Paso 8. 

Si después de utilizar esta segunda heurística tampoco hemos encontrado planos de corte, consideremos $V_1$ como el conjunto de vértices de $G_{x^*}$ incidentes con alguna arista cuyo peso $k_{uv}=1$. En tal caso, eliminamos de $G_{x^*}$ el conjunto de vértices $V_1$ así como todas las aristas que les sean incidentes. Denotamos por $G'_{x^*} =(V',E'_{x^*})$ al grafo resultante de esta operación. Padberg y Rao presentaron un método para encontrar un plano de corte adecuado \cite{rao} basado en determinar el árbol de Gomory-Hu de $G'_{x^*}$ y encontrar un eje de este árbol de peso mínimo entre aquellos que al ser eliminados dividen el árbol en dos componentes de cardinalidad impar. Notemos que, en particular, $G'_{x^*}$ tiene un número par de nodos.
Una vez encontrado el plano de corte, continuamos con el Paso 8.

Notemos que el algoritmo de Padberg y Rao que utilizan los autores para encontrar el plano de corte, a pesar de ser polinomial, es $O(n^4)$. Para grafos de gran tamaño, este procedimiento también puede resultar prohibitivo, razón por la cuál se intentan encontrar planos de corte con heurísticas más sencillas. 

\item[Paso 8.] Añadimos los planos de corte encontrados en el paso anterior a nuestro problema y volvemos al Paso 4.
\item[Paso 9.] Hemos obtenido una solución entera que induce un matching perfecto e $(V,E')$, pero naturalmente no tiene por qué ser un matching óptimo en $K_n$. Para comprobarlo, calculamos el coste reducido a cada eje $(u,v)\in E\setminus E'$. Sea $VAR$ el conjunto de variables que podrían mejorar la solución óptima. Si todos los costes tienen el signo adecuado, entonces $VAR=\emptyset$ y nuestra solución actual es de hecho óptima en $K_n$, por lo que hemos terminado. Si $VAR\neq \emptyset$, entonces establecemos $E'=E'\cup VAR$ y añadimos las correspondientes variables. Volvemos al Paso 4.
\end{itemize}
\section{Aplicaciones y problemas relacionados}
En esta sección presentamos algunas aplicaciones, como la resolución del problema de la asignación o un algoritmo de aproximación del problema del viajante, y otros problemas interesantes, como el problema de la cobertura por aristas, el de la coloración de aristas o los T-enlaces, que guardan cierta relación con el problema del emparejamiento.
\subsection{El problema de la asignación}
Para ilustrar las distintas aplicaciones que tiene el problema del matchig, comenzamos con un problema clásico de la Investigación Operativa. El problema de asignación o \textit{assigment assignment problem} consiste en encontrar la forma de asignar recursos (máquinas, empleados, etc.) a un cierto conjunto de teareas determinadas con coste mínimo. Se supone que cada recurso se destina a una sola tarea, y que cada tarea es ejecutada por uno solo de los recursos. Es claro que este problema puede formularse como un problema de matching de coste mínimo. Notemos que, además, la formulación natural del mismo se realiza sobre un grafo bipartito. 
\subsection{Aproximación al problema del viajante}
El problema del viajante o \textit{travelling salesman problem} (TSP) en un grafo no dirigido consiste en encontrar el ciclo hamiltoniano de peso mínimo. Computacionalmente hablando, se sabe que es NP-duro. Este problema aparece para responder a la pregunta: ¿Dada una lista de ciudades y las distancias entre cada pareja de ciudades, cuál es la ruta más corta para visitar todas las ciudades y volver a la ciudad de origen? Este problema puede formularse como problema de programación entera, como puede encontrarse en \cite{papa}. 

Aunque no puede utilizarse el problema del matching para resolver este problema, Nicos Christofides encontró un algoritmo \cite{nico} para obtener, bajo ciertas condiciones, una aproximación no mayor que $1.5$ el valor óptimo del TSP. Para este resultado es necesario imponer que las distancias estén en el marco de un espacio métrico, es decir, han de ser simétricas y verificar la desigualdad triangular. El algoritmo consiste en:
\begin{align*}
\text{Paso 1: }&\text{Obtener el árbol recubrido mínimo $T$ de $G$.}\\
\text{Paso 2: }&\text{Consideremos $O$ el conjunto de vértices con grado impar}\\
&\text{Por el Lema del apretón de manos, $O$ tiene cardinalidad par.}\\
&\text{Hallar el MCPM sobre el grafo inducido por $O$.}\\
\text{Paso 3: }& \text{Combinar los ejes del árbol y del matching obteniendo un}\\
&\text{multigrafo euleriano. Encontrar un circuito euleriano.}\\
\text{Paso 4: }&\text{Obtener un circuito hamiltoniano descartando nodos ya visitados}
\end{align*} 
\subsection{Cobertura de aristas}
Sea $G=(V,E)$ un grafo. Una \textit{cobertura por aristas} del grafo $G$ es un conjunto $C$ de aristas de manera que todo vértice de $G$ es incidente con, al menos, un arista de $C$. Podemos definir además el \textit{número de cobertura de aristas} de un grafo $G$ como el tamaño de una cobertura de aristas de tamaño mínimo. Se verifica el siguiente resultado.
\begin{prop}
Sea $M$ un matching de cardinalidad máxima en $G$ y sea $C$ una cobertura de aristas de cardinalidad mínima en el grafo $G=(V,E)$. Entonces $|M|+|C|=|V|$.
\end{prop}
\begin{dem}
La demostración de este resultado es sencilla. Sea $U$ el conjunto de nodos con grado $0$ relativo a $M$, entonces $|U|=|V|-2|M|$. Dado que añadiendo las aristas de $U$ a $M$ obtendríamos una cobertura por aristas de $G$, necesariamente
$$
|C| \leq |M|+|U| = |M| + |V|-2|M| = |V|-|M|
$$
Por otro lado, consideremos el grafo inducido por $C$, y sea $M'$ un matching de cardinalidad máxima en este subgrafo. Sea $U'$ el conjunto de nodos con grado $0$ relativo a $M'$ en este subgrafo, entonces
$$
|C| = |M'| +|U'| = |V|-|M'|
$$
y, por tanto,
$$
|M| \geq |M'| = |V|-|C|
$$
de donde se deduce el resultado.
\end{dem}
Además de la relación anterior, tenemos un teorema análogo al Teorema 1.1.1. para la cobertura de aristas.
\begin{thm}
La envolvente convexa de las coberturas en un grafo $G=(V,E)$ está dada por
\begin{align*}
\sum_{(u,v)\in\delta(u)} x_{uv} &\geq 1, \quad \forall u\in V\\
x_{uv} &\geq 0\\
\sum_{(u,v)\in \gamma(S)\cup\delta(S)} x_{uv}& \geq \frac{1}{2}(|S|+1)\quad \forall S \in B	
\end{align*}
\end{thm}


\subsection{Coloración de aristas}
Dado un grafo $G=(V,E)$, una \textit{coloración de las aristas} de $G$ con el mínimo número de colores posibles, de manera que aristas que sean incidentes en un mismo vértice tengan colores diferentes. 

El problema de la coloración por aristas está relacionado con el problema del matching dado que una coloración es factible si y solo si cada conjunto de ejes de un mismo color forma un matching. Por tanto, podemos formular el problema de la cobertura de aristas como el problema de cubrir las aristas de $E$ con el mínimo número posible de matching maximales. 

Consideremos $\chi(G)$ como el número de colores mínimo con el que se puede colorear $G$. $\chi(G)$ se conoce usualmente como \textit{número cromático}. Sea además $\Delta(G) = \max_{v\in V} |\delta(v)|$. Veamos algunos resultados relacionados con estos conceptos.
\begin{prop}
Si $G$ es un grafo bipartito, entonces $\chi(G)=\Delta(G)$. 
\end{prop}
Más generalmente, Vinzing probó el siguiente teorema.
\begin{thm}
En general, para cualquier grafo $G$, $\chi(G)$ es $\Delta(G)$ o $\Delta(G)+1$. 
\end{thm}
La demostración del mismo presenta un algoritmo -bastante rápido- para colorear $G$ con $\Delta(G)+1$ colores. Esperaríamos que el teorema anterior nos permitiese decidir con facilidad si $\chi(G)$ es o no $\Delta(G)$, pero en realidad este problema es NP-completo. Es más, determinar si $\chi(G)$ es relativamente pequeño puede ser realmente costoso, como muestra el siguiente resultado.
\begin{thm}
El problema de decidir si $\chi(G)\leq 3$ es NP-completo.
\end{thm}
Ahora bien, sea $A$ la matriz cuyas filas corresponden a los matching maximales de $G$, entonces podemos formular
\begin{align*}
\chi(G)=\min_{y} &\; \sum_{i=1}^m  y_{i}  \nonumber\\ 
s.a.\;  &  yA_{\cdot i} \geq  1 \qquad \forall i=1,\dotsc,m\\
& y\in\{0,1\}^{m}\nonumber
\end{align*}
La relajación lineal de este problema, así como el dual de la misma, están dados por
\begin{align*}
\chi_{LP}(G)=\min_{y} &\; \sum_{i=1}^m  y_{i}  \nonumber\\ 
s.a.\;  &  yA_{\cdot i} \geq  1 \qquad \forall i=1,\dotsc,m\\
& y_i \geq 0  \qquad \forall i=1,\dotsc,m 
\end{align*}
\begin{align*}
\Delta_{LP}(G)=\max_{x} &\; \sum_{i=1}^n  x_i  \nonumber\\ 
s.a.\;  &  Ax_{\cdot i} \leq  1 \qquad \forall i=1,\dotsc,n\\
& x_i\geq 0  \qquad \forall i=1,\dotsc,n
\end{align*}
donde $|E|=n$. El problema dual puede ser resuelto en tiempo polinomial, pues $x^*$ ($0\leq x^*_i \leq 1$) es una solución factible del mismo si y solo si un matching de coste máximo en $G$ tomando como pesos $x^*$ tiene valor no mayor que 1.
\subsection{T-Enlaces y el problema del cartero chino}
Continuamos presentando la relación entre los emparejamientos y los $T$-enlaces, el problema de emparejamiento y el problema del cartero chino.
\begin{defi}
Dado un grafo $G=(V,E)$, consideremos $T\subset V$ con cardinalidad par y $E'\subset E$. Decimos que $E'$ es un $T$-enlace si en $G'=(V,E')$ si $v\in V$ tiene grado impar si y solo si $v\in T$.
\end{defi}
\begin{prop}
Los $T$-enlaces de cardinalidad mínima son árboles.
\end{prop}
\begin{prop}
El problema de encontrar un $T$-enlace de peso mínimo puede resolverse en tiempo polinomial.
\end{prop}
\begin{dem}
Encontrar un $T$-enlace de coste mínimo. Para verlo, veamos que podemos reducir este problema al de encontrar un matching perfecto. Sea $G=(V,E)$ y $T$, reemplazamos cada $v\in V$ por un grafo completo $C_v$ con $\delta(v)+\alpha_v$ vértices, donde
$$
\alpha_v = \begin{cases}
0 & \text{Si $v\in T$ y $|\delta(v)|$ es impar}\\
0 & \text{Si $v\notin T$ y $|\delta(v)|$ es par}\\
1 & \text{En otro caso}
\end{cases}
$$
Entonces, para cada $(u,v)\in E$, unimos un nodo de $C_v$ a otro de $C_i$ de manera que dos de estos nuevos ejes no sean incidentes en un mismo vértice. Sea $G'=(V',E\cup E')$ este nuevo grafo, donde $E'$ son las aristas de los grafos completos. Tenemos entonces
\begin{prop}
Si $M\subset E\cup E'$ es un matching perfecto en $G'$, entonces $M\cap E$ es un $T$-enlace en $G$. Recíprocamente, si $E^*$ es un $T$-enlace en $G$, entonces existe $K\subset E'$ tal que $E^*\cup K$ es un matching perfecto en $G'$.
\end{prop}
\end{dem}
Veamos ahora la importancia de encontrar estos conjuntos de cara a la resolución del problema conocido como \textbf{problema del cartero chino}. Este problema trata de encontrar un el circuito de menor coste que visite todas las aristas. Si el grafo es euleriano, necesariamente la solución ha de ser cualquier circuito euleriano. En caso de que nuestro grafo no sea euleriano, podemos utilizar el siguiente algoritmo polinomial basado precisamente en el $T$-enlace.
\begin{align*}
\text{Paso 1: }& \text{Si el grafo es euleriano, entonces tomamos un circuito euleriano}\\
&\text{como solución. En otro caso, vamos al Paso 2.}\\
\text{Paso 2: }&\text{Obtener el $T$-enlace de coste mínimo. Como hemos visto, este}\\
&\text{puede obtenerse usando un algoritmo que resuelva el problema}\\
&\text{del emparejamiento.}\\
\text{Paso 3: }&\text{Consideramos el multigrafo obtenido a partir de duplicar las}\\
&\text{aristas de los vértices de $T$. Por el lema del apretón de manos.}\\
&\text{Este nuevo grafo es euleriano. Obtenemos un circuito euleriano.}\\
\text{Paso 4: }& \text{El circuito euleriano de este grafo induce una solución del problema.}\\
\end{align*} 
\subsection{Envolvente convexa del $b$-matching}
Aunque el objetivo de este trabajo es analizar el problema relativo al $1$-emparejamiento, podríamos hablar más generalmente, tal y como dijimos al inicio de este capítulo, del problema del $b$-matching. De manera ilustrativa, presentamos el teorema análogo al Teorema 1.1.1. para esta formulación más general del problema. 
\begin{thm}
La envolvente convexa de los $b$-emparejamientos, donde $b$ es un vector de coordenadas enteras no negativas, de un grafo $G=(V,E)$ está dada por
\begin{align*}
\sum_{(u,v)\in\delta(u)} x_{uv} &\leq b_u, \quad \forall u\in V\\
x_{uv} &\geq 0 \quad \forall(u,v)\in E\\
\sum_{(u,v)\in \gamma(S)} x_{uv}& \leq \frac{1}{2}\left(\sum_{v\in S} b_v -1\right) \quad \forall S \text{ con $\sum_{v\in S} b_v$ impar}
\end{align*}
\end{thm}

\chapter{Algoritmo SAP}
En este capítulo vamos a explicar el funcionamiento del método SAP o \textit{shortest augmenting path} a partir del cuál puede obtenerse un matching perfecto de mínimo coste y, además, nos permitirá realizar una reoptimización eficiente.
\section{Propiedades teóricas}
\begin{defi}
Sea $G=(V,E)$ un grafo, $M$ un matching en $G$ y $v\in V$. Diremos que $v$ es \textbf{expuesto} con respecto a $M$ si no es extremo de ninguna arista de $M$.
\end{defi}
\begin{defi}
Sea $G=(V,E)$ un grafo y $M$ un matching en $G$, un \textbf{camino alternante} o \textit{alternating path} con respecto a $M$ es un camino en el cual se van alternando aristas que están en $M$ y fuera del matching. Análogamente pueden definirse \textbf{árboles y ciclos alternantes}.
\end{defi}
\begin{figure}[h!]
\centering
\begin{tikzpicture}[scale=.8,auto=left,every node/.style={circle,fill=blue!10}]
	\node (nA) at (1,10) {};
	\node (nB) at (3,10) {};
	\node (nC) at (5,10) {};
	\node (nD) at (7,10) {};
	\node (nE) at (9,10) {};

  \path[every node/.style={sloped,anchor=south,auto=false}]
        (nA) edge              node {\scriptsize $e\in M$} (nB)            
        (nB) edge              node {\scriptsize  $e\in E\setminus M$} (nC)
        (nC) edge              node {\scriptsize $e\in M$} (nD)
        (nD) edge              node {\scriptsize $e\in E\setminus M$} (nE);     
\end{tikzpicture}
\caption{Camino alternante}
\end{figure}

\begin{defi}
En las condiciones de la definición anterior, se define un \textbf{camino de aumento} o \textit{augmenting path} como un camino alternante cuyos nodos extremos son expuestos.
\end{defi}
Resulta claro que si tenemos un matching $M$ y un camino de aumento con respecto a $M$, podemos obtener un matching de mayor cardinalidad intercambiando el rol que tienen. Las que pertenecen al matching pasan a no pertencer y recíprocramente.

\begin{defi} 
Dado un matching $M$ y un camino de aumento $P$, definimos la operación 
$$
M\oplus P = (M\setminus P)\cup (P\setminus M)
$$
\end{defi}
El concepto de camino de aumento es realmente interesante para nuestros propósitos, como se desprende del siguiente resultado.
\begin{prop}
Un matching $M$ contiene el máximo número de ejes si y solo si no existe ningún camino de aumento relativo a $M$.
\end{prop}
\begin{dem}
Veamos la demostración del contrarrecíproco, es decir, $M$ no contiene el número máximo de ejes si y solo sí existe un camino de aumento. Una de las implicaciones es clara, pues si $P$ es un camino de aumento, entonces $M\oplus P$ es un matching y tiene cardinalidad estrictamente mayor que la de $M$. 

Supongamos que $M$ no es de máxima cardinalidad. En particular, ha de existir $M'$ tal que $|M'| = |M| +1$. Consideremos $D$ la diferencia simétrica de $M$ y $M'$. Entonces
$$
|D| = |M|+|M'| -2|M\cap M'| = 2|M|+1-2|M\cap M'|
$$
Por lo que $D$ tiene cardinalidad impar. Sea $G' = (V,D)$. Dado que $M$ y $M'$ son emparejamientos, el grado de cualquier vértice de $G'$ es a lo sumo $2$. De hecho, si existiese algún vértice de grado $2$ entonces una de las aristas pertenece a $M$ y la otra $M'$. Por tanto, cada componente conexa de $G'$ debe ser vértices aislados, ciclos de cardinalidad par o un camino de alternantes relativos a $M$ y a $M'$. Dado que $|D|$ es impar, debe haber al menos un camino alternante. Es más, dado que $|M'|=|M|+1$ uno de esos caminos debe ser de aumento con respecto a $M$.zD,4 1
\end{dem}

Dejamos los caminos a un lado y pasamos a definir otro tipo de conceptos relacionados con distintas familias de grafos y subgrafos.
\begin{defi}
Sea $G=(V,E)$ un grafo y $B \subset V$. Definimos el \textbf{grafo inducido por }$B$ como $G[B] = (B,\gamma(B))$.  
\end{defi}
\begin{defi}
Sea $G=(V,E)$ y $H$ un subgrafo de $G$. Si $H$ tiene el mismo conjunto de nodos que $V$, decimos que \textbf{abarca }$G$.
\end{defi}
\begin{defi}
Sea $G=(V,E)$ un grafo y $B\subset V$. Definimos $G \times B=(V_B,E_B)$ como \textbf{el grafo obtenido por la contracción de} $B$, donde $V_B = (V \setminus B)\cup \{v_B\}$ y $E_B$ es el conjunto $E$ salvo que, cada arista con un único extremo en algún vértice de $B$ en $G$, ahora este vértice es $v_B$. Si existen varias solo se mantiene una para no tener un multigrafo. El vértice $v_B$ es llamado \textbf{pseudonodo}. Podemos definir además $M_B:= M\cap E_B$.
\end{defi}

\begin{defi}
Sea $A\subset V$, decimos que es un conjunto \textbf{anidado} si $|A|\geq 3$ y $\forall W,Z\subset A$ se verifica que 
$$
W\cap Z \neq \emptyset \Rightarrow W\subset Z \text{ o bien } Z\subset W
$$
Además, sea $W \in A$, definimos $A[W] =\{Z\in A \mid Z\subset W, Z\neq W\}$.
\end{defi}
\begin{defi}
En las condiciones de la definición anterior, sean $\{W_1,\dotsc,W_n\}$ el conjunto de los elementos maximales de $A$. Entonces definimos
$$
G\times A = (\cdots((G\times W_1)\times W_2)\times \cdots \times W_n)
$$
Notemos que el orden no relevante. Si denotamos $E_A$ al conjunto de aristas del grafo anterior, podemos definir $M_A = M\cap E_A$. 
\end{defi}
\begin{defi}
Diremos que $A \subset V$ anidado es una \textbf{familia de contracción} si verifica además que
$$
G[W]\times A[W] \text{ es abarcado por un ciclo impar $\forall W \in A$}
$$ 
Además, si un elemento $W$ (maximal) de una familia de contracción $A$ es tal que $|M\cap\gamma(W)| = \frac{1}{2}(|W|-1)$ diremos que un \textbf{blossom} (exterior).
\end{defi}
Una propiedad fundamental al respecto fue probada también por Edmonds en \cite{edmon2}.
\begin{thm}
Sea $M$ un matching en $G$ y sea $A$ una familia de contracción tal que todo $W\in A$ es un blossom con respecto a $M$. Entonces, todo camino de aumento en $G\times A$ con respecto a $M_A$ induce un camino de aumento $P$ con respecto a $M$.
\end{thm}
\begin{cor}
Sea $A$ una familia de contracción de $G$ y $M_A$ un matching (perfecto) en $G\times A$. Entonces $M_A$ induce (de manera única) un matching (perfecto) $M$ en $G$ tal que $\forall W\in A$ es un blossom con respecto a $M$.
\end{cor}
\subsection{El problema del matching y las familias de contracción}
Las familias de contracción también son conocidas en la literatura como \textit{hypomatchable set} o \textit{shrinkable set} y Edmonds en \cite{edmon3} caracteriza estos conjuntos de la siguiente manera.
\begin{thm}
Sea $A \subset V$. Entonces $A$ es una familia de contracción si y solo si $\forall u \in A$ $\exists M_i$ matching tal que 
$$
|M_i\cap \gamma(A)| = \frac{1}{2}(|A|-1)
$$
\end{thm}
Esta caracterización es esencial, pues nos permite reformular el Teorema 1.1.1 de la siguiente forma, tal y como prueban Edmonds y Pulleyblank también en \cite{edmon3}.
\begin{thm}
Sea $G=(V,E)$ un grafo y sea $\mathcal{A}(G)$ el conjunto de todas las familias de contracción de $G$. Entonces, la envolvente convexa del politopo del matching viene dada por
\begin{align*}
\sum_{(u,v)\in\delta(u)} x_{uv} &\leq 1, \quad \forall u\in V\\
x_{uv} &\geq 0\\
\sum_{(u,v)\in \delta(S)} x_{uv}& \geq 1 \quad \forall S \in \mathcal{A}(G)	
\end{align*}
Además, si nos restringimos a los emparejamientos perfectos, entonces basta considerar únicamente la igualdad en el primer bloque de restricciones.
\end{thm}
Utilizando este teorema, podemos formular el MCPM como
\begin{align*}
\min_x & \sum_{(u,v) \in E}x_{uv}c_{uv}\\
s.a.&\;\sum_{(u,v)\in\delta(u)} x_{uv} \leq 1, \quad \forall u \in V\\
&\sum_{(u,v)\in \delta(S)} x_{uv} \geq 1\quad \forall S \in \mathcal{A}(G)	\\
&x_{uv} \geq 0 \qquad \forall(u,v)\in E
\end{align*}
Con esta formulación del MCPM vamos a considerar el problema dual asociado.
\begin{align*}
\max_{y} &\; \sum_{v\in V} y_v - \sum_{S\in \mathcal{A}(G)} y_S\\
s.a.&\;y_u+y_v + \sum_{S\in B,(u,v)\in \delta(S)}y_S  \geq c_{uv} \quad \forall (u,v)\in E\\
&y_S\geq 0 \quad \forall S\in \mathcal{A}(G)
\end{align*}
\begin{defi}
Sea $y$ una solución factible del problema dual anterior. Definimos el coste reducido de un eje $(u,v)$ con respecto a $y$
$$
c_{uv}'(y) = c_{uv} - y_u -y_v - \sum_{S\in \mathcal{A}(G),(u,v)\in \delta(S)} y_S
$$ 

\end{defi}
\section{Fundamentos del algoritmo}
\begin{defi}
Sea $M$ un matching en $G$ y sea $P$ un camino (o ciclo) alternante con respecto a $M$. Definimos la longitud de $P$ respecto a $M$ como
$$
l(P) = \sum_{(u,v)\in P\setminus M}c_{uv} - \sum_{(u,v)\in P\cap M}c_{uv}
$$
\end{defi}
\begin{defi}
Sea $M$ un matching en $G$ y sea $s\in V$ expuesto con respecto a $M$. Definimos $\mathcal{P}(M)$ como el conjunto de todos los caminos de aumento con respecto a $M$.
\end{defi}
\begin{prop}
En las condiciones de la definición anterior se tiene que
$$
c(M\oplus P) = l(P) + c(M)
$$
\end{prop}

Usando estas definiciones podemos probar ahora una primera caracterización de optimalidad para el MCPM.
\begin{thm}
Un matching perfecto es de coste mínimo si y solo si no existe un ciclo alternante de longitud negativa.
\end{thm}
\begin{dem}
Demostremos el enunciado equivalente, un matching perfecto no es de coste mínimo si y solo si existe un ciclo alternante de longitud negativa.

Si existe un ciclo alternante de longitud negativa $P$ con respecto a un matching $M$ perfecto, entonces se deduce de la proposición anterior que $c(M\oplus P) < c(M)$. 

Sea un matching $M$ perfecto que no sea de coste mínimo y sea entonces $M'$ un matching de coste mínimo. Sabemos por teoría de grafos que existe un conjunto finito de ciclos alternantes $P_1,\dotsc,P_r$ con respecto a $M$ tales que
$$
M' = M \oplus P_1 \oplus \cdots \oplus P_r
$$
Si $l(P_i) \geq 0$ $\forall i=1,\dotsc, r$ entonces $c(M') \geq c(M)$, pero por hipótesis $c(M')<c(M)$. Por tanto, ha de existir $i \in \{1,\dotsc,r\}$ tal que $l(P_i)<0$. \qed
\end{dem}
A continuación enunciamos y probamos un resultado crucial para el algoritmo.
\begin{thm}
Sea $M$ un matching que no admite ningún ciclo alternante de longitud negativa y sea $P$ el camino alternante de menor coste (\textit{shortest alternanting path}) entre todos los relativos a $M$. Entonces $M\oplus P$ no admite ningún ciclo alternante de longitud negativa.\qed
\end{thm}
\begin{dem}
Sea $K$ un ciclo alternante de longitud negativa con respecto a $M\oplus P$. Podemos suponer que $K\cap P \neq \emptyset$, pues en otro caso $K$ sería un ciclo alternante de longitud negativa con respecto a $M$. Definamos entonces
$$
P' = (P\setminus K)\cup (K\setminus P)
$$
Entonces $P'$ es un camino de aumento con respecto a $M$ y
$$
c(M \oplus P') = c(M \oplus P \oplus K) < c(M \oplus P)
$$
Por tanto $l(P')<l(P)$ (con respecto a $M$), lo cuál es una contradicción.
\qed
\end{dem}
\subsection{Transformaciones admisibles}
\begin{defi}
Sea $M$ un matching y sea $\mathcal{P}(M)$ definido en la Definición 2.2.2. Una transformación $T:c_{uv}\to c_{uv}'$ es llamado \textbf{admisible} si
$$
l'(P)\geq 0\quad \forall P \in \mathcal{P}(M)
$$
Siendo $l'(P)$ la longitud de $P$ con respecto a $M$ y los costes $c'_{uv}$. Además, se define $d(P):= l(P)-l'(P)$.
\end{defi}
A partir de la definición de transformación admisible podemos probar un criterio de optimalidad para que un camino de aumento sea de longitud mínima.
\begin{thm}
Sea $T$ una transformación admisible y $P_0 \in \mathbb{P}(M)$ con $l'(P_0) =0$ y 	$d(P_0)\leq d(P)$ $\forall P \in \mathcal{P}(M)$. Entonces $P_0$ es el camino de aumento con menor longitud con respecto a la función de coste $c$ y matching $M$.
\end{thm}
\begin{dem}
Se deduce inmediatamente a partir de la definición y las hipótesis a través de la siguiente cadena de igualdades / desigualdades.
\begin{align*}
l(P_0) &= l'(P_0) + d(P_0)\\
&=d(P_0)\leq d(P) \leq l'(P) + d(P) = l(P)  \quad \forall P \in \mathcal{P}(M)
\end{align*}
\qed
\end{dem}
\begin{defi}
Sea $B=\{S\subset V \mid |S| \text{ es impar, }|S|\geq 3\}$. A cada $u\in V$ y $S\in B$ podemos asociar unos pesos $y_u \in \R$ e $y_S \in \R_{\geq 0}$. Definimos la transformación $T:c_{uv}\to c'_{uv}$ tal que
$$
c_{uv}' =c_{uv}-y_u-y_v - \sum_{S\in B, (u,v)\in \delta(S)}y_S
$$
\end{defi}
\begin{nota}
Notemos que si consideramos $y$ una solución factible del problema dual del matching, la transformación asocia a los costes de cada arista los costes reducidos asociados a $y$, pues puede establecerse a través de las condiciones de holgura complementaria que si $S$ no pertenece a una familia de contracción, $y_S =0$.
\end{nota}
\begin{defi}
En las condiciones de la definición anterior, definimos el \textbf{grafo de admisibilidad} $G' = (V,E')$ donde $E' = \{(u,v)\in E\mid c'_{uv}=0\}$.
\end{defi}
\begin{defi}
Una transformación admisible $T$ del tipo definida en la Definición 2.2.4 se dice \textbf{adecuada} con respecto a $M$ si existe una familia de contracción $A$ en $G'$ tal que
\begin{align*}
(u,v)\in M &\Rightarrow c'_{uv} = 0\\
y_S>0&\Rightarrow S\in A \text{ y } |M\cap \gamma(S)| = \frac{1}{2}(|S|-1)
\end{align*}
\end{defi}
Con estas definiciones en mente pasamos a probar el criterio de optimalidad que utilizaremos en el nuestro algoritmo, cuya demostración puede encontrarse en \cite{derigs}.
\begin{thm}
Sea $M$ un matching y sea $T$ una transformación adecuada. Sea $A = \{S \in B\mid y_S>0\}$. Si existe $P\in \mathbb{P}(M)$ tal que $c'(P)=0$ y $P_A$ es un camino de aumento respecto $M_A$, entonces $P$ es el camino alternante de menor costo en $\mathcal{P}(M)$.
\end{thm}
\begin{dem}
Sea $s\in V$, consideremos todos los caminos de aumento para $M$ que tienen como nodo inicial $s$. Sea $P'$ un camino de este tipo. Sea $S\in A$, entonces
\begin{align*}
s\in S \Rightarrow & |(P'\cap M)\cap \delta(S)| = 0 \text{ y}\\
& |(P'\setminus M)\cap \delta(S)| = 1\\
s\notin S \Rightarrow & |(P'\setminus M)\cap \delta(S)| - |(P'\cap M)\cap \delta(S)| \geq 0
\end{align*}

En particular, $l(P')\geq y_s + \sum_{s\in S}y_S$. Dado que que $P_A$  es un camino de aumento con respecto a $M_A$, se verifica
\begin{align*}
s\in S \Rightarrow & |(P\setminus M)\cap \delta(S)|\\
s\notin S \Rightarrow & |(P\setminus M)\cap \delta(S)| = |(P\cap M)\cap \delta(S)| =1
\end{align*}
y, por tanto, $l(P) = y_s  + \sum_{s\in S} y_S$. Aplicando el Teorema 2.2.3 tenemos el resultado.
\end{dem}
\section{Algoritmo del camino de aumento más corto}
Durante el procedimiento un emparejamiento $M$, una familia de contracción $A$ y unos peso $y_u$ ($u \in V$) e $y_S$ ($S$ conjunto impar de vértices) definen una transformación admisible $T:c_{uv}\to c'_{uv}$ que verifica además que $c'_{uv}=0$ si $(u,v)\in M$. Todos los pasos del algoritmo están constituidos en el grafo $G\times A$. En \cite{inte}, utilizan distintas notaciones para los costes $\pi_v$ asociados a los nodos e $y_S$ para los asociados conjuntos impares. Sin embargo, los costes no nulos $y_S$ asociados con un pseudonodo $v_S$ representa el respecto $S\in A$. Que estos costes puedan entenderse como asociados a un pseudonodo justifica que vayamos a utilizar la notación $y$ para ambos pesos. 

Antes de dar el algoritmo es preciso notar que utilizaremos un etiquetado de manera que todo nodo de $G\times A$ tendrá tres etiquetas temporales $d_v^+,d_v^-$ y $p(v)$ con la información:
\begin{itemize}
\item $d_v^+$ es el límite superior de la longitud del camino alternante más corto que conecta $v$ con un cierto nodo $s$ y tiene un número par de ejes.
\item $d_v^-$ es análogo para un número impar de ejes.
\item $p(v)$ es el predecesor de $v$ en el camino alternante que define $d_v^-$. 
\end{itemize}
Estas distancias y medidas están tomadas respecto de los costes $c'_{uv}$. Pasemos a describir el algoritmo.
\begin{align*}
\text{Inicialización: }& y_u:= y_S:= 0 \quad \forall u \in V,\; S\in B\\
& M:=A:=\emptyset\\
& c_{uv}' = c_{uv} \quad \forall (u,v)\in E\\
\text{Paso 1: }& \text{Determinar si existe un nodo $s$ no emparejado en $G\times A$ con respecto a $M_A$.}\\
&\text{Si no existe ninguno, ir al Paso 8.}\\
&\text{En caso contrario, ir al Paso 2.}
\end{align*}
\begin{align*}
\text{Paso 2: }& \text{[Inicialización de una fase.]}\\
&d_v^+ := \infty \quad \forall j\in G\times A,\; v\neq s\\
&d_s^+:=0\\
&d_v^-:=c'_{sv},\;p(v):=s\quad \forall j\in G\times A,\; v\neq s\\
&d^-_s:=\infty\\
&\text{Etiquetamos $s$ con una $S$ y avanzamos al Paso 3.}\\
\text{Paso 3: }& \text{Determinamos:}\\
&\delta_1 = \min\{d_v^-\mid \text{$v$ no está etiqetado ni emparejado}\},\\
&\delta_2 =\min\{d_v^-\mid \text{$v$ no está etiquedato pero sí emparejado}\},\\
&\delta_3 =\min\{\frac{1}{2}(\delta_v^+ + \delta_v^-)\mid \text{$v$ está $S$-etiquetado}\}\\
&\delta_4 =\min\{y_B+d_B^- \mid \text{$v_B$ es un pseudonodo $T$-etiquetado}\}\\
&\delta\;= \min\{\delta_1,\delta_2,\delta_3,\delta_4\}\\
&\text{Si $\delta=\delta_1$ ir al Paso 7}\\
&\text{Si $\delta=\delta_2$ ir al Paso 4}\\
&\text{Si $\delta=\delta_3$ ir al Paso 5}\\
&\text{Si $\delta=\delta_4$ ir al Paso 6}\\
\text{Paso 4: }&\text{[Crecimiendo de un arbol alternante.]}\\
&\text{Sea $u$ el nodo que define $\delta$ y sea $(u,v)\in M_A$. Entonces el árbol crece}\\
&\text{asignando un $T$-etiquetado a $u$ y un $S$ etiquetado a $v$.}\\
&\text{Establecemos $d^+_v:=\delta$ y escaneamos $v$, es decir,}\\
&d^-_k:= \min\{d_k^-,d_j^++c'_{vk}\}\quad \forall k\neq u,v\\
&p(k):=v \text{ si } d_k^-= d_v^++c'_{vk}\\
&\text{Ir al Paso 3}\\
\text{ Paso 5: }&\text{[Contracción de un blossom.]}\\
&\text{Sea $u$ el nodo que define $\delta$ y sea $v:=p(u)$. Introduciendo la arista $(u,v)$}\\
&\text{al arbol alternante genera un blossom $B$. Este blossom se contrae en un}\\
&\text{pseudono $v_B$. Establecemos:}\\
&A:=A-\{B\},\;d^+_B:=\delta,\;y_B:=0 \text{ y asignamos una $S$-etiqueda a $v_B$}\\
&\text{Para todo $i \in B$ $S$-etiqueado}\\
&y_i:=y_i + (\delta-d_i^+)\\
&c_{ij}' := c_{ij}' -(\delta-d_i^+) \quad \forall j\neq i\\
&\text{Para todo $i \in B$ $T$-etiquedado}\\
&y_i:=y_i + (d^-_i-\delta)\\
&c_{ij}' := c_{ij}' -(d_i^- -\delta) \quad \forall j\neq i\\
&\text{Escaneamos $v_B$ (como en el Paso 4 para el nodo $v$) y vamos al Paso 3.}
\end{align*}
\begin{align*}
\text{Paso 6: }&\text{[Expansión de un pseudonodo.]}\\
&\text{Sea $v_B$ el pseudonodo que define $\delta$. Establecemos}\\
&c_{uv}':=c_{uv}'+y_B\quad \forall (u,v)\in\delta(B)\\
&y_B:=0\\
&\text{y extendemos el matching $M_A$. El blossom $B$ se particiona en dos}\\
&\text{caminos, uno tiene un número par de ejes y el otro un número impar}\\
&\text{Todos los nodos $v$ del camino par se etiquetan con $S$ y $T$ alternativamente}\\
&\text{y $d_v^+:=d_v^-:=\delta$. Los nodos $S$-etiqueados son escaneados. Los nodos $u$}\\
&\text{del camino impar no se etiquetan. Establecemos}\\
&d^-_v=\min\{d_i^++c_{iu}'\mid i\text{ está $S$ etiquetado}\}\\
&\text{y $p(v):=i$ si el mínimo se alcanza en $i$. Si no existe, $d_v^-:=\infty$.}\\
&\text{Finalmente, $A:=A\setminus\{B\}$. Vamos al Paso 3.}\\
\text{Paso 7: }&\text{[Aumento.]}\\
&\text{Hemos detectado un camino de aumento $P_A$ con respecto a $M_A$.}\\
&\text{Actualizamos $M_A = M_A \oplus P_A$ y definimos y definimos para todos los}\\
&\text{(pseudo)nodos $S$-etiquetados $v$:}\\
&y_v:=y_v+(\delta-d_v^+)\\
&c_{uv}' := c_{uv}' - (\delta-d_i^+) \quad u\neq v\\
&\text{y para los $T$-etiquedados}\\
&y_v:=y_v-(\delta-d_i^-)\\
&c_{uv}'=c_{uv}'+(\delta-d_i^-)\\
&\text{Eliminamos todas las etiquetas y vamos al Paso 1}\\
\text{Paso 8: }& \text{Expandimos todos los pseudonodos y extendemos el matching $M_A$ a un}\\
&\text{matching perfecto $M$ en $G$. Este matching es óptimo.} 
\end{align*}
\newpage
\section{Método de reoptimización}
En esta sección vamos a presentar un método de reoptimización para el problema del MCPM y cómo aplicarlo al problema biobjetivo. Antes de presentar este método hemos revisado diversos análsis de postoptimalidad como el de Ball y Taberna \cite{balltab} o el de Weber \cite{webber}, pero el que presenta mejores resultados computacionales y supone una simplificación importante en el número de casos a considerar es el que presenta Derigs. 

Nuestro objetivo en esta sección es, suponiendo que tenemos un matching óptimo $M$ para el MCPM, vamos a alterar el coste de ciertos ejes. Denotemos $c_{uv}^{new}$ los costes nuevos para estas aristas y sea $c_{uv}'^{new}$ los correspondientes costes reducidos respecto de una solución factible dual $y$. Antes de proseguir, precisemos algunas definiciones.
\begin{defi}
Sea $y$ una solución factible del dual descrito gracias al Teorema 2.1.3. Consideremos $c'_{uv}:=c'_{uv}(y)$, es decir, los costes reducidos asociados a $y$. Consideremos $G'$ el grafo de admisibilidad asociado. Diremos que $y$ es \textbf{fuértemente dual factible} si $y_S>0$ implica que $S$ es contractible en $G'$.
\end{defi}
\begin{defi}
En las mismas condiciones de la definición anterior, diremos que $(M,y)$ es un \textbf{par compatible} si $y$ es fuertemente dual factible y se verifica
\begin{align*}
(u,v)\in M &\Rightarrow c'_{uv} = 0\\
y_S>0&\Rightarrow |M\cap \gamma(S)| = \frac{1}{2}(|S|-1)
\end{align*}
\end{defi}
Se puede probar que el método que hemos establecido anteriormente comienza con un par compatible $(M,y)$ y encuentra el menor camino de aumento $P$ y una solución fuertemente dual factible $y'$ de manera que $(M\oplus P,y')$ es un par compatible.
\begin{prop}
Sea $M$ un matching óptimo y sea $u\in V$ un nodo arbitrario. Entonces existe una solución $y$ fuertamente dual factible tal que $(M,y)$ es un par compatible y además $y_S=0$ $\forall S \in \mathcal{A}(G)$ tal que $u \in S$.
\end{prop}
\begin{dem}
La demostración es constructiva. Podemos suponer que tenemos $(M,y)$ un par compatible obtenido por el algoritmo SAP. Si $y$ verifica la propiedad adicional, no hay nada que demostrar. En caso contrario, sea $v$ tal que $(u,v)\in M$. Vamos a construir un grafo $\overline{G}$ introduciendo dos nuevos nodos artificiales $r,t\notin V$ y aristas $(r,u)$ y $(v,t)$ con costes 
$$
c_{ru}:= y_u+\sum_{u\in S}y_S \qquad c_{vt} = y_v + \sum_{u\in S}y_S  +\sum_{v\in S}y_S 
$$
Además, extendemos $y$ al nuevo grafo $\overline{G}$ de manera que $y_r:=y_t:=0$ e $y_S = 0$ $\forall S \in \mathcal{A}(\overline{G})$ tal que $S\cap\{r,t\}\neq \emptyset$. Esta extensión hace que $(M,y)$ sea un par compatible en $\overline{G}$. Aplicamos ahora el algoritmo del SAP. Como $M$ era óptimo en $G$, el único camino de aumento que puede generarse es $\{(r,u),(u,v),(v,t)\}$. Dado que $c_{vt}>\sum_{u\in S}y_S$, antes de que se detecte este camino la solución dual será alterada a $y'$ de manera que $y'_S = 0$ $\forall S \in \mathcal{A}(G)$ tal que $u \in S$. Obtenemos por tanto un par $(M,y')$ que verifica la propiedad deseada. \qed
\end{dem}
Es importante hacer una consideración para entender por qué es importante el teorema anterior. Sea $y$ una solución fuertemente dual factible, definimos $\mathcal{A}(y)=\{S\in \mathcal{A}(G)\mid y_S>0\}$. Naturalmente, si existe algún $S \in \mathcal{A}(G)$ con $v\in S$ tal que $y_S >0$, el nodo $v$ no estará -estará contraído en algún pseudonodo- en el grafo $G\times \mathcal{A}(y)$. Por tanto, siguiendo el proceso de la demostración anterior, siempre podemos encontrar un par compatible $(M,y)$ de manera que un determinado vértice $v$ esté en el grafo $G\times \mathcal{A}(y)$. Pasamos a ver el algoritmo de repotimización. 

Supongamos que todas las aristas cuyos costes son alteradas están en $\delta(v)$. Naturalmente, Si $c_{ij}'^{new} \geq 0$ $\forall (u,v) \in \delta(v)$, nuestro matching sigue siendo óptimo, luego podemos suponer que hay alguno que es negativo.
\begin{align*}
\text{Inicialización: }& \text{Sea $(M,y)$ un par compatible óptimo para MCPM}\\
\text{Paso 1: }& \text{Si $v$ no está contraído en $G\times \mathcal{A}(y)$ vamos al Paso 2.}\\
&\text{En otro caso, aplicamos el procedimiento anterior}\\
&\text{ y denotamos $y:=y'$. Vamos al Paso 2.}\\
\text{Paso 2: }& \text{Sea $(u,v) \in M$ establecemos}\\
&\overline{M}:=M\setminus\{(u,v)\}\\
&\delta:=\min\{c_{vk}'^{new}\mid (v,k)\in \delta(v)\}<0\\
&c_{vk}:=c_{vk}^{new} \quad \forall (v,k)\in \delta (v)\\
&y_v:=y_v+\delta\\
&\text{Ir al Paso 3.}\\
\text{Paso 3: }&\text{Aplicar el algoritmo SAP a $(\overline{M},y)$ y determinar el menor}\\
&\text{camino de aumento $P$ y una solución dual $y'$ tal que el par}\\
&\text{$(\overline{M}\oplus P,y')$ sea un par compatible}
\end{align*}
A continuación explicamos explícitamente cómo aplicar el oráculo anterior para reoptimizar en general un problema de matching y añadimos además una forma algorítima del método.

El proceso de reoptimización que muestra Derigs solo nos permite alterar los costes de las aristas relativas a un vértice en concreto. Por tanto, tenemos que invocarlo secuencialmente. Partimos de un par compatible $(M,y)$ óptimo. Calculamos $V' \subset V$ tal que $\delta(V') \cup \gamma(V')= E$. 	

Enumeramos los vértices de $V' = \{u_1,\dotsc,u_s\}$. En el primer paso, llamamos al método de repotimización inicializandolo mediante $(M,y)$, el vértice $u_1$ y los costes $c_{uv}^{new}$ $\forall (u,v) \in \delta(u_1)$. El oráculo devuelve un par compatible $(M_1,y_1)$. En el paso general, partimos del par compatble $(M_i,y_i)$ y tomamos el vértice $u_{n+1}$ e inicializamos el proceso de repotimización junto con los costes $c_{uv}^{new}$ $\forall (u,v) \in \delta(u_{i+1})$. 
De manera algoritmica, podemos escribir:
\begin{align*}
\text{Inicialización: }& \text{Sea $(M,y)$ un par compatible óptimo para MCPM y}\\
& \text{sea $V'=\{u_1,\dotsc,u_s\}$ un conjunto de vértices tales que }\\
& \text{$\delta(V')\cup\gamma(V')=E$.}\\
\text{Paso 1: }& \text{Invocamos el proceso de repotimización con (M,y), el vértice $u_1$}\\
&\text{y la función de costes $k^1$, que mantiene los mismos costes}\\
&\text{$k^1_{uv} = c_{uv}$ $\forall (u,v)\notin \delta(u)$ y $k^1_{uv}=c^{new}_{uv}$ $\forall(u,v) \in \delta(u)$.}\\
&\text{Obtenemos un par compatible óptimo $(M_1,y_1)$.}\\
&\text{Avanzamos al Paso 2}\\
\text{Paso 2: }& \text{Para todo $n =2,\dotsc,s$}\\
&\text{		Invocamos el proceso de repotimización partiendo de}\\
&\text{		el par compatible $(M_{n-1},y_{n-1})$, el vértice $u_n$ y la función}\\
&\text{		de costes $k^n$ definida como $k^n_{uv} =k^{n-1}_{uv}$ si $(u,v)\notin \delta(u_n)$ y}\\
&\text{		$k^n_{uv}=c^{new}_{uv}$ si $(u,v)\in \delta(u_n)$. Al final de cada iteración }\\
&\text{obtenemos un par compatible $(M_n,y_n)$ con el cuál se}\\
&\text{inicializa la siguiente iteración, hasta que $n=s$. En este caso}\\
&\text{avanzamos al Paso 3.}\\
\text{Paso 3: }&\text{El par compatible $(M_s,y_s)$ es óptimo para $G$ con la función}\\
&\text{de coste $c^{new}$}
\end{align*}
Notemos que es posible que algunos o todos las aristas cuyos costes alteramos en un paso general pueden haber sido alteras en otros anteriores, pero tiene un efecto inocuo en el desarrollo del algoritmo. 

Los conjuntos $W$ que verifican que $\delta(W)\cup\gamma(W)=E$ se denominan \textit{coberturas por vértices} de $G$. En general, el problema de decidir si un grafo $G$ tiene una cobertura por vértices de tamaño $k$ es NP-completo. A pesar de ello, por la observación anterior, tampoco es estrictamente necesario que obtengamos una cobertura de tamaño mínimo para inicializar el algoritmo. 

\subsection{Problema biobjetivo}
En el matching paramétrico, tal y como vimos en la Subsección 1.1.1, la función de coste es $c_{uv}(\lambda) = c^1_{uv} + \lambda c^2_{uv}$. Por tanto, para aplicar el procedimiento anterior, tenemos que considerar $c_{uv}^{new} = c^1_{uv} + (\lambda + \Delta) c^2_{uv}= c_{uv}(\lambda+\Delta)$. Notemos que $\Delta$ no tendría por qué ser un parámetro real, sino que podríamos considerar $\Delta \in \R^{|E|}$. En ese caso, $c_{uv}^{new} = c^1_{uv} + (\lambda + \Delta_{uv}) c^2_{uv}$ 
\section{Mantenimiento de la optimalidad}
Estamos especialmente interesados en el caso del matching biobjetivo en saber cuánto puede variar $\Delta$ de manera que nuestro matching $M$ continúe siendo óptimo. Para realizar este análisis, consideremos el problema de Programación Lineal:
\begin{align*}
\min_x & \sum_{(u,v) \in E}x_{uv}(c^1_{uv} + \lambda c^2_{uv})\\
s.a.&\;\sum_{(u,v)\in\delta(u)} x_{uv} \leq 1, \quad \forall u \in V\\
&\sum_{(u,v)\in \delta(S)} x_{uv} \geq 1\quad \forall S \in \mathcal{A}(G)	\\
&x_{uv} \geq 0 \qquad \forall(u,v)\in E
\end{align*}
Sea $A$ la matriz de restricciones sin considerar la introducción de las variables de holgura. Si $|V|=2n$, una base $B$ óptima tendrá, en primer lugar, $n$ columnas de $A$ correspondientes a las aristas del matching según el indexado que hayamos hecho de las mismas; y otras $s$ columnas de la base canónica para las holguras cuyas desigualdades se verifiquen de manera estricta. Sean $i_1,\dotsc,i_n$ los índices asociados a las aristas del matching, sean $j_1,\dotsc,j_k$ los índices asociados a dichas variables de holgura, sea $N$ el número total de restricciones del problema tras la introducción de las holguras y denotemos por $e_p$ el $p$-ésimo vector de la base canónica de $R^N$, entonces la matriz $B$ es precisamente
$$ B= 
\begin{pmatrix}
A_{\cdot i_1} & \cdots & A_{\cdot i_n} & e_{j_1} & \cdots & e_{j_k}
\end{pmatrix}
$$
Si denotamos por $b$ la columna de términos independientes y por $N$ las columnas no básicas de $A$, sabemos que la condiciones que certifican que $B$ es una base óptima son
\begin{align*}
B^{-1}b\geq0\\
\overline{c_R}^t = c_N^t - c_B^t B^{-1}N\geq 0
\end{align*}
\subsection{Problema biobjetivo}
Realizamos el análisis para el caso en $\lambda$ es un único parámetro escalar común. Alterar $\lambda$ no modifica en ningún caso la primera condición, luego nos centraremos solo en la segunda. 
\begin{align*}
\overline{c_R}(\lambda+\Delta)^t&= c_N(\lambda + \Delta)^t  - c_B(\lambda + \Delta)^t B^{-1}N \\
& = c_N(\lambda)^t  - c_B(\lambda)^t B^{-1}N + \Delta {c_N^2}^t  - \Delta {c_B^2}^t B^{-1}N \\
&= \overline{c_R}(\lambda)^t + \Delta \overline{c_R^2}^t \geq 0
\end{align*}
Si simplemente utilizamos la aproximación clásica a la reoptimización de un problema de Programación Lineal, sabemos que $B$ se mantiene como base óptima siempre que
$$
\max_{(u,v):\overline{c^2}_{uv}>0} -\frac{\overline{c}(\lambda)_{uv}}{\overline{c^2}_{uv}} \leq \Delta \leq \min_{j:\overline{c^2}_{uv}<0} - \frac{\overline{c}(\lambda)_{uv}}{\overline{c^2}_{uv}}
$$
Sea $y$ una solución del problema dual asociado para $\lambda$, entonces sabemos que para cualquier arista $(u,v)\in E$, en particular para una arista representada por una variable no básica, el coste reducido es precisamente 
$$
\overline{c}(\lambda)_{uv} =c_{uv}+\lambda-y_u-y_v - \sum_{S\in B, (u,v)\in \delta(S)}y_S
$$

Una vez estudiado el caso anterior, consideremos un caso más genérico en el que $\Delta \in \R^{|E|}$. Para simplificar la notación, sea $\lambda \in \R$ y $\Delta \in \R^{|E|}$ podemos escribir $c(\lambda+\Delta)$, donde
$$
c(\lambda+\Delta)_{uv} = c^1_{uv}+(\lambda+\Delta)c^2_{uv}
$$
y análogamente para $c^2$. Podemos considerar entonces
\begin{align*}
\overline{c_R}(\lambda+\Delta)^t&= c_N(\lambda + \Delta_N)^t  - c_B(\lambda + \Delta_B)^t B^{-1}N \\
& = c_N(\lambda)^t  - c_B^2(\lambda)^t B^{-1}N + c_N(\Delta_N)^t  - c_B^2(\Delta_B)^t B^{-1}N \\
&=\overline{c_R}(\lambda)^t + \overline{c_R}^2(\Delta)^t \geq 0
\end{align*}
Cada variable no básica nos da una desigualdad sobre $\Delta$, de manera que tenemos un poliedro en $\R^{|E|}$ de valores admisibles. En ambos casos, al llegar a uno de los vértices del poliedro (el extremo del intervalo en el caso más simple), podemos escoger una nueva base y repetir el análisis.

\subsection{Lema de Farkas}
En Programación Lineal el Lema de Farkas es un resultado capital. 
\begin{lem}
Sea $A\in\R^{m\times n}$ y $b\in \R^m$. Entonces exactamente uno de los siguientes enunciados es cierto:
\begin{itemize}
\item Existe $x\in\R^n$ tal que $Ax=b$ y $x\geq 0$.
\item Existe $y\in \R^m$ tal que $y^t A \leq0$ e $y^t b>0$. 
\end{itemize}
\end{lem}
Por otro lado, sabemos que una base $B$ es óptima si la función objetivo, es decir, $c(\lambda)$ pertenece al cono que genera $B$. Se tiene que
$$
c(\lambda) \in cone(B) \Leftrightarrow \exists x\geq0 \mid Bx = c(\lambda) 
$$  
Usando Farkas, esto es equivalente a $\not\exists y \mid u^tB \leq 0$ y $u'c(\lambda)>0$. Por tanto, $B$ es base óptima para $c(\lambda)$ si y solo si
$$u'B \leq 0 \Rightarrow u^t c(\lambda) \leq 0 \qquad u^t c(\lambda)>0 \Rightarrow u^tB >0$$









\chapter{Análisis computacional}
En esta sección presentamos los resultados del análisis de sensitividad del problema del emparejamiento biobjetivo sobre distintos grafos. Para los distintos valores de $\lambda$ hemos resuelto en Python 3.6 utilizando el software Gurobi 8.0 los correspondientes problemas de Programación Lineal. Seguidamente se adjunta dicho código.
\newline
\begin{pythone}
 # Comenzamos el codigo para resolver el problema del matching mediante resolucion directa del problema de Programacion Lineal.
# Utilizamos Python 3.6 como interfaz para el software Gurobi.
# El siguiente codigo genera una solucion y valor objetivo asociado para cada elemento de un mallado de valores de lambda, de manera que podemos observar como varian ambos en funcion de dicho parametro.
from gurobipy import *
import numpy as np
import itertools 
import operator

# Numeramos los vertices comenzando en 0.
		
# Matriz de adyacencias del grafo.
A = np.array([[0,1,1,1],[1,0,1,0],[1,1,0,0],[1,0,0,0]])

# Numero de vertices
N = len(A)

# Comencemos creando el conjunto de indices que vamos a utilizar para las variables
def indices(n,a):
    s = []
    for i in range(n):
        for j in range(i+1,n):
            if (a[i,j] == 1):
                s.append((i,j))
    return(s)

# Calculamos una vez el conjunto de indices
IND = indices(N,A)


# Definimos la funcion delta, que nos da los vertices
# ejes adyacentes a un vertice dado.
def delta(v):
    s = []
    for (a,b) in IND:
        if (a==v or b==v):
            s.append((a,b))
    return(s)

# Calculamos una vez el conjunto delta
deltaV = [delta(v) for v in range(N)]

# A continuacion, definimos la funcion gamma, que dado un conjunto
# de vertices, debe devolvernos las aristas que tienen extremos dentro
# de dicho conjunto.
def gamma(S):
    m = []
    s = []
    for v in S:
        s = s + delta(v)
    seti = set(s)
    for (a,b) in seti:
        if (a in S and b in S):
            m.append((a,b))
    return(m)

def subconj(N,m):
    S = list(range(N))
    return list(itertools.combinations(S, m))

# Calculamos los conjuntos impares que dan lugar a restricciones
IMPARES = []
for i in range(3,N,2):
    IMPARES = IMPARES + subconj(N,i)


# Sea M el numero de puntos en el que queremos dividir el intervalo [0,1]
M = 10
lambdas = np.array(range(0,M+1,1))/M

def lambdafun(lam):
    # Creamos un modelo
    m = Model("escenarios");
    
    # Creamos las variables. El tipo puede ser 
    # 'C' para continuas, 'B' para binarias, 'I' para enteras,
    # 'S' para semicontinuas, or 'N' for semienteras.
    # El parametro lb nos da una cota inferior a las variables.
    x = m.addVars(IND, lb = 0, vtype='C', name = "x");
    
    # Damos la siguiente función para ayudarnos a escribir las restricciones
    def suma(S):
        s = 0
        for (a,b) in S:
            s = s + x[a,b]
        return(s)
    # También diseñamos una para escribir las funciones objetivo
    def objind(c):
        s = 0
        for i in range(len(IND)):
            (a,b) = IND[i]
            s = s + c[i]*x[a,b] 
        return(s)
    
    # De esta forma, generamos las dos funciones objetivo. Notemos que 
    # escribimos directamente los costes sobre las variables aristas
    # del modelo, pues si una arista no está en el grafo, no tiene sentido
    # generar una variable para ella.       
    c1 = [20,0,0,20,20,20,20,0]
    c2 = [0,20,20,20,20,0,0,20]       
    
    cos1 = objind(c1)
    cos2 = objind(c2)
    
    # Definimos la funcion objetivo y queremos maximizar/minimizar.
    m.ModelSense = GRB.MINIMIZE
    
    # Añadimos la funcion objetivo
    m.setObjective((1-lam)*cos1+lam*cos2); 
    
    # Pasamos a generar las restricciones.
    # Imponemos que cada vertice tenga una pareja.
    for i in range(N):
        m.addConstr(suma(delta(i)) == 1);
    
    # Imponemos las condiciones sobre los conjuntos impares.
    for a in IMPARES:
        S = gamma(a)
        card = len(a)-1
        m.addConstr(suma(S) <= card/2)
    m.optimize();
    
    # Preparamos la salida por pantalla
    s=[]
    for i in range(len(x)):
        q = m.getVars()
        v = q[i]
        s.append(v.x)
      
    return([m.objVal,s])

    
\end{pythone}
\begin{tikzpicture}[scale=5,
    vertex/.style={draw,circle},
    arc/.style={draw=blue!#10,thick,->},
    arc label/.style={fill=white, font=\tiny, inner sep=1pt}
    ]
    \foreach [count=\i] \coord in {(0.809,0.588),(0.309,0.951),(-0.309,0.951),(-0.809,0.588),(-1.,0.),(-0.809,-0.588),(-0.309,-0.951),(0.309,-0.951),(0.809,-0.588),(1.,0.)}{
        \node[vertex] (p\i) at \coord {\i};
    }
\end{tikzpicture}

\begin{thebibliography}{9}
\bibitem{edmond} 
Edmonds, J. (1965). Maximum matching and a polyhedron with 0, 1-vertices. Journal of Research of the National Bureau of Standards B, 69, 125--130. 

 
\bibitem{holland} 
Grötschel, M.; Holland, O. Mathematical Programming (1985) 33: 243. https://doi.org/10.1007/BF01584376

\bibitem{pruebaed} 
Schrijver, Alexander. (1983). Short proofs on the matching polyhedron. Journal of Combinatorial Theory, Series B. 34. 104-108. 10.1016/0095-8956(83)90011-4. 
 
\bibitem{balltab}
Ball, M. O.,  Taverna, R. (1985). Sensitivity analysis for the matching problem and its use in solving matching problems with a single side constraint. Annals of Operations Research, 4(1), 25-56.

\bibitem{papa}
Papadimitriou, C.H.; Steiglitz, K. (1998), Combinatorial optimization: algorithms and complexity, Mineola, NY: Dover, pp.308-309.

\bibitem{nico}
Nicos Christofides, Worst-case analysis of a new heuristic for the travelling salesman problem, Report 388, Graduate School of Industrial Administration, CMU, 1976

\bibitem{inte}
George L. Nemhauser and Laurence A. Wolsey. 1988. Integer and Combinatorial Optimization. Wiley-Interscience, New York, NY, USA.
\bibitem{edmon2}
J. Edmonds, “Paths, Trees, and Flowers,” Can. J. Math., 17,449-467 (1965).

\bibitem{edmon3}
Pulleyblank, William and Edmonds, Jack. (1970). Facets of 1-Matching Polyhedra. 10.1007/BFb0066196. 

\bibitem{derigs}
Derigs, U. (1981), A shortest augmenting path method for solving minimal perfect matching problems. Networks, 11: 379-390. doi:10.1002/net.3230110407

\bibitem{webber}
Weber, G. M. (1981), Sensitivity analysis of optimal matchings. Networks, 11: 41-56. doi:10.1002/net.3230110105

\bibitem{rao}
Padberg, Manfred W., and M. R. Rao. “Odd Minimum Cut-Sets and b-Matchings.” Mathematics of Operations Research, vol. 7, no. 1, 1982, pp. 67–80. JSTOR, www.jstor.org/stable/3689360.
\end{thebibliography}
\end{document} 
