\documentclass[twoside,a4paper,openright,12pt]{book}
\usepackage{makeidx}
\usepackage{capt-of}
\usepackage[T1]{fontenc}
\usepackage{amsfonts}
\usepackage{mathtools,amscd,amsthm}
\usepackage{tabularx}
\usepackage{amssymb,eucal,bezier,graphicx}
\usepackage{times}
\usepackage{subfig}
\usepackage[svgnames]{xcolor}
\usepackage{fancybox}
\usepackage{fancyhdr}
\usepackage{hyperref}
\usepackage{enumerate}
\usepackage{array}
\usepackage{comment}
\usepackage[spanish]{babel}
\usepackage[utf8]{inputenc}
\usepackage[colorinlistoftodos]{todonotes}
\usepackage{anysize}
\usepackage{booktabs}
\usepackage{listings}
\usepackage{listingsutf8}
\usepackage{etoolbox}
% \usepackage{slashbox}
% \usepackage{verbatim}
% \usepackage[font=small]{caption}
% \usepackage{framed}
% \usepackage{cancel}
\usepackage{tikz}
\usetikzlibrary{snakes}
% \usepackage{epstopdf}
% \usepackage{float}
% Plantillas de código
\newcommand{\rstyle}{\lstset{ 
  language=R,                     % the language of the code
  basicstyle=\small\ttfamily, % the size of the fonts that are used for the code
  numbers=left,                   % where to put the line-numbers
  numberstyle=\tiny\color{Blue},  % the style that is used for the line-numbers
  stepnumber=1,                   % the step between two line-numbers. If it is 1, each line
                                  % will be numbered
  numbersep=5pt,                  % how far the line-numbers are from the code
  backgroundcolor=\color{white},  % choose the background color. You must add \usepackage{color}
  showspaces=false,               % show spaces adding particular underscores
  showstringspaces=false,         % underline spaces within strings
  showtabs=false,                 % show tabs within strings adding particular underscores
  frame=single,                   % adds a frame around the code
  rulecolor=\color{black},        % if not set, the frame-color may be changed on line-breaks within not-black text (e.g. commens (green here))
  tabsize=2,                      % sets default tabsize to 2 spaces
  captionpos=b,                   % sets the caption-position to bottom
  breaklines=true,                % sets automatic line breaking
  breakatwhitespace=false,        % sets if automatic breaks should only happen at whitespace
  keywordstyle=\color{RoyalBlue},      % keyword style
  commentstyle=\color{YellowGreen},   % comment style
  stringstyle=\color{ForestGreen},     % string literal style
     literate=%
         {á}{{\'a}}1
         {í}{{\'i}}1
         {é}{{\'e}}1
         {ý}{{\'y}}1
         {ú}{{\'u}}1
         {ó}{{\'o}}1
         {ñ}{{\~n}}1}}
         
\lstnewenvironment{erre}[1][]
{
\rstyle
\lstset{#1}
}
{}         

\definecolor{deepblue}{rgb}{0,0,0.5}
\definecolor{deepred}{rgb}{0.6,0,0}
\definecolor{deepgreen}{rgb}{0,0.5,0}

\newtoggle{InString}{}% Keep track of if we are within a string
\togglefalse{InString}% Assume not initally in string
\definecolor{majo}{HTML}{CD2626}
\newcommand*{\ColorIfNotInString}[1]{\iftoggle{InString}{#1}{\color{majo}#1}}%
\newcommand*{\ProcessQuote}[1]{#1\iftoggle{InString}{\global\togglefalse{InString}}{\global\toggletrue{InString}}}%


% Default fixed font does not support bold face
\DeclareFixedFont{\ttb}{T1}{txtt}{bx}{n}{12} % for bold
\DeclareFixedFont{\ttm}{T1}{txtt}{m}{n}{11}  % for normal

\newcommand{\pythonstyle}{\lstset{
  language=Python,                     % the language of the code
  basicstyle=\small\ttm, % the size of the fonts that are used for the code
  numbers=left,                   % where to put the line-numbers
  numberstyle=\tiny\color{DarkBlue},  % the style that is used for the line-numbers
  stepnumber=1,                   % the step between two line-numbers. If it is 1, each line
                                  % will be numbered
  numbersep=5pt,                  % how far the line-numbers are from the code
  backgroundcolor=\color{white},  % choose the background color. You must add \usepackage{color}
  showspaces=false,               % show spaces adding particular underscores
  showstringspaces=false,         % underline spaces within strings
  showtabs=false,                 % show tabs within strings adding particular underscores
  frame=single,                   % adds a frame around the code
  rulecolor=\color{black},        % if not set, the frame-color may be changed on line-breaks within not-black text (e.g. commens (green here))
  tabsize=2,                      % sets default tabsize to 2 spaces
  captionpos=b,                   % sets the caption-position to bottom
  breaklines=true,                % sets automatic line breaking
  breakatwhitespace=false,        % sets if automatic breaks should only happen at whitespace
  emph={range,len,print},          
  emphstyle=\ttb\color{deepred},
  keywordstyle=\color{RoyalBlue},      % keyword style
  commentstyle=\color{Grey},   % comment style
  stringstyle=\color{LimeGreen},     % string literal style
     literate=%
         {á}{{\'a}}1
         {í}{{\'i}}1
         {é}{{\'e}}1
         {ý}{{\'y}}1
         {ú}{{\'u}}1
         {ó}{{\'o}}1
         {ñ}{{\~n}}1
         {"}{{{\ProcessQuote{"}}}}1% Disable coloring within double q
         {'}{{{\ProcessQuote{'}}}}1% Disable coloring within single 
         {0}{{{\ColorIfNotInString{0}}}}1
    	 {1}{{{\ColorIfNotInString{1}}}}1
   	 	 {2}{{{\ColorIfNotInString{2}}}}1
   		 {3}{{{\ColorIfNotInString{3}}}}1
   		 {4}{{{\ColorIfNotInString{4}}}}1
   		 {5}{{{\ColorIfNotInString{5}}}}1
   		 {6}{{{\ColorIfNotInString{6}}}}1
   		 {7}{{{\ColorIfNotInString{7}}}}1
   		 {8}{{{\ColorIfNotInString{8}}}}1
   		 {9}{{{\ColorIfNotInString{9}}}}1
  }}
         
\lstnewenvironment{pythone}[1][]
{
\pythonstyle
\lstset{#1}
}
{}  

\marginsize{3.25cm}{3.25cm}{3cm}{3cm}

\newtheorem{defi}{Definici\'on}[section]
\newtheorem{ej}{Ejemplo}[section]
\newtheorem{ejs}{Ejemplos}[section]
\newtheorem{prop}{Proposici\'on}[section]
\newtheorem{nota}{Nota}[section]
\newtheorem{notac}{Notación}[section]
\newtheorem{rem}{Observaci\'on}[section]
\newtheorem{thm}{Teorema}[section]
\newtheorem{cor}{Corolario}[section]
\newtheorem{lem}{Lema}[section]
\newtheorem*{dem}{Demostración}

\providecommand{\abs}[1]{\left|{#1}\right|}
\providecommand{\conv}[1]{\overset{#1}{\longrightarrow}}
\providecommand{\convcs}{\xrightarrow{CS}}
\providecommand{\conve}{\xrightarrow{e}}
\providecommand{\func}[2]{\colon{#1}\longrightarrow{#2}}
\newcommand{\efe}{\hat{f}}
\newcommand{\R}{\mathbb{R}}
\newcommand{\D}{\mathbb{D}}
\newcommand{\Z}{\mathbb{Z}}
\newcommand{\N}{\mathbb{N}}
\newcommand{\E}{\mathbb{E}}
\newcommand{\fn}{\hat{f}_{0N}}
\newcommand{\X}{\overline{X}}
\newcommand{\dis}{\displaystyle}
\providecommand{\norm}[1]{\left\lVert#1\right\rVert}
\providecommand{\posi}[1]{\left[#1\right]^+}

% IRENITA--------------------------------
\renewcommand{\headrulewidth}{0.4pt} 
\fancyhead[RO,LE]{\thepage} 
\fancyhead[LO]{\nouppercase{\leftmark}}
\fancyhead[RE]{\nouppercase{\rightmark}}
\fancyfoot{}
\newcommand{\va}{\hat{\vartheta}_N}
\pagestyle{fancy}
% --------------------------------------

\setcounter{secnumdepth}{3}
\setcounter{tocdepth}{3}


\begin{document}
% ---------------------PORTADA
\begin{titlepage}

\vspace*{1in}
\begin{center}
\vspace*{-1in}
\begin{figure}[htb]
\begin{center}
\begin{large}
TRABAJO FIN DE MÁSTER\\
\end{large}
\rule{80mm}{0.1mm}\\
\vspace*{0.1in}
\end{center}
\end{figure}
\begin{large}
\end{large}

\vspace*{0.2in}
\begin{Large}
{\huge \bfseries El problema paramétrico del emparejamiento en grafos y
problema de emparejamiento con dos objetivos}\\[2cm]
\end{Large}

\begin{center} \Large
\emph{Presentado por:}\\
\textsc{ \bf{Rafael González López}}
\end{center}

\vspace*{0.2in}
\begin{center} \large
\emph{Supervisado por:} \\
\textsc{Dr.~Justo Puerto Albondoz}\\
\end{center}
\vspace*{0.2in}

\centering
\includegraphics[width =7cm]{logo}



\begin{large}
\centering
FACULTAD DE MATEMÁTICAS \\
\end{large} 

\begin{large}
Departamento de Estadística e Investigación Operativa\\
\end{large}


\begin{large}
\centering
Sevilla, Junio 2018\\
 \end{large}
\end{center}


\end{titlepage}

\newpage
\thispagestyle{empty}
%------------------------------------------------------------------------

\tableofcontents
\newpage
\thispagestyle{empty}

\chapter*{Abstract}
\addcontentsline{toc}{chapter}{Abstract}
The sample average approximation (SAA) method is an approach for solving stochastic optimization problems by using Monte Carlo simulation. The basic idea of such method is that we can approximate the expected objetive function by the corresponding sample average function using a random sample. We solve the obtained sample average approximating problem by deterministic optimization techniques, and the process is repeated several times with different samples to obtain candidate solutions along with statistical estimates of their optimality gaps until a stopping criterion is satisfied.


In section 1 we describe the expected value and sample average approximation problems and give a few examples of real cases in which it can be useful. In section 2 we show many results related to convergence of estimators (objective value, optimal solution, etc) under certain assumptions. In section 3  we discuss convergence rates of objetive values. In section 4 we implement the method to study two problems (that involve different random variables) to illustrate the power of the method.
\newpage
\thispagestyle{empty}

\chapter*{Introducci\'on}\label{cap.introduccion}
\addcontentsline{toc}{chapter}{Introducción}
En ocasiones, cuando tratamos de resolver problemas de programación estocástica que involucran valores esperados, tenemos que lidiar un tamaño del espacio muestral excesivamente grande o con cálculos de esperanzas que pueden ser tremendamente costosos desde un punto de vista computacional. En este trabajo desarrollamos una técnica que nos permite en una gran variedad de casos aproximar estos problemas mediante mediante el método de Monte Carlo. Esta técnica se conoce como Método de aproximación por media muestral.

Tras una presentación inicial del método, proseguiremos dando resultados relacionados con la convergencia, bajo hipótesis adecuadas, de las soluciones, valor objetivo y otros estimadores de los problemas muestrales a los correspondientes del problema real, así como una primera aproximación al orden de convergencia del método. 

Finalmente, ejemplificaremos la eficacia del método con experimentos reales. Expondremos dos casos donde utilizaremos una normal multivariante y una mixtura de normales para tener una primera impresión sobre la capacidad de aproximación del método y la utilidad para casos donde, amén de heurística, resulta infactible resolver los problemas por fuerza bruta.


\newpage
\thispagestyle{empty}
%\pagenumbering{arabic} % para empezar la numeración con números
\chapter{El problema del emparejamiento}
\section{El problema del $b$-emparejamiento}
Dado que nuestro objetivo es el estudio paramétrico del problema del emparejamiento y la versión multiobjetivo del mismo, comenzamos este trabajo exponiendo en qué consiste el problema del ejemparejamiento o problema del \textit{matching}. Para ello, comenzamos definiendo algunos conceptos básicos en el marco de la teoría de grafos.
\begin{defi}
Sea $G=(V,E)$ un grafo y sea $S \subset V$, definimos $\delta(S)$ como el conjunto de aristas con un único extremo en $S$. En el caso de un conjunto unitario $\{i\}$, denotamos $\delta(i):=\delta(\{i\})$. Usualmente se denomina \textbf{grado del vértice $i$} al cardinal de $\delta(i)$. Definimos además $\gamma(S)$ como el conjunto de aristas que tienen ambos extremos en $S$.
\end{defi}

En adelante consideramos siempre $G=(V,E)$ un grafo no dirigido. El problema del \textit{matching} consiste en encontrar un subconjunto $M\subset E$ con la propiedad de que en el subgrafo inducido $G(M)=(V,M)$ ningún vértice tenga grado mayor que $1$, es decir, que ninguna arista tenga vértices en común. Naturalmente, este problema es fácilmente generalizable al problema del $b$-emparejamiento o $b$-\textit{matching}, en el cual cada vértice $v$ debe tener un grado no mayor que $b_v$, donde $b_v$ es un entero positivo. El problema original pasaría a ser el caso particular en el que $b_v = 1$ $\forall v \in V$.

\begin{defi}
Sea $G=(V,E)$ un grafo y sea $M\subset E$ un $b$-emparejamiento. Diremos que $M$ es un \textbf{emparejamiento perfecto} si $|\delta(v)|=b_v$ $\forall v \in V$, es decir, si las restricciones se verifican con igualdad.
\end{defi}
Para cada $(u,v)\in E$ podemos considerar el peso o coste $c_{uv}$ asociado. Dependiendo del contexto en el que estemos trabajando estos pesos pueden ser números reales, reales positivos, enteros no negativos, etc. En este trabajo consideraremos que los costes son reales no negativos. Dado un conjunto de aristas $E'\subset E$, tiene sentido considerar
$$
c(E')=\sum_{(u,v)\in E'} c_{uv}
$$
El problema del $b$-emparejamiento de coste máximo o \textit{weighted $b$-matching problem} consiste en encontrar el $b$-emparejamiento que maximiza la función $c(\cdot)$. Si nos ceñimos únicamente a los emparejamientos perfectos, también tiene sentido considerar el problema de encontrar el que tiene peso mínimo. En general, cuando $c_{uv}=1$ $\forall (u,v)\in E$, el problema se denomina de cardinalidad o \textit{cardinality problem}.

El problema del matching puede ser formulado como un problema de programación entera
\begin{align*}
\max_{x} &\; \sum_{(u,v)\in E} x_{uv}c_{uv}  \nonumber\\ 
s.a.\;  &  Ax\leq b \\
& x\in\{0,1\}^n\nonumber
\end{align*}
donde $c$ es el vector de pesos, $A$ es la matriz de incidencia del grafo, $|E|=n$ y la variable $x_{uv}=1$ si la arista $(u,v)$ está en el emparejamiento y $0$ en caso contrario. Nótese que si $G$ es un grafo bipartito entonces $A$ es una matriz totalmente unimodular y, en ese caso, los puntos extremos del poliedro $\{x \in \R^n_+\mid Ax\leq b\}$ son precisamente los $b$-emparejamientos.

Es claro que la formulación como problema de programación entera no suele resultar la más conveniente. La mayoría de técnicas y algoritmos para resolver este problema de manera eficiente utilizan instrumentos basados en la dualidad de la Programación Lineal. Esto es posible gracias a un importante resultado que probó Edmond \cite{edmond}, que pasamos a enunciar.
\begin{thm}
Sea $G=(V,E)$ un grafo. Sea $B$ el conjunto
$$
B = \{S\subset V \mid |S| \text{ es impar},\;|S|\geq 3\}
$$
Entonces, la envolvente convexa del politopo del matching viene dada por
\begin{align*}
\sum_{(u,v)\in\delta(u)} x_{uv} &\leq 1, \quad \forall u\in V\\
x_{uv} &\geq 0\\
\sum_{(u,v)\in \gamma(S)} x_{uv}& \leq \frac{1}{2}(|S|-1)\quad \forall S \in B	
\end{align*}
Además, si nos restringimos a los emparejamientos perfectos, entonces basta considerar únicamente la igualdad en el primer bloque de restricciones.
\end{thm}

Resumiendo nuestras asunciones, consideraremos que $c_{uv}\geq 0$ $\forall (u,v)\in E$, es decir, los pesos son no negativos. Además, imponemos que los grafos estudiados admiten un matching perfecto. Principalmente, vamos a estudiar técnicas para resolver y reoptimiar el el problema del emparejamiento perfecto de coste mínimo o \textit{minimum-cost perfect matching problem} (MCPM). Usando el teorema anterior, este puede ser furmulado como
\begin{align*}
\min_x & \sum_{(u,v) \in E}x_{uv}c_{uv}\\
s.a.&\;\sum_{(u,v)\in\delta(u)} x_{uv} \leq 1, \quad \forall u \in V\\
&\sum_{(u,v)\in \gamma(S)} x_{uv} \leq \frac{1}{2}(|S|-1)\quad \forall S \in B	\\
&x_{uv} \geq 0 \qquad \forall(u,v)\in E
\end{align*}
A pesar de que esta no será nuestra formulación final, pues para ello debemos profundizar en conceptos más complejos de teoría de grafos, tenemos una primera aproximación al problema del matching con una formulación propia de la Programación Lineal. 
\subsection{El problema del emparejamiento paramétrico y biobjetivo}
Una vez que hemos presentado los conceptos básicos para entender el MCPM, podemos definir una clase más general de problemas a los cuáles dedicaremos especial atención durante los procesos de reoptimización. Definimos el \textit{problema del emparejamiento paramétrico respecto de $R\subset E$} con parámetro $\lambda$ como
\begin{align*}
\min_x & \sum_{(u,v) \in E}x_{uv} (c_{uv} + \lambda d_{uv})\\
s.a.&\;\sum_{(u,v)\in\delta(u)} x_{uv} \leq 1, \quad \forall u \in V\\
&\sum_{(u,v)\in \gamma(S)} x_{uv} \leq \frac{1}{2}(|S|-1)\quad \forall S \in B	\\
&x_{uv} \geq 0 \qquad \forall(u,v)\in E
\end{align*}
donde 
$$
d_{uv} = \begin{cases}
1 & (u,v)\in R\\
0 & (u,v)\notin R
\end{cases}
$$
Este problema puede verse como la relajación lagrangiana del problema del matching con una cota superior y es ampliamente estudiado por Ball y Taverna en \cite{balltab}. 

Más generalmente, si consideramos dos funciones de coste $c^1$ y $c^2$ cualesquiera y un parámetro $\lambda$, podemos formular el problema del emparejamiento biobjetivo
\begin{align*}
\min_x & \sum_{(u,v) \in E}x_{uv}(c^1_{uv}+\lambda c^2_{uv})\\
s.a.&\;\sum_{(u,v)\in\delta(u)} x_{uv} \leq 1, \quad \forall u \in V\\
&\sum_{(u,v)\in \gamma(S)} x_{uv} \leq \frac{1}{2}(|S|-1)\quad \forall S \in B	\\
&x_{uv} \geq 0 \qquad \forall(u,v)\in E
\end{align*}
\section{Aplicaciones}
\subsection{El problema de la asignación}
Para ilustrar las distintas aplicaciones que tiene el problema del matchig, comenzamos con un problema clásico de la Investigación Operativa. El problema de asignación o \textit{assigment assignment problem} consiste en encontrar la forma de asignar recursos (máquinas, empleados, etc.) a un cierto conjunto de teareas determinadas con coste mínimo. Se supone que cada recurso se destina a una sola tarea, y que cada tarea es ejecutada por uno solo de los recursos. Es claro que este problema puede formularse como un problema de matching de coste mínimo sobre un grafo bipartito.
\subsection{Aproximación al problema del viajante}
El problema del viajante o \textit{travelling salesman problem} (TSP) en un grafo no dirigido consiste en encontrar el ciclo hamiltoniano de peso mínimo. Computacionalmente hablando, se sabe que es NP-duro. Este problema aparece para responder a la pregunta: ¿Dada una lista de ciudades y las distancias entre cada pareja de ciudades, cuál es la ruta más corta para visitar todas las ciudades y volver a la ciudad de origen? Este problema puede formularse como problema de programación entera, como puede encontrarse en \cite{papa}. 

Aunque no puede utilizarse el problema del matching para resolver este problema, Nicos Christofides encontró un algoritmo \cite{nico} para obtener, bajo ciertas condiciones, una aproximación no mayor que $1.5$ el valor óptimo del TSP. Para este resultado es necesario imponer que las distancias estén en el marco de un espacio métrico, es decir, han de ser simétricas y verificar la desigualdad triangular.

\subsection{•}

\chapter{Algoritmo SAP}
En este capítulo vamos a explicar el funcionamiento del método SAP o \textit{shortest augmenting path} a partir del cuál puede obtenerse un matching perfecto de mínimo coste y, además, nos permitirá realizar una reoptimización eficiente.
\section{Propiedades teóricas}
\begin{defi}
Sea $G=(V,E)$ un grafo, $M$ un matching en $G$ y $v\in V$. Diremos que $v$ es textbf{expuesto} con respecto a $M$ si no es extremo de ninguna arista de $M$.
\end{defi}
\begin{defi}
Sea $G=(V,E)$ un grafo y $M$ un matching en $G$, un \textbf{camino alternante} o \textit{alternating path} con respecto a $M$ es un camino en el cual se van alternando aristas que están en $M$ y fuera del matching. Análogamente pueden definirse \textbf{árboles y ciclos alternantes}.
\end{defi}
\begin{figure}[h!]
\centering
\begin{tikzpicture}[scale=.8,auto=left,every node/.style={circle,fill=blue!10}]
	\node (nA) at (1,10) {};
	\node (nB) at (3,10) {};
	\node (nC) at (5,10) {};
	\node (nD) at (7,10) {};
	\node (nE) at (9,10) {};

  \path[every node/.style={sloped,anchor=south,auto=false}]
        (nA) edge              node {\scriptsize $e\in M$} (nB)            
        (nB) edge              node {\scriptsize  $e\in E\setminus M$} (nC)
        (nC) edge              node {\scriptsize $e\in M$} (nD)
        (nD) edge              node {\scriptsize $e\in E\setminus M$} (nE);     
\end{tikzpicture}
\caption{Camino alternante}
\end{figure}

\begin{defi}
En las condiciones de la definición anterior, se define un \textbf{camino de aumento} o \textit{augmenting path} como un camino alternante cuyos nodos extremos son expuestos.
\end{defi}
Resulta claro que si tenemos un matching $M$ y un camino de aumento con respecto a $M$, podemos obtener un matching de mayor cardinalidad intercambiando el rol que tienen. Las que pertenecen al matching pasan a no pertencer y recíprocramente.

\begin{defi} 
Dado un matching $M$ y un camino de aumento $P$, definimos la operación 
$$
M\oplus P = (M\setminus P)\cup (P\setminus M)
$$
\end{defi}
En \cite{inte} podemos encontrar una demostración del siguiente resultado.
\begin{prop}
Un matching $M$ contiene el máximo número de ejes si y solo si no existe ningún camino de aumento relativo a $M$.
\end{prop}
Dejamos los caminos a un lado y pasamos a definir otro tipo de conceptos relacionados con distintas familias de grafos y subgrafos.
\begin{defi}
Sea $G=(V,E)$ un grafo y $B \subset V$. Definimos el \textbf{grafo inducido por }$B$ como $G[B] = \{B,\gamma(B)\}$.  
\end{defi}
\begin{defi}
Sea $G=(V,E)$ y $H$ un subgrafo de $G$. Si $H$ tiene el mismo conjunto de nodos que $V$, decimos que \textbf{abarca }$G$.
\end{defi}
\begin{defi}
Sea $G=(V,E)$ un grafo y $B\subset V$. Definimos $G \times B=(V_B,E_B)$ como \textbf{el grafo obtenido por la contracción de} $B$, donde $V_B = (V \setminus B)\cup \{v_B\}$ y $E_B$ es el conjunto $E$ salvo que, cada arista con un único extremo en algún vértice de $B$ en $G$, ahora este vértice es $v_B$. Si existen varias solo se mantiene una para no tener un multigrafo. El vértice $v_B$ es llamado \textbf{pseudonodo}. Podemos definir además $M_B:= M\cap E_B$.
\end{defi}

\begin{defi}
Sea $A\subset V$, decimos que es un conjunto \textbf{anidado} si $|A|\geq 3$ y $\forall W,Z\subset A$ se verifica que 
$$
W\cap Z \neq \emptyset \Rightarrow W\subset Z \text{ o bien } Z\subset W
$$
Además, sea $W \in A$, definimos $A[W] =\{Z\in A \mid Z\subset W, Z\neq W\}$.
\end{defi}
\begin{defi}
En las condiciones de la definición anterior, sean $\{W_1,\dotsc,W_n\}$ el conjunto de los elementos maximales de $A$. Entonces definimos
$$
G\times A = (\cdots((G\times W_1)\times W_2)\times \cdots \times W_n)
$$
Notemos que el orden no relevante. Si denotamos $E_A$ al conjunto de aristas del grafo anterior, podemos definir $M_A = M\cap E_A$. 
\end{defi}
\begin{defi}
Diremos que $A \subset V$ anidado es una \textbf{familia de contracción} si verifica además que
$$
G[W]\times A[W] \text{ es abarcado por un ciclo impar $\forall W \in A$}
$$ 
Además, si un elemento $W$ (maximal) de una familia de contracción $A$ es tal que $|M\cap\gamma(W)| = \frac{1}{2}(|W|-1)$ diremos que un \textbf{blossom} (exterior).
\end{defi}
Como veremos posteriormente, estas familias son un concepto capital a la hora de desarrollar un algoritmo para encontrar un matching perfecto cuando el grafo involucrado es no bipartito, que es realmente más simple. Una propiedad fundamental al respecto fue probada también por Edmonds en \cite{edmon2}.
\begin{thm}
Sea $M$ un matching en $G$ y sea $A$ una familia de contracción tal que todo $W\in A$ es un blossom con respecto a $M$. Entonces, todo camino de aumento en $G\times A$ con respecto a $M_A$ induce un camino de aumento $P$ con respecto a $M$.
\end{thm}
\begin{cor}
Sea $A$ una familia de contracción de $G$ y $M_A$ un matching (perfecto) en $G\times A$. Entonces $M_A$ induce (de manera única) un matching (perfecto) $M$ en $G$ tal que $\forall W\in A$ es un blossom con respecto a $M$.
\end{cor}
\subsection{El problema del matching y las familias de contracción}
Los las familias de contracción también son conocidas en la literatura como \textit{hypomatchable set} o \textit{shrinkable set} y en \cite{edmon3} se usan para caracterizarlos de la siguiente manera.
\begin{thm}
Sea $A \subset V$. Entonces $A$ es una familia de contracción si y solo si $\forall u \in A$ $\exists M_i$ matching tal que 
$$
|M_i\cap \gamma(A)| = \frac{1}{2}(|A|-1)
$$
\end{thm}
Esta caracterización es esencial, pues nos permite reformular el Teorema 1.1.1 de la siguiente forma, tal y como prueban Edmonds y Pulleyblank también en \cite{edmon3}.
\begin{thm}
Sea $G=(V,E)$ un grafo y sea $\mathcal{A}(G)$ el conjunto de todos las familias de contracción de $G$. Entonces, la envolvente convexa del politopo del matching viene dada por
\begin{align*}
\sum_{(u,v)\in\delta(u)} x_{uv} &\leq 1, \quad \forall u\in V\\
x_{uv} &\geq 0\\
\sum_{(u,v)\in \delta(S)} x_{uv}& \geq 1 \quad \forall S \in \mathcal{A}(G)	
\end{align*}
Además, si nos restringimos a los emparejamientos perfectos, entonces basta considerar únicamente la igualdad en el primer bloque de restricciones.
\end{thm}
Utilizando este teorema, podemos formular el MCPM como
\begin{align*}
\min_x & \sum_{(u,v) \in E}x_{uv}c_{uv}\\
s.a.&\;\sum_{(u,v)\in\delta(u)} x_{uv} \leq 1, \quad \forall u \in V\\
&\sum_{(u,v)\in \delta(S)} x_{uv} \geq 1\quad \forall S \in \mathcal{A}(G)	\\
&x_{uv} \geq 0 \qquad \forall(u,v)\in E
\end{align*}
Con esta formulación del MCPM vamos a considerar el problema dual asociado.
\begin{align*}
\max_{y} &\; \sum_{v\in V} y_v - \sum_{S\in \mathbb{A}(G)} y_S\\
s.a.&\;y_u+y_v + \sum_{S\in B,(u,v)\in \delta(S)}y_S  \geq c_{uv} \quad \forall (u,v)\in E\\
&y_S\geq 0 \quad \forall S\in \mathbb{A}(G)
\end{align*}
\begin{defi}
Sea $y$ una solución factible del problema dual anterior. Definimos el coste reducido de un eje $(u,v)$ con respecto a $y$
$$
c_{uv}'(y) = c_{uv} - y_u -y_v - \sum_{S\in \mathcal{A}(G),(u,v)\in \delta(S)} y_S
$$ 
\end{defi}
\section{Fundamentos del algoritmo}
\begin{defi}
Sea $M$ un matching en $G$ y sea $P$ un camino (o ciclo) alternante con respecto a $M$. Definimos la longitud de $P$ respecto a $M$ como
$$
l(P) = \sum_{(u,v)\in P\setminus M}c_{uv} - \sum_{(u,v)\in P\cap M}c_{uv}
$$
\end{defi}
\begin{defi}
Sea $M$ un matching en $G$ y sea $s\in V$ expuesto con respecto a $M$. Definimos $\mathcal{P}(M)$ como el conjunto de todos los caminos de aumento con respecto a $M$.
\end{defi}
\begin{prop}
En las condiciones de la definición anterior se tiene que
$$
c(M\oplus P) = l(P) + c(M)
$$
\end{prop}

Usando estas definiciones podemos probar ahora una primera caracterización de optimalidad para el MCPM.
\begin{thm}
Un matching perfecto es de coste mínimo si y solo si no existe un ciclo alternante de longitud negativa.
\end{thm}
\begin{dem}
Demostremos el enunciado equivalente, un matching perfecto no es de coste mínimo si y solo si existe un ciclo alternante de longitud negativa.

Si existe un ciclo alternante de longitud negativa $P$ con respecto a un matching $M$ perfecto, entonces se deduce de la proposición anterior que $c(M\oplus P) < c(M)$. 

Sea un matching $M$ perfecto que no sea de coste mínimo y sea entonces $M'$ un matching de coste mínimo. Sabemos por teoría de grafos que existe un conjunto finito de ciclos alternantes $P_1,\dotsc,P_r$ con respecto a $M$ tales que
$$
M' = M \oplus P_1 \oplus \cdots \oplus P_r
$$
Si $l(P_i) \geq 0$ $\forall i=1,\dotsc, r$ entonces $c(M') \geq c(M)$, pero por hipótesis $c(M')<c(M)$. Por tanto, ha de existir $i \in \{1,\dotsc,r\}$ tal que $l(P_i)<0$. \qed
\end{dem}
A continuación enunciamos y probamos un resultado crucial para el algoritmo.
\begin{thm}
Sea $M$ un matching que no admite ningún ciclo alternante de longitud negativa y sea $P$ el camino alternante de menor coste (\textit{shortest alternanting path}) entre todos los relativos a $M$. Entonces $M\oplus P$ no admite ningún ciclo alternante de longitud negativa.\qed
\end{thm}
\begin{dem}
Sea $K$ un ciclo alternante de longitud negativa con respecto a $M\oplus P$. Podemos suponer que $K\cap P \neq \emptyset$, pues en otro caso $K$ sería un ciclo alternante de longitud negativa con respecto a $M$. Definamos entonces
$$
P' = (P\setminus K)\cup (K\setminus P)
$$
Entonces $P'$ es un camino de aumento con respecto a $M$ y
$$
c(M \oplus P') = c(M \oplus P \oplus K) < c(M \oplus P)
$$
Por tanto $l(P')<l(P)$ (con respecto a $M$), lo cuál es una contradicción.
\qed
\end{dem}
\subsection{Transformaciones admisibles}
\begin{defi}
Sea $M$ un matching y sea $\mathcal{P}(M)$ definido en la Definición 2.2.2. Una transformación $T:c_{uv}\to c_{uv}'$ es llamado \textbf{admisible} si
$$
l'(P)\geq 0\quad \forall P \in \mathcal{P}(M)
$$
Además, se define $d(P):= l(P)-l'(P)$.
\end{defi}
A partir de la definición de transformación admisible podemos probar un criterio de optimalidad para que un camino de aumento sea de longitud mínima.
\begin{thm}
Sea $T$ una transformación admisible y $P_0 \in \mathbb{P}(M)$ con $l'(P_0) =0$ y 	$d(P_0)\leq d(P)$ $\forall P \in \mathcal{P}(M)$. Entonces $P_0$ es el camino de aumento con menor longitud con respecto a la función de coste $c$ y matching $M$.
\end{thm}
\begin{dem}
Se deduce inmediatamente a partir de la definición y las hipótesis a través de la siguiente cadena de igualdades / desigualdades.
\begin{align*}
l(P_0) &= l'(P_0) + d(P_0)\\
&=d(P_0)\leq d(P) \leq l'(P) + d(P) = l(P)  \quad \forall P \in \mathcal{P}(M)
\end{align*}
\qed
\end{dem}
\begin{defi}
Sea $B=\{S\subset V \mid |S| \text{ es impar, }|S|\geq 3\}$. A cada $u\in V$ y $S\in B$ podemos asociar unos pesos $y_u \in \R$  e $y_S \in \R_{\geq 0}$. Definimos la transformación $T:c_{uv}\to c'_{uv}$ tal que
$$
c_{uv}' =c_{uv}-y_u-y_v - \sum_{S\in B, (u,v)\in \delta(S)}y_S
$$
\end{defi}
\begin{nota}
Notemos que si consideramos $y$ una solución factible del problema dual del matching, la transformación asocia a los costes de cada arista los costes reducidos asociados a $y$, pues puede establecerse a través de las condiciones de holgura complementaria que si $S$ no pertenece a una familia de contracción, $y_S =0$.
\end{nota}
\begin{defi}
En las condiciones de la definición anterior, definimos el \textbf{grafo de admisibilidad} $G' = (V,E')$ donde $E' = \{(u,v)\in E\mid c'_{uv}=0\}$.
\end{defi}
\begin{defi}
Una transformación admisible $T$ del tipo definida en la Definición 2.2.4 se dice \textbf{adecuada} con respecto a $M$ si existe una familia de contracción $A$ en $G'$ tal que
\begin{align*}
(u,v)\in M &\Rightarrow c'_{uv} = 0\\
y_S>0&\Rightarrow S\in A \text{ y } |M\cap \gamma(S)| = \frac{1}{2}(|S|-1)
\end{align*}
\end{defi}
Con estas definiciones en mente pasamos a probar el criterio de optimalidad que utilizaremos en el nuestro algoritmo, cuya demostración puede encontrarse en \cite{derigs}.
\begin{thm}
Sea $M$ un matching y sea $T$ una transformación adecuada. Sea $A = \{S \in B\mid y_S>0\}$. Si existe $P\in \mathbb{P}(M)$ tal que $c'(P)=0$ y $P_A$ es un camino de aumento respecto $M_A$, entonces $P$ es el camino alternante de menor costo en $\mathcal{P}(M)$.
\end{thm}
\section{Algoritmo del camino de aumento más corto}
Antes de dar el algoritmo es preciso notar que utilizaremos un etiquetado de manera que todo nodo de $G\times A$ tendrá tres etiquetas temporales $d_v^+,d_v^-$ y $p(v)$ con la información:
\begin{itemize}
\item $d_v^+$ es el límite superior de la longitud del camino alternante más corto que conecta $v$ con un cierto nodo $s$ y tiene un número par de ejes.
\item $d_v^-$ es análogo para un número impar de ejes.
\item $p(v)$ es el predecesor de $v$ en el camino alternante que define $d_v^-$. 
\end{itemize}
Estas distancias y medidas están tomadas respecto de los costes $c'_{uv}$. Pasemos a describir el algoritmo.
\begin{align*}
\text{Inicialización: }& y_u:= y_S:= 0 \quad \forall u \in V,\; S\in B\\
& M:=A:=\emptyset\\
& c_{uv}' = c_{uv} \quad \forall (u,v)\in E\\
\text{Paso 1: }& \text{Determinar si existe un nodo $s$ no emparejado en $G\times A$ con respecto a $M_A$.}\\
&\text{Si no existe ninguno, ir al Paso 8.}\\
&\text{En caso contrario, ir al Paso 2.}
\end{align*}
\begin{align*}
\text{Paso 2: }& \text{[Inicialización de una fase.]}\\
&d_v^+ := \infty \quad \forall j\in G\times A,\; v\neq s\\
&d_s^+:=0\\
&d_v^-:=c'_{sv},\;p(v):=s\quad \forall j\in G\times A,\; v\neq s\\
&d^-_s:=\infty\\
&\text{Etiquetamos $s$ con una $S$ y avanzamos al Paso 3.}\\
\text{Paso 3: }& \text{Determinamos:}\\
&\delta_1 = \min\{d_v^-\mid \text{$v$ no está etiqetado ni emparejado}\},\\
&\delta_2 =\min\{d_v^-\mid \text{$v$ no está etiquedato pero sí emparejado}\},\\
&\delta_3 =\min\{\frac{1}{2}(\delta_v^+ + \delta_v^-)\mid \text{$v$ está $S$-etiquetado}\}\\
&\delta_4 =\min\{y_B+d_B^- \mid \text{$v_B$ es un pseudonodo $T$-etiquetado}\}\\
&\delta\;= \min\{\delta_1,\delta_2,\delta_3,\delta_4\}\\
&\text{Si $\delta=\delta_1$ ir al Paso 7}\\
&\text{Si $\delta=\delta_2$ ir al Paso 4}\\
&\text{Si $\delta=\delta_3$ ir al Paso 5}\\
&\text{Si $\delta=\delta_4$ ir al Paso 6}\\
\text{Paso 4: }&\text{[Crecimiendo de un arbol alternante.]}\\
&\text{Sea $u$ el nodo que define $\delta$ y sea $(u,v)\in M_A$. Entonces el árbol crece}\\
&\text{asignando un $T$-etiquetado a $u$ y un $S$ etiquetado a $v$.}\\
&\text{Establecemos $d^+_v:=\delta$ y escaneamos $v$, es decir,}\\
&d^-_k:= \min\{d_k^-,d_j^++c'_{vk}\}\quad \forall k\neq u,v\\
&p(k):=v \text{ si } d_k^-= d_v^++c'_{vk}\\
&\text{Ir al Paso 3}\\
\text{ Paso 5: }&\text{[Contracción de un blossom.]}\\
&\text{Sea $u$ el nodo que define $\delta$ y sea $v:=p(u)$. Introduciendo la arista $(u,v)$}\\
&\text{al arbol alternante genera un blossom $B$. Este blossom se contrae en un}\\
&\text{pseudono $v_B$. Establecemos:}\\
&A:=A-\{B\},\;d^+_B:=\delta,\;y_B:=0 \text{ y asignamos una $S$-etiqueda a $v_B$}\\
&\text{Para todo $i \in B$ $S$-etiqueado}\\
&y_i:=y_i + (\delta-d_i^+)\\
&c_{ij}' := c_{ij}' -(\delta-d_i^+) \quad \forall j\neq i\\
&\text{Para todo $i \in B$ $T$-etiquedado}\\
&y_i:=y_i + (d^-_i-\delta)\\
&c_{ij}' := c_{ij}' -(d_i^- -\delta) \quad \forall j\neq i\\
&\text{Escaneamos $v_B$ (como en el Paso 4 para el nodo $v$) y vamos al Paso 3.}
\end{align*}
\begin{align*}
\text{Paso 6: }&\text{[Expansión de un pseudonodo.]}\\
&\text{Sea $v_B$ el pseudonodo que define $\delta$. Establecemos}\\
&c_{uv}':=c_{uv}'+y_B\quad \forall (u,v)\in\delta(B)\\
&y_B:=0\\
&\text{y extendemos el matching $M_A$. El blossom $B$ se particiona en dos}\\
&\text{caminos, uno tiene un número par de ejes y el otro un número impar}\\
&\text{Todos los nodos $v$ del camino par se etiquetan con $S$ y $T$ alternativamente}\\
&\text{y $d_v^+:=d_v^-:=\delta$. Los nodos $S$-etiqueados son escaneados. Los nodos $u$}\\
&\text{del camino impar no se etiquetan. Establecemos}\\
&d^-_v=\min\{d_i^++c_{iu}'\mid i\text{ está $S$ etiquetado}\}\\
&\text{y $p(v):=i$ si el mínimo se alcanza en $i$. Si no existe, $d_v^-:=\infty$.}\\
&\text{Finalmente, $A:=A\setminus\{B\}$. Vamos al Paso 3.}\\
\text{Paso 7: }&\text{[Aumento.]}\\
&\text{Hemos detectado un camino de aumento $P_A$ con respecto a $M_A$.}\\
&\text{Actualizamos $M_A = M_A \oplus P_A$ y definimos y definimos para todos los}\\
&\text{(pseudo)nodos $S$-etiquetados $v$:}\\
&y_v:=y_v+(\delta-d_v^+)\\
&c_{uv}' := c_{uv}' - (\delta-d_i^+) \quad u\neq v\\
&\text{y para los $T$-etiquedados}\\
&y_v:=y_v-(\delta-d_i^-)\\
&c_{uv}'=c_{uv}'+(\delta-d_i^-)\\
&\text{Eliminamos todas las etiquetas y vamos al Paso 1}\\
\text{Paso 8: }& \text{Expandimos todos los pseudonodos y extendemos el matching $M_A$ a un}\\
&\text{matching perfecto $M$ en $G$. Este matching es óptimo.} 
\end{align*}
\chapter{Análisis de sensitividad}
\section{Método de reoptimización}
En esta sección vamos a presentar un método de reoptimización para el problema del MCPM y cómo aplicarlo al problema biobjetivo. Antes de presentar este método hemos revisado diversos análsis de postoptimalidad como el de Ball y Taberna \cite{balltab} o el de Weber \cite{webber}, pero el que ofrece mejores resultados computacionales y supone una simplificación importante en el número de casos a considerar es el que ofrece Derigs en \cite{derigs2}***. 

Nuestro objetivo en esta sección es, suponiendo que tenemos un matching óptimo $M$ para el MCPM, vamos a alterar el coste de ciertos ejes. Denotemos $c_{uv}^{new}$ los costes nuevos para estas aristas y sea $c_{uv}'^{new}$ los correspondientes costes reducidos respecto de una solución factible dual $y$. Antes de proseguir, precisemos algunas definiciones.
\begin{defi}
Sea $y$ una solución factible del dual descrito gracias al Teorema 2.1.3. Consideremos $c'_{uv}:=c'_{uv}(y)$, es decir, los costes reducidos asociados a $y$. Consideremos $G'$ el grafo de admisibilidad asociado. Diremos que $y$ es \textbf{fuértemente dual factible} si $y_S>0$ implica que $S$ es contractible en $G'$.
\end{defi}
\begin{defi}
En las mismas condiciones de la definición anterior, diremos que $(M,y)$ es un \textbf{par compatible} si $y$ es fuertemente dual factible y verifican
\begin{align*}
(u,v)\in M &\Rightarrow c'_{uv} = 0\\
y_S>0&\Rightarrow |M\cap \gamma(S)| = \frac{1}{2}(|S|-1)
\end{align*}
\end{defi}
Se puede probar que el método que hemos establecido anteriormente comienza con un par compatible $(M,y)$ y encuentra el menor camino de aumento $P$ y una solución fuertemente dual factible $y'$ de manera que $(M\oplus P,y')$ es un par compatible.
\begin{prop}
Sea $M$ un matching óptimo y sea $u\in V$ un nodo arbitrario. Entonces existe una solución $y$ fuertamente dual factible tal que $(M,y)$ es un par compatible y además $y_S=0$ $\forall S \in \mathcal{A}(G)$ tal que $u \in S$.
\end{prop}
\begin{dem}
La demostración es constructiva. Podemos suponer que tenemos $(M,y)$ un par compatible obtenido por el algoritmo SAP. Si $y$ verifica la propiedad adicional, no hay nada que demostrar. En caso contrario, sea $v$ tal que $(u,v)\in M$. Vamos a construir un grafo $\overline{G}$ introduciendo dos nuevos nodos artificiales $r,t\notin V$ y aristas $(r,u)$ y $(v,t)$ con costes 
$$
c_{ru}:= y_u+\sum_{u\in S}y_S \qquad c_{vt} = y_v + \sum_{u\in S}y_S  +\sum_{v\in S}y_S 
$$
Además, extendemos $y$ al nuevo grafo $\overline{G}$ de manera que $y_r:=y_t:=0$ e $y_S = 0$ $\forall S \in \mathcal{A}(\overline{G})$ tal que $S\cap\{r,t\}\neq \emptyset$. Esta extensión hace que $(M,y)$ sea un par compatible en $\overline{G}$. Aplicamos ahora el algoritmo del SAP. Como $M$ era óptimo en $G$, el único camino de aumento que puede generarse es $\{(r,u),(u,v),(v,t)\}$. Dado que $c_{vt}>\sum_{u\in S}y_S$, antes de que se detecte este camino la solución dual será alterada a $y'$ de manera que $y'_S = 0$ $\forall S \in \mathcal{A}(G)$ tal que $u \in S$. Obtenemos por tanto un par $(M,y')$ que verifica la propiedad deseada. \qed
\end{dem}
Es importante hacer una consideración para entender por qué es importante el teorema anterior. Sea $y$ una solución fuerfuertemente dual factible, definimos $\mathcal{A}(y)=\{S\in \mathcal{A}(G)\mid y_S>0\}$. Naturalmente, si existe algún $S \in \mathcal{A}(G)$ con $v\in S$ tal que $y_S >0$, el nodo $v$ no estará -estará contraído en algún pseudonodo- en el grafo $G\times \mathcal{A}(y)$. Por tanto, siguiendo el proceso de la demostración anterior, siempre podemos encontrar un par compatible $(M,y)$ de manera que un determinado vértice $v$ esté en el grafo $G\times \mathcal{A}(y)$. Pasamos a ver el algoritmo de repotimización. 

Supongamos que todas las aristas cuyos costes son alteras están en $\delta(v)$.
\begin{align*}
\text{Inicialización: }& \text{Sea $(M,y)$ un par compatible óptimo para MCPM}\\
\text{Paso 1: }& \text{Si $v$ no está contraído en $G\times \mathcal{A}(y)$ vamos al Paso 2.}\\
&\text{En otro caso, aplicamos el procedimiento anterior}\\
&\text{ y denotamos $y:=y'$. Vamos al Paso 2.}\\
\text{Paso 2: }& \text{Para todo $(u,v) \in M$ establecemos}\\
&\overline{M}:=M\setminus\{(u,v)\}\\
&\delta:=\min\{c_{vk}'^{new}\mid (v,k)\in \delta(v)\}<0\\
&c_{vk}:=c_{vk}^{new} \quad \forall (v,k)\delta (v)\\
&y_v:=y_v+\delta\\
&\text{Aplicamos el algoritmo SAP y obtenemos }\\
&\text{PENDIENTEEEEEEEEEEEEEE}
\end{align*}
\subsection{Reoptimización del problema biobjetivo}
En el matching paramétrico, tal y como vimos en la Subsección 1.1.1, la función de coste es $c_{uv}(\lambda) = c^1_{uv} + \lambda c^2_{uv}$. Por tanto, para aplicar el procedimiento anterior, tenemos que considerar $c_{uv}^{new} = c^1_{uv} + (\lambda + \Delta\lambda) c^2_{uv}= c_{uv}(\lambda+\Delta\lambda)$. 
\section{Mantenemiento de la optimalidad}
Estamos especialmente interesados en el caso del matching biobjetivo en saber cuánto puede variar $\Delta \Lambda$ de manera que nuestro matching $M$ continúe siendo óptimo. Para realizar este análisis, consideremos el problema de Programación Lineal:
\begin{align*}
\min_x & \sum_{(u,v) \in E}x_{uv}c_{uv}(\lambda)\\
s.a.&\;\sum_{(u,v)\in\delta(u)} x_{uv} \leq 1, \quad \forall u \in V\\
&\sum_{(u,v)\in \delta(S)} x_{uv} \geq 1\quad \forall S \in \mathcal{A}(G)	\\
&x_{uv} \geq 0 \qquad \forall(u,v)\in E
\end{align*}
Sea $A$ la matriz de restricciones sin considerar la introducción de las variables de holgura. Si $|V|=2n$, una base $B$ óptima tendrá, en primer lugar, $n$ columnas de $A$ correspondientes a las aristas del matching según el indexado que hayamos hecho de las mismas; y otras $s$ columnas de la base canónica para las holguras cuyas desigualdades se verifiquen de manera estricta. Sean $i_1,\dotsc,i_n$ los índices asociados a las aristas del matching, sean $j_1,\dotsc,j_k$ los índices asociados a dichas variables de holgura, sea $N$ el número total de restricciones del problema tras la introducción de las holguras y denotemos por $e_p$ el $p$-ésimo vector de la base canónica de $R^N$, entonces la matriz $B$ es precisamente
$$ B= 
\begin{pmatrix}
A_{\cdot i_1} & \cdots & A_{\cdot i_n} & e_{j_1} & \cdots & e_{j_k}
\end{pmatrix}
$$
\newpage
\begin{thebibliography}{9}
\bibitem{edmond} 
Edmonds, J. (1965). Maximum matching and a polyhedron with 0, 1-vertices. Journal of Research of the National Bureau of Standards B, 69, 125--130. 

 
\bibitem{pruebaed} 
Schrijver, Alexander. (1983). Short proofs on the matching polyhedron. Journal of Combinatorial Theory, Series B. 34. 104-108. 10.1016/0095-8956(83)90011-4. 
 
\bibitem{balltab}
Ball, M. O.,  Taverna, R. (1985). Sensitivity analysis for the matching problem and its use in solving matching problems with a single side constraint. Annals of Operations Research, 4(1), 25-56.

\bibitem{papa}
Papadimitriou, C.H.; Steiglitz, K. (1998), Combinatorial optimization: algorithms and complexity, Mineola, NY: Dover, pp.308-309.

\bibitem{nico}
Nicos Christofides, Worst-case analysis of a new heuristic for the travelling salesman problem, Report 388, Graduate School of Industrial Administration, CMU, 1976

\bibitem{inte}
George L. Nemhauser and Laurence A. Wolsey. 1988. Integer and Combinatorial Optimization. Wiley-Interscience, New York, NY, USA.
\bibitem{edmon2}
J. Edmonds, “Paths, Trees, and Flowers,” Can. J. Math., 17,449-467 (1965).

\bibitem{edmon3}
Pulleyblank, William and Edmonds, Jack. (1970). Facets of 1-Matching Polyhedra. 10.1007/BFb0066196. 

\bibitem{derigs}
Derigs, U. (1981), A shortest augmenting path method for solving minimal perfect matching problems. Networks, 11: 379-390. doi:10.1002/net.3230110407

\bibitem{webber}
Weber, G. M. (1981), Sensitivity analysis of optimal matchings. Networks, 11: 41-56. doi:10.1002/net.3230110105
\end{thebibliography}
\end{document} 
